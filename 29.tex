
Nelle applicazioni è di uso frequente una formula empirica per calcolare le perdite
chiamata formula di Steinmetz (1892)
$$
P_{\text{ist}} = K_\text{ist}\cdot B^\alpha_\text{max}\ f
$$
dove $\alpha$ è tipicamente $1.6$ mentre il coefficiente di isteresi $K_\text{ist}$ dipende
dalla grandezza specifica che si vuole indicare se $\si{\watt\per\meter^3}$ o $\si{\watt\per\kilo\gram}$. La seconda è tipicamente utilizzata in ingegneria elettrica
e denominata \textit{cifra di perdita}, dipende esclusivamente dal materiale.

Il materiale utilizzato nelle macchine elettriche, solitamente Fe-Si possiede 
un'area del ciclo di isteresi inferiore a quella del ferro puro a parità di $M_s$ 
magnetizzazione di saturazione.

\subsection{Misura del ciclo di isteresi di un materiale}
Si utilizzerà un mezzo sperimentale composto da vari blocchi, partendo dall'alimentazione
di rete si utilizza un VARIAC, un autotrasformatore abbassatore in questo caso
che fornisce al sistema una tensione minore o uguale a quella di alimentazione.

Si effettuano infine due avvolgimenti destrorsi al campione ferromagnetico da studiare
$N_1$ ed $N_2$ in base al numero di spire. Ai terminali di $N_2$ si collega un ``blocco
integratore'' costituito da un resistore in serie e un capacitore in derivazione.

19:35


