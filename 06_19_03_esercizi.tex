
\paragraph{Calcolo di antitrasformate di funzioni razionali}
Supponiamo che nei circuiti siano presenti soltanto alcune tipologie di generatori, come ad esempio
generatori stazionari, sinusoidali... In queste ipotesi le trasformate di Laplace sono funzioni 
razionali, del tipo:
$$
F(s) = \frac{N(s)}{D(s)}\ N(s),D(s) \text{ polinomi con grado } n\leq d
$$

In generale la $F(s)$ è esprimibile come:
$$
F(s) = K + \frac{N^*(s)}{D(s)}\ K \text{ costante, } n^* < d
$$
la sua antitrasformata sarà:
$$
L^{-1}[F(s)] = k\delta(t) + L^{-1}\left[\frac{N^*(s)}{D(s)}\right] 
$$
per determinare la soluzione vanno ricercate le radici del denominatore, chiamate \textit{poli}
della funzione, mentre le radici del numeratore vengono chiamati \textit{zeri}.

La funzione può avere \textbf{poli semplici} di molteplicità 1:
$$
F^*(s) = \frac{k_1}{s-p_1} + \frac{k_2}{s-p_2} + ... + \frac{k_n}{s-p_n} = \sum_{i=1}^{N}\frac{k_i}{s-p_i}
$$
$k_i$ viene chiamato residuo i-esimo della $F^*(s)$ e si calcola con:
\begin{equation}
\lim_{s\to p_i} \left[(s-p_1)F^*(s)\right]
\label{eq:formula_poli_semplici}
\end{equation}

Esempio:
$$
F^*(s) = \frac{1}{0.5s^2+2.7s+2} = \frac{2}{s^2+5.4s+4} = \frac{k_1}{s+0.886} + \frac{k_2}{s+4.514}
$$
Applicando la \ref{eq:formula_poli_semplici}:
$$
k_1 = \frac{2}{-0.886+4.514} = 0.551
$$
Una volta trovati tutti i poli semplici, l'antitrasformata della funzione, diviene una somma di
antitrasformate dei singoli rapporti:
$$
L^{-1}[F^*(s)] = L^{-1}\left[\sum_{i=1}^{N}\frac{k_i}{s-p_i}\right] = \sum_{i=1}^{N} k_ie^{p_i t}
$$
\textbf{Poli multipli}, ossia con molteplicità maggiore di 1, supponiamo ad esempio $p_1$ sia un polo multiplo:
$$
F^*(s) = \frac{k_{11}}{(s-p_1)^2} + \frac{k_{12}}{s-p_1} + \sum_{i=3}^{N} \frac{k_i}{s-p_i}
$$
$$
k_{11} = \lim_{s\to p_1} \left[(s-p_1)^2F^*(s)\right]
$$
per determinare il secondo residuo del polo invece si deve eseguire il limite della derivata:
$$
k_{12} = \lim_{s\to p_1} \frac{d}{ds} \left[(s-p_1)^2 F^*(s)\right]
$$
Esempio:
$$
F^*(s) = \frac{2s+5}{(s+2)^2} = \frac{k_{11}}{(s+2)^2} + \frac{k_{12}}{s+2}
$$
$$
k_{11} = \lim_{s\to-2} \frac{(2s+5)\cancel{(s+2)^2}}{\cancel{(s+2)^2}} = 1
$$
$$
k_{12} = \lim_{s \to -2} \frac{d}{ds} [2s+5] = 2
$$
$$
F^*(s) = \frac{1}{(s+2)^2}+\frac{2}{s+2},\ L^{-1}[F^*(s)] = te^{-2t}+2e^{-2t}
$$

\textbf{Poli complessi e coniugati:}
Troviamo le radici del denominatore della seguente funzione:
$$
F^*(s) = \frac{2s-1}{s^2+4s+5} = \frac{k_1}{s+2+j} + \frac{k_2}{s+2-j}
$$
$$
k_1 = \frac{2(-2-j)-1}{\cancel{-2}-j\cancel{+2}-j} = \frac{-4-2j-1}{-2j} = \frac{5+2j}{2j} = 1 - \frac{5}{2}j
$$
$$
k_2 = \bar{k_1} = 1+\frac{5}{2}j
$$
$$
F^*(s) = \frac{1-\frac{5}{2}j}{s+2+j} + \frac{1+\frac{5}{2}j}{s+2-j}
$$
\\
$$
L^{-1}[F^*(s)] = \left(1-\frac{5}{2}j\right)e^{-(2+j)t} + \left(1+\frac{5}{2}j\right)e^{-(2-j)t} =
$$
$$
= \left(1-\frac{5}{2}j\right)e^{-2t}e^{-jt} + \left(1+\frac{5}{2}j\right)e^{-2t}e^{jt} =
$$
$$
= e^{-2t}\left[\frac{\left(e^{-jt}+e^{jt}\right)}{2}\cdot 2 - \frac{5}{2}je^{-jt}+\frac{5}{2}je^{jt}\right] =
$$
$$
= e^{-2t}\left[\frac{\left(e^{-jt}+e^{jt}\right)}{2}\cdot 2 + \frac{5}{2j}e^{-jt}-\frac{5}{2j}e^{jt}\right] =
$$
$$
= e^{-2t}\left[2\cos(t)-5\sin(t)\right]
$$
\newpage
\textbf{Esercizio 1 :}

Determinare la risposta all'impulso e al gradino
\begin{figure}[h]
\centering
\begin{circuitikz}
\draw (0,0) to [I,l=$j(t)$] (0,2)
            to (2,2) to [R,l_=$R$,i_=$i_R$] (2,0)
            to (0,0);
\draw (2,2) to (4,2) to [L,l_=$L$,i_=$i_L$] (4,0) to (2,0);
\draw (4,2) to (6,2) to [C,l_=$C$,v^=$v(t)$,i_=$i_C$] (6,0) to (4,0);
\end{circuitikz}
\caption{Circuito RLC parallelo}
\end{figure}
$$
R = \SI{10}{\kilo\ohm}\ L = \SI{100}{\milli\henry}\ C = \SI{10}{\micro\farad}
$$
$$
j(t) = u(t)
$$
$$
j(t) = \frac{v}{r} + i_L + i_C,\ v = L\frac{di_L}{dt},\ i_c = C\frac{dv_c}{dt}
$$
$$\begin{cases}
C\frac{dv_c}{dt} = i_c = u(t) -\frac{v}{r} - i_L\\
L\frac{di_L}{dt} = v(t)
\end{cases}
$$
Si risolve l'integrale generale esprimendo la $i_L$ in funzione della $v(t)$:
$$
L\frac{d}{dt}\left[u(t) -\frac{v}{r} - C \frac{dv}{dt}\right] = v(t)
$$
Dato che si esegue l'analisi per $t > 0$ la derivata di $u(t)$ è nulla.
$$
\begin{cases}
\frac{d^2v}{dt^2} + \frac{1}{RC}\frac{dv}{dt} + \frac{v}{LC} = 0\\
v(0^+)=0\\
\frac{dv}{dt}(0^+) = \frac{1}{C}\left[j(0^+) - \frac{v}{r}(0^+) - i_L(0^+)\right] = \frac{1}{C}
\end{cases}
$$
Si determinano ora le frequenze naturali del sistema:

$$
\lambda^2 + \frac{1}{RC}\lambda + \frac{1}{LC} = 0
$$
$$
\lambda_{1,2} = -\frac{1}{2RC} \pm \sqrt{\left(\frac{1}{2RC}\right)^2-\frac{1}{LC}}
$$
$$
\lambda^2 + \frac{1}{10^4\cdot10^{-5}}\lambda + \frac{1}{0.1\cdot10^{-5}} = 0
$$
$$
\lambda_{1,2} = -5 \pm j 1000
$$
Quindi abbiamo modi naturali periodici smorzati ossia:
$$
v(t) = e^{-5t}(k_1\cos(1000t)+k_2\sin(1000t))
$$
Imponiamo ora le condizioni iniziali:
$$
v(0^+) = 0 \Rightarrow k_1 = 0
$$
$$
\frac{dv}{dt}(0^+) = \frac{1}{C} \Rightarrow \left[-5e^{-5t}k_2\sin(1000t) + k_2e^{-5t}\cos(1000t)\cdot1000\right]_{t=0} = \frac{1}{C}
$$
$$
k_2\cdot1000 = \frac{1}{10^{-5}} \Rightarrow k_2 = 100
$$
$$
v(t) = 100e^{-5t}\sin(1000t)\cdot u(t) = g(t)
$$
Risposta all'impulso:
$$
h(t) = \frac{dg}{dt} = -500e^{-5t}\sin(1000t) + 100e^{-5t}\cdot1000\cdot\cos(1000t) = 
$$
$$
500e^{-5t}\left[200\cos(1000t)-\sin(1000t)\right]\cdot u(t)
$$
Verifica della $h(t)$, utilizziamo la procedura per analizzare la risposta all'impulso, sostituendo
i bipoli dinamici, il condensatore diventerà un corto circuito e l'induttore un circuito aperto:
\begin{figure}[h]
\centering
\begin{circuitikz}
\draw (0,0) to [I,l=$\delta(t)$] (0,2)
            to (2,2) to [R,l_=$R$,i_=$i_R$] (2,0)
            to (0,0);
\draw (2,2) to (4,2) to [open,v_=$v_L$] (4,0) to (2,0);
\draw (4,2) to (6,2) to [short,i_=$i_C$] (6,0) to (4,0);
\end{circuitikz}
\caption{Circuito RLC con bipoli sostituiti}
\end{figure}

$i_C$ sarà chiaramente pari a $\delta(t)$ e $v_L = 0$.
Determiniamo la discontinuità di $v_C$:
$$
v_c(0^+) = \frac{1}{C}\int_{0^-}^{0^+}\delta(\tau)d\tau = \frac{1}{C}
$$
$$
i_L(0^+) = \frac{1}{L}\int_{0^-}^{0^+} v_L(\tau)d\tau = 0
$$
$$
\begin{cases}
v(0^+) = \frac{1}{C}\\
\frac{dv}{dt}(0^+) = \frac{1}{C}\left(-\frac{1}{RC}\right) = -\frac{1}{RC^2}
\end{cases}
$$
Le frequenze naturali sono le stesse, quindi l'equazione sarà dello stesso tipo: moto periodico smorzato
$$
v(0^+) = \frac{1}{C} = k_1 = 10^5
$$
$$
\frac{dv}{dt}(0^+) = 1000k_2 - \frac{5}{C} = -\frac{1}{RC^2} \Rightarrow 1000k_2 = \frac{5}{C} - \frac{1}{RC^2} = 5\cdot10^5 - \frac{10^{10}}{10^4} = 5\cdot10^5-10^6 = -5\cdot10^5
$$
$$
k_2 = -500
$$
In conclusione:
$$
h(t) = e^{-5t}\left[10^5\cos(1000t)-500\sin(1000t)\right] = 500e^{-5t}\left[200\cos(1000t)-\sin(1000t)\right]\cdot u(t)
$$

\textbf{Esercizio 2 :}
\begin{figure}[h]
\centering
\begin{circuitikz}
\draw (0,0) to [R,l=$R_1$] (0,2);
\draw (0,0) to (2,0) to [V,l=$e(t)$] (2,2) to (0,2);
\draw (2,2) to [C,l=$C$]  (4.5,2) to [L,l=$L$]  (4.5,0) to [R,l=$R_2$,v=$v(t)$]  (2,0);
\draw (4,2) to (6,2) to [R,l=$R_1$] (6,0) to (4,0);
\end{circuitikz}
\end{figure}

Determina la risposta impulsiva nel dominio del tempo con la trasformata di Laplace
$$
R_1 = \SI{1}{\ohm},\ R_2 = \SI{3}{\ohm},\ L = \SI{3}{\henry},\ C = \SI{1}{\farad}
$$
Determiniamo le condizioni iniziali dovute al generatore impulsivo $e(t) = \delta(t)$

%circuito ausiliario
\begin{figure}[h]
\centering
\begin{circuitikz}
\draw (0,0) to [R,l=$R_1$] (0,2);
\draw (0,0) to (2,0) to [V,l=$\delta(t)$] (2,2) to (0,2);
\draw (2,2) to [short,i=$i_C$]  (4.5,2) to [open,v=$v_L$]  (4.5,0) to [R,l=$R_2$,v=$v(t)$]  (2,0);
\draw (4,2) to (6,2) to [R,l=$R_1$] (6,0) to (4,0);
\end{circuitikz}
\caption{Circuito ausiliario}
\end{figure}
Determiniamo quindi le variabili di stato $v_L$ e $i_C$:
$$
v_L = -\frac{R_1}{R_1+R_2}\delta(t) = -\frac{1}{4}\delta(t)
$$
$$
i_C = -\frac{\delta(t)}{R_1+R_2} = -\frac{1}{4}\delta(t)
$$
$$
i_L(0^+) = \frac{1}{L} \int_{0^-}^{0^+} -\frac{\delta}{4}d\tau = \SI{-1/12}{\ampere} 
$$
$$
v_C(0^+) = \frac{1}{C} \int_{0^-}^{0^+}-\frac{\delta}{4}d\tau = -\frac{1}{4} = \SI{-0.25}{\volt}
$$
quindi
$$
v(t) = -\frac{R_2}{R_1+R_2}\delta(t) = -\frac{3}{4}\delta(t) = -0.75\cdot\delta(t)
$$

Determiniamo ora le  equazioni di stato sfruttando le condizioni iniziali appena ricavate:

\begin{figure}[h]
\centering
\begin{circuitikz}
\draw (0,0) to [R,l=$R_1$] (0,2);
\draw (0,0) to (2,0) to [short] (2,2) to (0,2);
\draw (2,2) to [V,v=$v_C$,i^>=$i_C$]  (4.5,2) to [I,v>=$v_L$,l=$i_L$]  (4.5,0) to [R,l_=$R_2$,v^=$v(t)$]  (2,0);
\draw (4,2) to (6,2) to [R,l=$R_1$] (6,0) to (4,0);
\end{circuitikz}
\caption{Circuito resistivo associato}
\end{figure}
Applicando il principio di sovrapposizione degli effetti:
$$
\begin{cases}
i_c = -\frac{1}{R_1+R_2}v_C +\frac{R_1}{R_1+R_2}i_L = C\frac{dv_C}{dt}\\
v_L = -\frac{R_1}{R_1+R_2}v_C - \frac{R_1R_2}{R_1+R_2}i_L = L\frac{di_L}{dt}
\end{cases}
$$
Variabile di uscita:
$$
v(t) = R_2i_c = -\frac{R_2}{R_1+R_2}v_c + \frac{R_1R_2}{R_1+R_2}i_L = -\frac{3}{4}v_c +\frac{3}{4}i_L
$$

Riscriviamo ora la ODE in forma matriciale:
$$
\begin{pmatrix}
 C & 0 \\
 0 & L
\end{pmatrix}
\frac{d}{dt}
\begin{bmatrix}
v_c \\
i_L
\end{bmatrix}
=
\begin{bmatrix}
-\frac{1}{R_1+R_2} & \frac{R_1}{R_1+R_2}\\
-\frac{R_1}{R_1+R_2} & -\frac{R_1R_2}{R_1+R_2}
\end{bmatrix}\cdot
\begin{bmatrix}
v_c \\
i_L
\end{bmatrix}
$$
La forma matriciale generica è:
$$
D\frac{d\underline{x}}{dt} = A\underline{x}
$$
In questo caso la matrice A è l'opposta della matrice di rappresentazione ibrida del doppio bipolo visto 
dalle porte alle quali sono collegati i generatori $v_C$ e $i_L$.
$$
\underline{x}(t) = \underline{v}e^{\lambda t} \Rightarrow \lambda D \underline{v} e ^{\lambda t} = A\cdot \underline{v}e^{\lambda t} \Rightarrow A\cdot \underline{v} = \lambda D\cdot\underline{v}
$$
Quello appena citato è un problema agli autovalori generalizzato per le matrici $A$ e $D$.
$$
D^{-1}A\cdot\underline{v} = \lambda\underline{v} \Rightarrow D^{-1}A = 
\begin{bmatrix}
-\frac{1}{(R_1+R_2)C} & \frac{R_1}{(R_1+R_2)C} \\
-\frac{R_1}{L(R_1+R_2)} & -\frac{R_1R_2}{(R_1+R_2)L}
\end{bmatrix}
=
\begin{bmatrix}
-\frac{1}{4} & \frac{1}{4} \\
-\frac{1}{12} & -\frac{1}{4}
\end{bmatrix}
$$
Calcoliamo quindi gli autovalori della matrice risolvendo il seguente problema:
$$
\text{det}\left(A-\lambda I\right) = 
\begin{vmatrix}
a_{11}-\lambda & a_{12}\\
a_{21} & a_{22}-\lambda
\end{vmatrix}
= (a_{11}-\lambda)(a_{22}-\lambda) -a_{12}a_{21} =
$$
$$
= a_{11}a_{22} - a_{11}\lambda - a_{22}\lambda + \lambda^2 - a_{12}a_{21} = \lambda^2 - tr(A) + det(A) =0
$$

$$
\lambda_{1,2} = -\frac{1}{4} \pm \sqrt{\frac{1}{16}+\left(\frac{1}{16}+\frac{1}{48}\right)} = -\frac{1}{4}\pm \frac{\sqrt{3}}{12} = - 0.25 \pm 0.1433j
$$
Abbiamo ancora una volta una soluzione sinusoidale smorzata
$$
v_c(t) = e^{-\sigma t}\left(k_1\cos(\omega t) + k_2\sin(\omega t)\right),\ \sigma = 0.25,\ \omega = 0.1443
$$
Calcoliamo direttamente anche la derivata in forma esplicita, necessaria al calcolo di $i_C$:
$$
\frac{dv_c}{dt}(t) = -\sigma e^{-\sigma t}\left(k_1\cos(\omega t)+k_2\sin(\omega t)\right) + e^{-\sigma t}(-k_1\omega \sin(\omega t) + k_2 \omega \cos(\omega t)) =
$$
$$
=e^{-\sigma t} \left[(k_2\omega -\sigma k_1)\cos(\omega t) - (\sigma k_2 + k_1\omega ) \sin(\omega t) \right]
$$
\begin{equation}
\frac{dv_c}{dt}(0^+)= -\sigma k_1 + \omega k_2
\label{eq:dvc_esercizio2}
\end{equation}

Imponiamo ora le condizioni iniziali:
$$
v_c(0^+) = \SI{-0.25}{\volt} \Rightarrow k_1 = -0.25
$$
$$
\frac{dv_c}{dt}(0^+) = -\frac{1}{4}v_c(0^+) + \frac{1}{4}i_L(0^+) = \frac{1}{16} + \frac{1}{4}\left(\frac{1}{12}\right) = \frac{1}{24} = 0.0417
$$
usando la \ref{eq:dvc_esercizio2}
$$
-0.25 k_1 + 0.1443k_2 = 0.0417
$$
$$
k_2 = -\frac{\sqrt{3}}{12} = -0.1443
$$
In definitiva:
$$
v_c(t) = e^{-0.25 t}\left[-0.25\cos(0.1443 t) - \frac{\sqrt{3}}{12}\sin(0.1443 t)\right]
$$
Applichiamo l'equazione determinata dal circuito resistivo associato:
$$
v(t) = R_2 i_c - \frac{R_2}{R_1+R_2}\delta(t) = R_2C\frac{dv_c}{dt} - \frac{R_2}{R_1+R_2}\delta(t) =
3\frac{dv_c}{dt} - \frac{3}{4}\delta(t)=
$$
$$
= \frac{1}{8}e^{-\frac{1}{4}t}\left[\cos\left(\frac{\sqrt{3}}{12}t\right) + \sqrt{3}\sin\left(\frac{\sqrt{3}}{12}t\right)\right] - \frac{3}{4}\delta(t)
$$
$$
v(t) = 0.125 e^{-0.25 t}\left[\cos(0.1443 t) + 1.732\sin(0.1443 t)\right] - 0.75\cdot\delta(t)
$$

\textbf{Soluzione con la L-trasformata}

Utilizziamo le impedenze operatoriali

$$
V = -V_s\frac{Z_{R2}}{Z_{R2}+Z_C+\frac{Z_L+Z_{R1}}{Z_L+Z_{R1}}} = -V_s\frac{R_2}{R_2 + \frac{1}{sC} + 
\frac{sLR_1}{R_1+sL}} =
$$
$$
= -\frac{3}{3+\frac{1}{s} + \frac{3 s}{3s + 1}} = \frac{-3 s (3s + 1)}{3s(3s+1) + 3s+1 + 3s^2} = 
$$
$$
= \frac{- 3s(3s+1)}{12s^2+6s+1} = H(s)
$$
Antitrasformiamo:
$$
L^{-1}\left[\frac{-3s(3s+1)}{12s^2+6s +1}\right] = -3 \frac{d}{dt}L^{-1} \left[\frac{3s+1}{12s^2+6s+1}\right] = -\frac{1}{4} \frac{d}{dt} L^{-1} \left[\frac{3s+1}{s^2+0.5s + \frac{1}{12}}\right]
$$
$$
F(s) = \frac{k_1}{s+0.25+0.1443j} + \frac{k_2}{s+0.25-0.1443j}
$$

...

La trasformata inversa viene eseguita con le funzioni esponenziali:
$$
L^{-1}[F(s)] = \frac{3+\sqrt{3}j}{2}e^{-0.25 t}e^{-0.1443 t} + \frac{3-\sqrt{3}j}{2}e^{-0.25 t}e^{j0.1443 t} =$$
$$
= e^{-0.125 t}\left[3\cos(0.1443 t) + \sqrt{3}\sin(0.1443 t)\right]\cdot u(t)
$$

$$
h(t) = -\frac{1}{4} \frac{d}{dt}L^{-1}[F(s)] = 0.125 e ^{-0.25 t}[\cos(0.1443 t) + \sqrt{3}\sin(0.1443 t)] - 0.75\delta(t)
$$
