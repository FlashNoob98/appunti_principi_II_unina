
\paragraph{Legge di Ampére-Maxwell (locale) su superfici di discontinuità}
Presa una superficie $S$ di normale $\hat{n}$ e indichiamo la regione al di sopra di $S$ con
(2) e quella al di sotto con (1).
Si considera una linea chiusa $\Gamma$ orientata di tipo \textit{rettangolare} ortogonale alla
superficie, e la superficie orlante è chiamata $S_\Gamma$ con versore normale $\hat{m}$ orientato 
secondo la regola della mano destra.

$$
\oint_\Gamma \vec{B}\cdot\hat{t} dl = \mu_0 i_{S_\Gamma} + \mu_0 \iint_{S_\Gamma} \varepsilon_0 \frac{\partial\vec{E}}{\partial t}\cdot\hat{m}dS 
$$
$i_{S_\Gamma}$ sono le correnti concatenate alla superficie $S_\Gamma$.
Ricordando che il tratto di linea elementare ha una lunghezza $L$ e un'altezza $h$, 
il vettore densità di corrente superficiale $\vec{J}_S$ e $\vec{J}$ un'eventuale
densità di corrente volumetrica, allora
$$
\oint_\Gamma\vec{B}\cdot\hat{t}dl = \vec{B}_1\cdot\hat{t}_1L + \vec{B}_2\cdot\hat{t}_2L +
\int_B^C\vec{B}\cdot\hat{t}dl + \int_D^A\vec{B}\cdot\hat{t}dl = \mu_0\vec{J}_S\cdot\hat{m} + 
\mu_0\varepsilon_0\frac{\partial\vec{E}}{\partial t}\cdot\hat{m}hL + \mu_0 \vec{J}\cdot\hat{m}hL
$$

passando al limite per $h/l \to 0,\ \hat{t}_1 = \hat{t} = \hat{t}_2,\ \hat{t} = \hat{n}\times\hat{m}$

$$
\hat{n}\times\hat{m}\cdot(\vec{B}_1-\vec{B}_2) = \mu_0 \vec{J}_S \cdot \hat{m} \Leftrightarrow 
\hat{n}\times(\vec{B}_2-\vec{B}_1)\cdot\hat{m} = \mu_0 \vec{J}\cdot\hat{m}\ 
\forall\ \hat{m},\Gamma,S_\Gamma
$$

$$
\hat{n}\times(\vec{B}_2-\vec{B}_1) = \mu_0\vec{J}_S\ \text{su} \ S
$$

Si conclude quindi che la componente tangenziale del campo $\vec{B}$ ad una superficie $S$ è discontinua
in presenza di correnti superficiali.

\paragraph{Principio di conservazione della carica}
Sia preso un volumetto $\Delta\Omega$ centrato attorno ad un punto $P$
$$
\frac{\oiint_{\partial\Delta\Omega} \vec{J}\cdot\hat{n} dS}{\text{Vol}(\Delta\Omega)} = -\frac{ \iiint_{\Delta\Omega} \frac{\partial\rho}{\partial t} dV}{\text{Vol}(\Delta\Omega)} \stackrel{\text{Vol}(\Delta\Omega)\to 0} {\Rightarrow}
\nabla\cdot\vec{J} = -\frac{\partial\rho}{\partial t} \in \Omega
$$
nei punti regolari.

Sulle superfici di discontinuità invece ripetendo i ragionamenti sulla superficie \textit{``monetina''}
si ottiene:
$$
\hat{n}\cdot\left(\vec{J}_2-\vec{J}_1\right) = -\frac{\partial\sigma}{\partial t} \in S
$$
si trascurano infatti i flussi laterali che tendono a 0 se $h\to 0$.

La componente normale del vettore densità di corrente è discontinua se su $S$ è presente una densità
di carica variabile nel tempo.

\paragraph{Sintesi delle equazioni di Maxwell in forma locale} 
\begin{align*}
&\text{Nei punti}\text{ regolari}  & &\text{Nei punti}\text{ irreg}\text{olari}\\
&\nabla\cdot\vec{E} = \frac{\rho}{\varepsilon_0} & &\hat{n}\cdot(\vec{E}_2-\vec{E}_1) = \frac{\sigma}{\varepsilon_0}\\
&\nabla\cdot\vec{B} = 0 & &\hat{n}\cdot(\vec{B}_2-\vec{B}_1) = 0\\
&\nabla\times\vec{E} = -\frac{\partial\vec{B}}{\partial t}& &\hat{n}\times(\vec{E}_2-\vec{E}_1) = 0\\
&\nabla\times\vec{B} = \mu_0\left(\vec{J} + \varepsilon_0\frac{\partial\vec{E}}{\partial t}\right)& &\hat{n}\times(\vec{B}_2-\vec{B}_1) = \mu_0\vec{J}_S \\
&\nabla\cdot\vec{J} = -\frac{\partial\rho}{\partial t} & &\hat{n}\cdot(\vec{J}_2-\vec{J}_1) = -\frac{\partial \sigma}{\partial t}
\end{align*}

\subsection{Elettrostatica}
Tutte le cariche sono ferme e invariabili nel tempo
$$
\oiint_\Sigma \vec{E}\cdot\hat{n}dS = \frac{Q_{\Omega_\Sigma}}{\varepsilon_0}\ \forall\Sigma\ \text{Legge di Gauss}
$$
$$
\oint_\Gamma \vec{E}\cdot\hat{t}dl = 0 \ \forall\Gamma\ \text{Legge di Faraday-Neumann}
$$
Si vuole vedere ora come ricavare le configurazioni di campo elettrico associati a distribuzioni
di cariche assegnate, utilizzando le equazioni di Maxwell in forma integrale.

\subparagraph{Distribuzione di carica volumetrica a simmetria sferica}

Sia $\Omega$ la sfera di raggio $R$, un punto $P(r,\theta,\varphi)$ al suo interno.

La densità di carica è quindi:
$$
\rho(P) = 
\begin{cases}
\rho_0,\ 0\leq r\leq R\\
0,\ r > R
\end{cases}
$$

Il campo elettrico in generale dipenderà da $r,\theta$ e $\varphi$ e si può scomporre in 
tre componenti lungo gli assi di queste variabili.

La distribuzione $\rho$ non dipenderà da $\theta,\varphi$, ossia sarà simmetrica rispetto agli assi.
Di conseguenza anche le componenti del campo $\vec{E}$ dipenderanno solo da $r$.
Le componenti del campo $E_\theta$ ed $E_\varphi$ seguiranno le rotazioni dei rispettivi angoli.
Supponendo che ci sia una componente $E_\theta$ diversa da 0 e si suppone di ruotare la sfera di
180\textdegree\ la componente cambierebbe di segno ma rimarrebbe invariata la distribuzione di cariche $\rho$,
ne consegue che $E_\theta = 0$; un discorso analogo può essere effettuato per la componente $E_\varphi$.

In definitiva si conclude che l'unica componente del campo elettrico è quella lungo 
il versore $r$ quindi $E_r(r)$.
Si può applicare la legge di Gauss ad una superficie sferica $\Sigma$ esterna di raggio maggiore del raggio
della sfera carica.

Ricordando che:
\begin{align*}
dl_1 &= dr \\
dl_2 &= rd\theta \\
dl_3 &= r\sin\theta d\varphi \\
dS &= dS_1 = dl_2dl_3 = r^2\sin\theta d\theta d \varphi
\end{align*}
$$
\oiint_\Sigma\vec{E}\cdot\hat{n}dS = \int_0^\pi d\theta \int_0^\pi d\varphi r^2 \sin\theta E_r = 
\frac{1}{\varepsilon_0} \int_0^\pi d\theta \int_0^\pi d\varphi \int_0^r \rho(r) r^3\sin\theta dr =
$$
$$
= \frac{1}{\varepsilon_0}\iiint_{\Omega_\Sigma} \rho d V = 2\pi r^2 \left[- \cos \theta\right]_0 ^\pi 
E_r (r) = 
\begin{cases}
  2 \pi [-\cos\theta]_0^\pi \frac{\rho_0}{3\varepsilon_0}r^3 & r<R \\
  2 \pi [-\cos \theta]_0^\pi \frac{\rho_0}{3\varepsilon_0}R^3 & r \geq R
\end{cases} =
$$
$$
= 4 \pi r^2 E_r(r) = \begin{cases}
\frac{4\pi r^3 \rho_0}{3\varepsilon_0}, & r < R \\
\frac{4\pi R^3 \rho_0}{3\varepsilon_0}, & r \geq R
\end{cases}
$$
$$
E_r(r) = \frac{\rho_0 r}{3 \varepsilon_0}, \ \ r < R
$$
$$
E_r(r) = \frac{\rho_0 R^3}{3 \varepsilon_0}\frac{1}{r^2}, \ \ r \geq R
$$
\begin{center} % plot di un grafico 2D andamento del campo elettrico
\begin{tikzpicture}
\begin{axis}[
axis lines = left,
xlabel = $r/R$,
ylabel = $\frac{E_r}{E_R}(r)$,
ymax = 3,
]
\addplot [
domain = 0:1,
samples = 2,
] {x};
\addplot[
 domain = 1:3,
 samples = 50,
] {x^-2};
\end{axis}
\end{tikzpicture}
\end{center}
Si osserva che se $r\geq R$ 
$$
\vec{E}(r) = \frac{Q}{4\pi\varepsilon_0} \frac{1}{r^2} \vec{e}_r,\ \ Q = \frac{4\pi }{3}R^3\rho_0
$$
Come se la carica $Q$ fosse puntiforme nell'origine.
Facendo tendere $R\to 0 $ e $\rho_0 \to \infty$ in modo da mantenere $Q$ finito, si ottiene
il campo della carica puntiforme.

\paragraph{Campo elettrico prodotto da una distribuzione di carica qualunque}
Sia $\rho(p')dV$ la carica elementare in $\Delta\Omega$, volumetto infinitesimo della regione $\Omega$.
Si prenda un punto $p$ all'esterno della regione, $\vec{r}_p$ e $\vec{r}_{p'}$ sono i rispettivi vettori
che puntano a $p$ e $p'$.
Si calcola il campo elettrico emesso dalla carica puntiforme in $p$.
$$
\vec{dE}(p) = \frac{\rho(p')dV}{4\pi\varepsilon_0}\cdot\frac{1}{|\vec{r}_p-\vec{r}_{p'}|^2} \cdot\frac{\vec{r}_p-\vec{r}_{p'}}{|\vec{r}_p-\vec{r}_{p'}|}
$$
L'ultimo termine è il versore diretto da $p'$ a $p$ indicando il verso del campo elettrico.
In forma più compatta:
$$
\vec{dE}(p) = \frac{\rho(p')dV}{4\pi \varepsilon_0} \frac{\vec{r}_p-\vec{r}_{p'}}{|\vec{r}_p-\vec{r}_{p'}|^3}
$$
Per calcolare il campo generato dall'intera regione $\Omega$:
$$
\vec{E}(p) = \frac{1}{4\pi\varepsilon_0} \iiint_\Omega \rho(p')  \frac{\vec{r}_p-\vec{r}_{p'}}{|\vec{r}_p-\vec{r}_{p'}|^3} dV
$$
L'espressione è valida sia se il punto $p$ è interno ad $\Omega$ o meno. Bisogna prestare attenzione
al caso in cui $p=p'$ dato che la funzione integranda risulta singolare ma resta finito l'integrale
perchè esteso ad un volume.
Se la regione $\Omega$ è al finito $(\text{diam}(\Omega)<\infty)$ si dimostra che 
$$
\lim_{|\vec{r}_p| \to \infty} |\vec{E}(p)| = 0,\ \ \vec{E}(p) \propto \frac{1}{|\vec{r}_p|^2}
$$
È semplice da osservare:
$$
|\vec{r}_p| \gg |\vec{r}_{p'}| \Rightarrow |\vec{r}_p -\vec{r}_{p'}| \simeq |\vec{r}_p| \Rightarrow
\vec{E}(p) \simeq \frac{1}{4 \pi \varepsilon_0}\frac{Q}{|\vec{r}_p|^2}
$$

\newpage
\paragraph{Distribuzione di carica con simmetria cilindrica}
Indefinita lungo $z$ e di raggio $R$.
Ogni punto $P= (r,\varphi,z)$ definito in coordinate cilindriche, la densità di carica $\rho(p)$:
$$
\rho(p) = \begin{cases}
\rho_0 & \text{se }  r \leq R\\
0 & \text{se }  r > R
\end{cases}
$$
Simmetrica di rotazione intorno l'asse $z$ ed è invariante per traslazione lungo l'asse $z$ (o $l_3$).

Per questo motivo il campo $\vec{E}$ dipende ancora da una sola componente, $\vec{E}(p) = \vec{E}(r)$
mentre $E_\varphi = E_z = 0$. $E_\varphi$ è nulla perchè una rotazione di 180\textdegree\ attorno
l'asse $z$ invertirebbe il segno del campo lasciando $\rho$ invariata; $E_z = 0$ perchè una rotazione
attorno a $\vec{e}_r$ invertirebbe il segno di $E_z$ lasciando $\rho$ invariata.
Queste affermazioni sono contraddittorie e l'unico modo per validarle è imporre che le componenti
siano nulle. In conclusione:
$$
\vec{E}(p) = E_r(r)\vec{e}_r
$$
Si ricorda che
$$
dS = dS_1 = dl_2 dl_3 = rd\varphi dz
$$
Applicando la legge di Gauss ad una superficie cilindrica con distribuzione di raggio $R$ e distribuzione $\rho$ mediante un cilindro esterno o interno di raggio $r$:
$$
\oiint_\Sigma\vec{E}\cdot\hat{n}dS = \int_0^{2\pi}rd\varphi \int_0^L dz E_r(r) = \frac{1}{\varepsilon_0}
\iiint_{\Omega_\Sigma}\rho dV = \frac{1}{\varepsilon_0} \int_{0}^{r}dr\int_{0}^{2\pi} rd\varphi\int_0^L \rho_0 dz 
$$
$$
\oiint_\Sigma\vec{E}\cdot\hat{n}dS = 2\pi L E_r(r) r
$$
1:17
Quando $r > R$ 
$$
E_r(r) = \frac{\rho_0}{2\varepsilon_0} \frac{\pi R^2}{\pi} \frac{1}{r} = 
\frac{\lambda}{2\pi\varepsilon_0} \frac{1}{r}
$$

\paragraph{Distribuzione di carica a simmetria piana}

$$
\oiint_\Sigma \vec{E}\cdot\hat{n}dS = \iint_{A_1} dxdy E_z(z)\hat{n}\cdot\vec{e}_z + 
\iint_{A_2} dxdy E_z(-z)\hat{n}\cdot\vec{e}_z = SE_z(z) -SE_z(-z) = 2 S E_z(z) =
$$
$$
=\frac{1}{\varepsilon_0} \iiint_{\Omega_\Sigma} \rho dV = \frac{1}{\varepsilon_0} \int_{-z}^{z}dz \iint_A dxdy\rho_0
$$




