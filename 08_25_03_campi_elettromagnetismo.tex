\section{Campi conservativi e rotazionali}
\paragraph{Rotore} di un campo vettoriale $\vec{v}(P)$:

Consideriamo una linea chiusa $\Gamma$ all'interno del dominio $\Omega$ ed una superficie
$S_\Gamma$ (di orlo $\Gamma$).

Supponiamo che questa superficie passi su un punto $P$ e consideriamo il versore normale alla 
superficie orientato con la regola della mano destra rispetto al versore tangente della linea $\Gamma$.
Consideriamo inoltre:
$$
\lim_{\text{Area}(S_\Gamma)\to 0}\left(\frac{\oint_{\Gamma} \vec{v}\cdot\hat{t}dl}{\text{Area}(S_\Gamma)}\right) \stackrel{\text{def}}{=} \left(\nabla\times\vec{v}(P)\right)\cdot\hat{n}
$$
Nel caso in cui questo limite esista e sia finito e indipendente dalla forma di $\Gamma$, per definizione
diciamo che questa è la componente normale di un campo vettoriale chiamato \textit{rotore} di $\vec{v}$,
$(\nabla \times \vec{v})\cdot \hat{n}$.

In coordinate cartesiane si ricava il rotore con:

$$
\nabla \times \vec{v} =
\begin{vmatrix}
\vec{e_x} & \vec{e_y} & \vec{e_z} \\
\frac{\partial}{\partial x} & \frac{\partial}{\partial y} & \frac{\partial}{\partial z} \\
v_x & v_y & v_z
\end{vmatrix}
$$

$$
\nabla\times{\vec{v}}(P) = \left(\frac{\partial v_z}{\partial y} - \frac{\partial v_y}{\partial z}\right)\vec{e_x} -
\left(\frac{\partial v_z}{\partial x} - \frac{\partial v_x}{\partial z}\right)\vec{e_y} +
\left(\frac{\partial v_y}{\partial x} - \frac{\partial v_x}{\partial y}\right)\vec{e_z}
$$

\paragraph{Teorema di Stokes}
Dato un campo $\vec{v} \in \Omega$ sufficientemente regolare, $\Gamma$ chiusa in $\Omega$.

Se calcolo la circuitazione alla linea gamma
$$
\oint_\Gamma \vec{v}\cdot\hat{t} dl = \iint_{S_\Gamma} \nabla\times \vec{v} \cdot \hat{n} dS
$$

Si possono dunque introdurre i campi conservativi e irrotazionali:
$\vec{v}(P) \in \Omega$ si dice conservativo (per la circuitazione) quando
$$
\oint_\Gamma \vec{v}\cdot\hat{t} dl = 0\ \ \forall\ \Gamma \in \Omega
$$

Questa condizione equivale a dire che presi due punti nello spazio $A$ e $B$ e prese due linee aperte che connettono questi due punti, allora
$$
\int_{A\gamma'B} \vec{v}\cdot\hat{t}dl = \int_{A\gamma''B} \vec{v}\cdot\hat{t}dl\ \ \forall A,B\in\Omega,\ \forall \gamma',\gamma'' \in \Omega
$$

Un campo conservativo può essere espresso come gradiente di una funzione potenziale:
$$
\vec{v}(P) = -\nabla\varphi(P)
$$
con $\varphi(P)$ funzione potenziale, questo equivale a dire che:
$$
\int_{A\gamma B} \vec{v}\cdot \hat{t} dl = \int_{A\gamma B} -\nabla\varphi\cdot\hat{t} dl =
-\int_{A}^B d\varphi = \varphi(A) - \varphi(B)
$$
Il gradiente di una funzione per uno spostamento infinitesimo equivale al differenziale della funzione.

\paragraph{Campi irrotazionali}
$\vec{v}(P) \in C^1$ irrotazionale se 
$$
\nabla\times \vec{v}(P) = 0\  \forall P \in \Omega
$$
I campi irrotazionali soddisfano dunque il principio di deformazione del contorno, in analogia
ai campi solenoidali che soddisfano il principio di deformazione della superficie.
Preso un dominio $\Omega$ e due linee in esso contenute $\gamma_1$ e $\gamma_2$,
se ipotizzo che 
$$
\vec{v}(P) : \nabla\times\vec{v}(P) =0\ \forall P \in \Omega
$$
ipotizzata una superficie $S\gamma_1\gamma_2$ compresa tra le due curve allora utilizzando
il teorema di Stokes
$$
\iint_{S\gamma_1\gamma_2} \nabla\times\vec{v}\cdot\hat{n}dS = 0 = \oint_{\gamma_1} \vec{v}\cdot\hat{t} dl - \oint_{\gamma_2} \vec{v}\cdot\hat{t}dl = 0\ \forall\gamma_1,\gamma_2\in \Omega
$$

Il segno ``$-$'' all'interno dell'equazione è dovuto al diverso riferimento della superficie rispetto alla
curva $\gamma_2$.

Quando conservatività e irrotazionalità sono equivalenti?
Se $\vec{v}(P)$ è conservativo in $\Omega $ tutti gli integrali lungo linee chiuse sono nulli, 
allora $\nabla\times\vec{v}=0$

Il viceversa:
$$
\begin{cases}
\vec{v}(P) \text{ irrotazionale in }\Omega \\
\Omega \text{ a connessione lineare semplice}
\end{cases}
\Rightarrow \vec{v}(P) \text{ è conservativo in }\Omega
$$

\paragraph{Teorema di decomposizione di Helmhotz} (in tutto lo spazio)

Preso un campo $\vec{v}(P),\ P \in R^3$ normale all'infinito ossia che il modulo del campo tenda
a 0 quando la distanza tende all'infinito, in tale ipotesi può essere decomposto nella somma
di un campo solenoidale e un campo conservativo (entrambi normali all'infinito).

$$
\vec{v}(P) = \vec{v}_\text{sol} + \vec{v}_\text{cons},\ 
\oint_{\Gamma}\vec{v}_\text{cons}\cdot\hat{t} dl = 0\ \forall\ \Gamma,\ \oiint_{\Sigma}
\vec{v}_\text{sol}\cdot\hat{n}dS = 0\ \forall\ \Sigma 
$$
\textit{Se sono note tutte le circuitazioni su linee chiuse dello spazio e tutti i flussi uscenti da
superfici chiuse per un campo $\vec{v}(P)$ normale all'infinito, allora $\vec{v}(P)$ è univocamente
determinato}

\textbf{Operatore NABLA} ($\nabla$)
$$
\nabla = \frac{\partial}{\partial x}\vec{e_x} + \frac{\partial}{\partial y}\vec{e_y} + \frac{\partial}{\partial z}\vec{e_z}
$$
Si esprime il gradiente di un campo scalare $f(P)$ con: $\nabla f$ come se facessi il prodotto
termine a termine dei componenti dell'operatore $\nabla$
$$
\text{grad} f(P) = \nabla f = \frac{\partial f}{\partial x}\vec{e_x}
$$

\textbf{Divergenza}
$$
\text{div}\vec{v}(P) = \nabla\cdot\vec{v} = \frac{\partial v_x}{\partial x} + 
$$

\textbf{Rotore}
$$
\text{rot}\vec{v}(P) = \nabla \times \vec{v}
$$

\textbf{Identità vettoriali:}
\begin{align*}
\nabla\cdot(f\vec{v}) &= f\nabla\cdot\vec{v} + \vec{v}\cdot \nabla f \\
\nabla(f g) &= f\nabla g + g\nabla f \\
\nabla \times (f\vec{v}) &= f\nabla\times\vec{v} + \nabla f \vec{v} \\
\nabla\cdot (\vec{v}\times\vec{w}) &= \vec{w}\cdot\nabla\times\vec{v} - \vec{v}\cdot\nabla\times\vec{w}
= \vec{w}\cdot(\nabla\times\vec{v}) + \vec{v}\cdot (\vec{w}\times\nabla)
\end{align*}

Operatori differenziali del secondo ordine:
$ \vec{v}(P) \in C^2(\Omega) $ e considero combinazioni possibili di grad,div e rot per ottenere
operatori del secondo ordine:
\begin{align*}
\nabla f &\text{ scalare in un vettore}\\
\nabla\cdot\vec{v} &\text{ vettore in uno scalare}\\
\nabla\times\vec{v} &\text{ vettore in un vettore}
\end{align*}
quindi:
\begin{align*}
\nabla\times\nabla f = 0 &\text{ perchè $\nabla f$ è conservativo $\Rightarrow$ irrotazionale}\\
\nabla\cdot\nabla\times\vec{v} = 0 &\text{ dimostra in coordinate cartesiane}\\
\nabla\cdot \nabla f = \nabla^2 f &\text{ laplaciano}
\end{align*}
\textbf{Operatore laplaciano}
$$
\nabla^2 f = \frac{\partial^2f}{\partial x^2} + \frac{\partial^2f}{\partial y^2} + \frac{\partial^2f}{\partial z^2}
$$
Laplaciano vettore:
$$
\nabla\nabla\cdot\vec{v} - \nabla\times\nabla\times\vec{v} = \vec{\nabla}^2\vec{v}
$$

$$
\vec{\nabla}^2\vec{v} = \vec{\nabla}^2v_x\vec{e_x} + ...
$$

\section{Richiami di Elettromagnetismo}
L'elettromagnetismo è una delle 4 interazioni fondamentali che governano l'universo, le altre
sono le interazioni forti, riguardanti i quark, le interazioni deboli riguardano gli elettroni e i 
neutrini, le interazioni elettromagnetiche (che riguardano le cariche) e le interazioni gravitazionali
che coinvolgono le masse.

\subsection{La carica elettrica}
Possono essere di due specie, positive e negative e sono associate a interazioni di tipo repulsivo tra
cariche della stessa specie e attrattive tra cariche di segno opposto.
Si parla in questo caso di cariche ferme, altirmenti sarebbero soggette ad altre tipologie di forze.
L'Unità di misura è il Coulomb [\si{\coulomb}].
La carica è invariante per una particella in quiete o in moto.

La carica è quantizzata, ossia considerando una vportzione di materia in una certa regione 
dello spazio vuoto, la carica contenuta $\Delta Q = N^+q_p + N^-q_e\ \ N^+,N^- \in N$
e i valori di $q_p$ e $q_e$ 

Il principio di conservazione della carica per un sistema chiuso elettricamente afferma che in un tale 
sistema la quantità di carica netta $Q(t) = Q^+ + Q^- = Q_0\ \forall\ t$

\paragraph{Forza agente su una carica in moto}

Sia dato un sistema di riferimento nello spazio, in un certo punto P dello spazio ci sia una carica $q$
che sta percorrendo una traiettoria con un certo vettore di velocità $\vec{v}(P,t)$, questa
carica è sottoposta ad una forza elettromagnetica chiamata \textbf{forza di Lorentz}
$$
\vec{F}(p,t) = q\left[\vec{E}(P,t)+\vec{v}(P,t)\times\vec{B}(P,t)\right]
$$
dove $\vec{B}$ è il campo di induzione magentica anche detto densità di flusso magnetico.

Ricordando la dinamica newtoniana: $\vec{F} = m\vec{a}$, integrando l'equazione del moto si può
ricavare la legge orario di ciascuna carica, mediante la legge di Lorentz ammesso che si conoscano 
il campo elettrico e il campo di induzione magnetica.
Viene riconosciuto come modello di Maxwell-Lorentz.

Un blocco restituisce i campi elettromagnetici ($\vec{E},\vec{B}$) con i quali, si ricavano in un altro
``blocco'' le equazioni del moto sfruttando la forza di Lorentz.
In uscita al secondo blocco mediante la legge oraria, si può calcolare la distribuzione di cariche e 
corrente $\left(\rho,\vec{J}\right)$, ossia le sorgenti dei due campi, con queste si rientra nel primo blocco.

\paragraph{Definizioni operative}
Campo elettrico:
si esegue una misura in condizioni statiche, ossia la carica ha velocità nulla, con un ipotetico 
dinamometro si può misurare la forza F e facendo il rapporto con la carica $q$:
$$
\left.\frac{\vec{F}}{q}\right|_{\vec{v}=0} \stackrel{\text{def}}{=} \vec{E}(P,t)
$$
Si effettuano poi due misure dinamiche a velocità non nulla
$$
\vec{v}\times\vec{B} = \vec{F} - q\vec{E}
$$
Attraverso queste due misure si possono ricavare le componenti di $\vec{E}(P,t)$ e $\vec{B}(P,t)$.


Le unità di misura:
$[F] = \si{\newton}\ \  [E] = \si{\volt\per\meter} \ \ [B] = \si{\tesla}$

Si introduce una descrizione della materia nella teoria del continuo, facendo riferimento alle 
distribuzioni di cariche:
Presa una regione $\Omega$ contenente un punto $P$ attorno al quale consideriamo un volumetto elementare 
$\Delta\Omega$, all'interno del volumetto sono contenute diverse cariche che si muovono a diverse 
velocità. Nell'ipotesi del continuo si deve supporre che $\Delta \Omega$ deve essere sufficientemente ...
Ma deve essere anche sufficientemente grande in modo da contenere un numero di particelle che definiscano
grandezze che variano con continuità.

Si definisce quindi la funzione 
$$
\rho(P,t) \stackrel{\text{def}}{=} \frac{N^+q_p+N^-q_e}{\text{Vol}(\Delta\Omega)}
$$
densità volumetrica di carica, campo scalare $[\rho] = \si{\coulomb\per\meter}$.
Si possono inoltre definire densità di carica parziali $\rho^+$ e $\rho^-$ associate alla densità
di cariche di quello specifico segno, ovviamente $\rho = \rho^+ + \rho^-$.

\textbf{Densità di corrente elettrica}
Si definiscono con $v^+$ e $v^-$ le velocità medie (di \textit{drift}) dei portatori di carica positivi e negativi.

$$
\vec{J}(P,t) \stackrel{\text{def}}{=} \frac{N^+q_p\vec{v}^+ + }{\text{Vol}(\Delta \Omega)} = ...
$$

$$
[J] = \frac{A}{m^2}
$$
Nell'unità di tempo ci sarà un flusso di cariche attraverso il volumetto di controllo, mediante la sua 
frontiera.

\textbf{Intensità di corrente} attraverso una superficie aperta $S$ orientata con normale $\hat{n}$.

Sia $P$ un punto sulla superficie circondato da un volumetto di controllo, nello stesso punto
è definito il campo \textit{densità di corrente $J$}.

La quantità di carica netta che attraversa la superficie S nell'intervallo di tempo $[t,t+\delta t]$
$$
\delta Q_S = \vec{J}\cdot \hat{n}\ dS\ dt
$$
01:35:00



\textbf{Principio di conservazione della carica per sistemi aperti}
Sia $\Sigma$ una superficie chiusa che racchiude una superficie
$\Omega_\Sigma:\ \partial\Omega_\Sigma=\Sigma$
$$
i_{\Sigma(t)} = - \frac{dQ_{\Omega_\Sigma}}{dt}
$$

\paragraph{Materiali isolanti e conduttori}
Sia una regione $\Omega$ contenente il punto P con il suo volumetto elementare associato, allora
$\Omega$ è isolante se $\vec{J}(P,t) = 0\ \forall\ t \forall\ p \in \Omega$.

Conduttore ohmico se rispetta la legge di Ohm in forma locale, ossia:
$\vec{J} = \gamma\vec{E}$ dove $\gamma$ è la conducibilità elettrica [\si{\siemens\per\meter}]
o viceversa
$$
\vec{E} = \eta \vec{J}
$$
Con $\eta$ la resistività $[\eta] = \si{\ohm\cdot\meter}$, ad esempio per il rame è pari a 
\SI{e-7}{\siemens\per\meter}
