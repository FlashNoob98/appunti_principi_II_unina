
\subparagraph{Potenziale vettore di una spira a grande distanza}
Si ricorda l'espressione del vettore potenziale
$$
\vec{A}(p) = \frac{\mu_0}{4\pi}\hat{m}\times\frac{\vec{e}_r}{r_p^2} = \frac{\mu_0}{4\pi}\vec{m}\times\frac{\vec{r}_p}{r_p^3} = -\frac{\mu_0}{4\pi}\vec{m}\times\nabla_p \left(\frac{1}{r_p} \right)
$$
È quindi possibile calcolare il campo $\vec{B}$ effettuando il rotore di $\vec{A}$
$$
\vec{B}(p) = \nabla\times\vec{A} = -\frac{\mu_0}{4\pi}\nabla_p\times\left(\vec{m}\times\nabla_p\left(\frac{1}{r_p}\right)\right)
$$
Identità vettoriale: $\nabla\times(f\vec{v}) = f\nabla\times\vec{v} + \nabla f \times \vec{v}$
$$
f = \frac{1}{r_p} \qquad \vec{v}=\vec{m}
$$
quindi
$$
\nabla\times\left(\frac{\vec{m}}{r_p}\right) = \frac{1}{r_p}\nabla\times\vec{m} + \nabla\left(\frac{1}{r_p}\right)\times \vec{m} = \frac{1}{r_p}\nabla\times\vec{m} - \vec{m}\times\nabla\left(\frac{1}{r_p}\right)
$$
L'ultimo termine può essere sostituito nell'espressione del campo ma il rotore rispetto
a $p$ del rotore di $\vec{m}$ è nullo quindi
$$
\vec{B}(p) = \frac{\mu_0}{4\pi} \nabla_p\times\nabla_p\times\left(\frac{\vec{m}}{r_p}\right)
$$
ma il rotore del rotore è il gradiente della divergenza meno il laplaciano vettore
$$
\vec{B}(p) = \frac{\mu_0}{4\pi} \left[\nabla\left(\nabla\cdot\left(\frac{\vec{m}}{r_p}\right)\right) - \cancel{\nabla^2\left(\frac{\vec{m}}{r_p}\right)}\right]
$$
Il laplaciano di $\frac{\vec{m}}{r_p}$ è nullo in quanto il momento magnetico $\vec{m}$ è 
costante come il potenziale di una carica puntiforme
$$
\vec{B}(p) = \frac{\mu_0}{4\pi}\nabla\left(\nabla\cdot\left(\frac{\vec{m}}{r_p}\right)\right) = \frac{\mu_0}{4\pi}\nabla\left(\vec{m}\cdot\nabla\left(\frac{1}{r_p}\right)\right)
$$
l'espressione tra parentesi può essere nominata $\psi$, un caso in cui il campo magnetico
può essere espresso mediante un potenziale magnetico scalare, dato che $\psi$ è una funzione 
scalare.
$$
\psi(p) = \frac{\mu_0}{4\pi}\vec{m}\cdot\nabla\left(\frac{1}{r_p}\right) = -\vec{m}\cdot \frac{\vec{r_p}}{r_p^3}
$$
14:12
