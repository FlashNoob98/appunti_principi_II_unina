%\documentclass[a4paper,11pt]{article}
\documentclass[a4paper,11pt]{scrartcl}

\usepackage[utf8]{inputenc}
\usepackage[italian]{babel}

\usepackage{graphicx} %for includng images
\graphicspath{{img/}}

\usepackage{siunitx} % package for deciBel and other units

\usepackage{amsmath} % package for ``cases'' and other matemagical stuff

\usepackage[american]{circuitikz} % circuit drawer
\ctikzset{/tikz/circuitikz/bipoles/length=1cm} %dimensione componenti

\usepackage{subcaption}  %per immagini multiple

%\usepackage{xcolor} % Colori!
\usepackage{colortbl} %colori nelle tabelle!!

\usepackage{multirow} %righe doppie nelle tabelle

\usepackage{makecell} %multirow box

\title{Appunti di Principi di Ingegneria Elettrica II}
\author{Daniele Olivieri}
\date{}

\pdfinfo{%
  /Title    ()
  /Author   ()
  /Creator  ()
  /Producer ()
  /Subject  ()
  /Keywords ()
}
%\includeonly{}
\begin{document}
\maketitle
\setlength\arrayrulewidth{1.2pt} %larghezza righe tabelle
\section{Introduzione ai circuiti dinamici}
Tutti i circuiti che contengono componenti come
resistori, induttori e condensatori, generatori indipendenti di tensione e corrente ai quali associamo 
una tensione impressa o una corrente impressa vengono detti circuiti lineari tempo invarianti (LTI).

Se vi sono anche bipoli tempo-varianti come interruttori che si chiudono o si aprono, il circuito
si definisce lineare tempo variante (LTV).

Preso un generico circuito, vogliamo determinare una tensione o un'intensità di corrente di un generico
bipolo.
\textit{(Nella vita bisogna sempre porsi un obiettivo)}

\textbf{Strumenti a disposizione:}

\begin{itemize}
\item Equazioni di interconnessione:
\begin{equation} \label{eq:leggi_kirchooff}
\begin{cases}
        \text{LKC} & \forall \text{ nodo } \sum_{k} (\pm) i_k(t) = 0\ \forall\ t \\
        \text{LKT} & \forall \text{ maglia } \sum_{h} (\pm) v_n(t) = 0\ \forall\ t
\end{cases}
\end{equation}

\item Relazioni caratteristiche dei bipoli coerenti con la scelta dei versi delle grandezze (convenzione del generatore o dell'utilizzatore).
\begin{equation}
\begin{cases}
v_R  = R\cdot i \\
v_L  = L\frac{di_L}{dt} \\
i_C  = C\frac{dv_C}{dt} \\
v_e  = e(t)\\
i_j  = j(t) \\
\end{cases}
\end{equation}
\begin{equation*}
\begin{cases}
\text{interruttori in chiusura a t=0: }  {t<0,\ i = 0\ \forall\ v;\ t > 0,\ v = 0\ \forall\ i} \\
\text{interruttori in apertura a t=0: }  {t<0,\ v = 0\ \forall\ i;\ t > 0,\ i = 0\ \forall\ v} 
\end{cases}
\end{equation*}
\end{itemize}

Le equazioni di interconnessione non sono tutte indipendenti ma è sempre possibile
costruire un set di equazioni indipendenti scegliendo \textit{n-1} nodi o \textit{n-l-1} maglie fondamentali,
dove n ed l sono rispettivamente il numero di nodi e lati del grafo connesso.

NOTA: per ogni induttore la potenza assorbita:
$$ P^a(t) = v_l\cdot i_l = L \frac{di_L}{dt}\cdot i_l = \frac{d}{dt} [\frac{1}{2}Li_L^2] = \frac{dW_m}{dt} $$ 
L' energia immagazzinata nel campo magnetico dell'induttore invece:
$$\Delta W^a(t_1,t_2) = \int_{t_1}^{t_2} P^a(\tau)d\tau = W_m (t_2) - W_m(t_1) $$
Se in un certo istante di tempo l'induttore presenta una certa quantità di energia in Joule [\si{\joule}] quella sarà la massima energia estraibile.

Per ogni condensatore invece potenza ed energia sono così definite:
$$P^a(t) = \frac{d}{dt} W_e;\ W_e(t) = \frac{1}{2} CV_c^2$$
L'energia sarà immagazzinata mediante il campo elettrico.

Si ha quindi un sistema di equazioni circuitali in cui si ha una parte algebrica con le caratteristiche adinamiche, ossia con caratteristiche
non differenziali e non integrali, più una parte differenziale data dai bipoli dinamici come condensatori e induttori.
In letteratura un sistema simile si indica con DAE (Differential Algebric Equation), differente
dalla ODE (Ordinary Differential Equation).

Le tecniche utilizzate in teoria dei circuiti mirano a trasformare una DAE in una ODE, ossia 
per formulare le equazioni circuitali come equazioni differenziali ordinarie, per operare questa trasformazione si fa riferimento
alla dinamica delle sole {variabili di stato}: $i_l(t)\ v_c(t)$

La loro conoscenza permette di esprimere tutte le altre variabili attraverso relazioni algebriche, vengono definite
variabili \textit{slaved}, perchè subordinate alle prime.

Per eseguire ciò si richiama una procedura generale per l'analisi di circuiti lineari tempo varianti (LTV).
Questa procedura si basa su un'analisi a intervallo: si partiziona l'asse dei tempi in intervalli in ciascuno dei quali esiste un circuito 
tempo invariante (LTI) equivalente a quello di partenza.

\begin{figure}[h] %tre stati diversi dello stesso circuito
\centering
 \begin{subfigure}{.3\textwidth}
  \centering
  \caption{configurazione generica}
  \begin{circuitikz}
   \draw (0,0) to [R] (1,0);
   \draw (0,0.6) to [L] (1,0.6);
   \draw (0,1.2) to [C] (1,1.2);
   \draw (0,1.8) to [V] (1,1.8);
   \draw (0,2.4) to [I] (1,2.4);
   \draw (0,3) to [closing switch] (1,3);
   \draw (0,3.6) to [opening switch] (1,3.6);
   \draw (-0.3,-0.3) rectangle (1.3,3.9);
  \end{circuitikz}
 \end{subfigure}
 \begin{subfigure}{.3\textwidth}
  \centering
  \caption{$t<0$}
  \begin{circuitikz}
   \draw (0,0) to [R] (1,0);
   \draw (0,0.6) to [L] (1,0.6);
   \draw (0,1.2) to [C] (1,1.2);
   \draw (0,1.8) to [V] (1,1.8);
   \draw (0,2.4) to [I] (1,2.4);
   \draw (0,3) to [open,o-o] (1,3);
   \draw (0,3.6) to [short] (1,3.6);
   \draw (-0.3,-0.3) rectangle (1.3,3.9);
  \end{circuitikz}
 \end{subfigure}
 \begin{subfigure}{.3\textwidth}
  \centering
  \caption{$t>0$}
  \begin{circuitikz}
   \draw (0,0) to [R] (1,0);
   \draw (0,0.6) to [L] (1,0.6);
   \draw (0,1.2) to [C] (1,1.2);
   \draw (0,1.8) to [V] (1,1.8);
   \draw (0,2.4) to [I] (1,2.4);
   \draw (0,3) to [short] (1,3);
   \draw (0,3.6) to [open,o-o] (1,3.6);
   \draw (-0.3,-0.3) rectangle (1.3,3.9);   
  \end{circuitikz}
 \end{subfigure}
\end{figure}

Supponiamo che i generatori indipendenti abbiano grandezze \textbf{limitate}, ossia le tensioni impresse $e(t)$ e le correnti impresse $j(t)$,
in questa ipotesi sappiamo che le variabili di stato sono funzioni \textbf{continue} ossia:
\begin{equation*}
\begin{split}
i_L (0^+) & = i_L(0^-) \\ 
v_C (0^+) & = v_C(0^-)
\end{split}
\end{equation*}

La soluzione si determina trovando la dinamica delle variabili di stato in ciascun circuito ausiliario, "incollando" le soluzioni utilizzando la proprietà di continuità delle variabili di stato.

Il problema di risolvere circuiti lineari tempo varianti si scompone nel risolvere tanti circuiti lineari tempo invarianti. Una categoria semplice di circuiti tempo invarianti sono i
circuiti lineari del primo ordine (circuiti RC o RL) con un solo elemento dinamico.
\begin{figure}[h] %circuiti RC ed RL equivalenti
\centering
 \begin{subfigure}{.3\textwidth}
  \centering
  \begin{circuitikz}
   \draw (0,0) rectangle (1.6,1.7);
   \draw (0.3,0.3) to [I] (1.3,0.3);
   \draw (0.3,0.9) to [V] (1.3,0.9);
   \draw (0.3,1.5) to [R] (1.3,1.5);
   \draw (1.6,1.5) to [short,i=$i_C$] (2,1.5)
   to [C,v^=$v_C $] (2,0.4) to [short] (1.6,0.4);
  \end{circuitikz}
  \caption{Circuito RC}
 \end{subfigure} 
  \begin{subfigure}{.3\textwidth}
  \centering
  \begin{circuitikz}
   \draw (0,0) rectangle (1.6,1.7);
   \draw (0.3,0.3) to [I] (1.3,0.3);
   \draw (0.3,0.9) to [V] (1.3,0.9);
   \draw (0.3,1.5) to [R] (1.3,1.5);
   \draw (1.6,1.5) to [short,i=$i_L $] (2,1.5)
   to [L,v^=$v_L $] (2,0.4) to [short] (1.6,0.4);
  \end{circuitikz}
  \caption{circuito RL}
 \end{subfigure}
\end{figure}

Un bipolo adinamico lineare e un bipolo dinamico fanno subito pensare all'utilizzo dei teoremi di Thévenin e Norton, permettendo la riduzione del circuito 
adinamico ad un semplice generatore con un resistore equivalente.

Applicando ad esempio la LKT $e_0(t) = R_{th}\cdot C \frac{dV_C}{dt} + V_C$ si ha l'equazione di stato
del circuito e supponendo di conoscere $V_C(t=0) = V_0 $ allora la soluzione dell'equazione è:
$$V_C(t) = [V_0-V_{C_p}(0)] e ^{-\frac{t}{\tau}} + V_{C_p}(t)$$ dove $\tau = R_{th}\cdot C$ e $V_{C_p}(t)$
è la soluzione a regime. 
 
Dopo un intervallo pari a $4\sim5\ \tau$ si assume il processo di carica o scarica terminato.

Per il circuito RL 
$$I_L(t) = [I_0-I_{L_p}(0)]e^{-\frac{t}{\tau}} + I_{L_p}(t)$$ 
con $\tau = \frac{L}{R_{th}},\ I_0 = I_L (t=0)$

Osservazione: la soluzione generale del circuito RC, ossia la dinamica di $V_c(t)$ può essere espressa 
come la somma di due termini $V_{C_{tr}}(t)$ e $V_{C_p}(t)$, il primo transitorio, che tende a svanire se 
attendiamo un tempo sufficiente lungo, porta con se la \textit{memoria} dello stato iniziale, memoria che 
viene
persa quando $t>4\sim5\tau$, ammesso che $\tau$ sia positiva; esistono infatti alcune combinazioni di 
elementi circuitali si comportano come un resistore negativo.

Il secondo termine è quello di regime permanente, che ovviamente non ha memoria dello stato iniziale ma 
dipende solo dalla nuova configurazione
del circuito (termine forzato).

La decomposizione in regime transitorio e permanente è una decomposizione generale che vale per qualsiasi circuito, a patto di complicare opportunamente la matematica.

Si parla inoltre di evoluzione libera ed evoluzione forzata 
$$\begin{cases}
e_0(t) = R_{th}C\frac{dV_c}{dt} + V_C \\
V_c(0) = V_0
\end{cases}$$

si scompone in:

$$\begin{cases} %evoluzione libera
0 &= R_{th}C\frac{dV_c}{dt} + V_C \\ 
V_c(0) &= V_0
\end{cases}$$

$$\begin{cases} %evoluzione forzata
e_0(t) &= R_{th}C\frac{dV_c}{dt} + V_C \\
V_c(0) &= 0
\end{cases}$$

Trattazione analoga (duale) per il circuito RL

\section{Circuiti lineari tempo invarianti del II ordine}
Divisi in (RC, RL, RLC), la procedura generale di risoluzione richiede tre step:
\begin{enumerate}
 \item Determinazione delle equazioni di stato
 \item Determinazione delle condizioni iniziali
 \item Soluzione del problema di Cauchy
\end{enumerate}

Esempio:
\begin{figure}[h]
\centering
\begin{circuitikz}
 \draw (0,2) to [V,l_=$E$] (0,0);
 \draw (0,2) to [closing switch=${t=0}$](1,2) 
             to [R,l=$R_1$] (2.5,2)
             to [R,l=$R_2$] (4,2)
             to [L,i=$i_L$,l=$L$] (4,0)
             to [short] (0,0);
 \draw (2.5,2)node[circ,color=red]{} to [C,v=$v_C$,l=$C$] (2.5,0);
\end{circuitikz}
\caption{Esempio di un circuito RLC del secondo ordine}
\end{figure}

La dinamica dello stato per $t<0$ è di facile risoluzione, $i_L(t) = 0$, $v_C(t) = 0$
per $t > 0$ invece si analizza il circuito:

Equazioni di interconnessione: 
LKC nel nodo evidenziato in rosso (tra i due resistori)
\begin{equation*}
\begin{cases}
 I_1 &= I_C+I_L \ \text{LKC}\\
 E &= R_1\cdot I_1 + V_C\ \text{LKT} \\
V_C &= R_2 \cdot I_L + V_L \\
V_L &= L\frac{dI_L}{dt} \\
I_C &= C\frac{dV_C}{dt}
\end{cases}
\end{equation*}
Ricaviamo $I_1$ dalla seconda e sostituiamola nella prima,
ora vanno ricavate le variabili di stato ottenendo
$$
\begin{cases}
i_C = \frac{E}{R_2} - \frac{v_C}{R_1} - i_L \\
v_L = v_C - R_2 i_L
\end{cases}
$$
Sostituendo le equazioni caratteristiche dei bipoli dinamici:

$$
\begin{cases}
C\frac{dv_C}{dt} = \frac{E}{R_2} - \frac{v_C}{R_1} - i_L \\
L\frac{di_L}{dt} = v_C - R_2 i_L
\end{cases}
$$

Le condizioni iniziali sono 
$$\begin{cases}
v_C(0^+) = v_C(0^-) = 0\\
i_L(0^+) = i_L(0^-) = 0
\end{cases}
$$
Per risolvere queste equazioni conviene ridurre l'equazione del secondo ordine ad una sola delle incognite,
ad esempio si sostituisce nell'equazione che definisce $i_C$, $v_C$ ricavata dalla equazione di
$v_L$, in questo modo si ha un'unica equazione in cui compaiono le grandezze relative all'induttore.

Si ottiene 
\begin{equation}
LC \frac{d^2i_L}{dt^2} + CR_2\frac{di_L}{dt} = \frac{E}{R_1} - \frac{L}{R_1}\frac{di_L}{dt} - \frac{R_2}{R_1}i_L-i_L
\end{equation}
%ora raccogli i termini
Controlli da eseguire: controllo dimensionale, positività dei coefficienti altrimenti il circuito non sarebbe dissipativo.

L'integrale generale è scritto come somma dell'integrale dell'omogenea associata e dell'integrale particolare.
Polinomio caratteristico:
\begin{equation}
 \lambda^2 + (\frac{R_2}{L} + \frac{1}{R_1 C})\lambda + (1+\frac{R_2}{R1})\frac{1}{LC} = 0
\end{equation}
La soluzione del polinomio può essere di tre tipi:

Caso 1, radici reali e distinte $\Rightarrow$
Vedi grafico 01:38:00 modi naturali aperiodici smorzati

Caso 2 radici reali e coincidenti, caso piuttosto patologico
la radice viene determinata con $-\sigma$ avremo un modo esponenziale decrescente $e^{-\sigma t} $ con $\tau = \frac{1}{\sigma}$
e un modo pari a  $te^{-\sigma t}$

Caso 3 modo periodico smorzato
$\lambda_{1,2} = -\sigma \pm j\omega d$, la distanza tra due picchi è pari a $\frac{2\pi}{\omega}$
$$i_{L_0}(t) = e^{-\sigma t} [K_1 \cos (\omega_d t) + K_2 \sin(\omega_d t)]$$

Restano da determinare le costanti di integrazione imponendo le condizioni iniziali,
\begin{equation*}
\begin{cases}
i_L(0^+) = i_L(0^-) \\
\frac{d_{i_L}}{dt}(0^+) = \frac{1}{L}[v_C(0^+) - R_2I_L(0^+)] = 0
\end{cases}
\end{equation*}

%Lezione 2 i minuti si riferiranno a quelli visibili su teams e non quelli del file registrato con OBS
\subsection{Determinazione equazioni di stato di un circuito qualsiasi}

Si riprende la classe di circuiti lineari tempo invarianti (LTI), si suppone di conoscere
le variabili di stato $i_L(t) $ e $v_C(t) $ assumendole note, si sostituisce ogni \textit{condensatore}
con un generatore di tensione di valore pari alla $v_C(t)$, ripetendo il procedimento
per ciascun \textit{induttore} che viene sostituito con un generatore di corrente con corrente impressa
pari ad $i_L(t)$.

In queste condizioni, la soluzione del circuito resta formalmente invariata, 
il nuovo circuito sarà di tipo adinamico, non presenterà più alcun componente dinamico,
prende il nome di \textit{circuito resistivo associato al circuito di partenza}.

Il vantaggio di questa operazione è la possibilità di ricavare $v_L$ e $i_C$ utilizzando il principio
di \textit{sovrapposizione degli effetti} (PSE).

Si ricavano le equazioni di stato per il seguente circuito:
\begin{figure}[H]
\centering
\begin{circuitikz}
\draw
(0,0) to [voltage source,invert,l=$E$] (0,2)
to [R=$R_1$] (2,2)
to [C,v^=$v_C$,l_=$C$,i_=$i_C$] (2,0) -- (0,0)
;\draw
(2,2) to [R=$R_2$] (4,2)
to [L=$L$,i=$i_L$,v=$v_L$] (4,0) -- (2,0)
;
\end{circuitikz}
\end{figure}

Si applica il PSE, per trovare $i_C$ e $v_L$:
$$\begin{aligned}
i_C &= i_c' + i_C'' + i_C'''\\
v_L &= v_L' + v_L'' + v_L'''
\end{aligned}$$

\begin{figure}[H]
\centering
\begin{subfigure}{.49\linewidth} %Circuito C'
\centering
\begin{circuitikz}
\draw
(0,0) to [voltage source,invert,l=$E$] (0,2)
to [R=$R_1$] (2,2)
to [short,i_=$i_C'$] (2,0) -- (0,0)
;\draw
(2,2) to [R=$R_2$] (4,2)
to [open,v=$v_L'$] (4,0) -- (2,0)
;
\end{circuitikz}
\caption{Circuito 1}
\end{subfigure}
\begin{subfigure}{.49\linewidth} %Circuito C''
\centering
\begin{circuitikz}
\draw
(0,0) to [short] (0,2)
to [R=$R_1$] (2,2)
to [voltage source,v^=$v_C$,i>_=$i_C''$] (2,0) -- (0,0)
;\draw
(2,2) to [R=$R_2$] (4,2)
to [open,v=$v_L''$] (4,0) -- (2,0)
;
\end{circuitikz}
\caption{Circuito 2}
\end{subfigure}
\begin{subfigure}{.49\linewidth} %Circuito C'''
\centering
\begin{circuitikz}
\draw
(0,0) to [short] (0,2)
to [R=$R_1$] (2,2)
to [short,i_=$i_C'''$] (2,0) -- (0,0)
;\draw
(2,2) to [R=$R_2$] (4,2)
to [current source,i_=$i_L$,v^>=$v_L'''$] (4,0) -- (2,0)
;
\end{circuitikz}
\caption{Circuito 3}
\end{subfigure}
\end{figure}
\newpage
\begin{itemize}
\item Circuito 1)
$$
v_L' = 0;\ i_C' = \frac{E}{R_1} 
$$
\item Circuito 2)
$$
i_C'' = -\frac{v_C}{R_1};\ v_L'' = v_C
$$
\item Circuito 3)
$$
i_C''' = -i_L;\ v_L''' = -R_2\cdot i_L
$$
\end{itemize}
Sommando i tre contributi e aggiungendo il vincolo di continuità delle variabili
di stato si ottiene il problema di Cauchy del sistema:
$$\left\{\begin{aligned}
&i_C = \frac{E}{R_1} - \frac{v_C}{R_1} - i_L = C\frac{dv_C}{dt}\\
&v_L = v_C - R_2\cdot i_L = L\frac{di_L}{dt}\\
&i_L(0^+) = i_L(0^-)\\
&v_C(0^+) = v_C(0^-)
\end{aligned}\right.$$

\subsection{Circuiti lineari con generazioni impulsivi}
Si analizza ora un circuito che presenta generatori di tipo impulsivo, ad esempio la risposta di 
una linea elettrica a seguito di una fulminazione.
Si definisce quindi l'impulso rettangolare di ampiezza $\Delta$, la funzione viene chiamata 
$\Pi_\Delta(t)$, (funzione porta) è costante nell'intervallo $[-\frac{\Delta}{2},\frac{\Delta}{2}]$,
l'area del rettangoloide sotteso alla funzione è pari a
$$
\int_{-\Delta/2}^{\Delta/2}\Pi_\Delta(\tau)d\tau = 1\ \forall\ \Delta \in\ ]0,+\infty[
$$
Ha senso considerare la successione di funzioni ottenute per valori $\Delta$ decrescenti,
ma dimezzando la base, per mantenere l'area unitaria, va raddoppiata l'altezza.

Passando al limite per $\Delta \rightarrow 0^+$ la successione tende in maniera non ordinaria
ad un limite che non è una funzione ma può essere definita come funzione generalizzata,
ossia distribuzione, che prende il nome di \textbf{Delta di Dirac} ($\delta(t)$).

Proprietà della $\delta(t)$:
\begin{itemize}
\item È nulla $\forall\ t \neq 0$
\item Ha integrale unitario
\item Proprietà di campionamento $\int_{-\infty}^{+\infty}f(\tau)\delta(\tau-t_0)d\tau = f(t_0)$
\end{itemize}
\newpage
La proprietà di campionamento calcolata come l'integrale della $\delta(t_0)$ per una funzione ordinaria
permette di valutare la funzione ordinaria nel punto $t_0$ (pari a 0 in questo caso per semplicità)
in cui è centrata la $\delta(t_0)$, si dimostra
utilizzando la definizione di $\delta(t)$ mediante successioni di funzioni porta $\Pi_\Delta(t)$
$$
\int_{-\Delta/2}^{\Delta/2} f(\tau)\Pi_\Delta(\tau-t_0)d\tau = \frac{1}{\Delta} \int_{-\Delta/2}^{\Delta/2}  f(\tau)d\tau = \frac{1}{\Delta}f(\vartheta^*)\cdot\Delta= f(\vartheta^*)
$$
con $\vartheta^* \in\ \left]-\frac{\Delta}{2},\frac{\Delta}{2}\right[$. Eseguendo il limite $\Delta\to 0$ si ha
$\vartheta^* \to 0 $, in conclusione
$$
\int_{0^-}^{0^+} f(\tau)\delta(\tau)d\tau = f(0)
$$

Si analizza la funzione $U_\Delta(t)$ definita come segue:
\begin{equation*}
\begin{cases}
0 & t  < -\frac{\Delta}{2} \\
1 & t  > \frac{\Delta}{2} \\
\frac{1}{2}+\frac{t}{\Delta} & -\frac{\Delta}{2} \leq t \leq \frac{\Delta}{2}
\end{cases}
\end{equation*}
\begin{figure}[H]\centering
\begin{tikzpicture}
\begin{axis}[
    axis lines = center,
    xlabel = \(t\),
    ylabel = {\(U_\Delta(t)\)},
    ylabel style = {at={(axis cs: -0.9,0.95)}},
    xtick = {-2,-1,0,1,2},
    xticklabels = { , $-\frac{\Delta}{2}$, , $\frac{\Delta}{2}$,},
    ytick = {0 ,0.5 ,1},
    yticklabels = { , ,1},
    xmin =  -2, xmax = 2,
    yscale = 0.5,
]
%Below the red parabola is defined
\addplot [
    domain=-1:1, 
    samples=2, 
    color=black,
]
{0.5 + x/2};
\draw [thick] (axis cs:1,1) -- (axis cs:2,1);
\draw [thick] (axis cs:-2,0) -- (axis cs:-1,0);
\draw [dashed] (axis cs:0,1) -- (axis cs:0.5,1);
\addplot [
    domain = -0.5:0.5,
    samples = 2,
    color = green,
    style = thick,
]{0.5 + x};
\draw [thick,dashed,color = green] (axis cs:0.5,1) -- (axis cs:1,1);
\end{axis}
\end{tikzpicture}
\end{figure}

La derivata temporale nei punti regolari coincide quasi ovunque con la funzione $\Pi_\Delta(t)$.
Eseguendo il limite su $U_\Delta(t)$
di $\Delta \rightarrow 0^+$ come si nota dall'andamento del segmento in verde, si ottiene la funzione definita 
``gradino'' o funzione di Heaviside $u(t)$.
$$
u(t) = \begin{cases}
0 & t\leq 0\\
1 & t > 0
\end{cases}
$$


Un ulteriore modo per definire la $\delta(t)$ è appunto quella di derivata della funzione gradino $u(t)$.
$$
\delta(t) = \frac{d}{dt}u(t) \Leftrightarrow \int_{-\infty}^t \delta(\tau)d\tau = u(t)
$$
\newpage
\paragraph{Esempio con generatore impulsivo}
Si consideri un circuito RC serie
\begin{figure}[H]\centering
\begin{circuitikz}
\draw
(0,0) to [voltage source,invert,l=$e(t)$] (0,2)
      to [R=$R$] (2,2)
      to [C,l_=$C$,v^=$v_C$] (2,0) -- (0,0)
;
\end{circuitikz}
\end{figure}

una funzione 
$$
e(t) = \begin{cases}
E_0 & 0 < t < T\\
0 &\text{altrimenti}
\end{cases}
$$
Si vuole ricavare $v_C(t)$:

Si suppone che la condizione iniziale, per $t < 0 $, ossia la tensione sul condensatore sia nulla.
$$\begin{aligned}
&\begin{cases}
e(t) &= RC\frac{dv_C}{dt} + v_C \\
v_C(0) &= \SI{0}{\volt}
\end{cases}\\
&0 \leq t \leq T\\
&\begin{cases}
E_0 &= RC\frac{dv_C}{dt} + v_C \\
v_C^{(1)}(0) &= \SI{0}{\volt} 
\end{cases}\\
&t > T\\
&\begin{cases}
0 &= RC\frac{dv_C}{dt} + v_C \\
v_C^{(2)}(0) &= v_C^{(1)}(T) 
\end{cases}
\end{aligned}
$$
Conoscendo la forma della soluzione vanno solo sostituiti i termini nelle varie configurazioni
$$
v_C^{(1)}(t) = -E_0 e^{-\frac{t}{\tau}} + E_0,\ v_C^{(1)}(T) = E_0 \left(1-e^{-\frac{T}{\tau}}\right),\ \tau = RC
$$
$$
v_C^{(2)}(t) = E_0\left(1-e^{-\frac{T}{\tau}}\right) e^{-\frac{t-T}{\tau}} = v_C^{(1)}(T)\cdot e^{-\frac{t-T}{\tau}}
$$
In conclusione
$$
v_C(t) = \left\{
\begin{aligned}
&0 \qquad\qquad\qquad\qquad\qquad\ t\leq 0\\
&E_0 \left(1-e^{-\frac{T}{\tau}}\right) \qquad\qquad\ \ 0 < t < T\\
& E_0\left(1-e^{-\frac{T}{\tau}}\right) e^{-\frac{t-T}{\tau}} \qquad t \geq T
\end{aligned}\right.
$$
\begin{figure}[H]
\centering
\includegraphics[width = 0.4\linewidth]{v_C_lezione_2}
\end{figure}

Si ottiene una funzione esponenziale crescente fino a $T$ e poi decrescente fino a 0 all'infinito. 
Diminuendo il valore di $T$ si vede che il ``picco'' della funzione sarà più basso, al limite 
di $T \rightarrow 0$ la soluzione si annulla.
Se si impone il prodotto $E_0\cdot T = 1$ ossia $E_0 = \frac{1}{T}$ e si esegue il limite invece:
$$
\lim_{T\rightarrow0^+} v_C(t)
$$
Il limite viene eseguito sulle singole parti della funzione, ad esempio nel tratto $0\leq t \leq T$
la funzione ha sempre un tempo inferiore per raggiungere il valore massimo, mentre l'ampiezza
del generatore aumenta.

Si suppone di sviluppare la funzione esponenziale con la sua serie di Taylor:
$$
e^x = 1 + x + \frac{x^2}{2!} + \frac{x^3}{3!} + ...\simeq 1 + x \Rightarrow 1-e^{-\frac{t}{\tau}} \simeq \frac{t}{\tau} 
$$
L'equazione della tensione diventa
$$v_C(t) = 
\begin{cases}
0\ & t\leq 0 \\
\frac{1}{T}\frac{t}{\tau}\  & 0 \leq t\leq T \\
\frac{1}{T}\frac{T}{\tau} e^{-\frac{t-T}{\tau}}\ & t\geq T
\end{cases}
$$
Per $T\rightarrow 0^+$
$$v_C(t) =
\begin{cases}
0\ & t\leq 0\\
\frac{1}{\tau}e^{\frac{-t}{\tau}}\ & t \geq 0
\end{cases}
$$

Il primo tratto dell'equazione si approssima quindi ad un tratto lineare fino a T, arrivando ad 
un'altezza di $\frac{1}{\tau}$, al limite raggiunge questo valore in un tempo infinitesimo.
\begin{figure}[H]
\centering
\includegraphics[width = 0.4\linewidth]{v_C_limite_lezione_2}
\end{figure}
Si otterrebbe la stessa soluzione ponendo $e(t) = \Pi_\Delta(t)$ e svolgendo il limite
$$\Pi_\Delta(t) \stackrel{\Delta\rightarrow0^+}{\rightarrow} \delta(t) \Rightarrow v_C(t) \rightarrow h(t)$$
$h(t)$ è chiamata risposta all'impulso del circuito dinamico.

In alternativa è possibile utilizzare direttamente un generatore impulsivo e risolvere il circuito

\begin{figure}[H]\centering
\begin{circuitikz}
\draw
(0,0) to [voltage source,invert,l=$\delta(t)$] (0,2)
      to [R=$R$] (2,2)
      to [C,l_=$C$,v^=$v_C$,i_=$i_C$] (2,0) -- (0,0)
;
\end{circuitikz}
\end{figure}

$$
\delta(t) = RC\frac{dv_C}{dt} + v_C \Rightarrow i_C = \frac{\delta(t)-v_C}{R}
$$
Se la tensione del generatore è impulsiva anche la corrente nel condensatore sarà di tipo impulsivo
mentre la tensione ai capi del condensatore si può ricavare integrando la sua equazione caratteristica
$$
i_c = C\frac{dv_C}{dt} \Rightarrow \int_{0^-}^{0^+} i_C(\tau)d\tau = C[v_C(0^+)-v_C(0^-)]
$$
$$
v_C(0^+) - v_C(0^-) = v_C(0^+) = \frac{1}{RC} \int_{0^-}^{0^+} \delta(\tau)d\tau - \cancel{\frac{1}{RC} \int_{0^-}^{0^+} v_C(\tau)d\tau} = 
$$
$$
= \frac{1}{RC} = \frac{1}{\tau}  \Rightarrow v_C(0^+) = \frac{1}{RC} = \frac{1}{\tau}
$$
Il secondo integrale vale zero perché esteso ad un intervallo infinitesimo di una funzione limitata,
perché legata all'energia immagazzinata nel condensatore che non può essere infinita.
% CONSIDERAZIONE PERSONALE
%dunque non potrebbe essere infinita a meno
%di non supporre un generatore dotato di potenza infinita (non possibile in questo caso data la 
%resistenza inserita nel circuito).
%
È infinita invece la potenza assorbita dal condensatore nell'istante $0^+$ che gli permette di avere
una tensione ai suoi capi discontinua.

\newpage
\subsection{Risposta al gradino unitario di un circuito dinamico LTI}
Stesso circuito del precedente, ma si utilizza come forzamento il gradino unitario di
Heaviside, la soluzione è più semplice della precedente:

$$
v_C(t) = \left(1-e^{-\frac{t}{\tau}}\right) u(t)
$$
viene chiamata $g(t)$ e si afferma che sia la risposta al gradino, ricordando la relazione
tra la funzione $\Pi_\Delta(t)$ e $U_\Delta(t)$ si ha che:
$$
\Pi_\Delta(t) = \frac{U_\Delta\left(t+\frac{\Delta}{2}\right)-U_\Delta\left(t-\frac{\Delta}{2}\right)}{\Delta} = e(t)
$$
per $\Delta \rightarrow 0^+$ si ottiene $h(t) = v_C(t)$ ossia la risposta all'impulso.

Essendo il circuito tempo invariante, si può trovare la risposta alla funzione $\Pi_\Delta(t)$
come combinazione lineare delle risposte delle due $U_\Delta$ opportunamente traslate, ossia
la risposta al gradino traslata.
$$
\text{Risp}\left\{ \Pi_\Delta(t)\right\} = \frac{\text{Risp}\left\{U_\Delta\left(t+\frac{\Delta}{2}\right)\right\} - 
\text{Risp}\left\{U_\Delta\left(t-\frac{\Delta}{2}\right)\right\}}{\Delta} = 
\frac{g\left(t+\frac{\Delta}{2}\right) - g\left(t-\frac{\Delta}{2}\right)}{\Delta}
$$
Tutto si trasforma nella funzione rapporto incrementale della funzione $g(t)$ ossia
$$
\lim_{\Delta\rightarrow0^+}\text{Risp}\left\{\Pi_\Delta(t)\right\} = h(t) = \frac{dg}{dt}
$$

$$
h(t) = \frac{dg}{dt} =\begin{cases}
0& t < 0\\
\frac{1}{\tau}e^{-\frac{t}{\tau}} & t\geq0 
\end{cases}
$$
La risposta all'impulso del circuito è quindi pari alla derivata della risposta al gradino dello stesso
circuito.

Si consideri un circuito RC serie
\begin{figure}[H]\centering
\begin{circuitikz}
\draw
(0,0) to [voltage source,invert,l=$u(t)$] (0,2)
      to [R=$R$] (2,2)
      to [C,l_=$C$,v^=$v_C$] (2,0) -- (0,0)
;
\end{circuitikz}
\end{figure}
si suppone che la tensione imposta al generatore sia un gradino unitario $u(t)$,
l'equazione di stato sarà:
$$
\begin{cases}
u(t) = RC \frac{dv_C}{dt} + v_C \\
v_C(0^+) = 0
\end{cases}
\Rightarrow v_C(t) = \left.
\begin{cases}
0 & t<0 \\
1-e^{-\frac{t}{\tau}} & t\geq 0
\end{cases}\right] = g(t)
$$

Si considera la funzione $U_\Delta$
$$U_\Delta(t) = 
\begin{cases}
0 & t< -\frac{\Delta}{2}\\
\frac{1}{2} + \frac{t}{\Delta} & -\frac{\Delta}{2} < t < \frac{\Delta}{2} \\
1 & t > \frac{\Delta}{2}
\end{cases}
$$
se si esegue la differenza di due funzioni $U_\Delta$ traslate di $\pm\frac{\Delta}{2}$ si ottiene 
una porta trapezoidale
$$
f(t) = \frac{U_\Delta\left(t+\frac{\Delta}{2}\right) - U_\Delta\left(t-\frac{\Delta}{2}\right)}{\Delta}
$$
tende ad una $\delta(t)$ delta di Dirac per $\Delta \rightarrow 0$.
Semplicemente si può invece definire la porta come differenza di due gradini traslati, in questo modo
si elimina il problema dei segmenti obliqui.

La linearità del sistema e la tempo-invarianza delle grandezze dei bipoli implica che la
risposta ad una combinazione lineare di funzioni traslate nel tempo si ottiene come combinazione lineare 
delle risposte dei singoli termini traslati.
$$
\text{Risp}\left\{\Pi_\Delta(t)\right\} = \frac{\text{Risp}\left\{u\left(t+\frac{\Delta}{2}\right) \right\} -\text{Risp}\left\{ u\left(t-\frac{\Delta}{2}\right) \right\}}{\Delta} = \frac{g\left(t+\frac{\Delta}{2}\right)-
g\left(t-\frac{\Delta}{2}\right)}{\Delta}
$$
Eseguendo il limite per $\Delta\to 0^+ $ si vede che quello appena presentato è un rapporto incrementale e
quindi
$$
\lim_{\Delta\to0^+} \Rightarrow \text{Risp} \left\{\delta(t)\right\} =h(t) = \frac{dg}{dt}
$$
Ricordando le funzioni $h(t)$ e $g(t)$ si vede la relazione
$$
h(t) = \begin{cases}
0 & t<0\\
\frac{1}{\tau}e^{-\frac{t}{\tau}} & t\geq 0
\end{cases}\qquad
g(t) = \begin{cases}
0 & t<0\\
1 - e^{-\frac{t}{\tau}} & t\geq 0
\end{cases} \Rightarrow
\frac{dg}{dt} = h(t)
$$
è possibile studiare la risposta all'impulso sfruttando quella al gradino, che è una funzione limitata e 
più semplice da analizzare.

\paragraph{Circuiti RC ed RL semplici con generatori impulsivi}
Si considerino 2 circuiti modello: il circuito RC parallelo e il circuito RL serie.

\begin{figure}[H]\centering
\begin{subfigure}{.4\textwidth}\centering
\begin{circuitikz}
\draw
(0,0) to [current source,l=$j(t)$] (0,2)
      to (1,2) to [R=$R$,i>^=$i_R$] (1,0) to (0,0);
\draw
(1,2) to (2.5,2)  to [C,l_=$C$,v^=$v_C(t)$,i>_=$i_C$] (2.5,0) -- (1,0)
;
\end{circuitikz}
\subcaption{RC parallelo}
\end{subfigure}
\begin{subfigure}{.4\textwidth}\centering
\begin{circuitikz}
\draw
(0,0) to [voltage source,invert,l=$e(t)$] (0,2)
      to [R,l_=$R$] (2.5,2) to [L,l_=$L$,i_>=$i_L(t)$,v^=$v_L$] (2.5,0) -- (0,0)
;
\end{circuitikz}
\subcaption{RL serie}
\end{subfigure}
\end{figure}

Siano i generatori impulsivi: $j(t) = Q\delta(t)$ ed $e(t) = \Phi\delta(t)$.
$$
j(t) = \frac{v_C}{R} + i_C = Q\delta(t) = \frac{v_C}{R} + C\frac{dv_C}{dt}
$$
Si integra la funzione nell'istante in cui è centrata la $\delta(t)$ ossia:
$$
\int_{0^-}^{0^+} Q\delta(\tau)d\tau  = \cancel{\int_{0^-}^{0^+}\frac{v_C}{R}d\tau} + C[v_C(0^+)-\cancel{v_C(0^-)}]
\Leftrightarrow Q = Cv_C(0^+) \ \ [Q] = \si{\coulomb} \text{ (Coulomb)}
$$
$$
\left[\int_{0^-}^{0^+} Q\delta(\tau)d\tau\right] = \text{ Coulomb}
$$
Tirando la costante $C$ fuori dall'integrale, che ha la dimensione di Coulomb, l'integrale rimanente
deve essere adimensionale.
Ciò significa che la $\delta(t)$ ha la dimensione di \si{\per\second} per essere coerente con l'integrale
e restituire una quantità finita.

\subparagraph{Caso duale con circuito RL:}

$$
e(t) = R\cdot i_L + v_L
$$
$$
\Phi\delta(t) = R\cdot i_L + L\frac{di_L}{dt}
$$
$$
\Phi\int_{0^-}^{0^+} \delta(\tau)d\tau = L i_L(0^+)\ , \ i_L(0^-) = 0
$$

Anche in questo caso per avere la dimensione del flusso in \si{\weber} per $\Phi$ allora la $\delta(t)$ 
avrà le dimensioni di \si{\per\second}.
\newpage
\subsection{Procedura generale per la risoluzione di circuiti con generatori impulsivi}
Si ha un circuito dinamico semplice al quale è collegato un generatore impulsivo,
si determina il circuito resistivo associato, ossia vengono sostituiti i condensatori con 
generatori di tensione e gli induttori con generatori di corrente.

Si ottiene un circuito parziale in cui si spengono i generatori interni (anche quelli equivalenti ai bipoli dinamici) e si lascia agire solo il generatore impulsivo.

Un ulteriore circuito è ottenuto facendo l'esatto contrario e spegnendo quindi il generatore impulsivo.

\begin{figure}[H]\centering
\begin{subfigure}{0.45\linewidth}\centering
\includegraphics[width=\linewidth]{lezione_03_circuito_A}
\subcaption{Circuito iniziale}
\end{subfigure}
\begin{subfigure}{0.45\linewidth}\centering
\includegraphics[width=\linewidth]{lezione_03_circuito_B}
\subcaption{Circuito resistivo associato}
\end{subfigure}
\begin{subfigure}{0.45\linewidth}\centering
\includegraphics[width=\linewidth]{lezione_03_circuito_C-}
\subcaption{Circuito C'}
\end{subfigure}
\begin{subfigure}{0.4\linewidth}\centering
\includegraphics[width=\linewidth]{lezione_03_circuito_C--}
\subcaption{Circuito C"}
\end{subfigure}
\end{figure}
È sempre necessario determinare la dinamica dello stato, in modo da ridurre tutte le equazioni circuitali
ad un'equazione differenziale e non algebrico-differenziale, in seguito si ricavano le restanti 
variabili mediante operazioni algebriche.

Si applica il PSE (Principio di Sovrapposizione degli effetti)
$$
\begin{cases}
v_L = v_L' + v_L''\\
i_C = i_C' + i_C''
\end{cases}
$$

Dal circuito C' si ricavano $v_L'$ e $i_C'$ in funzione dei generatori impulsivi, si determineranno le 
discontinuità delle variabili di stato $i_L$ e $v_C$ a $t=0$.

Viceversa dal circuito C'' si valutano le variabili di stato utilizzando le condizioni iniziali
ricavate in C'.

$$
v_C(0^+) = \frac{1}{C} \int_{0^-}^{0^+} i_C'(\tau)d\tau \ , \ v_C(0^-) = 0
$$
$$
i_L(0^+) = \frac{1}{L} \int_{0^-}^{0^+} i_L'(\tau)d\tau \ , \ i_L(0^-) = 0
$$
L'integrale delle variabili di stato del circuito C'' è pari a 0 dato che la funzione integranda è limitata
e l'intervallo è infinitesimo.

Infine si risolve il circuito C'' usando le variabili di stato appena calcolate in C'.

In sintesi per analizzare un circuito con generatori impulsivi bisogna determinare le condizioni iniziali
utilizzando un circuito ausiliario in cui i generatori non impulsivi sono spenti e ogni induttore e condensatore
è sostituito da un circuito aperto o un corto circuito.

\subsection{Esempio risposta impulsiva circuito del secondo ordine}
Si vuole risolvere il seguente circuito del secondo ordine
\begin{figure}[H]\centering
\begin{circuitikz}
\draw
(0,0) to [R,l=$R_1$] (0,2) to (2,2)
      to [C,l_=$C$,i>_=$i_C$,v^=$v_C$] (2,0) to (0,0);
\draw
(2,2) to [R,l_=$R_2$,i^=$i(t)$] (4,2)
      to [L,l_=$L$,i=$i_L$] (4,0) to (2,0)
;
\draw
(4,2) to (5,2) to [current source,invert,l=$\delta(t)$] (5,0) to (4,0);
\end{circuitikz}
\caption*{Circuito C'}
\end{figure}

Si vuole determinare la risposta impulsiva per la variabile $i(t)$ che non è una variabile di stato.
Si procede con il metodo risolutivo sostituendo i bipoli dinamici con i rispettivi generatori chiusi 
o aperti
\begin{figure}[H]\centering
\begin{circuitikz}
\draw
(0,0) to [R,l=$R_1$] (0,2) to (2,2)
      to [short,i>_=$i_C'$] (2,0) to (0,0);
\draw
(2,2) to [R,l_=$R_2$,i^=$i'(t)$] (4,2)
      to [open,v=$v_L'$] (4,0) to (2,0)
;
\draw
(4,2) to (5,2) to [current source,invert,l=$\delta(t)$] (5,0) to (4,0);
\end{circuitikz}
\end{figure}

Il resistore $R_1$ è in parallelo ad un corto circuito, di conseguenza si vede che la corrente
$i_C'(t)$ è pari a $\delta(t)$ mentre $v_L' = R_2\delta(t)$

Si ricavano le variabili di stato
\begin{align*}
v_C(0^+) &= \frac{1}{C}\int_{0^-}^{0^+} i_C' d\tau = \frac{1}{C} \int_{0^-}^{0^+}\delta(\tau)d\tau = \frac{1}{C}\\
i_L(0^+) &= \frac{1}{L}\int_{0^-}^{0^+} v_L' d\tau = \frac{1}{L} \int_{0^-}^{0^+}R_2\delta(\tau)d\tau = \frac{R_2}{L}
\end{align*}

Si risolve il circuito C'' per determinare le equazioni di stato introducendo i generatori sostitutivi

\begin{figure}[H]\centering
\begin{circuitikz}
\draw
(0,0) to [R,l=$R_1$] (0,2) to (2,2)
      to [voltage source,i>_=$i_C''$,v^=$v_C$] (2,0) to (0,0);
\draw
(2,2) to [R,l_=$R_2$,i^=$i''(t)$] (4,2)
      to [current source,l_=$i_L$,v^>=$v_L''$] (4,0) to (2,0)
;
\end{circuitikz}
\caption*{Circuito C''}
\end{figure}

Applicando il PSE spegnendo prima il generatore di corrente e poi quello di tensione
\begin{align*}
i_C'' &= -\frac{v_C}{R_1} - i_L= C\frac{dv_C''}{dt}\\
v_L'' &= v_C - R_2i_L = L\frac{di_L''}{dt}
\end{align*}
Quelle appena ottenute sono le equazioni di stato sostituendo le equazioni caratteristiche dei bipoli dinamici.
Si ottiene infine il problema di Cauchy soluzione dello stato del circuito
$$\begin{cases}
C\frac{dv_C''}{dt} = -\frac{v_C}{R_1} - i_L \\
L\frac{di_L''}{dt} = v_C - R_2i_L \\
v_C(0^+) = \frac{1}{C} \\
i_L(0^+) = \frac{R_2}{L}
\end{cases}$$
La richiesta dell'esercizio è trovare la $i(t)$, si applica la LKC al nodo tra 
il resistore $R_2$ e l'induttore
$$
i(t) = i_L - \delta(t)
$$
è quindi sufficiente sottrarre l'impulso $\delta(t)$ alla corrente $i_L$ ricavata risolvendo le
equazioni di stato.

Per ricavare $i_L$ si isola $v_C$ dalla seconda equazione di stato (che contiene anche la derivata di $i_L$)
e la si sostituisce nella prima.
$$
v_C = R_2i_L + L\frac{di_L}{dt}
$$
$$
C \frac{d}{dt}\left[R_2i_L + L\frac{di_L}{dt} \right] = -\frac{R_2}{R_1}i_L - \frac{L}{R_1}\frac{di_L}{dt} - i_L
$$
raccogliendo
$$
LC\frac{d^2i_L}{dt^2} + \left(R_2C +\frac{L}{R_1}\right)\frac{di_L}{dt} + \left(1+\frac{R_2}{R_1}\right)i_L = 0
$$
Eseguendo il controllo dimensionale tutti i termini hanno le dimensioni di una corrente (Ampere).
Si suppone che il circuito sia dissipativo e dunque i resistori con resistenza maggiore di zero,
le radici del polinomio caratteristico hanno dunque parte reale negativa (il transitorio si estingue).
\newpage

Aggiungendo le condizioni iniziali si ottiene il problema di Cauchy
$$\left\{
\begin{aligned}
&LC\frac{d^2i_L}{dt^2} + \left(R_2C +\frac{L}{R_1}\right)\frac{di_L}{dt} + \left(1+\frac{R_2}{R_1}\right)i_L = 0
\\
&i_L(0^+) = \frac{R_2}{L}\\
&\frac{di_L}{dt}(0^+) = \frac{1}{L} \left[v_C(0^+) - R_2i_L(0^+)\right] = \frac{1}{LC} - \frac{R_2^2}{L^2}
\end{aligned}\right.
$$

Si valutano le frequenze naturali del circuito risolvendo il polinomio caratteristico, dividendo tutto per $LC$
$$
\lambda^2 + \left(\frac{R_2}{L} + \frac{1}{R_1C}\right)\lambda + \frac{\left(1+\frac{R_2}{R_1}\right)}{LC}=0
$$

Si suppone che $\lambda_1$ e $\lambda_2$ siano reali e distinte, la soluzione del problema sarà del tipo
$$\begin{aligned}
i_L(t) &= K_1e^{\lambda_1t} + K_2e^{\lambda_2t}\\
i_L(0^+) &=\frac{R_2}{L} = K_1 + K_2\\
\frac{di_L}{dt}(0^+) &= \frac{1}{LC} - \frac{R_2^2}{L^2} = \lambda_1K_1 + \lambda_2K_2
\end{aligned}
$$
La corrente incognita $i(t)$ sarà per quanto visto precedentemente mediante la LKC
$$
i(t) = i_L-\delta(t) = \left(K_1e^{\lambda_1t} + K_2e^{\lambda_2t}\right)u(t) -\delta(t)
$$

Si osserva che le variabili di stato possono essere discontinue ma comunque limitate, le altre
variabili possono invece essere anche impulsive.

\newpage
\paragraph{Risposta forzata di un circuito LTI}
Si suppone di avere un solo generatore esterno, se ne vogliono determinare le variabili ai capi
di un solo bipolo, si parla di filtro o sistema SISO \textit{(Single Input Single Output)},
sia $x(t)$ l'ingresso e $y(t)$ l'uscita, si deve supporre che il circuito sia a stato 0, ossia
le sue variabili di stato sono tutte nulle (circuito precedentemente spento).

Si parla di approssimazione \textit{PieceWise-Constant} ossia costante a tratti dell'ingresso
$x(t)$.
Si partiziona l'asse dei tempi in tanti intervalli di ampiezza $\Delta t$ centrati negli istanti
di tempo $t_k = k\Delta t\ ,\ k\ \in\ Z$.

Si costruisce un'approssimazione $x_\Delta(t)$ costante di valore $x(t_k)$ in $\left[t_k-\frac{\Delta t}{2}\ ,\ 
t_k+\frac{\Delta t}{2}\right]$.
La funzione $x(t)$ può quindi essere rappresentata come somma di impulsi rettangolari successivi,
sfruttando la funzione $\Pi_{\Delta}(t)$, sarà quindi:
$$
x_{\Delta}(t) = \sum_{k = -\infty}^{+\infty} x(t_k) \Pi_\Delta(t-t_k)\Delta t
$$
Eseguendo il limite per $\Delta t \rightarrow 0^+$ in ipotesi di sufficiente regolarità:
$$
\lim_{\Delta t \to 0^+} x_\Delta(t) = x(t) = \int_{-\infty}^{+\infty} x(\tau) \delta (t-\tau)
d\tau
$$
Questa conclusione richiama la proprietà di campionamento della $\delta(t)$.

Se si vuole calcolare la risposta di $x(t)$ allora:
$$
\text{Risp}\left\{x(t)\right\} = y(t) = \text{Risp} \left\{\int_{-\infty}^{+\infty} x(\tau)\delta(t-\tau)
d\tau\right\}
$$
per linearità e tempo-invarianza si ottiene
$$
y(t) = \int_{-\infty}^{+\infty} x(\tau)h(t-\tau) d\tau
$$
chiamato integrale di convoluzione, $h(t)$ è la risposta impulsiva per la variabile di uscita.

Si parla di prodotto di convoluzione tra due funzioni $f,g \ \in\ \mathbb{R}$
$$
f * g(t) = \int_{-\infty}^{+\infty} f(\tau)g(t-\tau)d\tau
$$
Noto l'ingresso $x(t)$ si può calcolare la risposta forzata se si conosce la risposta impulsiva
dello stesso circuito, relativa alla medesima grandezza di uscita.


Data la risposta all'ingresso impulsivo, si può calcolare la risposta a qualsiasi ingresso, mediante
l'uso dell'integrale di convoluzione.
Osservando attentamente l'integrale si vede che la risposta impulsiva gode della proprietà per la quale
$$
h(t) = 0 \ t<0,\ h(t) \neq 0 \forall t\ \geq 0 \Rightarrow t - \tau \geq 0 \Rightarrow \tau \leq t
$$

Se l'ingresso $x(t) = 0,\ t< t_0$ allora possiamo affermare che la $y(t)$ avrà come estremi di integrazione
$t_0^-$ e $t$, con il - si sottintende la possibilità che siano presenti $\delta(t_0)$ in quell'istante 
di tempo.

\paragraph{Esempi dell'utilizzo dell'integrale di convoluzione}
Circuito RC parallelo con forzamento esponenziale, forzato da un generatore di corrente.
$$
j(t) = I e ^{\frac{t}{\tau}} u(t)
$$
Per studiare la risposta del circuito bisogna per prima cosa determinare la $V_{cf}(t)$, mediante
lo studio della risposta impulsiva.
Per determinare la risposta impulsiva si forza il circuito con una $\delta(t)$, utilizzando i metodi 
precedenti.

$i_c = \delta(t)$
$$
v_c(0^+) = \frac{1}{C}\int_{0^-}^{0^+} i_c(\tau)d\tau + V_c(0^-) = \frac{1}{C}
$$
Per determinare la risposta impulsiva si determina l'evoluzione libera spegnendo il generatore,
ci sarà un parallelo RC con la seguente equazione di stato:
$$
\begin{cases}
RC\frac{dV_c}{dt} + v_C = 0 \\
V_c(0^+) = \frac{1}{C}
\end{cases}
$$
Quindi 
$$
V_{c_{lib}} = h(t) = \frac{1}{C} e^{-\frac{t}{RC}},\ t\geq 0
$$
$$
h(t) = \frac{1}{C}e^{-\frac{t}{RC}}\cdot u(t) \forall t
$$

Utilizzando l'integrale di convoluzione per calcolare la risposta forzata:

$$
V_{cf}(t) = \int_{0^-}^{t}Ie^{\frac{\tau}{T}} \cdot \frac{1}{C} e^{-\frac{t-\tau}{RC}} d\tau = 
\frac{I}{C} \int_{0^-}^{t} e^{\frac{\tau}{t}}\cdot e^{-\frac{t}{RC}}\cdot e^{\frac{\tau}{RC}} d\tau
$$

$$
= \frac{I}{C} e^{-\frac{t}{RC}}\cdot \frac{e^{t\left(\frac{1}{T}+\frac{1}{RC}\right)}-1}{\frac{1}{RC}+\frac{1}{T}}
$$

Circuito RC serie con forzamento in tensione a rampa lineare $e(t)$
$$
e(t) = \begin{cases}
0\ t<0 \\
\frac{t}{T},\ 0\leq t\leq T\\
0\ t>0
\end{cases}
$$
Possiamo rappresentare questa funzione mediante l'uso di due funzioni $u(t)$

$$
e(t) = \frac{t}{T}\left[u(t) - u(t-T)\right]
$$
Determiniamo quindi la risposta $h(t)$ mediante la risposta al gradino $g(t)$
Le equazione di stato è:
$$
\begin{cases}
1 = RC\frac{dV_c}{dt} + V_c \\
V_c(0) = 0 
\end{cases}
\Rightarrow V_c(t) = 1 - e^{-\frac{t}{RC}},\ t\geq 0$$
Quindi 
$$
g(t) = (1 - e^{-\frac{t}{RC}})u(t) \forall t
$$
ma
$$
h(t) = \frac{dg}{dt} = \frac{1}{RC}e^{-\frac{t}{RC}},\ t\geq 0
$$
Determiniamo ora la risposta forzata $v_{cf}(t)$:
$$
V_{cf}(t) = \int_{0^-}^{t} e(\tau) h(t-\tau)d\tau = \int_{0^-}^{t}\frac{\tau}{T}\left[u(\tau) - u(t-\tau)\right]
\cdot \frac{1}{RC} e ^{-\frac{t-\tau}{RC}}\cdot u(t-\tau)d\tau
$$
Svolgiamo ora l'integrale osservando che $e(\tau) \neq 0 $ solo se $t \in [0,T]$, separiamo quindi
il calcolo dell'integrale in due eventualità:
$$
t\leq T:\ \int_{0^-}^{t}\frac{\tau}{T}\frac{1}{RC}e^{-\frac{t-\tau}{RC}}d\tau = \frac{1}{TRC}e^{-\frac{t}{RC}}\int_{0^-}^{t}\tau e^{\frac{\tau}{RC}}d\tau 
$$
utilizzando l'integrazione per parti:
$$
\left(f\cdot g\right)' = f'g + g'f
$$
$$
f = \tau\ g' = e^{\frac{\tau}{RC}}
$$
$$
\frac{1}{TRC}e^{-\frac{t}{}RC}\int_{0^-}^{t}\frac{d}{dt}\left[\tau e^{\frac{\tau}{RC}}\cdot RC \right] -
RC e^{\frac{\tau}{RC}} d\tau =
$$
$$
\frac{1}{TRC} e^{-\frac{t}{RC}} \left\{ \left[t e^{\frac{t}{RC}}\cdot RC \right] - \left[(RC)^2 e^{\frac{\tau}{RC}} \right]_{0^-}^{t}  \right\} =
$$
$$
\frac{t}{T} - \frac{RC}{T}\left(1-e^{-\frac{t}{RC}}\right)
$$

Secondo caso:
$$
t > T:\ \int_{0^-}^{T} \frac{\tau}{T}\frac{1}{RC} e^{-\frac{t-\tau}{RC}}d\tau =  e^{-\frac{t-T}{RC}}-\frac{RC}{T}\left(e^{-\frac{t-T}{RC}} -e^{-\frac{t}{RC}} \right)
$$

\section{Metodo dei fasori}
Il metodo dei fasori si basa sul concetto che conoscendo l'andamento di regime con grandezze isofrequenziali, si può risolvere il sistema traspotando le grandezze sinusoidali nel dominio simbolico
dei fasori, si rapprentano cioè le grandezze descrittive dei bipoli medianti numeri complessi.

Mediante la L-Trasformata si possono usare ancora una volta equazioni nel dominio complesso ma per 
studiare reti nel regime transitorio.

\paragraph{Analisi delle reti dinamiche LTI con Laplace}
Consideriamo un blocco contenenti tutte le equazioni circuitali nel dominio del tempo,
otteniamo tutte le equazioni di stato di induttori e condensatori.
Questo risultato si ottiene mediante i metodi di risoluzione delle ODE, per arrivare alla conoscenza
della dinamica delle \textit{variabili di stato}, se sono poi interessato ad altre variabili, attravarso
operazioni \textit{algebriche lineari} si conosce la dinamica di utte le variabili del circuito.

In alternativa, sfruttando la trasformata di Laplace, le equazioni ODE vengono trasformate nel dominio
di Laplace, saranno equazioni algebriche lineari anzichè differenziali, risolvendo quindi solo un sistema
di equazioni lineari si conosce direttamente la trasformata delle variabili di stato, effettuando quindi
il procedimento inverso di antitrasformazione si ricavano le variabili di stato nel dominio del tempo,
l'antitrasformazione può essere invece posticipata ed eseguita dopo aver ricavato le generiche variabili
del circuito, ancora una volta con operazioni algebriche.

La soluzione di un sistema di equazioni differenziali è notevolmente più complesso che risolvere un 
sistema lineare, ecco il vantaggio dell'utilizzo della trasformata di Laplace.

Definizione della trasformata di Laplace di una funzione $f(t)\ \in\ [0,+\infty[\to R$:
$$
L[f(t)] = F(s) = \int_{0^-}^{+\infty} f(t) e^{-st}dt \ F:s \in\ C \to C
$$
ammesso che l'integrale improprio converga, per assicurarci che ciò accade si pone la condizione sufficiente, verificata nella stragrande delle situazioni:
$$
\text{Se} \left|f(t) \right| \leq Me^{\alpha t} ,\ M,\alpha \in R \text{ costanti}
$$
allora l'integrale converge, dim.:
$$
\int_{0^-}^{+\infty} f(t) e^{-st}dt \leq \int_{0^-}^{+\infty} \left|f(t)\right| e^{-st}dt \leq
\int_{0^-}^{+\infty} Me^{\alpha t} e^{-st}dt = M\int_{0^-}^{+\infty} e^{(\alpha-s)t}dt
$$
condizione verificata per ogni $\Re \left\{ s\right\} > \alpha$.

Definizione dell'antitrasformata:
$$
L^{-1}\left[F(S) \right] = f(t) = \frac{1}{2\pi i} \lim_{T\to+\infty} \int_{\gamma-iT}^{\gamma+iT}
e^{st}F(s)ds
$$
con $\Re\left\{s\right\} = \gamma$ tale che tutte le singolarità di $F(s)$ si trovino a sinistra di $\gamma$.

\paragraph{Proprietà della L-trasformata} Ricordando quelle utilizzate nel metodo dei fasori:
\begin{itemize}
\item Unicità: $\forall f(t) \in [0,+\infty[\ \exists!\ F(s) = L[F(s)]$

considerate $F(s)$ e $G(s)$, se $F(s) = G(s) \Rightarrow f(t) = g(t)$ quasi ovunque, ossia:
$$\int_{0^-}^{+\infty}|f(t) - g(t)|dt = 0$$

\item Linearità: date $f_1(t)$ ed $f_2(t) :\ [0,+\infty[ \rightarrow R,\ k_1,k_2 \in R$ allora
$$L[k_1f_1(t) + k_2f_2(t)] = k_1F_1(s) + k_2F_2(s) $$
si dimostra con la proprietà di linearità dell'integrale

\item Traslazione nel dominio di Laplace: sia data $f(t) \in [0^-,+\infty[,\ L[f(t)] = F(s)$ e
consideriamo $F(s-\lambda),\ \lambda\ \in\ C$ 
$$F(s-\lambda) = \int_{0^-}^{+\infty}f(t) e^{-(s-\lambda)t}dt = \int_{0^-}^{+\infty} (f(t)e^{\lambda t})e^{-st} dt \Rightarrow F(s-\lambda) = L[f(t) e^{\lambda t}]$$ con $\Re\{s\} > \lambda$ 

\item Derivazione: $f'(t)  = \frac{d}{dt}f(t),\ L[f(t)] = F(s)$ allora 
$$L\left[\frac{d}{dt}f(t)\right] = \int_{0^-}^{+\infty}f'e^{-st}dt \stackrel{\text{x parti}}{=} 
\left[fe^{-st}\right]_0^{+\infty} - \int_{0^-}^{+\infty}(-s)e^{-st}\cdot f dt =$$
$$= \left(\lim_{t\to\infty}\left[fe^{-st}\right]-f(0^-)\right) + sF(s) = sF(s) - f(0^-)$$
$$
L\left[\frac{d}{dt}f(t)\right] = sF(s) - f(0^-)
$$
\item Prodotto di convoluzione nel dominio di Laplace, facendo leva sul teorema di Borel:
$$
L\left[f*g(t)\right] =\int_{0^-}^{+\infty}\left(\int_{0^-}^{t}f(\tau)g(t-\tau)d\tau\right) e^{-st}dt = F(s)\cdot G(s)
$$
per dimostrare questo teorema si scambiano le variabili di integrazione $t$ e $\tau$ sfruttando i teoremi
di \href{https://it.wikipedia.org/wiki/Teorema_di_Fubini}{Fubini} e Tonelli
\end{itemize}
\newpage
Trasformate notevoli di frequente utilizzo nei circuiti:
\begin{itemize}
\item Esponenziale: $f(t) = e^{\lambda t},\ \lambda\ \in R $ 
$$
L\left[e^{\lambda t}\right] = \int_{0^-}^{+\infty} e^{-(s-\lambda) t} dt = \left[\frac{1}{\lambda -s}e^{(\lambda -s)t}\right]_{0^-}^{+\infty} = \frac{1}{\lambda -s} \left[\lim_{t\to\infty}e^{(\lambda-s)t}-1\right] =
\frac{1}{s-\lambda}
$$
\item Funzione gradino $u(t)$:
$$
L[u(t)] = \int_{0^-}^{+\infty}e^{0t}e^{-st}dt = \frac{1}{s}
$$
\item Delta di Dirac $\delta(t)$:
$$
L[\delta(t)] = \int_{0^-}^{+\infty}\delta(t) e^{-st}dt = e^{-s\cdot 0} = 1
$$
\item Funzioni sinusoidali, $\cos(\omega t)=\frac{e^{j\omega t}+e^{-j\omega t}}{2},\ \sin(\omega t) = \frac{e^{j\omega t}-e^{-j\omega t}}{2j}$
$$
L[\cos(\omega t)] = \frac{1}{2} L[e^{j\omega t}] + \frac{1}{2} L[e^{-j\omega t}] = \frac{s}{s^2+\omega^2},\ \Re\{s\} > 0
$$
$$
L[\sin(\omega t)] = \frac{1}{2j}\left(\frac{1}{s-j\omega}-\frac{1}{s+j\omega}\right) = \frac{\omega}{s^2+\omega^2},\ \Re\{ s\} > 0
$$
\end{itemize}


\end{document}
