%\documentclass[a4paper,11pt]{article}
\documentclass[a4paper,11pt]{scrartcl}

\usepackage[utf8]{inputenc}
\usepackage[italian]{babel}

\usepackage{graphicx} %for includng images
\graphicspath{{img/}}

\usepackage{siunitx} % package for deciBel and other units
\usepackage{amsmath} % package for ``cases'' and other matemagical stuff
\usepackage[american]{circuitikz} % circuit drawer
\ctikzset{/tikz/circuitikz/bipoles/length=1cm} %dimensione componenti
\usepackage{subcaption}  %per immagini multiple
%\usepackage{xcolor} % Colori!
\usepackage{colortbl} %colori nelle tabelle!!
\usepackage{multirow} %righe doppie nelle tabelle
\usepackage{makecell} %multirow box
\usepackage{hyperref}
\usepackage{cancel} %utile a cancellare termini nelle equazioni
\hypersetup{ % vedi https://it.overleaf.com/learn/latex/hyperlinks
    colorlinks=true,
    linkcolor=blue,
    filecolor=magenta,      
    urlcolor=blue
}
\usepackage{esint} %utile a plottare integrali doppi chiusi
\usepackage{tikz-3dplot} %per fare plot in 3 assi
\tdplotsetmaincoords{70}{110} %rotazione globale per tutte le rappresentazioni 3D

\usepackage{pgfplots} % plot 2D
\pgfplotsset{compat=1.17}

\usepackage{amssymb} % simboli come R^3

\usepackage{float} %gestire le posizioni dei 'float' es tabella lez 16

\usepackage{listings} %inserire codice MATLAB colorato
\usepackage{xcolor}
\usepackage[framed]{matlab-prettifier}

\title{Appunti di Principi di Ingegneria Elettrica II}
\author{Daniele Olivieri}
\date{}

\pdfinfo{%
  /Title    ()
  /Author   ()
  /Creator  ()
  /Producer ()
  /Subject  ()
  /Keywords ()
}
\includeonly{17_condensatori_con_dielettrici,18}
\begin{document}
\maketitle
Ti piacciono i miei appunti? Saranno sempre liberi e gratuiti ma puoi sostenermi con una donazione cliccando \href{https://www.paypal.com/donate?hosted_button_id=7KELP768NJSYW}{QUI}

Puoi accedere ai codici sorgente seguendo questo \href{https://github.com/FlashNoob98/appunti_principi_II_unina}{link} invece.
\setcounter{tocdepth}{2}
\tableofcontents
\setlength\arrayrulewidth{1.2pt} %larghezza righe tabelle
\section{Introduzione ai circuiti dinamici}
Tutti i circuiti che contengono componenti come
resistori, induttori e condensatori, generatori indipendenti di tensione e corrente ai quali associamo 
una tensione impressa o una corrente impressa vengono detti circuiti lineari tempo invarianti (LTI).

Se vi sono anche bipoli tempo-varianti come interruttori che si chiudono o si aprono, il circuito
si definisce lineare tempo variante (LTV).

Preso un generico circuito, vogliamo determinare una tensione o un'intensità di corrente di un generico
bipolo.
\textit{(Nella vita bisogna sempre porsi un obiettivo)}

\textbf{Strumenti a disposizione:}

\begin{itemize}
\item Equazioni di interconnessione:
\begin{equation} \label{eq:leggi_kirchooff}
\begin{cases}
        \text{LKC} & \forall \text{ nodo } \sum_{k} (\pm) i_k(t) = 0\ \forall\ t \\
        \text{LKT} & \forall \text{ maglia } \sum_{h} (\pm) v_n(t) = 0\ \forall\ t
\end{cases}
\end{equation}

\item Relazioni caratteristiche dei bipoli coerenti con la scelta dei versi delle grandezze (convenzione del generatore o dell'utilizzatore).
\begin{equation}
\begin{cases}
v_R  = R\cdot i \\
v_L  = L\frac{di_L}{dt} \\
i_C  = C\frac{dv_C}{dt} \\
v_e  = e(t)\\
i_j  = j(t) \\
\end{cases}
\end{equation}
\begin{equation*}
\begin{cases}
\text{interruttori in chiusura a t=0: }  {t<0,\ i = 0\ \forall\ v;\ t > 0,\ v = 0\ \forall\ i} \\
\text{interruttori in apertura a t=0: }  {t<0,\ v = 0\ \forall\ i;\ t > 0,\ i = 0\ \forall\ v} 
\end{cases}
\end{equation*}
\end{itemize}

Le equazioni di interconnessione non sono tutte indipendenti ma è sempre possibile
costruire un set di equazioni indipendenti scegliendo \textit{n-1} nodi o \textit{n-l-1} maglie fondamentali,
dove n ed l sono rispettivamente il numero di nodi e lati del grafo connesso.

NOTA: per ogni induttore la potenza assorbita:
$$ P^a(t) = v_l\cdot i_l = L \frac{di_L}{dt}\cdot i_l = \frac{d}{dt} [\frac{1}{2}Li_L^2] = \frac{dW_m}{dt} $$ 
L' energia immagazzinata nel campo magnetico dell'induttore invece:
$$\Delta W^a(t_1,t_2) = \int_{t_1}^{t_2} P^a(\tau)d\tau = W_m (t_2) - W_m(t_1) $$
Se in un certo istante di tempo l'induttore presenta una certa quantità di energia in Joule [\si{\joule}] quella sarà la massima energia estraibile.

Per ogni condensatore invece potenza ed energia sono così definite:
$$P^a(t) = \frac{d}{dt} W_e;\ W_e(t) = \frac{1}{2} CV_c^2$$
L'energia sarà immagazzinata mediante il campo elettrico.

Si ha quindi un sistema di equazioni circuitali in cui si ha una parte algebrica con le caratteristiche adinamiche, ossia con caratteristiche
non differenziali e non integrali, più una parte differenziale data dai bipoli dinamici come condensatori e induttori.
In letteratura un sistema simile si indica con DAE (Differential Algebric Equation), differente
dalla ODE (Ordinary Differential Equation).

Le tecniche utilizzate in teoria dei circuiti mirano a trasformare una DAE in una ODE, ossia 
per formulare le equazioni circuitali come equazioni differenziali ordinarie, per operare questa trasformazione si fa riferimento
alla dinamica delle sole {variabili di stato}: $i_l(t)\ v_c(t)$

La loro conoscenza permette di esprimere tutte le altre variabili attraverso relazioni algebriche, vengono definite
variabili \textit{slaved}, perchè subordinate alle prime.

Per eseguire ciò si richiama una procedura generale per l'analisi di circuiti lineari tempo varianti (LTV).
Questa procedura si basa su un'analisi a intervallo: si partiziona l'asse dei tempi in intervalli in ciascuno dei quali esiste un circuito 
tempo invariante (LTI) equivalente a quello di partenza.

\begin{figure}[h] %tre stati diversi dello stesso circuito
\centering
 \begin{subfigure}{.3\textwidth}
  \centering
  \caption{configurazione generica}
  \begin{circuitikz}
   \draw (0,0) to [R] (1,0);
   \draw (0,0.6) to [L] (1,0.6);
   \draw (0,1.2) to [C] (1,1.2);
   \draw (0,1.8) to [V] (1,1.8);
   \draw (0,2.4) to [I] (1,2.4);
   \draw (0,3) to [closing switch] (1,3);
   \draw (0,3.6) to [opening switch] (1,3.6);
   \draw (-0.3,-0.3) rectangle (1.3,3.9);
  \end{circuitikz}
 \end{subfigure}
 \begin{subfigure}{.3\textwidth}
  \centering
  \caption{$t<0$}
  \begin{circuitikz}
   \draw (0,0) to [R] (1,0);
   \draw (0,0.6) to [L] (1,0.6);
   \draw (0,1.2) to [C] (1,1.2);
   \draw (0,1.8) to [V] (1,1.8);
   \draw (0,2.4) to [I] (1,2.4);
   \draw (0,3) to [open,o-o] (1,3);
   \draw (0,3.6) to [short] (1,3.6);
   \draw (-0.3,-0.3) rectangle (1.3,3.9);
  \end{circuitikz}
 \end{subfigure}
 \begin{subfigure}{.3\textwidth}
  \centering
  \caption{$t>0$}
  \begin{circuitikz}
   \draw (0,0) to [R] (1,0);
   \draw (0,0.6) to [L] (1,0.6);
   \draw (0,1.2) to [C] (1,1.2);
   \draw (0,1.8) to [V] (1,1.8);
   \draw (0,2.4) to [I] (1,2.4);
   \draw (0,3) to [short] (1,3);
   \draw (0,3.6) to [open,o-o] (1,3.6);
   \draw (-0.3,-0.3) rectangle (1.3,3.9);   
  \end{circuitikz}
 \end{subfigure}
\end{figure}

Supponiamo che i generatori indipendenti abbiano grandezze \textbf{limitate}, ossia le tensioni impresse $e(t)$ e le correnti impresse $j(t)$,
in questa ipotesi sappiamo che le variabili di stato sono funzioni \textbf{continue} ossia:
\begin{equation*}
\begin{split}
i_L (0^+) & = i_L(0^-) \\ 
v_C (0^+) & = v_C(0^-)
\end{split}
\end{equation*}

La soluzione si determina trovando la dinamica delle variabili di stato in ciascun circuito ausiliario, "incollando" le soluzioni utilizzando la proprietà di continuità delle variabili di stato.

Il problema di risolvere circuiti lineari tempo varianti si scompone nel risolvere tanti circuiti lineari tempo invarianti. Una categoria semplice di circuiti tempo invarianti sono i
circuiti lineari del primo ordine (circuiti RC o RL) con un solo elemento dinamico.
\begin{figure}[h] %circuiti RC ed RL equivalenti
\centering
 \begin{subfigure}{.3\textwidth}
  \centering
  \begin{circuitikz}
   \draw (0,0) rectangle (1.6,1.7);
   \draw (0.3,0.3) to [I] (1.3,0.3);
   \draw (0.3,0.9) to [V] (1.3,0.9);
   \draw (0.3,1.5) to [R] (1.3,1.5);
   \draw (1.6,1.5) to [short,i=$i_C$] (2,1.5)
   to [C,v^=$v_C $] (2,0.4) to [short] (1.6,0.4);
  \end{circuitikz}
  \caption{Circuito RC}
 \end{subfigure} 
  \begin{subfigure}{.3\textwidth}
  \centering
  \begin{circuitikz}
   \draw (0,0) rectangle (1.6,1.7);
   \draw (0.3,0.3) to [I] (1.3,0.3);
   \draw (0.3,0.9) to [V] (1.3,0.9);
   \draw (0.3,1.5) to [R] (1.3,1.5);
   \draw (1.6,1.5) to [short,i=$i_L $] (2,1.5)
   to [L,v^=$v_L $] (2,0.4) to [short] (1.6,0.4);
  \end{circuitikz}
  \caption{circuito RL}
 \end{subfigure}
\end{figure}

Un bipolo adinamico lineare e un bipolo dinamico fanno subito pensare all'utilizzo dei teoremi di Thévenin e Norton, permettendo la riduzione del circuito 
adinamico ad un semplice generatore con un resistore equivalente.

Applicando ad esempio la LKT $e_0(t) = R_{th}\cdot C \frac{dV_C}{dt} + V_C$ si ha l'equazione di stato
del circuito e supponendo di conoscere $V_C(t=0) = V_0 $ allora la soluzione dell'equazione è:
$$V_C(t) = [V_0-V_{C_p}(0)] e ^{-\frac{t}{\tau}} + V_{C_p}(t)$$ dove $\tau = R_{th}\cdot C$ e $V_{C_p}(t)$
è la soluzione a regime. 
 
Dopo un intervallo pari a $4\sim5\ \tau$ si assume il processo di carica o scarica terminato.

Per il circuito RL 
$$I_L(t) = [I_0-I_{L_p}(0)]e^{-\frac{t}{\tau}} + I_{L_p}(t)$$ 
con $\tau = \frac{L}{R_{th}},\ I_0 = I_L (t=0)$

Osservazione: la soluzione generale del circuito RC, ossia la dinamica di $V_c(t)$ può essere espressa 
come la somma di due termini $V_{C_{tr}}(t)$ e $V_{C_p}(t)$, il primo transitorio, che tende a svanire se 
attendiamo un tempo sufficiente lungo, porta con se la \textit{memoria} dello stato iniziale, memoria che 
viene
persa quando $t>4\sim5\tau$, ammesso che $\tau$ sia positiva; esistono infatti alcune combinazioni di 
elementi circuitali si comportano come un resistore negativo.

Il secondo termine è quello di regime permanente, che ovviamente non ha memoria dello stato iniziale ma 
dipende solo dalla nuova configurazione
del circuito (termine forzato).

La decomposizione in regime transitorio e permanente è una decomposizione generale che vale per qualsiasi circuito, a patto di complicare opportunamente la matematica.

Si parla inoltre di evoluzione libera ed evoluzione forzata 
$$\begin{cases}
e_0(t) = R_{th}C\frac{dV_c}{dt} + V_C \\
V_c(0) = V_0
\end{cases}$$

si scompone in:

$$\begin{cases} %evoluzione libera
0 &= R_{th}C\frac{dV_c}{dt} + V_C \\ 
V_c(0) &= V_0
\end{cases}$$

$$\begin{cases} %evoluzione forzata
e_0(t) &= R_{th}C\frac{dV_c}{dt} + V_C \\
V_c(0) &= 0
\end{cases}$$

Trattazione analoga (duale) per il circuito RL

\section{Circuiti lineari tempo invarianti del II ordine}
Divisi in (RC, RL, RLC), la procedura generale di risoluzione richiede tre step:
\begin{enumerate}
 \item Determinazione delle equazioni di stato
 \item Determinazione delle condizioni iniziali
 \item Soluzione del problema di Cauchy
\end{enumerate}

Esempio:
\begin{figure}[h]
\centering
\begin{circuitikz}
 \draw (0,2) to [V,l_=$E$] (0,0);
 \draw (0,2) to [closing switch=${t=0}$](1,2) 
             to [R,l=$R_1$] (2.5,2)
             to [R,l=$R_2$] (4,2)
             to [L,i=$i_L$,l=$L$] (4,0)
             to [short] (0,0);
 \draw (2.5,2)node[circ,color=red]{} to [C,v=$v_C$,l=$C$] (2.5,0);
\end{circuitikz}
\caption{Esempio di un circuito RLC del secondo ordine}
\end{figure}

La dinamica dello stato per $t<0$ è di facile risoluzione, $i_L(t) = 0$, $v_C(t) = 0$
per $t > 0$ invece si analizza il circuito:

Equazioni di interconnessione: 
LKC nel nodo evidenziato in rosso (tra i due resistori)
\begin{equation*}
\begin{cases}
 I_1 &= I_C+I_L \ \text{LKC}\\
 E &= R_1\cdot I_1 + V_C\ \text{LKT} \\
V_C &= R_2 \cdot I_L + V_L \\
V_L &= L\frac{dI_L}{dt} \\
I_C &= C\frac{dV_C}{dt}
\end{cases}
\end{equation*}
Ricaviamo $I_1$ dalla seconda e sostituiamola nella prima,
ora vanno ricavate le variabili di stato ottenendo
$$
\begin{cases}
i_C = \frac{E}{R_2} - \frac{v_C}{R_1} - i_L \\
v_L = v_C - R_2 i_L
\end{cases}
$$
Sostituendo le equazioni caratteristiche dei bipoli dinamici:

$$
\begin{cases}
C\frac{dv_C}{dt} = \frac{E}{R_2} - \frac{v_C}{R_1} - i_L \\
L\frac{di_L}{dt} = v_C - R_2 i_L
\end{cases}
$$

Le condizioni iniziali sono 
$$\begin{cases}
v_C(0^+) = v_C(0^-) = 0\\
i_L(0^+) = i_L(0^-) = 0
\end{cases}
$$
Per risolvere queste equazioni conviene ridurre l'equazione del secondo ordine ad una sola delle incognite,
ad esempio si sostituisce nell'equazione che definisce $i_C$, $v_C$ ricavata dalla equazione di
$v_L$, in questo modo si ha un'unica equazione in cui compaiono le grandezze relative all'induttore.

Si ottiene 
\begin{equation}
LC \frac{d^2i_L}{dt^2} + CR_2\frac{di_L}{dt} = \frac{E}{R_1} - \frac{L}{R_1}\frac{di_L}{dt} - \frac{R_2}{R_1}i_L-i_L
\end{equation}
%ora raccogli i termini
Controlli da eseguire: controllo dimensionale, positività dei coefficienti altrimenti il circuito non sarebbe dissipativo.

L'integrale generale è scritto come somma dell'integrale dell'omogenea associata e dell'integrale particolare.
Polinomio caratteristico:
\begin{equation}
 \lambda^2 + (\frac{R_2}{L} + \frac{1}{R_1 C})\lambda + (1+\frac{R_2}{R1})\frac{1}{LC} = 0
\end{equation}
La soluzione del polinomio può essere di tre tipi:

Caso 1, radici reali e distinte $\Rightarrow$
Vedi grafico 01:38:00 modi naturali aperiodici smorzati

Caso 2 radici reali e coincidenti, caso piuttosto patologico
la radice viene determinata con $-\sigma$ avremo un modo esponenziale decrescente $e^{-\sigma t} $ con $\tau = \frac{1}{\sigma}$
e un modo pari a  $te^{-\sigma t}$

Caso 3 modo periodico smorzato
$\lambda_{1,2} = -\sigma \pm j\omega d$, la distanza tra due picchi è pari a $\frac{2\pi}{\omega}$
$$i_{L_0}(t) = e^{-\sigma t} [K_1 \cos (\omega_d t) + K_2 \sin(\omega_d t)]$$

Restano da determinare le costanti di integrazione imponendo le condizioni iniziali,
\begin{equation*}
\begin{cases}
i_L(0^+) = i_L(0^-) \\
\frac{d_{i_L}}{dt}(0^+) = \frac{1}{L}[v_C(0^+) - R_2I_L(0^+)] = 0
\end{cases}
\end{equation*}

%Lezione 2 i minuti si riferiranno a quelli visibili su teams e non quelli del file registrato con OBS
\subsection{Determinazione equazioni di stato di un circuito qualsiasi}

Si riprende la classe di circuiti lineari tempo invarianti (LTI), si suppone di conoscere
le variabili di stato $i_L(t) $ e $v_C(t) $ assumendole note, si sostituisce ogni \textit{condensatore}
con un generatore di tensione di valore pari alla $v_C(t)$, ripetendo il procedimento
per ciascun \textit{induttore} che viene sostituito con un generatore di corrente con corrente impressa
pari ad $i_L(t)$.

In queste condizioni, la soluzione del circuito resta formalmente invariata, 
il nuovo circuito sarà di tipo adinamico, non presenterà più alcun componente dinamico,
prende il nome di \textit{circuito resistivo associato al circuito di partenza}.

Il vantaggio di questa operazione è la possibilità di ricavare $v_L$ e $i_C$ utilizzando il principio
di \textit{sovrapposizione degli effetti} (PSE).

Si ricavano le equazioni di stato per il seguente circuito:
\begin{figure}[H]
\centering
\begin{circuitikz}
\draw
(0,0) to [voltage source,invert,l=$E$] (0,2)
to [R=$R_1$] (2,2)
to [C,v^=$v_C$,l_=$C$,i_=$i_C$] (2,0) -- (0,0)
;\draw
(2,2) to [R=$R_2$] (4,2)
to [L=$L$,i=$i_L$,v=$v_L$] (4,0) -- (2,0)
;
\end{circuitikz}
\end{figure}

Si applica il PSE, per trovare $i_C$ e $v_L$:
$$\begin{aligned}
i_C &= i_c' + i_C'' + i_C'''\\
v_L &= v_L' + v_L'' + v_L'''
\end{aligned}$$

\begin{figure}[H]
\centering
\begin{subfigure}{.49\linewidth} %Circuito C'
\centering
\begin{circuitikz}
\draw
(0,0) to [voltage source,invert,l=$E$] (0,2)
to [R=$R_1$] (2,2)
to [short,i_=$i_C'$] (2,0) -- (0,0)
;\draw
(2,2) to [R=$R_2$] (4,2)
to [open,v=$v_L'$] (4,0) -- (2,0)
;
\end{circuitikz}
\caption{Circuito 1}
\end{subfigure}
\begin{subfigure}{.49\linewidth} %Circuito C''
\centering
\begin{circuitikz}
\draw
(0,0) to [short] (0,2)
to [R=$R_1$] (2,2)
to [voltage source,v^=$v_C$,i>_=$i_C''$] (2,0) -- (0,0)
;\draw
(2,2) to [R=$R_2$] (4,2)
to [open,v=$v_L''$] (4,0) -- (2,0)
;
\end{circuitikz}
\caption{Circuito 2}
\end{subfigure}
\begin{subfigure}{.49\linewidth} %Circuito C'''
\centering
\begin{circuitikz}
\draw
(0,0) to [short] (0,2)
to [R=$R_1$] (2,2)
to [short,i_=$i_C'''$] (2,0) -- (0,0)
;\draw
(2,2) to [R=$R_2$] (4,2)
to [current source,i_=$i_L$,v^>=$v_L'''$] (4,0) -- (2,0)
;
\end{circuitikz}
\caption{Circuito 3}
\end{subfigure}
\end{figure}
\newpage
\begin{itemize}
\item Circuito 1)
$$
v_L' = 0;\ i_C' = \frac{E}{R_1} 
$$
\item Circuito 2)
$$
i_C'' = -\frac{v_C}{R_1};\ v_L'' = v_C
$$
\item Circuito 3)
$$
i_C''' = -i_L;\ v_L''' = -R_2\cdot i_L
$$
\end{itemize}
Sommando i tre contributi e aggiungendo il vincolo di continuità delle variabili
di stato si ottiene il problema di Cauchy del sistema:
$$\left\{\begin{aligned}
&i_C = \frac{E}{R_1} - \frac{v_C}{R_1} - i_L = C\frac{dv_C}{dt}\\
&v_L = v_C - R_2\cdot i_L = L\frac{di_L}{dt}\\
&i_L(0^+) = i_L(0^-)\\
&v_C(0^+) = v_C(0^-)
\end{aligned}\right.$$

\subsection{Circuiti lineari con generazioni impulsivi}
Si analizza ora un circuito che presenta generatori di tipo impulsivo, ad esempio la risposta di 
una linea elettrica a seguito di una fulminazione.
Si definisce quindi l'impulso rettangolare di ampiezza $\Delta$, la funzione viene chiamata 
$\Pi_\Delta(t)$, (funzione porta) è costante nell'intervallo $[-\frac{\Delta}{2},\frac{\Delta}{2}]$,
l'area del rettangoloide sotteso alla funzione è pari a
$$
\int_{-\Delta/2}^{\Delta/2}\Pi_\Delta(\tau)d\tau = 1\ \forall\ \Delta \in\ ]0,+\infty[
$$
Ha senso considerare la successione di funzioni ottenute per valori $\Delta$ decrescenti,
ma dimezzando la base, per mantenere l'area unitaria, va raddoppiata l'altezza.

Passando al limite per $\Delta \rightarrow 0^+$ la successione tende in maniera non ordinaria
ad un limite che non è una funzione ma può essere definita come funzione generalizzata,
ossia distribuzione, che prende il nome di \textbf{Delta di Dirac} ($\delta(t)$).

Proprietà della $\delta(t)$:
\begin{itemize}
\item È nulla $\forall\ t \neq 0$
\item Ha integrale unitario
\item Proprietà di campionamento $\int_{-\infty}^{+\infty}f(\tau)\delta(\tau-t_0)d\tau = f(t_0)$
\end{itemize}
\newpage
La proprietà di campionamento calcolata come l'integrale della $\delta(t_0)$ per una funzione ordinaria
permette di valutare la funzione ordinaria nel punto $t_0$ (pari a 0 in questo caso per semplicità)
in cui è centrata la $\delta(t_0)$, si dimostra
utilizzando la definizione di $\delta(t)$ mediante successioni di funzioni porta $\Pi_\Delta(t)$
$$
\int_{-\Delta/2}^{\Delta/2} f(\tau)\Pi_\Delta(\tau-t_0)d\tau = \frac{1}{\Delta} \int_{-\Delta/2}^{\Delta/2}  f(\tau)d\tau = \frac{1}{\Delta}f(\vartheta^*)\cdot\Delta= f(\vartheta^*)
$$
con $\vartheta^* \in\ \left]-\frac{\Delta}{2},\frac{\Delta}{2}\right[$. Eseguendo il limite $\Delta\to 0$ si ha
$\vartheta^* \to 0 $, in conclusione
$$
\int_{0^-}^{0^+} f(\tau)\delta(\tau)d\tau = f(0)
$$

Si analizza la funzione $U_\Delta(t)$ definita come segue:
\begin{equation*}
\begin{cases}
0 & t  < -\frac{\Delta}{2} \\
1 & t  > \frac{\Delta}{2} \\
\frac{1}{2}+\frac{t}{\Delta} & -\frac{\Delta}{2} \leq t \leq \frac{\Delta}{2}
\end{cases}
\end{equation*}
\begin{figure}[H]\centering
\begin{tikzpicture}
\begin{axis}[
    axis lines = center,
    xlabel = \(t\),
    ylabel = {\(U_\Delta(t)\)},
    ylabel style = {at={(axis cs: -0.9,0.95)}},
    xtick = {-2,-1,0,1,2},
    xticklabels = { , $-\frac{\Delta}{2}$, , $\frac{\Delta}{2}$,},
    ytick = {0 ,0.5 ,1},
    yticklabels = { , ,1},
    xmin =  -2, xmax = 2,
    yscale = 0.5,
]
%Below the red parabola is defined
\addplot [
    domain=-1:1, 
    samples=2, 
    color=black,
]
{0.5 + x/2};
\draw [thick] (axis cs:1,1) -- (axis cs:2,1);
\draw [thick] (axis cs:-2,0) -- (axis cs:-1,0);
\draw [dashed] (axis cs:0,1) -- (axis cs:0.5,1);
\addplot [
    domain = -0.5:0.5,
    samples = 2,
    color = green,
    style = thick,
]{0.5 + x};
\draw [thick,dashed,color = green] (axis cs:0.5,1) -- (axis cs:1,1);
\end{axis}
\end{tikzpicture}
\end{figure}

La derivata temporale nei punti regolari coincide quasi ovunque con la funzione $\Pi_\Delta(t)$.
Eseguendo il limite su $U_\Delta(t)$
di $\Delta \rightarrow 0^+$ come si nota dall'andamento del segmento in verde, si ottiene la funzione definita 
``gradino'' o funzione di Heaviside $u(t)$.
$$
u(t) = \begin{cases}
0 & t\leq 0\\
1 & t > 0
\end{cases}
$$


Un ulteriore modo per definire la $\delta(t)$ è appunto quella di derivata della funzione gradino $u(t)$.
$$
\delta(t) = \frac{d}{dt}u(t) \Leftrightarrow \int_{-\infty}^t \delta(\tau)d\tau = u(t)
$$
\newpage
\paragraph{Esempio con generatore impulsivo}
Si consideri un circuito RC serie
\begin{figure}[H]\centering
\begin{circuitikz}
\draw
(0,0) to [voltage source,invert,l=$e(t)$] (0,2)
      to [R=$R$] (2,2)
      to [C,l_=$C$,v^=$v_C$] (2,0) -- (0,0)
;
\end{circuitikz}
\end{figure}

una funzione 
$$
e(t) = \begin{cases}
E_0 & 0 < t < T\\
0 &\text{altrimenti}
\end{cases}
$$
Si vuole ricavare $v_C(t)$:

Si suppone che la condizione iniziale, per $t < 0 $, ossia la tensione sul condensatore sia nulla.
$$\begin{aligned}
&\begin{cases}
e(t) &= RC\frac{dv_C}{dt} + v_C \\
v_C(0) &= \SI{0}{\volt}
\end{cases}\\
&0 \leq t \leq T\\
&\begin{cases}
E_0 &= RC\frac{dv_C}{dt} + v_C \\
v_C^{(1)}(0) &= \SI{0}{\volt} 
\end{cases}\\
&t > T\\
&\begin{cases}
0 &= RC\frac{dv_C}{dt} + v_C \\
v_C^{(2)}(0) &= v_C^{(1)}(T) 
\end{cases}
\end{aligned}
$$
Conoscendo la forma della soluzione vanno solo sostituiti i termini nelle varie configurazioni
$$
v_C^{(1)}(t) = -E_0 e^{-\frac{t}{\tau}} + E_0,\ v_C^{(1)}(T) = E_0 \left(1-e^{-\frac{T}{\tau}}\right),\ \tau = RC
$$
$$
v_C^{(2)}(t) = E_0\left(1-e^{-\frac{T}{\tau}}\right) e^{-\frac{t-T}{\tau}} = v_C^{(1)}(T)\cdot e^{-\frac{t-T}{\tau}}
$$
In conclusione
$$
v_C(t) = \left\{
\begin{aligned}
&0 \qquad\qquad\qquad\qquad\qquad\ t\leq 0\\
&E_0 \left(1-e^{-\frac{T}{\tau}}\right) \qquad\qquad\ \ 0 < t < T\\
& E_0\left(1-e^{-\frac{T}{\tau}}\right) e^{-\frac{t-T}{\tau}} \qquad t \geq T
\end{aligned}\right.
$$
\begin{figure}[H]
\centering
\includegraphics[width = 0.4\linewidth]{v_C_lezione_2}
\end{figure}

Si ottiene una funzione esponenziale crescente fino a $T$ e poi decrescente fino a 0 all'infinito. 
Diminuendo il valore di $T$ si vede che il ``picco'' della funzione sarà più basso, al limite 
di $T \rightarrow 0$ la soluzione si annulla.
Se si impone il prodotto $E_0\cdot T = 1$ ossia $E_0 = \frac{1}{T}$ e si esegue il limite invece:
$$
\lim_{T\rightarrow0^+} v_C(t)
$$
Il limite viene eseguito sulle singole parti della funzione, ad esempio nel tratto $0\leq t \leq T$
la funzione ha sempre un tempo inferiore per raggiungere il valore massimo, mentre l'ampiezza
del generatore aumenta.

Si suppone di sviluppare la funzione esponenziale con la sua serie di Taylor:
$$
e^x = 1 + x + \frac{x^2}{2!} + \frac{x^3}{3!} + ...\simeq 1 + x \Rightarrow 1-e^{-\frac{t}{\tau}} \simeq \frac{t}{\tau} 
$$
L'equazione della tensione diventa
$$v_C(t) = 
\begin{cases}
0\ & t\leq 0 \\
\frac{1}{T}\frac{t}{\tau}\  & 0 \leq t\leq T \\
\frac{1}{T}\frac{T}{\tau} e^{-\frac{t-T}{\tau}}\ & t\geq T
\end{cases}
$$
Per $T\rightarrow 0^+$
$$v_C(t) =
\begin{cases}
0\ & t\leq 0\\
\frac{1}{\tau}e^{\frac{-t}{\tau}}\ & t \geq 0
\end{cases}
$$

Il primo tratto dell'equazione si approssima quindi ad un tratto lineare fino a T, arrivando ad 
un'altezza di $\frac{1}{\tau}$, al limite raggiunge questo valore in un tempo infinitesimo.
\begin{figure}[H]
\centering
\includegraphics[width = 0.4\linewidth]{v_C_limite_lezione_2}
\end{figure}
Si otterrebbe la stessa soluzione ponendo $e(t) = \Pi_\Delta(t)$ e svolgendo il limite
$$\Pi_\Delta(t) \stackrel{\Delta\rightarrow0^+}{\rightarrow} \delta(t) \Rightarrow v_C(t) \rightarrow h(t)$$
$h(t)$ è chiamata risposta all'impulso del circuito dinamico.

In alternativa è possibile utilizzare direttamente un generatore impulsivo e risolvere il circuito

\begin{figure}[H]\centering
\begin{circuitikz}
\draw
(0,0) to [voltage source,invert,l=$\delta(t)$] (0,2)
      to [R=$R$] (2,2)
      to [C,l_=$C$,v^=$v_C$,i_=$i_C$] (2,0) -- (0,0)
;
\end{circuitikz}
\end{figure}

$$
\delta(t) = RC\frac{dv_C}{dt} + v_C \Rightarrow i_C = \frac{\delta(t)-v_C}{R}
$$
Se la tensione del generatore è impulsiva anche la corrente nel condensatore sarà di tipo impulsivo
mentre la tensione ai capi del condensatore si può ricavare integrando la sua equazione caratteristica
$$
i_c = C\frac{dv_C}{dt} \Rightarrow \int_{0^-}^{0^+} i_C(\tau)d\tau = C[v_C(0^+)-v_C(0^-)]
$$
$$
v_C(0^+) - v_C(0^-) = v_C(0^+) = \frac{1}{RC} \int_{0^-}^{0^+} \delta(\tau)d\tau - \cancel{\frac{1}{RC} \int_{0^-}^{0^+} v_C(\tau)d\tau} = 
$$
$$
= \frac{1}{RC} = \frac{1}{\tau}  \Rightarrow v_C(0^+) = \frac{1}{RC} = \frac{1}{\tau}
$$
Il secondo integrale vale zero perché esteso ad un intervallo infinitesimo di una funzione limitata,
perché legata all'energia immagazzinata nel condensatore che non può essere infinita.
% CONSIDERAZIONE PERSONALE
%dunque non potrebbe essere infinita a meno
%di non supporre un generatore dotato di potenza infinita (non possibile in questo caso data la 
%resistenza inserita nel circuito).
%
È infinita invece la potenza assorbita dal condensatore nell'istante $0^+$ che gli permette di avere
una tensione ai suoi capi discontinua.

\newpage
\subsection{Risposta al gradino unitario di un circuito dinamico LTI}
Stesso circuito del precedente, ma si utilizza come forzamento il gradino unitario di
Heaviside, la soluzione è più semplice della precedente:

$$
v_C(t) = \left(1-e^{-\frac{t}{\tau}}\right) u(t)
$$
viene chiamata $g(t)$ e si afferma che sia la risposta al gradino, ricordando la relazione
tra la funzione $\Pi_\Delta(t)$ e $U_\Delta(t)$ si ha che:
$$
\Pi_\Delta(t) = \frac{U_\Delta\left(t+\frac{\Delta}{2}\right)-U_\Delta\left(t-\frac{\Delta}{2}\right)}{\Delta} = e(t)
$$
per $\Delta \rightarrow 0^+$ si ottiene $h(t) = v_C(t)$ ossia la risposta all'impulso.

Essendo il circuito tempo invariante, si può trovare la risposta alla funzione $\Pi_\Delta(t)$
come combinazione lineare delle risposte delle due $U_\Delta$ opportunamente traslate, ossia
la risposta al gradino traslata.
$$
\text{Risp}\left\{ \Pi_\Delta(t)\right\} = \frac{\text{Risp}\left\{U_\Delta\left(t+\frac{\Delta}{2}\right)\right\} - 
\text{Risp}\left\{U_\Delta\left(t-\frac{\Delta}{2}\right)\right\}}{\Delta} = 
\frac{g\left(t+\frac{\Delta}{2}\right) - g\left(t-\frac{\Delta}{2}\right)}{\Delta}
$$
Tutto si trasforma nella funzione rapporto incrementale della funzione $g(t)$ ossia
$$
\lim_{\Delta\rightarrow0^+}\text{Risp}\left\{\Pi_\Delta(t)\right\} = h(t) = \frac{dg}{dt}
$$

$$
h(t) = \frac{dg}{dt} =\begin{cases}
0& t < 0\\
\frac{1}{\tau}e^{-\frac{t}{\tau}} & t\geq0 
\end{cases}
$$
La risposta all'impulso del circuito è quindi pari alla derivata della risposta al gradino dello stesso
circuito.

Si consideri un circuito RC serie
\begin{figure}[H]\centering
\begin{circuitikz}
\draw
(0,0) to [voltage source,invert,l=$u(t)$] (0,2)
      to [R=$R$] (2,2)
      to [C,l_=$C$,v^=$v_C$] (2,0) -- (0,0)
;
\end{circuitikz}
\end{figure}
si suppone che la tensione imposta al generatore sia un gradino unitario $u(t)$,
l'equazione di stato sarà:
$$
\begin{cases}
u(t) = RC \frac{dv_C}{dt} + v_C \\
v_C(0^+) = 0
\end{cases}
\Rightarrow v_C(t) = \left.
\begin{cases}
0 & t<0 \\
1-e^{-\frac{t}{\tau}} & t\geq 0
\end{cases}\right] = g(t)
$$

Si considera la funzione $U_\Delta$
$$U_\Delta(t) = 
\begin{cases}
0 & t< -\frac{\Delta}{2}\\
\frac{1}{2} + \frac{t}{\Delta} & -\frac{\Delta}{2} < t < \frac{\Delta}{2} \\
1 & t > \frac{\Delta}{2}
\end{cases}
$$
se si esegue la differenza di due funzioni $U_\Delta$ traslate di $\pm\frac{\Delta}{2}$ si ottiene 
una porta trapezoidale
$$
f(t) = \frac{U_\Delta\left(t+\frac{\Delta}{2}\right) - U_\Delta\left(t-\frac{\Delta}{2}\right)}{\Delta}
$$
tende ad una $\delta(t)$ delta di Dirac per $\Delta \rightarrow 0$.
Semplicemente si può invece definire la porta come differenza di due gradini traslati, in questo modo
si elimina il problema dei segmenti obliqui.

La linearità del sistema e la tempo-invarianza delle grandezze dei bipoli implica che la
risposta ad una combinazione lineare di funzioni traslate nel tempo si ottiene come combinazione lineare 
delle risposte dei singoli termini traslati.
$$
\text{Risp}\left\{\Pi_\Delta(t)\right\} = \frac{\text{Risp}\left\{u\left(t+\frac{\Delta}{2}\right) \right\} -\text{Risp}\left\{ u\left(t-\frac{\Delta}{2}\right) \right\}}{\Delta} = \frac{g\left(t+\frac{\Delta}{2}\right)-
g\left(t-\frac{\Delta}{2}\right)}{\Delta}
$$
Eseguendo il limite per $\Delta\to 0^+ $ si vede che quello appena presentato è un rapporto incrementale e
quindi
$$
\lim_{\Delta\to0^+} \Rightarrow \text{Risp} \left\{\delta(t)\right\} =h(t) = \frac{dg}{dt}
$$
Ricordando le funzioni $h(t)$ e $g(t)$ si vede la relazione
$$
h(t) = \begin{cases}
0 & t<0\\
\frac{1}{\tau}e^{-\frac{t}{\tau}} & t\geq 0
\end{cases}\qquad
g(t) = \begin{cases}
0 & t<0\\
1 - e^{-\frac{t}{\tau}} & t\geq 0
\end{cases} \Rightarrow
\frac{dg}{dt} = h(t)
$$
è possibile studiare la risposta all'impulso sfruttando quella al gradino, che è una funzione limitata e 
più semplice da analizzare.

\paragraph{Circuiti RC ed RL semplici con generatori impulsivi}
Si considerino 2 circuiti modello: il circuito RC parallelo e il circuito RL serie.

\begin{figure}[H]\centering
\begin{subfigure}{.4\textwidth}\centering
\begin{circuitikz}
\draw
(0,0) to [current source,l=$j(t)$] (0,2)
      to (1,2) to [R=$R$,i>^=$i_R$] (1,0) to (0,0);
\draw
(1,2) to (2.5,2)  to [C,l_=$C$,v^=$v_C(t)$,i>_=$i_C$] (2.5,0) -- (1,0)
;
\end{circuitikz}
\subcaption{RC parallelo}
\end{subfigure}
\begin{subfigure}{.4\textwidth}\centering
\begin{circuitikz}
\draw
(0,0) to [voltage source,invert,l=$e(t)$] (0,2)
      to [R,l_=$R$] (2.5,2) to [L,l_=$L$,i_>=$i_L(t)$,v^=$v_L$] (2.5,0) -- (0,0)
;
\end{circuitikz}
\subcaption{RL serie}
\end{subfigure}
\end{figure}

Siano i generatori impulsivi: $j(t) = Q\delta(t)$ ed $e(t) = \Phi\delta(t)$.
$$
j(t) = \frac{v_C}{R} + i_C = Q\delta(t) = \frac{v_C}{R} + C\frac{dv_C}{dt}
$$
Si integra la funzione nell'istante in cui è centrata la $\delta(t)$ ossia:
$$
\int_{0^-}^{0^+} Q\delta(\tau)d\tau  = \cancel{\int_{0^-}^{0^+}\frac{v_C}{R}d\tau} + C[v_C(0^+)-\cancel{v_C(0^-)}]
\Leftrightarrow Q = Cv_C(0^+) \ \ [Q] = \si{\coulomb} \text{ (Coulomb)}
$$
$$
\left[\int_{0^-}^{0^+} Q\delta(\tau)d\tau\right] = \text{ Coulomb}
$$
Tirando la costante $C$ fuori dall'integrale, che ha la dimensione di Coulomb, l'integrale rimanente
deve essere adimensionale.
Ciò significa che la $\delta(t)$ ha la dimensione di \si{\per\second} per essere coerente con l'integrale
e restituire una quantità finita.

\subparagraph{Caso duale con circuito RL:}

$$
e(t) = R\cdot i_L + v_L
$$
$$
\Phi\delta(t) = R\cdot i_L + L\frac{di_L}{dt}
$$
$$
\Phi\int_{0^-}^{0^+} \delta(\tau)d\tau = L i_L(0^+)\ , \ i_L(0^-) = 0
$$

Anche in questo caso per avere la dimensione del flusso in \si{\weber} per $\Phi$ allora la $\delta(t)$ 
avrà le dimensioni di \si{\per\second}.
\newpage
\subsection{Procedura generale per la risoluzione di circuiti con generatori impulsivi}
Si ha un circuito dinamico semplice al quale è collegato un generatore impulsivo,
si determina il circuito resistivo associato, ossia vengono sostituiti i condensatori con 
generatori di tensione e gli induttori con generatori di corrente.

Si ottiene un circuito parziale in cui si spengono i generatori interni (anche quelli equivalenti ai bipoli dinamici) e si lascia agire solo il generatore impulsivo.

Un ulteriore circuito è ottenuto facendo l'esatto contrario e spegnendo quindi il generatore impulsivo.

\begin{figure}[H]\centering
\begin{subfigure}{0.45\linewidth}\centering
\includegraphics[width=\linewidth]{lezione_03_circuito_A}
\subcaption{Circuito iniziale}
\end{subfigure}
\begin{subfigure}{0.45\linewidth}\centering
\includegraphics[width=\linewidth]{lezione_03_circuito_B}
\subcaption{Circuito resistivo associato}
\end{subfigure}
\begin{subfigure}{0.45\linewidth}\centering
\includegraphics[width=\linewidth]{lezione_03_circuito_C-}
\subcaption{Circuito C'}
\end{subfigure}
\begin{subfigure}{0.4\linewidth}\centering
\includegraphics[width=\linewidth]{lezione_03_circuito_C--}
\subcaption{Circuito C"}
\end{subfigure}
\end{figure}
È sempre necessario determinare la dinamica dello stato, in modo da ridurre tutte le equazioni circuitali
ad un'equazione differenziale e non algebrico-differenziale, in seguito si ricavano le restanti 
variabili mediante operazioni algebriche.

Si applica il PSE (Principio di Sovrapposizione degli effetti)
$$
\begin{cases}
v_L = v_L' + v_L''\\
i_C = i_C' + i_C''
\end{cases}
$$

Dal circuito C' si ricavano $v_L'$ e $i_C'$ in funzione dei generatori impulsivi, si determineranno le 
discontinuità delle variabili di stato $i_L$ e $v_C$ a $t=0$.

Viceversa dal circuito C'' si valutano le variabili di stato utilizzando le condizioni iniziali
ricavate in C'.

$$
v_C(0^+) = \frac{1}{C} \int_{0^-}^{0^+} i_C'(\tau)d\tau \ , \ v_C(0^-) = 0
$$
$$
i_L(0^+) = \frac{1}{L} \int_{0^-}^{0^+} i_L'(\tau)d\tau \ , \ i_L(0^-) = 0
$$
L'integrale delle variabili di stato del circuito C'' è pari a 0 dato che la funzione integranda è limitata
e l'intervallo è infinitesimo.

Infine si risolve il circuito C'' usando le variabili di stato appena calcolate in C'.

In sintesi per analizzare un circuito con generatori impulsivi bisogna determinare le condizioni iniziali
utilizzando un circuito ausiliario in cui i generatori non impulsivi sono spenti e ogni induttore e condensatore
è sostituito da un circuito aperto o un corto circuito.

\subsection{Esempio risposta impulsiva circuito del secondo ordine}
Si vuole risolvere il seguente circuito del secondo ordine
\begin{figure}[H]\centering
\begin{circuitikz}
\draw
(0,0) to [R,l=$R_1$] (0,2) to (2,2)
      to [C,l_=$C$,i>_=$i_C$,v^=$v_C$] (2,0) to (0,0);
\draw
(2,2) to [R,l_=$R_2$,i^=$i(t)$] (4,2)
      to [L,l_=$L$,i=$i_L$] (4,0) to (2,0)
;
\draw
(4,2) to (5,2) to [current source,invert,l=$\delta(t)$] (5,0) to (4,0);
\end{circuitikz}
\caption*{Circuito C'}
\end{figure}

Si vuole determinare la risposta impulsiva per la variabile $i(t)$ che non è una variabile di stato.
Si procede con il metodo risolutivo sostituendo i bipoli dinamici con i rispettivi generatori chiusi 
o aperti
\begin{figure}[H]\centering
\begin{circuitikz}
\draw
(0,0) to [R,l=$R_1$] (0,2) to (2,2)
      to [short,i>_=$i_C'$] (2,0) to (0,0);
\draw
(2,2) to [R,l_=$R_2$,i^=$i'(t)$] (4,2)
      to [open,v=$v_L'$] (4,0) to (2,0)
;
\draw
(4,2) to (5,2) to [current source,invert,l=$\delta(t)$] (5,0) to (4,0);
\end{circuitikz}
\end{figure}

Il resistore $R_1$ è in parallelo ad un corto circuito, di conseguenza si vede che la corrente
$i_C'(t)$ è pari a $\delta(t)$ mentre $v_L' = R_2\delta(t)$

Si ricavano le variabili di stato
\begin{align*}
v_C(0^+) &= \frac{1}{C}\int_{0^-}^{0^+} i_C' d\tau = \frac{1}{C} \int_{0^-}^{0^+}\delta(\tau)d\tau = \frac{1}{C}\\
i_L(0^+) &= \frac{1}{L}\int_{0^-}^{0^+} v_L' d\tau = \frac{1}{L} \int_{0^-}^{0^+}R_2\delta(\tau)d\tau = \frac{R_2}{L}
\end{align*}

Si risolve il circuito C'' per determinare le equazioni di stato introducendo i generatori sostitutivi

\begin{figure}[H]\centering
\begin{circuitikz}
\draw
(0,0) to [R,l=$R_1$] (0,2) to (2,2)
      to [voltage source,i>_=$i_C''$,v^=$v_C$] (2,0) to (0,0);
\draw
(2,2) to [R,l_=$R_2$,i^=$i''(t)$] (4,2)
      to [current source,l_=$i_L$,v^>=$v_L''$] (4,0) to (2,0)
;
\end{circuitikz}
\caption*{Circuito C''}
\end{figure}

Applicando il PSE spegnendo prima il generatore di corrente e poi quello di tensione
\begin{align*}
i_C'' &= -\frac{v_C}{R_1} - i_L= C\frac{dv_C''}{dt}\\
v_L'' &= v_C - R_2i_L = L\frac{di_L''}{dt}
\end{align*}
Quelle appena ottenute sono le equazioni di stato sostituendo le equazioni caratteristiche dei bipoli dinamici.
Si ottiene infine il problema di Cauchy soluzione dello stato del circuito
$$\begin{cases}
C\frac{dv_C''}{dt} = -\frac{v_C}{R_1} - i_L \\
L\frac{di_L''}{dt} = v_C - R_2i_L \\
v_C(0^+) = \frac{1}{C} \\
i_L(0^+) = \frac{R_2}{L}
\end{cases}$$
La richiesta dell'esercizio è trovare la $i(t)$, si applica la LKC al nodo tra 
il resistore $R_2$ e l'induttore
$$
i(t) = i_L - \delta(t)
$$
è quindi sufficiente sottrarre l'impulso $\delta(t)$ alla corrente $i_L$ ricavata risolvendo le
equazioni di stato.

Per ricavare $i_L$ si isola $v_C$ dalla seconda equazione di stato (che contiene anche la derivata di $i_L$)
e la si sostituisce nella prima.
$$
v_C = R_2i_L + L\frac{di_L}{dt}
$$
$$
C \frac{d}{dt}\left[R_2i_L + L\frac{di_L}{dt} \right] = -\frac{R_2}{R_1}i_L - \frac{L}{R_1}\frac{di_L}{dt} - i_L
$$
raccogliendo
$$
LC\frac{d^2i_L}{dt^2} + \left(R_2C +\frac{L}{R_1}\right)\frac{di_L}{dt} + \left(1+\frac{R_2}{R_1}\right)i_L = 0
$$
Eseguendo il controllo dimensionale tutti i termini hanno le dimensioni di una corrente (Ampere).
Si suppone che il circuito sia dissipativo e dunque i resistori con resistenza maggiore di zero,
le radici del polinomio caratteristico hanno dunque parte reale negativa (il transitorio si estingue).
\newpage

Aggiungendo le condizioni iniziali si ottiene il problema di Cauchy
$$\left\{
\begin{aligned}
&LC\frac{d^2i_L}{dt^2} + \left(R_2C +\frac{L}{R_1}\right)\frac{di_L}{dt} + \left(1+\frac{R_2}{R_1}\right)i_L = 0
\\
&i_L(0^+) = \frac{R_2}{L}\\
&\frac{di_L}{dt}(0^+) = \frac{1}{L} \left[v_C(0^+) - R_2i_L(0^+)\right] = \frac{1}{LC} - \frac{R_2^2}{L^2}
\end{aligned}\right.
$$

Si valutano le frequenze naturali del circuito risolvendo il polinomio caratteristico, dividendo tutto per $LC$
$$
\lambda^2 + \left(\frac{R_2}{L} + \frac{1}{R_1C}\right)\lambda + \frac{\left(1+\frac{R_2}{R_1}\right)}{LC}=0
$$

Si suppone che $\lambda_1$ e $\lambda_2$ siano reali e distinte, la soluzione del problema sarà del tipo
$$\begin{aligned}
i_L(t) &= K_1e^{\lambda_1t} + K_2e^{\lambda_2t}\\
i_L(0^+) &=\frac{R_2}{L} = K_1 + K_2\\
\frac{di_L}{dt}(0^+) &= \frac{1}{LC} - \frac{R_2^2}{L^2} = \lambda_1K_1 + \lambda_2K_2
\end{aligned}
$$
La corrente incognita $i(t)$ sarà per quanto visto precedentemente mediante la LKC
$$
i(t) = i_L-\delta(t) = \left(K_1e^{\lambda_1t} + K_2e^{\lambda_2t}\right)u(t) -\delta(t)
$$

Si osserva che le variabili di stato possono essere discontinue ma comunque limitate, le altre
variabili possono invece essere anche impulsive.

\newpage
\paragraph{Risposta forzata di un circuito LTI}
Si suppone di avere un solo generatore esterno, se ne vogliono determinare le variabili ai capi
di un solo bipolo, si parla di filtro o sistema SISO \textit{(Single Input Single Output)},
sia $x(t)$ l'ingresso e $y(t)$ l'uscita, si deve supporre che il circuito sia a stato 0, ossia
le sue variabili di stato sono tutte nulle (circuito precedentemente spento).

Si parla di approssimazione \textit{PieceWise-Constant} ossia costante a tratti dell'ingresso
$x(t)$.
Si partiziona l'asse dei tempi in tanti intervalli di ampiezza $\Delta t$ centrati negli istanti
di tempo $t_k = k\Delta t\ ,\ k\ \in\ Z$.

Si costruisce un'approssimazione $x_\Delta(t)$ costante di valore $x(t_k)$ in $\left[t_k-\frac{\Delta t}{2}\ ,\ 
t_k+\frac{\Delta t}{2}\right]$.
La funzione $x(t)$ può quindi essere rappresentata come somma di impulsi rettangolari successivi,
sfruttando la funzione $\Pi_{\Delta}(t)$, sarà quindi:
$$
x_{\Delta}(t) = \sum_{k = -\infty}^{+\infty} x(t_k) \Pi_\Delta(t-t_k)\Delta t
$$
Eseguendo il limite per $\Delta t \rightarrow 0^+$ in ipotesi di sufficiente regolarità:
$$
\lim_{\Delta t \to 0^+} x_\Delta(t) = x(t) = \int_{-\infty}^{+\infty} x(\tau) \delta (t-\tau)
d\tau
$$
Questa conclusione richiama la proprietà di campionamento della $\delta(t)$.

Se si vuole calcolare la risposta di $x(t)$ allora:
$$
\text{Risp}\left\{x(t)\right\} = y(t) = \text{Risp} \left\{\int_{-\infty}^{+\infty} x(\tau)\delta(t-\tau)
d\tau\right\}
$$
per linearità e tempo-invarianza si ottiene
$$
y(t) = \int_{-\infty}^{+\infty} x(\tau)h(t-\tau) d\tau
$$
chiamato integrale di convoluzione, $h(t)$ è la risposta impulsiva per la variabile di uscita.

Si parla di prodotto di convoluzione tra due funzioni $f,g \ \in\ \mathbb{R}$
$$
f * g(t) = \int_{-\infty}^{+\infty} f(\tau)g(t-\tau)d\tau
$$
Noto l'ingresso $x(t)$ si può calcolare la risposta forzata se si conosce la risposta impulsiva
dello stesso circuito, relativa alla medesima grandezza di uscita.


Data la risposta all'ingresso impulsivo, si può calcolare la risposta a qualsiasi ingresso, mediante
l'uso dell'integrale di convoluzione.
Osservando attentamente l'integrale si vede che la risposta impulsiva gode della proprietà per la quale
$$
h(t) = 0 \ t<0,\ h(t) \neq 0 \forall t\ \geq 0 \Rightarrow t - \tau \geq 0 \Rightarrow \tau \leq t
$$

Se l'ingresso $x(t) = 0,\ t< t_0$ allora possiamo affermare che la $y(t)$ avrà come estremi di integrazione
$t_0^-$ e $t$, con il - si sottintende la possibilità che siano presenti $\delta(t_0)$ in quell'istante 
di tempo.

\paragraph{Esempi dell'utilizzo dell'integrale di convoluzione}
Circuito RC parallelo con forzamento esponenziale, forzato da un generatore di corrente.
$$
j(t) = I e ^{\frac{t}{\tau}} u(t)
$$
Per studiare la risposta del circuito bisogna per prima cosa determinare la $V_{cf}(t)$, mediante
lo studio della risposta impulsiva.
Per determinare la risposta impulsiva si forza il circuito con una $\delta(t)$, utilizzando i metodi 
precedenti.

$i_c = \delta(t)$
$$
v_c(0^+) = \frac{1}{C}\int_{0^-}^{0^+} i_c(\tau)d\tau + V_c(0^-) = \frac{1}{C}
$$
Per determinare la risposta impulsiva si determina l'evoluzione libera spegnendo il generatore,
ci sarà un parallelo RC con la seguente equazione di stato:
$$
\begin{cases}
RC\frac{dV_c}{dt} + v_C = 0 \\
V_c(0^+) = \frac{1}{C}
\end{cases}
$$
Quindi 
$$
V_{c_{lib}} = h(t) = \frac{1}{C} e^{-\frac{t}{RC}},\ t\geq 0
$$
$$
h(t) = \frac{1}{C}e^{-\frac{t}{RC}}\cdot u(t) \forall t
$$

Utilizzando l'integrale di convoluzione per calcolare la risposta forzata:

$$
V_{cf}(t) = \int_{0^-}^{t}Ie^{\frac{\tau}{T}} \cdot \frac{1}{C} e^{-\frac{t-\tau}{RC}} d\tau = 
\frac{I}{C} \int_{0^-}^{t} e^{\frac{\tau}{t}}\cdot e^{-\frac{t}{RC}}\cdot e^{\frac{\tau}{RC}} d\tau
$$

$$
= \frac{I}{C} e^{-\frac{t}{RC}}\cdot \frac{e^{t\left(\frac{1}{T}+\frac{1}{RC}\right)}-1}{\frac{1}{RC}+\frac{1}{T}}
$$

Circuito RC serie con forzamento in tensione a rampa lineare $e(t)$
$$
e(t) = \begin{cases}
0\ t<0 \\
\frac{t}{T},\ 0\leq t\leq T\\
0\ t>0
\end{cases}
$$
Possiamo rappresentare questa funzione mediante l'uso di due funzioni $u(t)$

$$
e(t) = \frac{t}{T}\left[u(t) - u(t-T)\right]
$$
Determiniamo quindi la risposta $h(t)$ mediante la risposta al gradino $g(t)$
Le equazione di stato è:
$$
\begin{cases}
1 = RC\frac{dV_c}{dt} + V_c \\
V_c(0) = 0 
\end{cases}
\Rightarrow V_c(t) = 1 - e^{-\frac{t}{RC}},\ t\geq 0$$
Quindi 
$$
g(t) = (1 - e^{-\frac{t}{RC}})u(t) \forall t
$$
ma
$$
h(t) = \frac{dg}{dt} = \frac{1}{RC}e^{-\frac{t}{RC}},\ t\geq 0
$$
Determiniamo ora la risposta forzata $v_{cf}(t)$:
$$
V_{cf}(t) = \int_{0^-}^{t} e(\tau) h(t-\tau)d\tau = \int_{0^-}^{t}\frac{\tau}{T}\left[u(\tau) - u(t-\tau)\right]
\cdot \frac{1}{RC} e ^{-\frac{t-\tau}{RC}}\cdot u(t-\tau)d\tau
$$
Svolgiamo ora l'integrale osservando che $e(\tau) \neq 0 $ solo se $t \in [0,T]$, separiamo quindi
il calcolo dell'integrale in due eventualità:
$$
t\leq T:\ \int_{0^-}^{t}\frac{\tau}{T}\frac{1}{RC}e^{-\frac{t-\tau}{RC}}d\tau = \frac{1}{TRC}e^{-\frac{t}{RC}}\int_{0^-}^{t}\tau e^{\frac{\tau}{RC}}d\tau 
$$
utilizzando l'integrazione per parti:
$$
\left(f\cdot g\right)' = f'g + g'f
$$
$$
f = \tau\ g' = e^{\frac{\tau}{RC}}
$$
$$
\frac{1}{TRC}e^{-\frac{t}{}RC}\int_{0^-}^{t}\frac{d}{dt}\left[\tau e^{\frac{\tau}{RC}}\cdot RC \right] -
RC e^{\frac{\tau}{RC}} d\tau =
$$
$$
\frac{1}{TRC} e^{-\frac{t}{RC}} \left\{ \left[t e^{\frac{t}{RC}}\cdot RC \right] - \left[(RC)^2 e^{\frac{\tau}{RC}} \right]_{0^-}^{t}  \right\} =
$$
$$
\frac{t}{T} - \frac{RC}{T}\left(1-e^{-\frac{t}{RC}}\right)
$$

Secondo caso:
$$
t > T:\ \int_{0^-}^{T} \frac{\tau}{T}\frac{1}{RC} e^{-\frac{t-\tau}{RC}}d\tau =  e^{-\frac{t-T}{RC}}-\frac{RC}{T}\left(e^{-\frac{t-T}{RC}} -e^{-\frac{t}{RC}} \right)
$$

\section{Metodo dei fasori}
Il metodo dei fasori si basa sul concetto che conoscendo l'andamento di regime con grandezze isofrequenziali, si può risolvere il sistema traspotando le grandezze sinusoidali nel dominio simbolico
dei fasori, si rapprentano cioè le grandezze descrittive dei bipoli medianti numeri complessi.

Mediante la L-Trasformata si possono usare ancora una volta equazioni nel dominio complesso ma per 
studiare reti nel regime transitorio.

\paragraph{Analisi delle reti dinamiche LTI con Laplace}
Consideriamo un blocco contenenti tutte le equazioni circuitali nel dominio del tempo,
otteniamo tutte le equazioni di stato di induttori e condensatori.
Questo risultato si ottiene mediante i metodi di risoluzione delle ODE, per arrivare alla conoscenza
della dinamica delle \textit{variabili di stato}, se sono poi interessato ad altre variabili, attravarso
operazioni \textit{algebriche lineari} si conosce la dinamica di utte le variabili del circuito.

In alternativa, sfruttando la trasformata di Laplace, le equazioni ODE vengono trasformate nel dominio
di Laplace, saranno equazioni algebriche lineari anzichè differenziali, risolvendo quindi solo un sistema
di equazioni lineari si conosce direttamente la trasformata delle variabili di stato, effettuando quindi
il procedimento inverso di antitrasformazione si ricavano le variabili di stato nel dominio del tempo,
l'antitrasformazione può essere invece posticipata ed eseguita dopo aver ricavato le generiche variabili
del circuito, ancora una volta con operazioni algebriche.

La soluzione di un sistema di equazioni differenziali è notevolmente più complesso che risolvere un 
sistema lineare, ecco il vantaggio dell'utilizzo della trasformata di Laplace.

Definizione della trasformata di Laplace di una funzione $f(t)\ \in\ [0,+\infty[\to R$:
$$
L[f(t)] = F(s) = \int_{0^-}^{+\infty} f(t) e^{-st}dt \ F:s \in\ C \to C
$$
ammesso che l'integrale improprio converga, per assicurarci che ciò accade si pone la condizione sufficiente, verificata nella stragrande delle situazioni:
$$
\text{Se} \left|f(t) \right| \leq Me^{\alpha t} ,\ M,\alpha \in R \text{ costanti}
$$
allora l'integrale converge, dim.:
$$
\int_{0^-}^{+\infty} f(t) e^{-st}dt \leq \int_{0^-}^{+\infty} \left|f(t)\right| e^{-st}dt \leq
\int_{0^-}^{+\infty} Me^{\alpha t} e^{-st}dt = M\int_{0^-}^{+\infty} e^{(\alpha-s)t}dt
$$
condizione verificata per ogni $\Re \left\{ s\right\} > \alpha$.

Definizione dell'antitrasformata:
$$
L^{-1}\left[F(S) \right] = f(t) = \frac{1}{2\pi i} \lim_{T\to+\infty} \int_{\gamma-iT}^{\gamma+iT}
e^{st}F(s)ds
$$
con $\Re\left\{s\right\} = \gamma$ tale che tutte le singolarità di $F(s)$ si trovino a sinistra di $\gamma$.

\paragraph{Proprietà della L-trasformata} Ricordando quelle utilizzate nel metodo dei fasori:
\begin{itemize}
\item Unicità: $\forall f(t) \in [0,+\infty[\ \exists!\ F(s) = L[F(s)]$

considerate $F(s)$ e $G(s)$, se $F(s) = G(s) \Rightarrow f(t) = g(t)$ quasi ovunque, ossia:
$$\int_{0^-}^{+\infty}|f(t) - g(t)|dt = 0$$

\item Linearità: date $f_1(t)$ ed $f_2(t) :\ [0,+\infty[ \rightarrow R,\ k_1,k_2 \in R$ allora
$$L[k_1f_1(t) + k_2f_2(t)] = k_1F_1(s) + k_2F_2(s) $$
si dimostra con la proprietà di linearità dell'integrale

\item Traslazione nel dominio di Laplace: sia data $f(t) \in [0^-,+\infty[,\ L[f(t)] = F(s)$ e
consideriamo $F(s-\lambda),\ \lambda\ \in\ C$ 
$$F(s-\lambda) = \int_{0^-}^{+\infty}f(t) e^{-(s-\lambda)t}dt = \int_{0^-}^{+\infty} (f(t)e^{\lambda t})e^{-st} dt \Rightarrow F(s-\lambda) = L[f(t) e^{\lambda t}]$$ con $\Re\{s\} > \lambda$ 

\item Derivazione: $f'(t)  = \frac{d}{dt}f(t),\ L[f(t)] = F(s)$ allora 
$$L\left[\frac{d}{dt}f(t)\right] = \int_{0^-}^{+\infty}f'e^{-st}dt \stackrel{\text{x parti}}{=} 
\left[fe^{-st}\right]_0^{+\infty} - \int_{0^-}^{+\infty}(-s)e^{-st}\cdot f dt =$$
$$= \left(\lim_{t\to\infty}\left[fe^{-st}\right]-f(0^-)\right) + sF(s) = sF(s) - f(0^-)$$
$$
L\left[\frac{d}{dt}f(t)\right] = sF(s) - f(0^-)
$$
\item Prodotto di convoluzione nel dominio di Laplace, facendo leva sul teorema di Borel:
$$
L\left[f*g(t)\right] =\int_{0^-}^{+\infty}\left(\int_{0^-}^{t}f(\tau)g(t-\tau)d\tau\right) e^{-st}dt = F(s)\cdot G(s)
$$
per dimostrare questo teorema si scambiano le variabili di integrazione $t$ e $\tau$ sfruttando i teoremi
di \href{https://it.wikipedia.org/wiki/Teorema_di_Fubini}{Fubini} e Tonelli
\end{itemize}
\newpage
Trasformate notevoli di frequente utilizzo nei circuiti:
\begin{itemize}
\item Esponenziale: $f(t) = e^{\lambda t},\ \lambda\ \in R $ 
$$
L\left[e^{\lambda t}\right] = \int_{0^-}^{+\infty} e^{-(s-\lambda) t} dt = \left[\frac{1}{\lambda -s}e^{(\lambda -s)t}\right]_{0^-}^{+\infty} = \frac{1}{\lambda -s} \left[\lim_{t\to\infty}e^{(\lambda-s)t}-1\right] =
\frac{1}{s-\lambda}
$$
\item Funzione gradino $u(t)$:
$$
L[u(t)] = \int_{0^-}^{+\infty}e^{0t}e^{-st}dt = \frac{1}{s}
$$
\item Delta di Dirac $\delta(t)$:
$$
L[\delta(t)] = \int_{0^-}^{+\infty}\delta(t) e^{-st}dt = e^{-s\cdot 0} = 1
$$
\item Funzioni sinusoidali, $\cos(\omega t)=\frac{e^{j\omega t}+e^{-j\omega t}}{2},\ \sin(\omega t) = \frac{e^{j\omega t}-e^{-j\omega t}}{2j}$
$$
L[\cos(\omega t)] = \frac{1}{2} L[e^{j\omega t}] + \frac{1}{2} L[e^{-j\omega t}] = \frac{s}{s^2+\omega^2},\ \Re\{s\} > 0
$$
$$
L[\sin(\omega t)] = \frac{1}{2j}\left(\frac{1}{s-j\omega}-\frac{1}{s+j\omega}\right) = \frac{\omega}{s^2+\omega^2},\ \Re\{ s\} > 0
$$
\end{itemize}


Funzioni generiche $t\cdot f(t)$ con $f(t)$ le funzioni precedentemente analizzate.
Esponenziale:
$$
L[te^{\lambda t}] = \int_{0^-}^{+\infty} t e^{\lambda t}e^{-s t} dt = \int_{0^-}^{+\infty}t e^{(\lambda -s)t}
dt = 
$$
$$
= \left[\frac{e^{(\lambda -s )t }}{\lambda-s}\cdot t\right]_{0^-}^{+\infty} - \int_{0^-}^{+\infty}1\cdot\frac{1}{\lambda -s}e^{(\lambda-s)t}dt = 
$$
$$
= 0 + \frac{1}{s-\lambda}\cdot\frac{1}{s-\lambda} = \frac{1}{s-\lambda} ,\ \Re\{s\} > \lambda
$$

Coseno:
$$
L[t\cos(\omega t)] = L\left[t \frac{e^{j\omega t}+e^{-j\omega t}}{2}\right] = \frac{1}{2}\frac{1}{(s-j\omega)^2} + \frac{1}{2}\frac{1}{(s+j\omega)^2} = 
$$
$$
= \frac{1}{2}\frac{1}{s^2-\omega^2-2j\omega s} + \frac{1}{2}\frac{1}{s^2-\omega^2+2j\omega s} =
\frac{1}{2}\frac{s^2-\omega^2+2j\omega s +s^2 -\omega^2 -2j\omega s}{(s^2-\omega^2)^2+4\omega^4s^2} =
$$
$$
= \frac{s^2-\omega^2}{(s^2+\omega^2)^2}
$$

Seno:
$$%%Rivedi il rigo qui sotto
L[t\sin(\omega t)] = \frac{1}{2j} \left[\frac{1}{(s-j\omega)^2}-\frac{1}{s+j\omega}^2\right] =
\frac{1}{2j}\left[\frac{s^2-\omega^2+2j\omega s-s^2+\omega^2+2j\omega s}{(s^2-\omega^2)^2+4\omega^2s^2}\right] = 
$$
$$
= \frac{2\omega s}{(s^2+\omega^2)^2}
$$
Funzioni che descrivono moti periodici smorzati, sfruttando la proprietà di traslazione
$f(t) = e^{-\sigma t}\cos(\omega t),\  e^{-\sigma t}\sin(\omega t)$:
$$
L[e^{-\sigma t}\cos(\omega t)] = L[\cos(\omega t)](s+\sigma) = \frac{s+\sigma}{(s+\sigma)^2+\omega^2}\ \Re\{s\} > -\sigma
$$
$$
L[e^{-\sigma t}\sin(\omega t)] = L[\sin(\omega t)](s+\sigma) = \frac{\omega}{(s+\sigma)^2+\omega^2}\ \Re\{s\} > -\sigma
$$
\paragraph{Circuito RL serie con L-trasformata}
Sia dato il circuito RL serie, se ne voglia determinare la risposta impulsiva $h(t) = i_L(t)$ dato l'ingresso
$e(t) = \delta(t)$
$$
\begin{cases}
\delta(t) = R\cdot i_L + L\frac{di_L}{dt} \\
i_L(0^-) = 0
\end{cases}
$$
Si applica la trasformata di Laplace ad entrambi i membri della prima equazione, sfruttando anche
la proprietà di derivazione:
$$
1 = RI_L(s) + L(sI_L(s) - i_L(0^-))
$$
$$
L[i_L(t)] = I_L(s) = \frac{1}{R+sL} = \frac{1}{L}\frac{1}{\frac{R}{L}+s}
$$
L'antitrasformata sarà invece pari a:
$$
L^{-1}[I_L(s)] = \frac{1}{L} L^{-1}\left[\frac{1}{S+\frac{R}{L}}\right] = \frac{1}{L}e^{-\frac{t}{\tau}},\ 
\tau = \frac{L}{R}
$$

Si vede ora la risposta al gradino, $e(t) = u(t)$:
$$
u(t) = R\cdot i_L+ L\frac{di_L}{dt} \rightarrow \frac{1}{s} = RI_L + sLI_L 
$$
$$
I_L = \frac{1}{s(R+sL)} = \frac{A}{s} + \frac{B}{R+sL} = \frac{AR+sLA+sB}{s(R+sL)}
$$
Si ricavano i valori di $A$ e $B$ sfruttando il principio di identità dei polinomi:
$$
LA+B = 0,\ AR = 1
$$

$$
A = \frac{1}{R},\ B = -\frac{L}{R}
$$
In conclusione:
$$
I_L(s) = \frac{1}{R}\left[\frac{1}{s}-\frac{L}{R+SL}\right] = \frac{1}{R}\left[\frac{1}{s}-
\frac{L}{L\left(\frac{R}{L}+s\right)}\right]
$$
Antitrasformando si ottiene:
$$
i_L(t) = \frac{u(t)}{R} - \frac{1}{R}e^{-\frac{R}{L} \text{eccetera}}
$$

\paragraph{Equazioni circuitali nel dominio della L-trasformata}
Si considera ancora la classe dei circuiti dinamici lineari tempo-invarianti (LTI),
nel dominio del tempo sono necessarie le LKT e le LKC, le caratteristiche dei bipoli e dei generatori.
Questo sistema è un DAE, va trasformato in un ODE nelle variabili di stato per poter essere risolto.

Se trasformo tutte le equazioni che provengono dalle leggi di Kirchoof ottengo ancora le stesse equazioni
ma nel dominio di Laplace:
$$
\sum_{k} (\pm)I_k(s) = 0
$$
$$
\sum_k (\pm)V_k(s) = 0
$$
Ancora le equazioni dei bipoli:
$$
V_R(s) = RI_R(s)
$$
$$
V_L(s) = sLI_L(s) - Li_L(0^-)
$$
$$
I_C(s) = sCV_C(s) - Cv_c(0^-)
$$

Per trattare i circuiti in evoluzione forzata minuto(46) si può associare un'impedenza equivalente
ai bipoli, ad esempio l'impedenza operatoriale dell'induttore diventa
$$
Z_L(s) = sL,\ V_L(s) = Z_LI_L(s)
$$
$$
Z_C = \frac{1}{sC}
$$
Questo risultato ci permette di utilizzare tutti i metodi precedentemente visti per la risoluzione
di un circuito, si esegue un'antitrasformazione alla fine per ritornare nel dominio del tempo.
Abbiamo posto però l'ipotesi di trovarsi in evoluzione forzata, ossia supponendo uno stato iniziale 
nullo, cosa accade se ciò non è vero?

Stato non zero iniziale:
$$
V_L(s) = sLI_L - L i_L(0^-)
$$
$$
V_L(s) = Z_LI_L+E_0
$$
Si modella il bipolo dinamico con un generatore di tensione in serie, impulsivo.
Si può fare un ragionamento simile con il condensatore al quale si affianca un generatore di corrente
in parallelo $J_{cc}$
$$
I_C(s) = sCV_C(s) - Cv_C(0^-)
$$
$$
J_{cc} = Cv_C(0^-)
$$

\paragraph{Funzione di trasferimento e legame con la risposta impulsiva}
Sia preso un generico circuito LTI, viene forzato con un generatore di tensione, se ne analizzano
le grandezze su un singolo bipolo interno al circuito, considerato quindi come un SISO $h(t)$,
l'uscita del sistema può essere calcolata mediante l'integrale di convoluzione della risposta impulsiva.
$$
y(t) = \int_{0^-}^{t} x(\tau)h(t-\tau)d\tau
$$
Si trasporta il fenomeno nel dominio di Laplace e si considera lo stesso circuito sostituito dai bipoli
operatoriali, si può ancora considerare il sistema come un SISO con ingresso pari a $X(s)$ e uscita
pari a $Y(s)$.
Questo circuito è a-dinamico lineare con un solo generatore, quindi tutte le grandezze sono proporzionali
al singolo forzamento, posso quindi affermare che $Y(s) = H(s)X(s)$ con $H(s)$ un coefficiente di 
proporzionalità.

$H(s)$ viene ricavato, mediante il teorema di Borel:
$$
L[f*g(t)] = F(s)\cdot G(s)
$$
$$
y(t) = \int_{0^-}^{t} x(\tau)h(t-\tau) d\tau \Rightarrow H(s) = L[h(t)]
$$
$H(s)$ si chiama quindi \textit{funzione di trasferimento} del circuito ed è definita come il rapporto
tra la trasformata dell'uscita e quella dell'ingresso e coincide con la trasformata della risposta
impulsiva del circuito:
$$
H(s) \stackrel{\text{def}}{=} \frac{Y(s)}{X(s)} = L[h(t)]
$$

Esempio di applicazione, circuito LC forzato in risonanza:

Si ha un generatore ideale di tensione, un induttore ideale e un condensatore ideale senza perdite,
questo circuito rappresenta il limite ideale 
La risonanza è associata ad una specifica pulsazione pari a:
$$
\omega_r = \frac{1}{\sqrt{LC}}
$$
e il forzamento è pari ad un segnale con pulsazione pari alla pulsazione di risonanza
$$
e(t) = E_m \sin(\omega_r t)
$$
Il circuito non può essere analizzato con il metodo dei fasori dato che non è dissipativo e l'energia 
immagazzinata nei bipoli dinamici non tende a 0 in evoluzione libera per $t\to \infty$.

Si utilizza quindi l'analisi nel dominio di Laplace:
$$
I(s) = \frac{E(s)}{sL+ \frac{1}{SC}}
$$
ma
$$
E(s) = \frac{\omega_r}{s^2+\omega_r^2} = \frac{\omega_r}{s^2+\frac{1}{LC}} = L[e(t)]
$$
$$
I(s) = \frac{E_m\omega_r s C}{(s^2+\frac{1}{LC})(s^2+\frac{1}{LC})} = \frac{E_m\omega_r\frac{sC}{LC}}{(s^2+\frac{1}{LC})^2} = \frac{E_m\omega_r}{L} \frac{s}{(s^2+\frac{1}{LC})^2}
$$
Riprendendo la trasformata del seno:
$$
L[t\sin(\omega_r t)] = \frac{2\omega s}{(s^2+\omega^2)^2}
$$
riprendendo il calcolo di $I(s)$:
$$
= \frac{E_m\omega_r}{2L\omega_r}\cdot \frac{2\omega_r s}{(s^2+\omega_r^2)^2} = \frac{E_m}{2L}\frac{2\omega_r s}{(s^2+\frac{1}{LC})^2}
$$
Antitrasformando:
$$
i(t) = L^{-1}[I(s)] = \frac{E_m}{2L} t \sin(\omega_r t)
$$

\newpage
\paragraph{Calcolo della funzione di trasferimento per un circuito del secondo ordine}
Circuito minuto 1:40:00
Siamo interessati come uscita alla $v(t)$ dato l'ingresso $e(t)$, ci trasferiamo ancora una volta dal 
dominio del tempo a quello di Laplace.
Si assegnano i parametri del circuito:
$R_1 = \SI{5}{\ohm}\ R_2=R_3=\SI{10}{\ohm}\ C =\SI{0.5}{\farad}\ L=\SI{1}{\henry} $
Sostituiamo i bipoli con impedenze, troviamo la $Z_{eq}$ pari a $5 + s$ in serie con $\frac{1}{0.2+0.5s}$
La tensione in uscita sarà la partizione della tensione in ingresso tra queste due impedenze:
$$
V(s) = E(s)\frac{Z_{eq}^{(2)}}{Z_{eq}^{(2)}+Z_{eq}^{(1)}} = \frac{1}{(0.2+0.5 s)(\frac{1}{0.2+0.5 s}+5+s)} = \frac{1}{2+2.75+0.5s^2}
$$
Si trovano gli zeri del polinomio, saranno due radici reali e distinte,
la $H(s)$ sarà quindi:
$$
H(s) = \frac{k_1}{s+0.886} + \frac{k_2}{s+4.514} = \frac{k_1 5 +4.514 k_1 + k_2 s + 0.886 k_2 }{s^2+5.45+4} =
$$
$$
= \frac{2}{(s^2+5s+4)},\ 
\begin{cases}k_1+k_2 = 0\\
4.514k_1 + 0.886k_2 = 2
\end{cases}
$$
$$
k_1 = \frac{2}{4.514-0.886} = 0.551 = -k_2
$$
In conclusione 
$$
H(s) = \frac{0.551}{s+0.886} - \frac{0.551}{s+4.514}
$$
antitrasformando:
$$
L^{-1}[H(s)] = 0.551\left(e^{-0.886 t}-e^{-4.514 t}\right) = h(t)
$$


\paragraph{Calcolo di antitrasformate di funzioni razionali}
Supponiamo che nei circuiti siano presenti soltanto alcune tipologie di generatori, come ad esempio
generatori stazionari, sinusoidali... In queste ipotesi le trasformate di Laplace sono funzioni 
razionali, del tipo:
$$
F(s) = \frac{N(s)}{D(s)}\ N(s),D(s) \text{ polinomi con grado } n\leq d
$$

In generale la $F(s)$ è esprimibile come:
$$
F(s) = K + \frac{N^*(s)}{D(s)}\ K \text{ costante, } n^* < d
$$
la sua antitrasformata sarà:
$$
L^{-1}[F(s)] = k\delta(t) + L^{-1}\left[\frac{N^*(s)}{D(s)}\right] 
$$
per determinare la soluzione vanno ricercate le radici del denominatore, chiamate \textit{poli}
della funzione, mentre le radici del numeratore vengono chiamati \textit{zeri}.

La funzione può avere poli semplici di molteplicità 1:
$$
F^*(s) = \frac{k_1}{s-p_1} + \frac{k_2}{s-p_2} + ... + \frac{k_n}{s-p_n} = \sum_{i=1}^{N}\frac{k_i}{s-p_i}
$$
$k_i$ viene chiamato residuo i-esimo della $F^*(s)$ e si calcola con:
\begin{equation}
\lim_{s\to p_i} \left[(s-p_1)F^*(s)\right]
\label{eq:formula_poli_semplici}
\end{equation}

Esempio:
$$
F^*(s) = \frac{1}{0.5s^2+2.7s+2} = \frac{2}{s^2+5.4s+4} = \frac{k_1}{s+0.886} + \frac{k_2}{s+4.514}
$$
Applicando la \ref{eq:formula_poli_semplici}:
$$
k_1 = \frac{2}{-0.886+4.514} = 0.551
$$

Poli multipli, ossia con molteplicità maggiore di 1, supponiamo ad esempio $p_1$ e $p_2$ poli multipli:
$$
F^*(s) = \frac{k_{11}}{(s-p_1)^2} + \frac{k_{12}}{s-p_1} + \sum_{i=3}^{N} \frac{k_1}{s-p_i}
$$
$$
k_{11} = \lim_{s\to p_1} \left[(s-p_1)^2F^*(s)\right]
$$
per determinare il secondo residuo invece si deve eseguire il limite della derivata:
$$
k_{12} \lim_{s\to p_1} \frac{d}{ds} \left[(s-p_1)^2 F^*(s)\right]
$$
Esempio:
$$
F^*(s) = \frac{2s+5}{(s+2)^2} = \frac{k_{11}}{(s+2)^2} + \frac{k_{12}}{s+2}
$$
$$
k_{11} = \lim_{s\to-2} \frac{(2s+5)(s+2)^2}{(s+2)^2} = 1
$$
$$
k_{12} = \lim_{s \to -2} \frac{d}{ds} [2s+5] = 2
$$
$$
F^*(s) = \frac{1}{(s+2)^2}+\frac{2}{s+2}\ L^{-1}[F^*(s)] = te^{-2t}+2e^{-2t}
$$

\textbf{Poli complessi e coniugati:}
$$
F^*(s) \frac{2s-1}{s^2+4s+5} = \frac{k_1}{s+2+j} + \frac{k_2}{s+2-j}
$$
$$
k_1 = \frac{2(-2-j)-1}{-2-j+2-j} = \frac{-4-2j-1}{-2j} = \frac{5+2j}{2j} = 1 - \frac{5}{2}j
$$
$$
k_2 = k_1^* = 1+\frac{5}{2}j
$$
$$
F^*(s) = \frac{1-\frac{5}{2}j}{s+2+j} + \frac{1+\frac{5}{2}j}{s+2-j}
$$

$$
L^{-1}[F^*(s)] = (1-\frac{5}{2}j)e^{-(2+j)t} + (1+\frac{5}{2}j)e^{-(2-j)t} = e^{-2t}\left[\frac{e^{-jt}+e^{jt}}{2}\cdot 2 - \frac{5}{2}je^{-jt}+\frac{5}{2}je^{jt}\right] =
$$
$$
= e^{-2t}\left[2\cos(t)-5\sin(t)\right]
$$

\textbf{Esercizio 1 :}

Circuito RLC parallelo, determinare la risposta all'impulso e al gradino
$$
R = \SI{10}{\kilo\ohm}\ L = \SI{100}{\milli\henry}\ C = \SI{10}{\micro\farad}
$$
$$
j(t) = \frac{v}{r} + i_L + i_C,\ v = L\frac{di_L}{dt},\ i_c = C\frac{dv_c}{dt}
$$
$$\begin{cases}
C\frac{dv_c}{dt} = i_c = u(t) -\frac{v}{r} - i_L\\
L\frac{di_L}{dt} = v
\end{cases}
$$
$$
L\frac{d}{dt}\left[u(t) -\frac{v}{r} - C \frac{dv}{dt}\right] = v
$$

$$
\begin{cases}
\frac{d^2v}{dt^2} + \frac{1}{RC}\frac{dv}{dt} + \frac{v}{LC} = 0\\
v(0^+)=0\\
\frac{dv}{dt}0^+ = \frac{1}{C}\left[j(0^+) - \frac{v}{r}(0^+) - i_L(0^+)\right] = \frac{1}{C}
\end{cases}
$$
Si determinano ora le frequenze naturali del sistema:

$$
\lambda^2 + \frac{1}{RC}\lambda + \frac{1}{LC} = 0
$$
$$
\lambda_{1,2} = -\frac{1}{2RC} \pm \sqrt{\left(\frac{1}{2RC}\right)^2-\frac{1}{LC}}
$$
$$
\lambda_{1,2} = -5 \pm j 1000
$$
Quindi 
$$
v(t) = e^{-st}(k_1\cos(1000t)+k_2\sin(1000t))
$$
$$
v(0^+) = 0 \Rightarrow k_1 = 0
$$
$$
\frac{dv}{dt}(0^+) = \frac{1}{C} \Rightarrow \left[-5e^{-5t}k_2\sin(1000t) + k_2e^{-5t}\cos(1000t)\cdot1000\right]_{t=0} = \frac{1}{C}
$$
$$
k_2\cdot1000 = \frac{1}{10^{-5}} \Rightarrow k_2 = 100
$$
$$
v(t) = 100e^{-5t}\sin(1000t)\cdot u(t) = g(t)
$$
Risposta all'impulso:
$$
h(t) = \frac{dg}{dt} = -500e^{-5t}\sin(1000t) + 100e^{-5t}-1000-\cos(1000t) = 
$$
$$
500e^{-5t}\left[200\cos(1000t)-\sin(1000t)\right]\cdot u(t)
$$
Controprova:
$$
v_c(0^+) = \frac{1}{C}\int_{0^-}^{0^+}\delta(t)d\tau = \frac{1}{C}
$$
$$
i_L(0^+) = \frac{1}{L}\int_{0^-}^{0^+} v_L(t)d\tau = 0
$$
$$
\begin{cases}
v(0^+) = \frac{1}{C}\\
\frac{dv}{dt}(0^+) = \frac{1}{C}\left(-\frac{1}{RC}\right) = -\frac{1}{RC^2}
\end{cases}
$$

$$
v(0^+) = \frac{1}{C} = k_1 = 10^5
$$
$$
\frac{dv}{dt}(0^+) = 1000k_2 - \frac{5}{C} = -\frac{1}{RC^2} \Rightarrow 1000k_2 = \frac{5}{C} - \frac{1}{RC^2} = 5\cdot10^5 - \frac{10^10}{10^4} = 5\cdot10^5-10^6 = -5\cdot10^5
$$
$$
k_2 = -500
$$
In conclusione:
$$
h(t) = e^{-st}\left[10^5\cos(1000t)-500\sin(1000t)\right] = 500e^{-st}\left[200\cos(1000t)-\sin(1000t)\right]\cdot u(t)
$$

\textbf{Esercizio 2 :} 01:10:00

Determina la risposta impulsiva nel dominio del tempo con la trasformata di Laplace
$$
R_1 = \SI{1}{\ohm},\ R_2 = \SI{3}{\ohm},\ L = \SI{3}{\henry},\ C = \SI{1}{\farad}
$$
Determiniamo le condizioni iniziali dovute al generatore impulsivo $e(t) = \delta(t)$

Disegna circuito 2

$$
v_L = -\frac{R_1}{R_1+R_2}\delta(t) = -\frac{1}{4}\delta(t)
$$
$$
i_C = -\frac{\delta(t)}{R_1+R_2} = -\frac{1}{4}\delta(t)
$$
$$
i_L(0^+) = \frac{1}{L} \int_{0^-}^{0^+} -\frac{\delta}{4}d\tau = \SI{-1/12}{\ampere} 
$$
$$
v_c(0^+) = \frac{1}{C} \int_{0^-}^{0^+}-\frac{\delta}{4}d\tau = -\frac{1}{4} = \SI{-0.25}{\volt}
$$
quindi
$$
v = -\frac{R_2}{R_1+R_2}\delta(t) = -\frac{3}{4}\delta(t) = -0.75\delta(t)
$$

Determiniamo ora le  equazioni di stato sfruttando le condizioni iniziali appena ricavate:
Circuito 1:20
$$
\begin{cases}
i_c = -\frac{v_c}{R_1+R_2} +\frac{R_1}{R_1+R_2}i_L = C\frac{dv_C}{dt}\\
v_L = -\frac{R_1}{R_1+R_2}v_C - \frac{R_1R_2}{R_1+R_2}i_L = L\frac{di_L}{dt}
\end{cases}
$$
$$
v(t) = R_2i_c = -\frac{R_2}{R_1+R_2}v_c + \frac{R_1R_2}{R_1+R_2}i_L = -\frac{3}{4}v_c +\frac{3}{4}i_L
$$

Riscriviamo ora la ODE in forma matriciale:
$$
\begin{pmatrix}
 C & 0 \\
 0 & L
\end{pmatrix}
\frac{d}{dt}
\begin{pmatrix}
v_c \\
i_L
\end{pmatrix}
=
\begin{pmatrix}
-\frac{1}{R_1+R_2} & \frac{R_1}{R_1+R_2}\\
-\frac{R_1}{R_1+R_2} & -\frac{R_1R_2}{R_1+R_2}
\end{pmatrix}
\begin{pmatrix}
v_c \\
i_L
\end{pmatrix}
$$

$$
\vec{x}(t) = \vec{v}e^{\lambda t} \Rightarrow \lambda D \vec{v} e ^{\lambda t} = A\cdot \vec{v}e^{\lambda t} \Rightarrow A\cdot \vec{v} = \lambda D\cdot\vec{v}
$$

$$
D^{-1}A\cdot\vec{v} = \lambda\vec{v} = \lambda\vec{v} \Rightarrow 1:33
$$

$$
\text{det}\left(A-\lambda I\right) = a_{11}-\lambda ecc = a_{11}a_{22} - a_{11}\lambda - \lambda a_{22} + \lambda^2 - a_{12}a_{21} = 
$$

$$
\lambda_{1,2} = -\frac{1}{4} \pm \sqrt{\frac{1}{16}+\left(\frac{1}{16}+\frac{1}{48}\right)} =
$$
$$
v_c(t) = e^{-\sigma t}\left(k_1\cos(¬omega t) + k_2\sin(\omega t)\right)
$$
$$
\frac{dv_c}{dt}(t) = -\sigma e^{-\sigma t}\left(k_1\cos(\omega t)+k_2\sin(\omega t)\right) + e^{-\sigma t}(-k_1\omega \sin(\omega t) + k_2 \omega \cos(\omega t)) = e^{-\sigma t} \left[(k_2\omega -\sigma k_1)\cos(\omega t) - (\sigma k_2 + k_1\omega ) \sin(\omega t) \right]
$$
$$
\frac{dv_c}{dt}(0^+)= -\sigma k_1 + \omega k_2
$$

Imponiamo ora le condizioni iniziali:
$$
v_c(0^+) = \SI{-0.25}{\volt} \Rightarrow k_1 = -0.25
$$
$$
\frac{dv_c}{dt}(0^+) = -\frac{1}{4}v_c(0^+) + \frac{1}{4}i_L(0^+) = \frac{1}{16} + \frac{1}{4}\left(\frac{1}{12}\right) = \frac{1}{24} = 0.0417
$$
$$
k_2 = -\frac{\sqrt{3}}{12} = -0.1443
$$

$$
v_c(t) = e^{-0.25 t}\left[-0.25\cos(0.1443 t) - \frac{\sqrt{3}}{12}\sin(0.1443 t)\right]
$$
$$
v(t) = R_2 i_c - \frac{R_2}{R_1+R_2}\delta(t) = R_2C\frac{dv_c}{dt} - \frac{R_2}{R_1+R_2}\delta(t) =
3\frac{dv_c}{dt} - \frac{3}{4}\delta(t)=
$$
$$
= \frac{1}{8}e^{-\frac{1}{4}t}\left[\cos\left(\frac{\sqrt{3}}{12}t\right) + \sqrt{3}\sin\left(\frac{\sqrt{3}}{12}t\right)\right] - \frac{3}{4}\delta(t)
$$
$$
v(t) = 0.125 e^{-0.25 t}\left[\cos(0.1443 t) + 1.732\sin(0.1443 t)\right] - 0.75\delta(t)
$$

\textbf{Soluzione con la L-trasformata}

Utilizziamo le impedenze operatoriali

$$
V = -V_s\frac{Z_{R2}}{Z_{R2}+Z_C+\frac{Z_L+Z_{R1}}{Z_L+Z_{R1}}} = -V_s\frac{R_2}{R_2 + \frac{1}{sC} + 
\frac{sLR_1}{R_1+sL}} =
$$
$$
= -\frac{3}{3+\frac{1}{s} + \frac{3 s}{3s + 1}} = \frac{-3 s (3s + 1)}{3s(3s+1) + 3s+1 + 3s^2} = 
$$
$$
= \frac{- 3s(3s+1)}{12s^2+6s+1} = H(s)
$$
Antitrasformiamo:
$$
L^{-1}\left[\frac{-3s(3s+1)}{12s^2+6s +1}\right] = -3 \frac{d}{dt}L^{-1} \left[\frac{3s+1}{12s^2+6s+1}\right] = -\frac{1}{4} \frac{d}{dt} L^{-1} \left[\frac{3s+1}{s^2+0.5s + \frac{1}{12}}\right]
$$
$$
F(s) = \frac{k_1}{s+0.25+0.1443j} + \frac{k_2}{s+0.25-0.1443j}
$$

...

La trasformata inversa viene eseguita con le funzioni esponenziali:
$$
L^{-1}[F(s)] = \frac{3+\sqrt{3}j}{2}e^{-0.25 t}e^{-0.1443 t} + \frac{3-\sqrt{3}j}{2}e^{-0.25 t}e^{j0.1443 t} = e^{-0.125 t}\left[3\cos(0.1443 t) + \sqrt{3}\sin(0.1443 t)\right]\cdot u(t)
$$

$$
h(t) = -\frac{1}{4} \frac{d}{dt}L^{-1}[F(s)] = 0.125 e ^{-0.25 t}[\cos(0.1443 t) + \sqrt{3}\sin(0.1443 t)] - 0.75\delta(t)
$$


\section{Introduzione ai campi stazionari e instazionari}
\subsection{Richiami di analisi vettoriale}
\paragraph{Sistemi di riferimento e coordinate}
Per descrivere una proprietà nello spazio, si utilizza di solito un riferimento composto da 
una terna ortogonale di assi indicati con $L_1$ $L_2$ e $L_3$ con origine comune in $O$.
Si descrive un punto nello spazio $P$ con un raggio vettore che parte dall'origine e 
raggiunge il punto $P$.

\begin{figure}[H] %esempio punto P nello spazio
\centering
\begin{tikzpicture}[scale=3,tdplot_main_coords]
\draw [thick,->] (0,0,0) -- (1,0,0) node[anchor= north east]{$l_1$};
\draw [thick,->] (0,0,0) -- (0,1,0) node[anchor= north west]{$l_2$};
\draw [thick,->] (0,0,0) -- (0,0,1) node[anchor= south]{$l_3$};
\tdplotsetcoord{P}{0.8}{50}{45};
\coordinate (O) at (0,0,0);
\draw [-stealth,color=red] (O) -- (P) node[anchor = west]{$P$};

\draw [dashed,color=red] (O) -- (Px);
\draw [dashed,color=red] (O) -- (Py);
\draw [dashed,color=red] (O) -- (Pz);
\draw [dashed,color=red] (Px) node[anchor = south east]{$x$} -- (Pxy);
\draw [dashed,color=red] (Py) node[anchor = south west]{$y$} -- (Pxy);
%\draw [dashed,color=red] (Px) -- (Pxz);
%\draw [dashed,color=red] (Pz) -- (Pxz);
%\draw [dashed,color=red] (Py) -- (Pyz);
\draw [dashed,color=red] (O) -- (Pxy);
%\draw [dashed,color=red] (Pz) -- (Pyz);
\draw [dashed,color=red] (Pxy) -- (P);
%\draw [dashed,color=red] (Pxz) -- (P);
%\draw [dashed,color=red] (Pyz) -- (P);
\draw [dashed,color=red] (Pz) node[anchor = east]{$z$} -- (P);
\end{tikzpicture}
\end{figure}

Questo vettore può essere determinato con una terna di scalari $(u_1,u_2,u_3)$ che sono le 
coordinate del punto $P$.

La scelta più consona è quella di introdurre un sistema di coordinate cartesiane tali che 
$$
P \rightarrow (x,y,z)\ \ \vec{OP} = x\vec{e_x} + y\vec{e_y} + z\vec{e_z}
$$
con $\vec{e_x},\ \vec{e_y},\ \vec{e_z}$ i versori degli assi coordinati $l_1,\ l_2,\ l_3$.

In alternativa si possono utilizzare le coordinate \textbf{cilindriche}:

\begin{figure}[H] %esempio punto P coordinate cilindriche
\centering
\begin{tikzpicture}[scale=3,tdplot_main_coords]
\draw [thick,->] (0,0,0) -- (1,0,0) node[anchor= north east]{$l_1$};
\draw [thick,->] (0,0,0) -- (0,1,0) node[anchor= north west]{$l_2$};
\draw [thick,->] (0,0,0) -- (0,0,1) node[anchor= south]{$l_3$};
\tdplotsetcoord{P}{0.8}{50}{45};
\coordinate (O) at (0,0,0);
\draw [-stealth,color=red] (O) -- (P) node[anchor = west]{$P$};

\draw [color=red] (O) -- (Pxy) node[anchor = north]{$r$};
\tdplotdrawarc[color=blue,->]{(O)}{0.2}{0}{45}{anchor=north}{$\varphi$};
%\draw [dashed,color=red] (O) -- (Px);
%\draw [dashed,color=red] (O) -- (Py);
%\draw [dashed,color=red] (O) -- (Pz);
%\draw [dashed,color=red] (Px) node[anchor = south east]{$x$} -- (Pxy);
%\draw [dashed,color=red] (Py) node[anchor = south west]{$y$} -- (Pxy);
%\draw [dashed,color=red] (Px) -- (Pxz);
%\draw [dashed,color=red] (Pz) -- (Pxz);
%\draw [dashed,color=red] (Py) -- (Pyz);
%\draw [dashed,color=red] (Pz) -- (Pyz);
\draw [dashed,color=red] (Pxy) -- (P);
%\draw [dashed,color=red] (Pxz) -- (P);
%\draw [dashed,color=red] (Pyz) -- (P);
\draw [dashed,color=red] (Pz) node[anchor = east]{$z$} -- (P);
\end{tikzpicture}
\end{figure}

Il punto $P$ è ancora rappresentato da 3 scalari $(r,\varphi, z)$ e dato dalla combinazione di queste 
coordinate.
$$
\vec{OP} = r\vec{e_r} + \varphi\vec{e_{\varphi}} + z\vec{e_z}
$$

$$
\begin{cases}
r = \sqrt{x^2+y^2}\\
\sin\varphi = \frac{y}{\sqrt{x^2+y^2}},\ \cos\varphi = \frac{x}{\sqrt{x^2+y^2}}\\
z = z
\end{cases}
$$
Queste variabili possono essere ottenute in MATLAB con i seguenti comandi:
\verb|cart2pol| e $\varphi= $ \verb|atan2(y,x)| o viceversa \verb|pol2cart|.
$$
\begin{cases}
x = r\cos\varphi \\
y = r\sin\varphi \\
z = z
\end{cases}
$$

Un altro sistema di riferimento comunemente utilizzato è quello delle coordinate \textbf{sferiche}.

\begin{figure}[H] %esempio punto P coordinate sferiche
\centering
\begin{tikzpicture}[scale=3,tdplot_main_coords]
\draw [thick,->] (0,0,0) -- (1,0,0) node[anchor= north east]{$l_1$};
\draw [thick,->] (0,0,0) -- (0,1,0) node[anchor= north west]{$l_2$};
\draw [thick,->] (0,0,0) -- (0,0,1) node[anchor= south]{$l_3$};
\tdplotsetcoord{P}{0.8}{50}{45}; %coordinate punto P
\tdplotsetthetaplanecoords{45}; %coordinata phi per determinare il piano di theta
\tdplotdrawarc[tdplot_rotated_coords,color=blue,->]{(0,0,0)}{0.5}{0}{50}{anchor=south}{$\theta$};
\coordinate (O) at (0,0,0);
\draw [-stealth,color=red] (O) -- (P) node[anchor = west]{$P$};
\draw (0,0.18,0.13) node[color=red]{$r$};
\draw [dashed,color=red] (O) -- (Pxy);
\tdplotdrawarc[color=blue,->]{(O)}{0.2}{0}{45}{anchor=north}{$\varphi$};
%\draw [dashed,color=red] (O) -- (Px);
%\draw [dashed,color=red] (O) -- (Py);
%\draw [dashed,color=red] (O) -- (Pz);
%\draw [dashed,color=red] (Px) node[anchor = south east]{$x$} -- (Pxy);
%\draw [dashed,color=red] (Py) node[anchor = south west]{$y$} -- (Pxy);
%\draw [dashed,color=red] (Px) -- (Pxz);
%\draw [dashed,color=red] (Pz) -- (Pxz);
%\draw [dashed,color=red] (Py) -- (Pyz);
%\draw [dashed,color=red] (Pz) -- (Pyz);
\draw [dashed,color=red] (Pxy) -- (P);
%\draw [dashed,color=red] (Pxz) -- (P);
%\draw [dashed,color=red] (Pyz) -- (P);
%\draw [dashed,color=red] (Pz) node[anchor = east]{$z$} -- (P);
\end{tikzpicture}
\end{figure}

$$
P\rightarrow (r,\theta,\varphi)\ \ \vec{OP} = r\vec{e_r} + \theta\vec{e_\theta} + \varphi \vec{e_\varphi}
$$
con $(\vec{e_r},\vec{e_\theta},\vec{e_\varphi})$ terna levogira

$$
\begin{cases}
x = r\sin\theta\cos\varphi\\
y = r\sin\theta\sin\varphi\\
z = r\cos\theta
\end{cases}
$$
anche in questo caso è possibile utilizzare la funzione MATLAB \verb|sph2cart|.

Formule inverse:
$$
\begin{cases}
r &= \sqrt{x^2+y^2+z^2} \\
\cos\theta &= \frac{z}{\sqrt{x^2+y^2+z^2}}\\
\sin\theta &= \frac{\sqrt{x^2+y^2}}{\sqrt{x^2+y^2+z^2}} \\
\cos\varphi  &= \frac{x}{\sqrt{x^2+y^2}},\ \sin\varphi = \frac{y}{\sqrt{x^2+y^2}}
\end{cases}
$$

\subsection{Spostamenti elementari}
Si supponga uno spostamento lungo la direzione $l_1$ associando un coefficiente ``metrico'' pari 
alla distanza percorsa nel sistema di coordinate.

\begin{align*}
dl_1 &= h_1 du_1 \\
dl_2 &= h_2 du_2 \\
dl_3 &= h_3 du_3
\end{align*}

Questi fattori ``aggiustano'' le dimensioni delle relazioni in metri dovute a variazioni di 
coordinate differenti ad esempio in radianti.

Per le coordinate cartesiane i coefficienti metrici sono tutti uguali tra loro e pari ad 1.
$$
h_1 = h_2 = h_3 = 1 \Rightarrow
\begin{cases}
dl_1 = dx \\
dl_2 = dy \\
dl_3 = dz
\end{cases}
$$
Si suppone di costruire un volumetto elementare attorno il punto $P$, se ne può calcolare
l'area delle facce e il suo volume.
Si indica con $dS_1$ la superficie perpendicolare all'asse $l_1$, essa sarà pari a 
$dS_1 = dl_2\cdot dl_3 = dy\cdot dz$, si riportano per completezza le tre superfici:
\begin{align*}
dS_1 &= dl_2\cdot dl_3 = dy\cdot dz \\
dS_2 &= dl_3\cdot dl_1 = dz\cdot dx \\
dS_3 &= dl_1\cdot dl_2 = dx\cdot dy
\end{align*}
Il volume elementare invece si ricava con:
$$
dV = dl_1\cdot dl_2 \cdot dl_3 = dx\cdot dy\cdot dz
$$

Ripetiamo l'analisi per le \textbf{coordinate cilindriche:}
$ (u_1,u_2,u_3) = (r,\varphi,z)$

Supponiamo uno spostamento associato alla variazione della coordinata $r$, il raggio vettore
viene incrementato di una quantità $dr = dl_1 \Rightarrow h_1 =1$.

Effettuando una variazione $d\varphi$ invece il raggio vettore percorrerà un arco pari 
a $rd\varphi = dl_2 \Rightarrow h_2 = r$ per una variazione di arco ci sarà uno 
spostamento proporzionale alla distanza dall'origine $r$.
\begin{align*}
dr = dl_1 &\Rightarrow h_1 = 1\\
r d\varphi = dl_2 &\Rightarrow h_2 = r \\
dz = dl_3 &\Rightarrow h_3 = 1
\end{align*}
Superfici elementari:
\begin{align*}
dS_1 &= r\cdot d\varphi\cdot dz\\
dS_2 &= dz\cdot dr\\
dS_3 &= r\cdot dr\cdot d\varphi
\end{align*}
Volume infinitesimo:
$$
dV = r\cdot dr\cdot d\varphi\cdot dz
$$

In \textbf{coordinate sferiche} si ha $(u_1,u_2,u_3)=(r,\theta,\varphi)$

Ad una variazione $dr$ si ha uno spostamento lungo il raggio vettore $\vec{OP}$ quindi anche in questo
caso il fattore metrico sarà pari ad 1.

Ad una variazione della variabile $\theta$ detta anche co-latitudine, corrisponde una rotazione
pari a $dl_2 = r\cdot d\theta \Rightarrow h_2 = r$.

Ad una variazione di $\phi$ si ha un arco $dl_3 = r\cdot\sin\theta\cdot d \varphi \Rightarrow h_3 = r\cdot\sin\theta$
\begin{align*}
dl_1 &= dr & h_1 &=1 \\
dl_2 &= r\cdot d\theta & h_2 &=r \\
dl_3 &= r\cdot\sin\theta\cdot d\varphi & h_3 &= r\cdot \sin\theta 
\end{align*}

Superfici infinitesime:
\begin{align*}
dS_1 &= r^2\cdot\sin\theta\cdot d\theta\cdot d\varphi \\
dS_2 &= r\cdot \sin\theta \\
dS_3 &= r\cdot d \theta\cdot dr
\end{align*}
Volume infinitesimo:
$$
dV = r^2\cdot \sin\theta\cdot dr\cdot d\theta\cdot d\varphi
$$

\subsection{Richiami sui campi}
\paragraph{Definizione}
Si intende con campo \textbf{scalare} una funzione $f:\Omega \in R^3 \to R$ con $\Omega$ 
sufficientemente regolare.

Un campo \textbf{vettoriale} invece è una funzione $\vec{v} : \Omega \in R^3 \to R^3$
che può essere espressa mediante 3 campi scalari $v_1(P),v_2(P),v_3(P)$ che sono le componenti
di $\vec{v}(P)$ nel sistema di coordinate scelto.

In coordinate \textit{cartesiane}:
$$
\vec{v}(x,y,z) = v_x(x,y,z)\vec{e_x} + v_y(x,y,z)\vec{e_y} + v_z(x,y,z)\vec{e_z}
$$

In coordinate \textit{sferiche}:
$$
\vec{v}(r,\theta,\varphi) = v_r(r,\theta,\varphi)\vec{e_r} + 
v_\theta(r,\theta,\varphi)\vec{e_\theta} + v_\varphi(r,\theta,\varphi)\vec{e_\varphi}
$$

Richiamiamo per i campi scalari la superficie (curva) di livello per il caso bidimensionale 
(tridimensionale):
$$
f(P) \in C^1(\Omega)
$$
una superficie di livello è definita da:
$$
f(x,y,z) = f_0
$$
In ogni punto $P$ passa una ed una sola superficie di livello.

Nel caso bidimensionale $f(x,y) = f_0$.


Si definisce la \textbf{linea di forza} di un campo vettoriale.
Sia $\vec{v(P)} : \Omega \to R^3$ e sia la linea $\gamma$ tangente in ogni punto a $\vec{v(P)}$,
è per definizione una linea vettoriale di $\vec{v}$.
L'orientamento dei versori tangenti della linea $\gamma$ descrivono direzione e verso di $\vec{v}$ 
(non il modulo).

Tracciamento delle linee vettoriali di un campo $\vec{v}(P)$ assegnato:

Riferendosi alle componenti del campo vettoriale secondo gli assi cartesiani appena definiti
si vede che lo spostamento differenziale $dx$ e $dy$ associato al campo vettoriale soddisfa la 
relazione:
$$
\frac{v_y(P_0)}{v_x(P_0)} = \frac{dy}{dx}(P_0)
$$
Estendendo questo concetto per tutti i punti della linea che si vuole tracciare
possiamo risolvere un problema ai valori iniziali per l'equazione differenziale ordinaria nella
forma:
$$
\begin{cases}
\frac{dy}{dx} = \frac{v_y(x,y,z)}{v_x(x,y,z)} \\
\frac{dz}{dx} = \frac{v_z(x,y,z)}{v_x(x,y,z)}\\
y(x_0) = y_0 \\
z(x_0) = z_0
\end{cases}
$$
Questa equazione restituisce la linea di campo a partire dal punto $P_0$ in forma parametrica
rispetto ad $x$.

Si definisce una \textbf{superficie vettoriale} di $\vec{v}(P)$
$$
S\in \Omega: \hat{n}(P)\cdot\vec{v}(P) = 0\ \forall\ P \in S
$$
Sia data una superficie aperta in cui è contenuto un campo vettoriale $\vec{v}$,
i versori normali $\hat{n}$ saranno ortogonali in ogni punto della superficie $S$ e quindi al 
campo $\vec{v}$.

Il \textbf{tubo di flusso} di un campo $\vec{v}$: 
sia $\Gamma$ una linea chiusa che non sia una linea di forza
sulla quale vengono definiti dei punti, si suppone che ci siano delle linee vettoriali che attraversano
questi punti. Si definisce questo ``tubo'' in modo tale che le linee vettoriali siano tangenti sulla sua faccia laterale.
\newpage
\subsection{Circuitazioni e flussi di campi vettoriali}

\textbf{Circuitazione} (o circolazione) di $\vec{v}$ sulla linea chiusa $\Gamma$, si suppone un punto $Q$ sulla 
linea e il versore tangente $\hat{t}(Q)$ e una linea vettoriale $\vec{v}(Q)$ con un certo angolo
$\alpha$ rispetto a $\hat{t}(Q)$ allora:

$$
\oint_{\Gamma}  \vec{v}\cdot\hat{t}dl = 
\oint_{\Gamma}  \left|\vec{v}(Q)\right|\cdot\cos\alpha\ dl
$$
con $\hat{t}dl$ elemento di linea orientata.

\textbf{Flusso} di $\vec{v}$ uscente da una superficie chiusa $\Sigma$

$$
\oiint_{\Sigma}\vec{v}\cdot\hat{n}\ dS = \oiint_{\Sigma} \left|\vec{v}(Q)\right|\cos\alpha\ dS
$$

\subsection{Operatori differenziali}

\textbf{Gradiente} siano date due superfici di livello $f(P) = f_0 = S_0$ e
$f(P) = f_0 + \Delta f = S_1$, consideriamo la distanza $d(P,P_0)$ con 
$P\in S_1$ lungo la retta normale alla superficie $S_0$ passante per il punto 
$P_0$ ed individuo un punto $P$ su $S_1$, si considera ora il rapporto $\frac{f(P)\cdot f(P_0)}{d(p,p_0)}$ è un rapporto incrementale che con il limite:
$$
\lim_{P\to P_0} \left[\frac{f(P)\cdot f(P_0)}{d(p,p_0)}\right] \stackrel{\text{\text{def}}}{=} \left.\frac{\partial f}{\partial n}\right|_{P_0} 
$$
definisce la derivata parziale lungo la direzione normale passante per il punto $P_0$.

Si definisce il \textbf{gradiente} di $f$ (indicato con $\nabla f$) il vettore con modulo pari a 
$\left.\frac{\partial f}{\partial n}\right|_{P_0}$, verso pari a quello della
normale $n$ nella direzione delle $f$ crescenti.
È quindi possibile definire una generica variazione di $f$ con:
$$
df = \nabla f(P_0)\cdot d\vec{r}
$$
con $\vec{dr}$ uno generico spostamento $\vec{r_q} - \vec{r_{p_0}},\ q\in\Omega$.

\paragraph{Gradiente nei sistemi di coordinate curvilinee}
Si consideri uno spostamento elementare lungo ciascuna delle direzioni coordinate
$dl_1,\ dl_2,\ dl_3$ allora la variazione
$$
df = \nabla f (P_0) \cdot dl_i\vec{e_i} = \nabla f(P_0) \cdot h_i du_i\vec{e_i} 
$$
ma ricordando che il prodotto tra il gradiente e il versore della coordinata i-esima è pari
alla derivata direzionale lungo quella coordinata
$$
\frac{1}{h_i} \frac{\partial f}{\partial u_i} = \left[\nabla f (P_0)\right]\cdot \vec{e_i}
$$
Ad esempio nelle coordinate cartesiane:
$$
\nabla f(P) = \frac{\partial f}{\partial x}\vec{e_x} + \frac{\partial f}{\partial y}\vec{e_y} + 
\frac{\partial f}{\partial z}\vec{e_z}
$$

Oppure nelle coordinate cilindriche:
$$
\nabla f(P) = \frac{\partial f}{\partial r}\vec{e_r} + \frac{1}{r}\frac{\partial f}{\partial \varphi}\vec{e_\varphi} + \frac{\partial f}{\partial z}\vec{e_z}
$$

Coordinate sferiche:
$$
\nabla f(P) = \frac{\partial f}{\partial r}\vec{e_r} + \frac{1}{r}\frac{\partial f}{\partial \theta}\vec{e_\theta} + \frac{1}{r\sin\theta} \frac{\partial f}{\partial \varphi}\vec{e_\varphi}
$$


Si aggiunge inoltre l'operatore \textbf{divergenza} di un campo vettoriale $\nabla\cdot\vec{v}$.

Sia $\Sigma$ una superficie chiusa che delimita una regione di spazio $\Omega_\Sigma$ contenente 
il punto $P\in\Omega$, sia $\hat{n}$ il versore normale alla
superficie, se ne definisce il flusso uscente alla superficie $\Phi_\Sigma$
$$
\oiint_\Sigma \vec{v}\cdot \hat{n}dS = \Phi_\Sigma
$$
Effettuando il limite sul volume 
$$
\lim_{\text{Vol}(\Omega_\Sigma)\to 0} \frac{\oiint_\Sigma \vec{v}\cdot \hat{n}dS}{\text{Vol}(\Omega_\Sigma)}
\stackrel{\text{def}}{=} \nabla\cdot \vec{v}(P)
$$
con la condizione che il volume infinitesimo contenga ancora il punto $P$ ed il limite esista e 
sia finito.

In coordinate cartesiane la divergenza di $\vec{v}(P)$
$$
\nabla\cdot\vec{v}(P) = \frac{\partial v_x}{\partial x} + \frac{\partial v_y}{\partial y} + \frac{\partial v_z}{\partial z}  
$$


\paragraph{Teorema della divergenza}
Il flusso uscente dalla superficie $\Sigma$ è pari all'integrale esteso al volume 
$\Omega_\Sigma$ della divergenza.
$$
\oiint_\Sigma \vec{v}\cdot \hat{n} dS = \iiint_{\Omega_\Sigma} \nabla\cdot \vec{v} d V
$$

\paragraph{Campi solenoidali e indivergenti}

Un campo $\vec{V}(P)$ si dice \textbf{solenoidale} nel dominio $\Omega$ se $\forall$ superficie chiusa $\Sigma$
$$
\oiint_\Sigma \vec{v}\cdot \hat{n} dS = 0\ \ \forall\ \Sigma \in \Omega
$$

Per i campi solenoidali, il flusso attraverso una superficie aperta dipende solo dall'orlo
della superficie (flusso concatenato ad una linea).
$$
\iint_{S_\Gamma'} \vec{v}\cdot\hat{n}dS = \iint_{S_\Gamma''} \vec{v}\cdot\hat{n}dS 
$$
con $\Gamma$ l'orlo delle due superfici $S_\Gamma'$ e $S_\Gamma''$.

Campo \textbf{indivergente}: $\vec{v}(P)$ indivergente in $\Omega$ se $div \vec{v}(P) = 0 \ \forall P \in \Omega$

I campi indivergenti soddisfano il principio di deformazione della superficie:

Si supponga di avere un dominio $\Omega$ contenente due superfici, $\Sigma_1$ contenente $\Sigma_2$
allora:
$$
\iint_{\Sigma_1} \vec{v}\cdot\hat{n}dS = \iint_{\Sigma_2}\vec{v}\cdot\hat{n}dS 
$$
le due superfici sono deformabili con continuità una nell'altra senza mai uscire da $\Omega$.

Determinata la regione di spazio compresa tra le due superfici $\Omega_{\Sigma_1\Sigma_2}$,
effettuando l'integrale della divergenza esso sarà pari a 0 (perché nulla la divergenza) ma per 
il teorema della divergenza esso sarà pari all'integrale sulle 2 superfici:
$$
\iiint_{\Omega_{\Sigma_1\Sigma_2}} \nabla\cdot\vec{v}dV = 0 = \oiint_{\Sigma_1}\vec{v}\cdot\hat{n}dS - \oiint_{\Sigma_2}\vec{v}\cdot\hat{n}dS
$$

Quando la \textbf{solenoidalità} è equivalente alla divergenza?

Se $\vec{v}(P)$ è solenoidale in $\Omega$ allora il flusso è pari a zero in ogni $\Sigma$, $\oiint_\Sigma \vec{v}\cdot\hat{n}dS = 0 \ \forall \Sigma $ allora
anche il flusso sulle superfici utilizzate per calcolare il limite che conduce alla divergenza del campo vettoriale ricade in questa circostanza, quindi $\nabla\cdot\vec{v}$ per $p\to 0$ è pari a 0 in ogni punto
del dominio $\Omega$, quindi la solenoidalità implica l'indivergenza.
L'indivergenza diventa una condizione necessaria per la solenoidalità.

Per dimostrare il viceversa è necessario richiedere un'ipotesi sul dominio, ossia che esso sia
a connessione superficiale semplice (semplicemente connesso)
ogni superficie appartenente ad $\Omega$ può essere ridotta ad un punto con continuità senza uscire da $\Omega$,
allora questa condizione è sufficiente per dire che $\vec{v}(P)$ è solenoidale in $\Omega$.

Un esempio di campo non solenoidale è quello con un dominio multi-connesso, ossia un dominio
al quale viene sottratta un'intera regione di spazio al suo interno.
Se anche la divergenza del campo in un punto di questo dominio è pari a 0, il campo non è 
solenoidale perché l'integrale del flusso non è pari a 0.

Nella regione non inclusa nel dominio potrebbero essere infatti presenti delle cariche
elettriche, e la divergenza non sarebbe nulla.

\section{Campi conservativi e rotazionali}
\paragraph{Rotore} di un campo vettoriale $\vec{v}(P)$:

Consideriamo una linea chiusa $\Gamma$ all'interno del dominio $\Omega$ ed una superficie
$S_\Gamma$ (di orlo $\Gamma$).

Supponiamo che questa superficie passi su un punto $P$ e consideriamo il versore normale alla 
superficie orientato con la regola della mano destra rispetto al versore tangente della linea $\Gamma$.
Consideriamo inoltre:
$$
\lim_{\text{Area}(S_\Gamma)\to 0}\left(\frac{\oint_{\Gamma} \vec{v}\cdot\hat{t}dl}{\text{Area}(S_\Gamma)}\right) \stackrel{\text{def}}{=} \left(\nabla\times\vec{v}(P)\right)\cdot\hat{n}
$$
Nel caso in cui questo limite esista e sia finito e indipendente dalla forma di $\Gamma$, per definizione
diciamo che questa è la componente normale di un campo vettoriale chiamato \textit{rotore} di $\vec{v}$,
$(\nabla \times \vec{v})\cdot \hat{n}$.

In coordinate cartesiane si ricava il rotore con:

$$
\nabla \times \vec{v} =
\begin{vmatrix}
\vec{e_x} & \vec{e_y} & \vec{e_z} \\
\frac{\partial}{\partial x} & \frac{\partial}{\partial y} & \frac{\partial}{\partial z} \\
v_x & v_y & v_z
\end{vmatrix}
$$

$$
\nabla\times{\vec{v}}(P) = \left(\frac{\partial v_z}{\partial y} - \frac{\partial v_y}{\partial z}\right)\vec{e_x} -
\left(\frac{\partial v_z}{\partial x} - \frac{\partial v_x}{\partial z}\right)\vec{e_y} +
\left(\frac{\partial v_y}{\partial x} - \frac{\partial v_x}{\partial y}\right)\vec{e_z}
$$

\paragraph{Teorema di Stokes}
Dato un campo $\vec{v} \in \Omega$ sufficientemente regolare, $\Gamma$ chiusa in $\Omega$.

Se calcolo la circuitazione alla linea gamma
$$
\oint_\Gamma \vec{v}\cdot\hat{t} dl = \iint_{S_\Gamma} \nabla\times \vec{v} \cdot \hat{n} dS
$$

Si possono dunque introdurre i campi conservativi e irrotazionali:
$\vec{v}(P) \in \Omega$ si dice conservativo (per la circuitazione) quando
$$
\oint_\Gamma \vec{v}\cdot\hat{t} dl = 0\ \ \forall\ \Gamma \in \Omega
$$

Questa condizione equivale a dire che presi due punti nello spazio $A$ e $B$ e prese due linee aperte che connettono questi due punti, allora
$$
\int_{A\gamma'B} \vec{v}\cdot\hat{t}dl = \int_{A\gamma''B} \vec{v}\cdot\hat{t}dl\ \ \forall A,B\in\Omega,\ \forall \gamma',\gamma'' \in \Omega
$$

Un campo conservativo può essere espresso come gradiente di una funzione potenziale:
$$
\vec{v}(P) = -\nabla\varphi(P)
$$
con $\varphi(P)$ funzione potenziale, questo equivale a dire che:
$$
\int_{A\gamma B} \vec{v}\cdot \hat{t} dl = \int_{A\gamma B} -\nabla\varphi\cdot\hat{t} dl =
-\int_{A}^B d\varphi = \varphi(A) - \varphi(B)
$$
Il gradiente di una funzione per uno spostamento infinitesimo equivale al differenziale della funzione.

\paragraph{Campi irrotazionali}
$\vec{v}(P) \in C^1$ irrotazionale se 
$$
\nabla\times \vec{v}(P) = 0\  \forall P \in \Omega
$$
I campi irrotazionali soddisfano dunque il principio di deformazione del contorno, in analogia
ai campi solenoidali che soddisfano il principio di deformazione della superficie.
Preso un dominio $\Omega$ e due linee in esso contenute $\gamma_1$ e $\gamma_2$,
se ipotizzo che 
$$
\vec{v}(P) : \nabla\times\vec{v}(P) =0\ \forall P \in \Omega
$$
ipotizzata una superficie $S\gamma_1\gamma_2$ compresa tra le due curve allora utilizzando
il teorema di Stokes
$$
\iint_{S\gamma_1\gamma_2} \nabla\times\vec{v}\cdot\hat{n}dS = 0 = \oint_{\gamma_1} \vec{v}\cdot\hat{t} dl - \oint_{\gamma_2} \vec{v}\cdot\hat{t}dl = 0\ \forall\gamma_1,\gamma_2\in \Omega
$$

Il segno ``$-$'' all'interno dell'equazione è dovuto al diverso riferimento della superficie rispetto alla
curva $\gamma_2$.

Quando conservatività e irrotazionalità sono equivalenti?
Se $\vec{v}(P)$ è conservativo in $\Omega $ tutti gli integrali lungo linee chiuse sono nulli, 
allora $\nabla\times\vec{v}=0$

Il viceversa:
$$
\begin{cases}
\vec{v}(P) \text{ irrotazionale in }\Omega \\
\Omega \text{ a connessione lineare semplice}
\end{cases}
\Rightarrow \vec{v}(P) \text{ è conservativo in }\Omega
$$

\paragraph{Teorema di decomposizione di Helmhotz} (in tutto lo spazio)

Preso un campo $\vec{v}(P),\ P \in R^3$ normale all'infinito ossia che il modulo del campo tenda
a 0 quando la distanza tende all'infinito, in tale ipotesi può essere decomposto nella somma
di un campo solenoidale e un campo conservativo (entrambi normali all'infinito).

$$
\vec{v}(P) = \vec{v}_\text{sol} + \vec{v}_\text{cons},\ 
\oint_{\Gamma}\vec{v}_\text{cons}\cdot\hat{t} dl = 0\ \forall\ \Gamma,\ \oiint_{\Sigma}
\vec{v}_\text{sol}\cdot\hat{n}dS = 0\ \forall\ \Sigma 
$$
\textit{Se sono note tutte le circuitazioni su linee chiuse dello spazio e tutti i flussi uscenti da
superfici chiuse per un campo $\vec{v}(P)$ normale all'infinito, allora $\vec{v}(P)$ è univocamente
determinato}

\textbf{Operatore NABLA} ($\nabla$)
$$
\nabla = \frac{\partial}{\partial x}\vec{e_x} + \frac{\partial}{\partial y}\vec{e_y} + \frac{\partial}{\partial z}\vec{e_z}
$$
Si esprime il gradiente di un campo scalare $f(P)$ con: $\nabla f$ come se facessi il prodotto
termine a termine dei componenti dell'operatore $\nabla$
$$
\text{grad} f(P) = \nabla f = \frac{\partial f}{\partial x}\vec{e_x}
$$

\textbf{Divergenza}
$$
\text{div}\vec{v}(P) = \nabla\cdot\vec{v} = \frac{\partial v_x}{\partial x} + 
$$

\textbf{Rotore}
$$
\text{rot}\vec{v}(P) = \nabla \times \vec{v}
$$

\textbf{Identità vettoriali:}
\begin{align*}
\nabla\cdot(f\vec{v}) &= f\nabla\cdot\vec{v} + \vec{v}\cdot \nabla f \\
\nabla(f g) &= f\nabla g + g\nabla f \\
\nabla \times (f\vec{v}) &= f\nabla\times\vec{v} + \nabla f \vec{v} \\
\nabla\cdot (\vec{v}\times\vec{w}) &= \vec{w}\cdot\nabla\times\vec{v} - \vec{v}\cdot\nabla\times\vec{w}
= \vec{w}\cdot(\nabla\times\vec{v}) + \vec{v}\cdot (\vec{w}\times\nabla)
\end{align*}

Operatori differenziali del secondo ordine:
$ \vec{v}(P) \in C^2(\Omega) $ e considero combinazioni possibili di grad,div e rot per ottenere
operatori del secondo ordine:
\begin{align*}
\nabla f &\text{ scalare in un vettore}\\
\nabla\cdot\vec{v} &\text{ vettore in uno scalare}\\
\nabla\times\vec{v} &\text{ vettore in un vettore}
\end{align*}
quindi:
\begin{align*}
\nabla\times\nabla f = 0 &\text{ perchè $\nabla f$ è conservativo $\Rightarrow$ irrotazionale}\\
\nabla\cdot\nabla\times\vec{v} = 0 &\text{ dimostra in coordinate cartesiane}\\
\nabla\cdot \nabla f = \nabla^2 f &\text{ laplaciano}
\end{align*}
\textbf{Operatore laplaciano}
$$
\nabla^2 f = \frac{\partial^2f}{\partial x^2} + \frac{\partial^2f}{\partial y^2} + \frac{\partial^2f}{\partial z^2}
$$
Laplaciano vettore:
$$
\nabla\nabla\cdot\vec{v} - \nabla\times\nabla\times\vec{v} = \vec{\nabla}^2\vec{v}
$$

$$
\vec{\nabla}^2\vec{v} = \vec{\nabla}^2v_x\vec{e_x} + ...
$$

\section{Richiami di Elettromagnetismo}
L'elettromagnetismo è una delle 4 interazioni fondamentali che governano l'universo, le altre
sono le interazioni forti, riguardanti i quark, le interazioni deboli riguardano gli elettroni e i 
neutrini, le interazioni elettromagnetiche (che riguardano le cariche) e le interazioni gravitazionali
che coinvolgono le masse.

\subsection{La carica elettrica}
Possono essere di due specie, positive e negative e sono associate a interazioni di tipo repulsivo tra
cariche della stessa specie e attrattive tra cariche di segno opposto.
Si parla in questo caso di cariche ferme, altirmenti sarebbero soggette ad altre tipologie di forze.
L'Unità di misura è il Coulomb [\si{\coulomb}].
La carica è invariante per una particella in quiete o in moto.

La carica è quantizzata, ossia considerando una vportzione di materia in una certa regione 
dello spazio vuoto, la carica contenuta $\Delta Q = N^+q_p + N^-q_e\ \ N^+,N^- \in N$
e i valori di $q_p$ e $q_e$ 

Il principio di conservazione della carica per un sistema chiuso elettricamente afferma che in un tale 
sistema la quantità di carica netta $Q(t) = Q^+ + Q^- = Q_0\ \forall\ t$

\paragraph{Forza agente su una carica in moto}

Sia dato un sistema di riferimento nello spazio, in un certo punto P dello spazio ci sia una carica $q$
che sta percorrendo una traiettoria con un certo vettore di velocità $\vec{v}(P,t)$, questa
carica è sottoposta ad una forza elettromagnetica chiamata \textbf{forza di Lorentz}
$$
\vec{F}(p,t) = q\left[\vec{E}(P,t)+\vec{v}(P,t)\times\vec{B}(P,t)\right]
$$
dove $\vec{B}$ è il campo di induzione magentica anche detto densità di flusso magnetico.

Ricordando la dinamica newtoniana: $\vec{F} = m\vec{a}$, integrando l'equazione del moto si può
ricavare la legge orario di ciascuna carica, mediante la legge di Lorentz ammesso che si conoscano 
il campo elettrico e il campo di induzione magnetica.
Viene riconosciuto come modello di Maxwell-Lorentz.

Un blocco restituisce i campi elettromagnetici ($\vec{E},\vec{B}$) con i quali, si ricavano in un altro
``blocco'' le equazioni del moto sfruttando la forza di Lorentz.
In uscita al secondo blocco mediante la legge oraria, si può calcolare la distribuzione di cariche e 
corrente $\left(\rho,\vec{J}\right)$, ossia le sorgenti dei due campi, con queste si rientra nel primo blocco.

\paragraph{Definizioni operative}
Campo elettrico:
si esegue una misura in condizioni statiche, ossia la carica ha velocità nulla, con un ipotetico 
dinamometro si può misurare la forza F e facendo il rapporto con la carica $q$:
$$
\left.\frac{\vec{F}}{q}\right|_{\vec{v}=0} \stackrel{\text{def}}{=} \vec{E}(P,t)
$$
Si effettuano poi due misure dinamiche a velocità non nulla
$$
\vec{v}\times\vec{B} = \vec{F} - q\vec{E}
$$
Attraverso queste due misure si possono ricavare le componenti di $\vec{E}(P,t)$ e $\vec{B}(P,t)$.


Le unità di misura:
$[F] = \si{\newton}\ \  [E] = \si{\volt\per\meter} \ \ [B] = \si{\tesla}$

Si introduce una descrizione della materia nella teoria del continuo, facendo riferimento alle 
distribuzioni di cariche:
Presa una regione $\Omega$ contenente un punto $P$ attorno al quale consideriamo un volumetto elementare 
$\Delta\Omega$, all'interno del volumetto sono contenute diverse cariche che si muovono a diverse 
velocità. Nell'ipotesi del continuo si deve supporre che $\Delta \Omega$ deve essere sufficientemente ...
Ma deve essere anche sufficientemente grande in modo da contenere un numero di particelle che definiscano
grandezze che variano con continuità.

Si definisce quindi la funzione 
$$
\rho(P,t) \stackrel{\text{def}}{=} \frac{N^+q_p+N^-q_e}{\text{Vol}(\Delta\Omega)}
$$
densità volumetrica di carica, campo scalare $[\rho] = \si{\coulomb\per\meter}$.
Si possono inoltre definire densità di carica parziali $\rho^+$ e $\rho^-$ associate alla densità
di cariche di quello specifico segno, ovviamente $\rho = \rho^+ + \rho^-$.

\textbf{Densità di corrente elettrica}
Si definiscono con $v^+$ e $v^-$ le velocità medie (di \textit{drift}) dei portatori di carica positivi e negativi.

$$
\vec{J}(P,t) \stackrel{\text{def}}{=} \frac{N^+q_p\vec{v}^+ + }{\text{Vol}(\Delta \Omega)} = ...
$$

$$
[J] = \frac{A}{m^2}
$$
Nell'unità di tempo ci sarà un flusso di cariche attraverso il volumetto di controllo, mediante la sua 
frontiera.

\textbf{Intensità di corrente} attraverso una superficie aperta $S$ orientata con normale $\hat{n}$.

Sia $P$ un punto sulla superficie circondato da un volumetto di controllo, nello stesso punto
è definito il campo \textit{densità di corrente $J$}.

La quantità di carica netta che attraversa la superficie S nell'intervallo di tempo $[t,t+\delta t]$
$$
\delta Q_S = \vec{J}\cdot \hat{n}\ dS\ dt
$$
01:35:00



\textbf{Principio di conservazione della carica per sistemi aperti}
Sia $\Sigma$ una superficie chiusa che racchiude una superficie
$\Omega_\Sigma:\ \partial\Omega_\Sigma=\Sigma$
$$
i_{\Sigma(t)} = - \frac{dQ_{\Omega_\Sigma}}{dt}
$$

\paragraph{Materiali isolanti e conduttori}
Sia una regione $\Omega$ contenente il punto P con il suo volumetto elementare associato, allora
$\Omega$ è isolante se $\vec{J}(P,t) = 0\ \forall\ t \forall\ p \in \Omega$.

Conduttore ohmico se rispetta la legge di Ohm in forma locale, ossia:
$\vec{J} = \gamma\vec{E}$ dove $\gamma$ è la conducibilità elettrica [\si{\siemens\per\meter}]
o viceversa
$$
\vec{E} = \eta \vec{J}
$$


\paragraph{Conduzione nei metalli}
I portatori di carica sono elettroni ``liberi'' ossia in grado di percorrere distanze
macroscopiche sotto l'azione di una forza sotto l'azione di un campo elettrico.
In un metallo a riposo c'è un moto dei portatori di carica ma la velocità media di deriva (drift)
$$
\vec{v}\ ^- = 0
$$
ciò non implica che le particelle siano ferme. Sono soggette infatti anche all'agitazione termica.

Nel caso in cui la velocità media sia diversa da 0 si ha una velocità delle particelle più bassa.

Considerando un conduttore cilindrico di sezione di area $S=\SI{1}{\milli\meter^2}$, considerando un'
intensità di corrente pari ad \SI{1}{\ampere}, e indicando con $n^-$ la densità di elettroni
solitamente pari a $\frac{10^{23}}{\si{\centi\meter}^3} = \SI{e29}{\meter^{-3}}$
$$
J = \frac{i}{S} = 10^6\si{\ampere\per\meter^2} = n^-q_e\vec{v}
$$

La densità $n^-$ di elettroni liberi è solitamente pari a $10^{23}/\si{\centi\meter^3}$
$$
v^- = \frac{J}{n^-q_e} =  \frac{\cancel{10^6}}{10^{23}\cdot\cancel{10^6}\cdot (-1.6\cdot10^{19})} \sim 10^{-6} \si{\meter\per\second} = \SI{1}{\micro\meter\per\second}
$$

\subsection{Equazioni di Maxwell nel vuoto}
Ci si possono esprimere tutti i fenomeni di elettromagnetismo macroscopico, in presenza di distribuzioni 
di carica e correnti nello spazio vuoto.

Forniscono tutte le possibili circuitazioni e i possibili flussi uscenti da superfici chiuse per i campi
$\vec{E}$ e $\vec{B}$.

In virtù del teorema di decomposizione di Helmoltz si garantisce che $\vec{E}$ e $\vec{B}$ 
sono univocamente determinati (assumendoli normali all'infinito).

\paragraph{Legge di gauss per il campo elettrico $\vec{E}$}
Consideriamo una superficie chiusa $\Sigma$ nello spazio con bordo $\Omega_\Sigma$
la prima equazione di Maxwell afferma: 
\begin{equation}
\label{eq:legge_gauss}
\oiint_\Sigma \vec{E}\cdot\hat{n}dS = \frac{1}{\varepsilon_0}Q_{\Omega_\Sigma} = \frac{1}{\varepsilon_0}
\iiint_{\Omega_\Sigma} \rho dV \ \ \forall \Sigma,\ \forall t
\end{equation}

$$
\left[\oiint \vec{E}\cdot\hat{n}dS\right] = \si{\volt\per\meter\cdot\meter^2} = \si{\volt\cdot\meter}
$$

\paragraph{Legge di Gauss per il campo $\vec{B}$} o solenoidalità del campo $\vec{B}$,
afferma che il flusso attraverso una superficie aperta dipende solo dall'orlo (flusso concatenato ad una
linea).
\begin{equation}
\oiint_{\Sigma} \vec{B}\cdot\hat{n}dS = 0\ \forall \Sigma, \forall t
\end{equation}

$$
\left[\oiint \vec{B}\cdot\hat{n}dS\right] = \si{\tesla\cdot\meter^2} = \si{\weber}
$$
Un'interpretazione di questa legge, in analogia a quella per il campo elettrico, possiamo affermare
che non è possibile isolare una ``carica'' magnetica netta, ossia un monopolo magnetico.

Le restanti equazioni di Maxwell riguardano le circuitazioni, definiscono i fenomeni di induzione
elettromagnetica ed induzione magnetoelettrica.

\paragraph{Legge di Faraday-Neumann-Lenz}
Consideriamo una linea chiusa qualunque $\Gamma$ ed una qualsiasi superficie $S_\Gamma$ di
orlo $\Gamma$.
\begin{equation}
\oint_{\Gamma} \vec{E}\cdot\hat{t}dl = - \iint_{S_\Gamma} \frac{\partial \vec{B}}{\partial t}
\cdot\hat{n} dS\ \ \forall\Gamma,\ S_\Gamma\in\Omega,\ \forall t
\end{equation}
$\Gamma$ ed $S_\Gamma$ possono anche essere variabili nel tempo.

Se in una regione di spazio c'è un campo $\vec{B}$ variabile nel tempo, esso induce necessariamente
un campo elettrico nella stessa regione di spazio.
Si può semplificare l'equazione supponendo che le superfici $\Gamma$ ed $S_\Gamma$ siano ferme
$$
\oint_{\Gamma} \vec{E}\cdot\hat{t}dl = -\frac{d}{dt} \iint_{S_\Gamma}\vec{b}\cdot\hat{t}dS = 
-\frac{d}{dt} \Phi_\Gamma
$$
Presi due punti qualsiasi $A$ e $B \in R^3$ supponiamo che esistono infinite linee $\gamma$, il terzo 
termine è il suo opposto.

La differenza fra tensione elettrica calcolata tra due linee che connettono gli stessi punti nello 
spazio è pari alla variazione di flusso di campo di induzione magnetica concatenato alle due linee.

$$
v_{AB\gamma} - v_{AB\gamma'} = -\frac{d\Phi_\Gamma}{dt}
$$
In generale il campo elettrico NON è conservativo.

\paragraph{Legge di Ampére-Maxwell} (Induzione magnetoelettrica)
\begin{equation}
\label{eq:legge_ampere_maxwell}
\oint_\Gamma \vec{B}\cdot\hat{t} dl = \mu_0 \iint_{S_\Gamma} \left(\vec{J} + \varepsilon_0\frac{\partial\vec{E}}{\partial t}\right)\cdot \hat{n} dS
\end{equation}

$\mu_0$ è la permeabilità magnentica del vuoto $\mu_0 = 4\pi\cdot10^{-7}\si{\henry\per\meter} $
anche se a partire dalla revisione delle unità di misura del 2019 è una quantità misurata e non più una costante universale (anche se coincide con la precedente definizione per le prime 10 cifre significative)

$$
\frac{1}{\sqrt{\varepsilon_0\mu_0}} = c \simeq 3\cdot10^{8}\ \si{\meter\per\second}
$$

questa legge è la duale della legge di Faraday-Neumann-Lenz. Si può riscrivere come:
$$
\oint_\Gamma\vec{B}\cdot\hat{t}dl = \mu_0 i_{S_\Gamma}(t) + \mu_0 \iint_{S_\Gamma} \frac{\partial}{\partial t}(\varepsilon_0 \vec{E})\cdot\hat{n}dS
$$
Il secondo termine prende il nome di intensità della corrente di spostamento, quindi il termine
$\left(\varepsilon_0 \vec{E}\right)$ è chiamato densità di corrente di spostamento.

Se le linee e le superfici sono ferme:
$$
\oint_\Gamma\vec{B}\cdot\hat{t}dl = \mu_0\iint_{S_\Gamma}\vec{J}\cdot\hat{n}dS + 
\mu_0\frac{d}{dt}\iint_{S_\Gamma} \varepsilon_0 \vec{E}\cdot\hat{n} dS
$$
Ricordando la \ref{eq:legge_gauss} (legge di Gauss per il campo elettrico), 
si vede che il flusso di $\varepsilon_0\vec{E}$ ha le dimensioni di una carica elettrica 
[\si{\coulomb}].


\paragraph{Principio di conservazione della carica}
Combinando linearmente le equazioni \ref{eq:legge_gauss} e \ref{eq:legge_ampere_maxwell} ossia 
la legge di Gauss e quella di Ampére-Maxwell si ottiene la descrizione del principio di conservazione
della carica per un sistema aperto.
Data una superficie chiusa $\Sigma$ contenente una regione $\Omega_\Sigma$ si ottiene

\begin{equation}
\oiint_\Sigma\vec{J}\cdot\hat{n}dS = - \iiint_{\Omega_\Sigma}\frac{\partial \rho}{\partial t} dV
\ \ \forall\Sigma,\ \forall t
\end{equation}

Se le superfici sono ferme
$$
i_{\Sigma}(t) = \oiint_\Sigma\vec{J}\cdot\hat{n}dS = -\frac{d}{dt} \iiint_{\Omega_\Sigma} \rho dV = -\frac{dQ_{\Omega_\Sigma}}{dt}\ \forall\Sigma,\ \forall t
$$

Consideriamo invece il campo di corrente totale, $\vec{J} + \varepsilon_0\frac{\partial\vec{E}}{\partial t}$, con superfici ferme si ha:
$$
\oiint_{\Sigma} \left(\vec{J} + \varepsilon_0\frac{\partial\vec{E}}{\partial t}\right)\cdot\hat{n}dS = 
-\frac{dQ_{\Omega_\Sigma}}{dt} + \frac{d}{dt} \oiint_{\Sigma} \varepsilon_0\vec{E}\cdot\hat{n}dS =
$$
$$
= -\frac{dQ_{\Omega_\Sigma}}{dt} + \varepsilon_0\frac{d}{dt}\left[\frac{Q_{\Omega_\Sigma}}{\varepsilon_0}\right] = 0\ \ \forall\Sigma
$$
Si conclude che il campo $\vec{J} + \varepsilon_0\frac{\partial\vec{E}}{\partial t} $ ossia
il campo di corrente totale è solenoidale.

\newpage
\subsection{Il limite stazionario delle equazioni di Maxwell}
Si ottengono i modelli quasi-stazionari, un'estensione del modello interamente stazionario.
Si ha condizione stazionaria se le grandezze (campi, distribuzioni di cariche e correnti) sono 
costanti nel tempo.
\begin{align*}
&\oiint_{\Sigma}\vec{E}\cdot\hat{n}dS = \frac{Q_{\Omega_\Sigma}}{\varepsilon_0} \ \ \forall \Sigma &
&\oiint_{\Sigma}\vec{B}\cdot\hat{n}dS = 0 \ \ \forall\Sigma \\
&\oint_{\Gamma} \vec{E}\cdot\hat{t}dl = 0 \ \ \forall \Gamma &
&\oint_{\Gamma}\vec{B}\cdot\hat{t} dl = \mu_0 i_{S_\Sigma}\ \ \forall \Sigma,S_\Sigma \\
&\oiint_\Sigma \vec{J}\cdot\hat{n}dS = 0\ \ \forall\Sigma
\end{align*}
Nel limite stazionario $\vec{E}$ e $\vec{B}$ sono disaccoppiati ossia si può determinare
indipendente l'uno dall'altro.

Osservando le equazioni per il campo elettrico e per quello di induzione magnetica si vede 
che sono equazioni lineari con un termine noto diverso da 0.
Sono presenti come termini noti la carica elettrica e le correnti, ossia le sorgenti
per il campo elettrico sono le cariche elettriche, mentre le sorgenti del campo di induzione magnetica
sono le cariche in moto, ossia le correnti elettriche.

Il campo $\vec{E}$ è conservativo per la circuitazione.

Il campo $\vec{J}$ diventa solenoidale, o conservativo per il flusso.

Questo riconduce a
$$
\vec{E} = -\nabla\varphi
$$
è possibile ricavare il campo elettrico utilizzando una sola incognita $(\varphi)$ piuttosto che
conoscere le funzioni di tutte le componenti di $\vec{E}$ nello spazio.

Consideriamo una linea aperta che connetta due punti $A$ e $B$, in condizioni stazionarie, è l'integrale di una forma differenziale esatta

$$
\int_{A\gamma B}\vec{E}\cdot\hat{t}dl \stackrel{\text{def}}{=} v_{AB\gamma} \stackrel{\text{staz.}}{=}
\int_{A\gamma B}-\nabla\varphi\cdot\hat{t}dl =\varphi(A)-\varphi(B)
$$
Ciò equivale a dire che la tensione di due punti dello spazio non dipende dalla linea $\gamma$ ed è
esprimibile come differenza di potenziale attraverso la $\varphi$.

\paragraph{Lavoro della forza elettromagnetica} Sulle cariche in moto in un volume elementare.
Sui portatori di carica agisce la forza di Lorentz
$$
\vec{F} = q\left(\vec{E}+\vec{v}\times\vec{B}\right)
$$

Se si considera il lavoro elementare $\delta L$ compiuto nello spostamento di una singola carica in
$\Delta\Omega$
$$
\delta L = \vec{F}\cdot\hat{t}dl = q\left(\vec{E}+\vec{v}\times\vec{B}\right)\cdot \hat{t} dl = q\vec{E}\cdot\vec{v}dt
$$

Per calcolare il lavoro ottenuto su tutte le particelle
$$
dL = \vec{E}\cdot\left(N^+q_p\vec{v}\ ^+ + N\ ^-q_e\vec{v}\ ^-\right)dt = \vec{E}\cdot\vec{J}\cdot \text{Vol}(\Delta\Omega)dt
$$
Quindi la variazione di lavoro nell'unità di tempo (ossia la potenza?) compiuto sui portatori di carica
per unità di volume è pari a
$$
\frac{1}{\text{Vol}(\Delta\Omega)}\frac{dL}{dt} = \vec{E}\cdot\vec{J} 
$$

Se si considera un materiale conduttore ohmico, ossia che soddisfa la legge di Ohm in forma locale
\begin{align*}
\begin{matrix}
\vec{J} = \gamma\vec{E} \\
\vec{E} = \eta\vec{J}
\end{matrix}\ 
\Rightarrow \vec{E}\cdot\vec{J} = \gamma\left|\vec{E}\right|^2
= \eta\left|\vec{J}\right|^2 \geq 0 \ \forall t
\end{align*}
Questa quantità indica su scala microscopica la dissipazione per effetto Joule in un materiale ohmico
dovuta agli urti tra le cariche in moto e il reticolo metallico che le sorregge, questa energia di
vibrazione viene quindi trasformata in calore.


\subparagraph{Sorgenti elementari dei campi elettromagnetici}
Ci si occupa ora del ``secondo'' blocco del modello Maxwell-Lorentz, si analizzano le ulteriori 
distribuzioni delle sorgenti:

Sia data una struttura con una grandezza preponderante rispetto alle latre due, viene definita 
\textit{trave}.
In elettromagnetismo ha senso considerare distribuzioni di cariche e correnti che abbiano una oppure
due dimensioni prevalenti.

Nel primo caso si hanno densità lineari di carica e corrente.

Nel secondo caso si hanno densità superficiali di carica e corrente.



\subsection{Sorgenti elementari dei campi elettromagnetici}
Ci si occupa ora del ``secondo'' blocco del modello Maxwell-Lorentz, si analizzano le ulteriori 
distribuzioni delle sorgenti:

Sia data una struttura con una grandezza preponderante rispetto alle altre due, viene definita 
\textit{trave}.
In elettromagnetismo ha senso considerare distribuzioni di cariche e correnti che abbiano una oppure
due dimensioni prevalenti.

Nel primo caso si hanno densità lineari di carica e corrente.

Nel secondo caso si hanno densità superficiali di carica e corrente.

Le particelle cariche in una descrizione nell'ambito del modello Maxwell-Lorentz possono 
essere viste come particelle puntiformi.

Le particelle cariche sono in realtà dei limiti ideali in cui la densità volumetrica 
$\rho\to \infty$ quando il volume della regione $\Omega \to 0$.

Preso un punto $P'$ nello spazio in cui si posiziona una carica $q$, questa è associata ad una densità
$\rho(P') = q\delta(P-P')$, con $\delta$ la Delta di Dirac,
la carica $q$ puntiforme può essere vista come:
$$
\iiint_{R^3} \rho dV = \iiint_{R^3} q\delta(P-P')dV = q
$$
Sfruttando la proprietà di campionamento della $[\delta]= \si{\meter^{-3}}$.

Le distribuzioni di cariche puntiformi in moto generano delle correnti e si possono trattare 
analogamente.

\paragraph{Densità di carica lineare}
Si introduce affermando che ha senso considerare delle regioni di spazio chiamate ``strutture''
ossia regioni in cui ci sono una o due dimensioni prevalenti rispetto alle altre.
Una struttura lineare può essere rappresentata come una linea media $\gamma$ attorno alla quale
si sviluppa una regione.

Sia preso l'elemento di linea $dL$ che definisce un ``cilindretto'' centrato attorno
al punto $P'$, si ipotizza che la sezione del cilindro sia pari ad $S$.
La carica $dQ$ contenuta nel cilindretto elementare $d\Omega$ è:

$$
dQ = \rho dV = \rho\ S\ dL
$$

Se si effettua il passaggio al limite per $S\to 0 $ mantenendo la carica $dQ$ finita, allora $\rho$
tenderà all'infinito.
Introduco la grandezza $\rho(P') S(P') = \lambda(P')$ e quindi:

$$
Q  = \int_\gamma \lambda(P') dl\ \ \ [\lambda] = \si{\coulomb/\meter}
$$
\subparagraph{Corrente filiforme (lineare)}

Si descrive con un vettore densità di corrente $J$ calcolato lungo la linea.

Si considera ancora una volta la linea media $\gamma$ e si calcola la corrente con:

$$
i(p') = \iint_\Sigma \vec{J}\cdot\hat{n}dS \simeq \vec{J}(P')\cdot\hat{t}(P')S
$$
Dato che $\hat{t}(P') \simeq \hat{n}(P')$ poiché il moto delle cariche è consentito solo 
all'interno del conduttore.

Il campo $\vec{J}(P') = \frac{i(P')}{S}\hat{t}(P')$ ma
$$
i(P') = \lim_{\stackrel{S\to 0}{J\to\infty}} \iint_\Sigma \vec{J}\cdot\hat{n}dS
$$
si ottiene con un'intensità di corrente non limitata e una sezione che tende a 0 affinché $J\cdot S$ 
resti finita.

\paragraph{Densità di carica superficiale}
Si immagina una struttura di tipo superficie di spessore piccolo ma non nullo, si indica con $S$ la
sua superficie ``media'' che la attraversa.
Intorno ad un punto $P'$ che giace sulla superficie media e si considera un cilindro che ha il
punto al suo centro ed ha altezza pari allo spessore della struttura.
Il cilindretto elementare ha un volume $dV$ con una superficie $dS$ e uno spessore $h(P')$

$$
dV = h(P') dS
$$
quindi
$$
dQ = \rho(P')dV = \rho(P')h(P')dS
$$
Passando al limite per $h(P') \to 0$ mantenendo la carica $dQ$ finita, necessariamente
$|\rho(P')| \to \infty$
Si definisce quindi la densità di carica superficiale $\sigma(P') = \rho(P')h(P')$
Si possono quindi descrivere i fenomeni di carica riferendosi solo alla superficie $S$
come se la carica fosse ``spalmata'' sulla superficie.

Quindi la carica sarà pari a:
$$
Q = \iint_S \sigma(P')dS
$$
\newpage
\subparagraph{Densità di corrente superficiale}
Permette di trattare correnti definite su superfici bidimensionali, anche curve ma descrivibili
mediante rappresentazione parametrica di 2 parametri.

Si suppone di avere una superficie identica alla precedente che compone la superficie media
di una struttura con un certo spessore. Nel punto $P' \in S$ si delimita una superficie di controllo
centrata attorno a questo punto, perpendicolare al piano di $S$ con normale $\hat{n}$, l'area
di questa superficie è $dS$, la lunghezza della linea appartenente al piano $S$ che interseca tale
superficie è $dl$, l'altezza è $h$. Si ripete il procedimento precedente.
$$
i(P') = \vec{J}(P')\cdot\hat{n}dS = \vec{J}(P')\cdot\hat{n}h(P')dl = \vec{J}(P')h(P')\cdot\hat{n}dl
$$
Il prodotto $\vec{J}(P')h(P')$ viene definito come densità di corrente superficiale 
$$
J_S(P')\stackrel{\text{def}}{=}\lim_{\stackrel{h(P')\to0}{|hJ|<\infty}} [\vec{J}(P')h(P')]\ \ \ [Js] = \si{\ampere/\meter}
$$
La densità di corrente superficiale è quindi un campo che giace nella superficie $S$.
Se si vuole calcolare l'intensità di corrente attraverso una linea $\gamma$ che giace 
sulla superficie $S$
$$
i_\gamma = \int_\gamma \vec{J_s}\cdot \hat{n} dl
$$

\subsection{Equazioni di Maxwell in forma locale (nel vuoto)}
Le equazioni in forma integrale sono potenti nelle analisi di strutture con particolari simmetrie.
(Ad esempio sferica, cilindrica, piana ecc...)

In condizioni generiche però risolvere equazioni integrali è più complesso che risolvere problemi di 
valori al contorno per equazioni differenziali a derivate parziali (PDE).
Le ultime si prestano inoltre a risoluzioni numeriche mediante software di calcolo.

È richiesta particolare attenzione se si parla di formulazione locale quando sono presenti
distribuzioni delle sorgenti singolari, ossia in cui le densità di carica o corrente tendono 
all'infinito.

\paragraph{Legge di Gauss}
Si considera un dominio $\Omega$ generalmente regolare e osservando l'intorno di
un punto $P$ interno ad $\Omega$ è costruito un volumetto elementare $\Delta\Omega$.
Sia $\Delta\Sigma = \partial\Delta\Omega$ la frontiera di $\Delta\Omega$.

Considerata la \textbf{Legge di Gauss} applicata ad una $\Delta\Sigma$ intorno al punto $P$
$$
\oiint_{\Delta\Sigma}\vec{E}\cdot\hat{n}dS = \frac{1}{\varepsilon_0} \iiint_{\Delta\Omega}\rho dV\ 
\forall\ \Delta\Sigma \text{ intorno a } P
$$
Dato che il volume $\Delta\Omega$ è infinitesimo si può scrivere la precedente approssimando il secondo
membro:
$$
\oiint_{\Delta\Sigma}\vec{E}\cdot\hat{n}dS \simeq \frac{\rho(P)}{\varepsilon_0}\text{Vol}(\Delta\Omega)
$$
si dividono ambo i membri per il volume di $\Delta\Omega$ e si ottiene:
$$
\frac{\oiint_{\Delta\Sigma}\vec{E}\cdot\hat{n}dS}{\text{Vol}(\Delta\Omega)} \simeq 
\frac{\rho(P)}{\varepsilon_0}
$$
Il secondo termine è una proprietà locale del campo, per adattare il termine di sinistra si esegue
il limite sul volume mantenendo $\Delta\Omega$ intorno a $P$
$$
\lim_{\text{Vol}(\Delta\Omega\to 0)} \frac{\oiint_{\Delta\Sigma}\vec{E}\cdot\hat{n}dS}{\text{Vol}(\Delta\Omega)} = \text{div}\vec{E}(P) = \nabla\cdot\vec{E}(P) = \frac{\rho(P)}{\varepsilon_0}\ \forall
\ P \in \Omega
$$
$P$ deve essere un punto \textit{regolare}, ossia il campo deve essere continuo in $P$.
La Legge di Gauss si può esprimere dunque con:
\begin{equation}
 \nabla\cdot\vec{E}(P) = \frac{\rho(P)}{\varepsilon_0}
 \label{eq:legge_gauss_locale}
\end{equation}
Si osserva che nei punti in cui la densità è nulla, il campo elettrico è \textit{indivergente} e
può diventare solenoidale se la regione è semplicemente connessa.

Si considera il caso in cui il volume $\Delta\Omega$ si trova in una zona di discontinuità
di $\vec{E}$, non è possibile definire la divergenza.
Si considera un cilindretto che contiene il punto $P$ e attraversa la superficie $S$ di discontinuità.
L'area di base è $A$ e lo spessore è pari ad $h$.
Le due superfici piane vengono chiamate $S_1$ ed $S_2$ ed hanno entrambe area pari ad $A$.
Si applica la Legge di Gauss al volume $\Delta\Omega$ la cui frontiera è appunto la superficie del 
cilindro.
$$
\oiint_{\partial\Delta\Omega} \vec{E}\cdot\hat{n}dS = \frac{1}{\varepsilon_0}Q_{\Delta\Omega} \Leftrightarrow \vec{E}_2\cdot\hat{n}_2A + \vec{E}_1\cdot\hat{n}_1A + 
\iint_{S_{\text{lat}}}\vec{E}\cdot\hat{n}dS
$$
allora
$$
\vec{E}_2\cdot\hat{n}_2A + \vec{E}_1\cdot\hat{n}_1A + 
\iint_{S_{\text{lat}}}\vec{E}\cdot\hat{n}dS = \frac{1}{\varepsilon_0} \sigma(P)\cdot A
$$
dove $\sigma(P)\cdot  A $ è la quantità di carica presente sulla traccia di $\Delta\Omega$ su $S$.

Si esegue il limite per $h(P)\to 0$ le normali delle due superfici saranno:
$$
\begin{matrix}
\hat{n}_2 & \to& \hat{n}(P) \\
\hat{n}_1 & \to& -\hat{n}(P)
\end{matrix}
\Rightarrow \vec{E}_2\cdot\hat{n}\cancel{A} - \vec{E}_1\cdot\hat{n}\cancel{A} = \frac{\sigma}{\varepsilon_0}\cancel{A} \ \ \forall A \Rightarrow
$$
$$
\Rightarrow \hat{n}\cdot(\vec{E}_2-\vec{E}_1) = \frac{\sigma}{\varepsilon_0}
$$

La discontinuità del campo in questo caso è associata alla sua componente normale e nella misura 
in cui il salto di discontinuità tra le due componenti all'interno o l'esterno della superficie sia 
pari a $\frac{\sigma}{\varepsilon}_0$.
Formalizzando la Legge di Gauss:
\begin{align*}
\nabla\cdot\vec{E} &= \frac{\rho}{\varepsilon_0}\text{ nei punti regolari}\\
\hat{n}\cdot(\vec{E}_2-\vec{E}_1) &= \frac{\sigma}{\varepsilon_0}
\text{ sulle superfici di discontinuità}
\end{align*}
Con $E_2$ ed $E_1$ il valore dei campi sopra e sotto la superficie $S$ di discontinuità.

\paragraph{La Legge di Gauss per il campo $\vec{B}$ (in forma locale)}
Sia il volume $\Delta\Omega : \Delta\Sigma = \partial\Delta\Omega$ allora
\begin{equation}
\oiint_{\Delta\Sigma}\vec{B}\cdot\hat{n}dS = 0 \ \ \forall\ \Delta \Sigma \Rightarrow
\begin{cases}
\nabla\cdot\vec{B} &= 0\text{ nei punti regolari}\\
\hat{n}\cdot(\vec{B}_2-\vec{B}_1) &= 0 \text{ sulle superfici di discontinuità}
\end{cases}
\end{equation}

la componente normale di $\vec{B}$ è sempre continua, se eventualmente sono presenti discontinuità,
queste riguardano la componente tangenziale alla superficie.

\subparagraph{Legge di Faraday-Neumann-Lenz}
Si considera un punto $P\in\Omega$ e si utilizza una linea chiusa $\Gamma$ intorno al punto $P$.

Si supponga inoltre che $P$ sia un punto regolare, applicando la legge di Faraday-Neumann:
$$
\oint_{\Gamma}\vec{E}\cdot\hat{t}dl = 
- \iint_{S_\Gamma} \frac{\partial\vec{B}}{\partial t}\cdot\hat{n}dS
$$
essendo l'area di $S_\Gamma$ infinitesima si può avere l'integrale nella seguente maniera:
$$
\oint_{\Gamma}\vec{E}\cdot\hat{t}dL \simeq - \frac{\partial\vec{B}}{\partial t} \cdot 
\hat{n}\text{ Area}(S_\Gamma)\Rightarrow
$$

$$
\Rightarrow
\lim_{\text{Area}(S_\Gamma)\to 0}
 \frac{\oint_{\Gamma}\vec{E}\cdot\hat{t}dL}{\text{Area}(S_\Gamma)} \simeq
- \frac{\partial\vec{B}}{\partial t}\cdot\hat{n}
$$
equivalentemente
$$
\hat{n}\cdot\nabla\times\vec{E}(P) = -\frac{\partial\vec{B}}{\partial t}\cdot\hat{n} 
\Leftrightarrow \hat{n}\cdot\left(\nabla\times\vec{E}+\frac{\partial\vec{B}}{\partial t}\right) = 0\ \ 
\forall \Gamma,S_\gamma\Rightarrow \forall \hat{n}
$$
quindi
\begin{equation}
\nabla\times\vec{E}(P) = -\frac{\partial\vec{B}}{\partial t} \text{ nei punti $P$ regolari}
\end{equation}

Si considera una superficie di discontinuità (per $\vec{E}$), si applica la Legge di 
Faraday-Neumann-Lenz a particolari linee.
Presa una superficie di discontinuità $S$ con normale $\hat{n}$ nel punto $P$, si racchiude una linea
$\Gamma$ rettangolare che attraversa la superficie, con due tratti paralleli alla superficie e due
perpendicolari. La linea chiude una superficie $S_\Gamma$ dotata anch'essa di versore normale $\hat{m}$
orientato anch'esso mediante la regola della mano destra.

I tratti superiore e inferiore hanno lunghezza $L$ e i tratti verticali altezza $h$.

$$
\oint_{\Gamma}\vec{E}\cdot\hat{t}dl \simeq \vec{E}_1 \cdot\hat{t}_1 L + \vec{E}_2 \cdot\hat{t}_2 L +
\int_B^C \vec{E}\cdot\hat{t} dl + \int_D^A\vec{E}\cdot\hat{t}dl
$$
I due integrali sono eseguiti lungo le curve perpendicolari alla superficie $S$ e variano linearmente 
con $h$.
La legge di Faraday-Neumann afferma che:
$$
\oint_{\Gamma}\vec{E}\cdot\hat{t}dl = 
-\iint_{S_\Gamma} \frac{\partial\vec{B}}{\partial t}\cdot \hat{m}dS \simeq 
- \frac{\partial\vec{B}}{\partial t} \cdot \hat{m} h L
$$
unendo i termini:
$$
\vec{E}_1 \cdot\hat{t}_1 \cancel{L} + \vec{E}_2 \cdot\hat{t}_2 \cancel{L} +
\frac{1}{L}\int_B^C \vec{E}\cdot\hat{t} dl + \frac{1}{L}\int_D^A\vec{E}\cdot\hat{t}dl
= - \frac{\partial \vec{B}}{\partial t}\cdot \hat{m} h \cancel{L}
$$
si esegue il limite per $h/L \to 0$ in modo che l'altezza tenda a 0 più rapidamente rispetto
alla base.

Il versore tangente $\hat{t}_1 \to \hat{t},\ \hat{t}_2\to -\hat{t}$, restano quindi:
$$
\hat{t}\cdot(\vec{E}_1\cdot\vec{E}_2) = 0
$$
Si conclude quindi dimostrando che la componente tangenziale di $\vec{E}$ è sempre continua.

Qualunque versore tangenziale ad una superficie è sicuramente perpendicolare alla normale della
superficie ossia $\hat{t}\cdot\hat{n} = 0$ quindi $\hat{t} = \hat{n}\times\hat{m}$.

Sostituendo 
$$
\hat{n}\times\hat{m}\cdot(\vec{E}_1-\vec{E}_2) = 0
$$
sfruttando la proprietà di scorrimento circolare del prodotto misto 
$$
\hat{n}\times(\vec{E}_2-\vec{E}_1)\cdot\hat{n} = 0 \ \ \forall\ \hat{m} \Rightarrow
\hat{n}\times(\vec{E}_2-\vec{E}_1) = 0
$$
La continuità della componente tangenziale di $\vec{E}$ ad una superficie si scrive in termini
del versore normale della superficie di discontinuità.

\paragraph{Legge di Ampére-Maxwell (in forma locale)}
Riferendosi ad una linea $\Gamma$ che circonda un punto $P$ nella superficie $S_\Gamma$
$$
\oint_\Gamma\vec{B}\cdot\hat{t}dl = \mu_0 \iint_{S_\Gamma} \left(\vec{J}+\varepsilon_0\frac{\partial \vec{E}}{\partial t}\right)\cdot \hat{m} dS
$$
allora
$$
\lim_{\text{Area}(S_\Gamma)\to 0} \frac{\oint_\Gamma \vec{B}\cdot\hat{t}dl}{\text{Area}(S_\Gamma)} =
\hat{m}\cdot\nabla\times\vec{B}(P) = \mu_0 \left(\vec{J}(P)+\varepsilon_0\frac{\partial\vec{E}}{\partial t}(P)\right)\cdot\hat{m}\ \ \forall\ \hat{m}
$$
Ciò significa che nei punti regolari la legge di Ampére-Maxwell si scrive
\begin{equation}
 \nabla\times\vec{B} = \mu_0\left(\vec{J}+\varepsilon_0\frac{\partial\vec{E}}{\partial t}\right)
\end{equation}



\paragraph{Legge di Ampére-Maxwell (locale) su superfici di discontinuità}
Presa una superficie $S$ di normale $\hat{n}$ e indichiamo la regione al di sopra di $S$ con
(2) e quella al di sotto con (1).
Si considera una linea chiusa $\Gamma$ orientata di tipo \textit{rettangolare} ortogonale alla
superficie, e la superficie orlante è chiamata $S_\Gamma$ con versore normale $\hat{m}$ orientato 
secondo la regola della mano destra.

$$
\oint_\Gamma \vec{B}\cdot\hat{t} dl = \mu_0 i_{S_\Gamma} + \mu_0 \iint_{S_\Gamma} \varepsilon_0 \frac{\partial\vec{E}}{\partial t}\cdot\hat{m}dS 
$$
$i_{S_\Gamma}$ sono le correnti concatenate alla superficie $S_\Gamma$.
Ricordando che il tratto di linea elementare ha una lunghezza $L$ e un'altezza $h$, 
il vettore densità di corrente superficiale $\vec{J}_S$ e $\vec{J}$ un'eventuale
densità di corrente volumetrica, allora
$$
\oint_\Gamma\vec{B}\cdot\hat{t}dl = \vec{B}_1\cdot\hat{t}_1L + \vec{B}_2\cdot\hat{t}_2L +
\int_B^C\vec{B}\cdot\hat{t}dl + \int_D^A\vec{B}\cdot\hat{t}dl = \mu_0\vec{J}_S\cdot\hat{m} + 
\mu_0\varepsilon_0\frac{\partial\vec{E}}{\partial t}\cdot\hat{m}hL + \mu_0 \vec{J}\cdot\hat{m}hL
$$

passando al limite per $h/l \to 0,\ \hat{t}_1 = \hat{t} = \hat{t}_2,\ \hat{t} = \hat{n}\times\hat{m}$

$$
\hat{n}\times\hat{m}\cdot(\vec{B}_1-\vec{B}_2) = \mu_0 \vec{J}_S \cdot \hat{m} \Leftrightarrow 
\hat{n}\times(\vec{B}_2-\vec{B}_1)\cdot\hat{m} = \mu_0 \vec{J}\cdot\hat{m}\ 
\forall\ \hat{m},\Gamma,S_\Gamma
$$

$$
\hat{n}\times(\vec{B}_2-\vec{B}_1) = \mu_0\vec{J}_S\ \text{su} \ S
$$

Si conclude quindi che la componente tangenziale del campo $\vec{B}$ ad una superficie $S$ è discontinua
in presenza di correnti superficiali.

\paragraph{Principio di conservazione della carica}
Sia preso un volumetto $\Delta\Omega$ centrato attorno ad un punto $P$
$$
\frac{\oiint_{\partial\Delta\Omega} \vec{J}\cdot\hat{n} dS}{\text{Vol}(\Delta\Omega)} = -\frac{ \iiint_{\Delta\Omega} \frac{\partial\rho}{\partial t} dV}{\text{Vol}(\Delta\Omega)} \stackrel{\text{Vol}(\Delta\Omega)\to 0} {\Rightarrow}
\nabla\cdot\vec{J} = -\frac{\partial\rho}{\partial t} \in \Omega
$$
nei punti regolari.

Sulle superfici di discontinuità invece ripetendo i ragionamenti sulla superficie \textit{``monetina''}
si ottiene:
$$
\hat{n}\cdot\left(\vec{J}_2-\vec{J}_1\right) = -\frac{\partial\sigma}{\partial t} \in S
$$
si trascurano infatti i flussi laterali che tendono a 0 se $h\to 0$.

La componente normale del vettore densità di corrente è discontinua se su $S$ è presente una densità
di carica variabile nel tempo.

\paragraph{Sintesi delle equazioni di Maxwell in forma locale} 
\begin{align*}
&\text{Nei punti}\text{ regolari}  & &\text{Nei punti}\text{ irreg}\text{olari}\\
&\nabla\cdot\vec{E} = \frac{\rho}{\varepsilon_0} & &\hat{n}\cdot(\vec{E}_2-\vec{E}_1) = \frac{\sigma}{\varepsilon_0}\\
&\nabla\cdot\vec{B} = 0 & &\hat{n}\cdot(\vec{B}_2-\vec{B}_1) = 0\\
&\nabla\times\vec{E} = -\frac{\partial\vec{B}}{\partial t}& &\hat{n}\times(\vec{E}_2-\vec{E}_1) = 0\\
&\nabla\times\vec{B} = \mu_0\left(\vec{J} + \varepsilon_0\frac{\partial\vec{E}}{\partial t}\right)& &\hat{n}\times(\vec{B}_2-\vec{B}_1) = \mu_0\vec{J}_S \\
&\nabla\cdot\vec{J} = -\frac{\partial\rho}{\partial t} & &\hat{n}\cdot(\vec{J}_2-\vec{J}_1) = -\frac{\partial \sigma}{\partial t}
\end{align*}

\section{Elettrostatica}
Tutte le cariche sono ferme e invariabili nel tempo
$$
\oiint_\Sigma \vec{E}\cdot\hat{n}dS = \frac{Q_{\Omega_\Sigma}}{\varepsilon_0}\ \forall\Sigma\ \text{Legge di Gauss}
$$
$$
\oint_\Gamma \vec{E}\cdot\hat{t}dl = 0 \ \forall\Gamma\ \text{Legge di Faraday-Neumann}
$$
Si vuole vedere ora come ricavare le configurazioni di campo elettrico associati a distribuzioni
di cariche assegnate, utilizzando le equazioni di Maxwell in forma integrale.

\subsection{Distribuzione di carica volumetrica a simmetria sferica}

Sia $\Omega$ la sfera di raggio $R$, un punto $P(r,\theta,\varphi)$ al suo interno.

La densità di carica è quindi:
$$
\rho(P) = 
\begin{cases}
\rho_0,\ 0\leq r\leq R\\
0,\ r > R
\end{cases}
$$

Il campo elettrico in generale dipenderà da $r,\theta$ e $\varphi$ e si può scomporre in 
tre componenti lungo gli assi di queste variabili.

La distribuzione $\rho$ non dipenderà da $\theta,\varphi$, ossia sarà simmetrica rispetto agli assi.
Di conseguenza anche le componenti del campo $\vec{E}$ dipenderanno solo da $r$.
Le componenti del campo $E_\theta$ ed $E_\varphi$ seguiranno le rotazioni dei rispettivi angoli.
Supponendo che ci sia una componente $E_\theta$ diversa da 0 e si suppone di ruotare la sfera di
180\textdegree\ la componente cambierebbe di segno ma rimarrebbe invariata la distribuzione di cariche $\rho$,
ne consegue che $E_\theta = 0$; un discorso analogo può essere effettuato per la componente $E_\varphi$.

In definitiva si conclude che l'unica componente del campo elettrico è quella lungo 
il versore $r$ quindi $E_r(r)$.
Si può applicare la legge di Gauss ad una superficie sferica $\Sigma$ esterna di raggio maggiore del raggio
della sfera carica.

Ricordando che:
\begin{align*}
dl_1 &= dr \\
dl_2 &= rd\theta \\
dl_3 &= r\sin\theta d\varphi \\
dS &= dS_1 = dl_2dl_3 = r^2\sin\theta d\theta d \varphi
\end{align*}
$$
\oiint_\Sigma\vec{E}\cdot\hat{n}dS = \int_0^\pi d\theta \int_0^\pi d\varphi r^2 \sin\theta E_r = 
\frac{1}{\varepsilon_0} \int_0^\pi d\theta \int_0^\pi d\varphi \int_0^r \rho(r) r^3\sin\theta dr =
$$
$$
= \frac{1}{\varepsilon_0}\iiint_{\Omega_\Sigma} \rho d V = 2\pi r^2 \left[- \cos \theta\right]_0 ^\pi 
E_r (r) = 
\begin{cases}
  2 \pi [-\cos\theta]_0^\pi \frac{\rho_0}{3\varepsilon_0}r^3 & r<R \\
  2 \pi [-\cos \theta]_0^\pi \frac{\rho_0}{3\varepsilon_0}R^3 & r \geq R
\end{cases} =
$$
$$
= 4 \pi r^2 E_r(r) = \begin{cases}
\frac{4\pi r^3 \rho_0}{3\varepsilon_0}, & r < R \\
\frac{4\pi R^3 \rho_0}{3\varepsilon_0}, & r \geq R
\end{cases}
$$
$$
E_r(r) = \frac{\rho_0 r}{3 \varepsilon_0}, \ \ r < R
$$
$$
E_r(r) = \frac{\rho_0 R^3}{3 \varepsilon_0}\frac{1}{r^2}, \ \ r \geq R
$$
\begin{center} % plot di un grafico 2D andamento del campo elettrico
\begin{tikzpicture}
\begin{axis}[
axis lines = left,
xlabel = $r/R$,
ylabel = $\frac{E_r}{E_R}(r)$,
ymax = 3,
]
\addplot [
domain = 0:1,
samples = 2,
] {x};
\addplot[
 domain = 1:3,
 samples = 50,
] {x^-2};
\end{axis}
\end{tikzpicture}
\end{center}
Si osserva che se $r\geq R$ 
$$
\vec{E}(r) = \frac{Q}{4\pi\varepsilon_0} \frac{1}{r^2} \vec{e}_r,\ \ Q = \frac{4\pi }{3}R^3\rho_0
$$
Come se la carica $Q$ fosse puntiforme nell'origine.
Facendo tendere $R\to 0 $ e $\rho_0 \to \infty$ in modo da mantenere $Q$ finito, si ottiene
il campo della carica puntiforme.

\paragraph{Campo elettrico prodotto da una distribuzione di carica qualunque}
Sia $\rho(p')dV$ la carica elementare in $\Delta\Omega$, volumetto infinitesimo della regione $\Omega$.
Si prenda un punto $p$ all'esterno della regione, $\vec{r}_p$ e $\vec{r}_{p'}$ sono i rispettivi vettori
che puntano a $p$ e $p'$.
Si calcola il campo elettrico emesso dalla carica puntiforme in $p$.
$$
\vec{dE}(p) = \frac{\rho(p')dV}{4\pi\varepsilon_0}\cdot\frac{1}{|\vec{r}_p-\vec{r}_{p'}|^2} \cdot\frac{\vec{r}_p-\vec{r}_{p'}}{|\vec{r}_p-\vec{r}_{p'}|}
$$
L'ultimo termine è il versore diretto da $p'$ a $p$ indicando il verso del campo elettrico.
In forma più compatta:
$$
\vec{dE}(p) = \frac{\rho(p')dV}{4\pi \varepsilon_0} \frac{\vec{r}_p-\vec{r}_{p'}}{|\vec{r}_p-\vec{r}_{p'}|^3}
$$
Per calcolare il campo generato dall'intera regione $\Omega$:
$$
\vec{E}(p) = \frac{1}{4\pi\varepsilon_0} \iiint_\Omega \rho(p')  \frac{\vec{r}_p-\vec{r}_{p'}}{|\vec{r}_p-\vec{r}_{p'}|^3} dV
$$
L'espressione è valida sia se il punto $p$ è interno ad $\Omega$ o meno. Bisogna prestare attenzione
al caso in cui $p=p'$ dato che la funzione integranda risulta singolare ma resta finito l'integrale
perchè esteso ad un volume.
Se la regione $\Omega$ è al finito $(\text{diam}(\Omega)<\infty)$ si dimostra che 
$$
\lim_{|\vec{r}_p| \to \infty} |\vec{E}(p)| = 0,\ \ \vec{E}(p) \propto \frac{1}{|\vec{r}_p|^2}
$$
È semplice da osservare:
$$
|\vec{r}_p| \gg |\vec{r}_{p'}| \Rightarrow |\vec{r}_p -\vec{r}_{p'}| \simeq |\vec{r}_p| \Rightarrow
\vec{E}(p) \simeq \frac{1}{4 \pi \varepsilon_0}\frac{Q}{|\vec{r}_p|^2}
$$

\newpage
\subsection{Distribuzione di carica con simmetria cilindrica}
Indefinita lungo $z$ e di raggio $R$.
Ogni punto $P= (r,\varphi,z)$ definito in coordinate cilindriche, la densità di carica $\rho(p)$:
$$
\rho(p) = \begin{cases}
\rho_0 & \text{se }  r \leq R\\
0 & \text{se }  r > R
\end{cases}
$$
Simmetrica di rotazione intorno l'asse $z$ ed è invariante per traslazione lungo l'asse $z$ (o $l_3$).

Per questo motivo il campo $\vec{E}$ dipende ancora da una sola componente, $\vec{E}(p) = \vec{E}(r)$
mentre $E_\varphi = E_z = 0$. $E_\varphi$ è nulla perchè una rotazione di 180\textdegree\ attorno
l'asse $z$ invertirebbe il segno del campo lasciando $\rho$ invariata; $E_z = 0$ perchè una rotazione
attorno a $\vec{e}_r$ invertirebbe il segno di $E_z$ lasciando $\rho$ invariata.
Queste affermazioni sono contraddittorie e l'unico modo per validarle è imporre che le componenti
siano nulle. In conclusione:
$$
\vec{E}(p) = E_r(r)\vec{e}_r
$$
Si ricorda che
$$
dS = dS_1 = dl_2 dl_3 = rd\varphi dz
$$
Applicando la legge di Gauss ad una superficie cilindrica con distribuzione di raggio $R$ e distribuzione $\rho$ mediante un cilindro esterno o interno di raggio $r$:
$$
\oiint_\Sigma\vec{E}\cdot\hat{n}dS = \int_0^{2\pi}rd\varphi \int_0^L dz E_r(r) = \frac{1}{\varepsilon_0}
\iiint_{\Omega_\Sigma}\rho dV = \frac{1}{\varepsilon_0} \int_{0}^{r}dr\int_{0}^{2\pi} rd\varphi\int_0^L \rho_0 dz 
$$
$$
\oiint_\Sigma\vec{E}\cdot\hat{n}dS = 2\pi L E_r(r) r =
\begin{cases}
\frac{1}{\varepsilon_0} 2 \pi L \frac{r^3}{2}\rho_0, & r < R \Rightarrow E_r(r) = \frac{\rho_0}{2\varepsilon_0}r\\
\frac{1}{\varepsilon_0} 2 \pi L \frac{R^2}{2}\rho_0, & r\geq R \Rightarrow R_r(r) = \frac{\rho_0}{2\varepsilon_0} \frac{R^2}{r}
\end{cases}
$$
Il diagramma del campo:
\begin{center} % plot di un grafico 2D andamento del campo elettrico
\begin{tikzpicture}
\begin{axis}[
axis lines = left,
xlabel = $r/R$,
ylabel = $\frac{E_r}{E_R}(r)$,
ymax = 2,
]
\addplot [
domain = 0:1,
samples = 2,
] {x};
\addplot[
 domain = 1:3,
 samples = 50,
] {x^-1};
\node[] at (axis cs: .5,1.15) {$E_R = \frac{\rho_0}{2\varepsilon_0}R$};
\node[] at (axis cs: 2,.7) {$\sim\frac{1}{r}$};
\addplot[
 domain = 0:1,
 samples = 10,
 dashed,
] {1};
\end{axis}
\end{tikzpicture}
\end{center}

Quando $r > R$ 
$$
E_r(r) = \frac{\rho_0}{2\varepsilon_0} \frac{\pi R^2}{\pi} \frac{1}{r} = 
\frac{\lambda}{2\pi\varepsilon_0} \frac{1}{r}
$$
Essendo $\pi R^2$ l'area trasversa del cilindro che moltiplicata per $\rho_0$ fornisce una densità di
carica lineare $\lambda = \rho_0 \pi R^2$. All'allontanarsi dalla distribuzione cilindrica, il campo si
comporta come quello generato da una distribuzione filiforme.

\subsection{Distribuzione di carica a simmetria piana}
Sia $\Delta$ lo spessore della lastra piana indefinita lungo le direzioni $l_1$ ed $l_2$.
Nella lastra è presente una densità di carica:
$$
\rho(p) = \rho_0, \ \  -\frac{\Delta}{2} \leq z \leq \frac{\Delta}{2}
$$
$\rho(p)$ è simmetrica rispetto al piano $(x,y)$ ed è invariante per traslazione lungo $x$ e $y$.
Si vede dunque che $\vec{E}(z) = E_z(z) = - E_z(-z)$ ossia le componenti godono della 
\textit{mirror simmetry} ossia la funzione è dispari.
Anche in questo caso le altre due componenti si vedono essere nulle dato che con eventuali
rotazioni attorno all'asse $z$ si invertirebbe il segno delle componenti senza variare
la distribuzione di carica. ($E_x=E_y=0$)

In conclusione:
$$
\vec{E}(p) = E_z(z)\vec{e}_z
$$

Si prenda una superficie $\Sigma$ di un parallelepipedo con facce parallele
ai piani coordinati, posto simmetricamente rispetto alla lastra, ipotizzando che lo spessore del 
parallelepipedo sia $2z$, le facce $A_1$ e $A_2$ appartenenti al parallelepipedo e parallele
alla superficie carica si troveranno a quota $z$ e $-z$.

Si applica la legge di Gauss alla superficie appena determinata:
$$
\oiint_\Sigma \vec{E}\cdot\hat{n}dS = \iint_{A_1} dxdy E_z(z)\hat{n}\cdot\vec{e}_z + 
\iint_{A_2} dxdy E_z(-z)\hat{n}\cdot\vec{e}_z = SE_z(z) -SE_z(-z) = 
$$
$$
=2 S E_z(z) =\frac{1}{\varepsilon_0} \iiint_{\Omega_\Sigma} \rho dV = \frac{1}{\varepsilon_0} \int_{-z}^{z}dz \iint_A dxdy\rho_0
$$
$$
2SE_z(z) = \begin{cases}
\frac{\rho_0}{\varepsilon_0} 2z S, & -\frac{\Delta}{2} \leq z \leq \frac{\Delta}{2} \Rightarrow E_z(z)  
= \frac{\rho_0}{\varepsilon_0}z,\ \ |z|\leq \frac{\Delta}{2} \\
\pm \frac{\rho_0}{\varepsilon_0} 2 \frac{\Delta}{2}S, & |z| > \frac{\Delta}{2} \Rightarrow E_z(z) =
\begin{cases}
\frac{\rho_0}{\varepsilon_0}\frac{\Delta}{2}, & z \geq \frac{\Delta}{2}\\
-\frac{\rho_0}{\varepsilon_0}\frac{\Delta}{2}, & z \leq -\frac{\Delta}{2}
\end{cases}
\end{cases}
$$
\begin{figure}[h] % plot di un grafico 2D andamento del campo elettrico
\centering
\begin{tikzpicture}
\begin{axis}[
%axis lines = left,
axis x line = center,
axis y line = middle,
xlabel = $z$,
ylabel = $E_z$,
ymax = 1.5,
ymin = -1.5,
xtick = {-2,-1,0,1,2},
xticklabels = {, ,$0$,$\frac{\Delta}{2}$},
ytick = {-2,-1,0,1,2},
yticklabels = { , , , $\frac{\rho_0\Delta}{2\varepsilon_0}$, }
]
\addplot [
domain = -2:-1,
samples = 2,
]{-1};
\addplot [
domain = -1:1,
samples = 2,
] {x};
\addplot [
domain = 1:2,
samples = 2,
]{1};

\addplot[
 domain = 0:1,
 samples = 10,
 dashed,
] {1};
\addplot [dashed] coordinates {(1,0) (1,1)};
\addplot [dashed] coordinates {(-1,0) (-1,-1)};
\addplot [dashed] coordinates {(0,-1) (-1,-1)};
\node[] at (axis cs: 0.4,-1) {$-\frac{\rho_0\Delta}{2\varepsilon_0}$};
\node[] at (axis cs: -1,0.2) {$-\frac{\Delta}{2}$};
\end{axis}
\end{tikzpicture}
\caption{Andamento del campo attraverso il piano}
\label{fig:campo_piano_solido}
\end{figure}

Definendo $\sigma = \rho_0\Delta$ si può esprimere il campo come:
$$
\vec{E}(z) = \begin{cases}
\frac{\sigma}{2\varepsilon_0}\vec{e}_z, & z> \frac{\Delta}{2}\\
-\frac{\sigma}{2\varepsilon_0}\vec{e}_z, & z < -\frac{\Delta}{2}
\end{cases}
$$

Passando al limite invece per $\Delta \to 0$ mantenendo $\sigma\Delta$ finito si ottiene
un campo elettrico discontinuo in figura \ref{fig:campo_piano_lastra}, proveniente dal limite di
quello in figura \ref{fig:campo_piano_solido}. Il salto di discontinuità è pari a 
$\frac{\sigma}{\varepsilon_0}$ chiamato strato semplice.

\begin{figure}[h] % plot di un grafico 2D andamento del campo elettrico
\centering
\begin{tikzpicture}
\begin{axis}[
%axis lines = left,
axis x line = center,
axis y line = middle,
xlabel = $z$,
ylabel = $E_z$,
ymax = 1.5,
ymin = -1.5,
xtick = {-2,-1,0,1,2},
xticklabels = {, ,$0$, ,},
ytick = {-2,-1,0,1,2},
yticklabels = { , , , $\frac{\sigma}{2\varepsilon_0}$, }
]
\addplot [
domain = -2:0,
samples = 2,
]{-1};

\addplot [
domain = 0:2,
samples = 2,
]{1};

\addplot[
 domain = 0:1,
 samples = 10,
 dashed,
] {1};

\node[] at (axis cs: 0.4,-1) {$-\frac{\sigma}{2\varepsilon_0}$};
\end{axis}
\end{tikzpicture}
\caption{Andamento del campo attraverso la lastra infinitesima}
\label{fig:campo_piano_lastra}
\end{figure}


\paragraph{Campo elettrico prodotto da due lastre parallele}

\begin{figure}[h] % plot di un grafico 2D andamento del campo elettrico
\centering
\begin{tikzpicture}
\begin{axis}[
%axis lines = left,
axis x line = center,
axis y line = middle,
xlabel = $E$,
ylabel = $x$,
ymax = 1.5,
ymin = -1.5,
xmax = 2,
xmin = -2,
xtick = {-2,-1,0,1,2},
xticklabels = {, , ,},
ytick = {-2,-1,0,1,2},
yticklabels = { , , , , }
]

\addplot [teal,thick] coordinates {(1,-1) (1,1)};
\addplot [teal,thick,dashed] coordinates {(1,1) (1,1.3)};
\addplot [teal,thick,dashed] coordinates {(1,-1) (1,-1.3)};

\addplot [red,thick] coordinates {(-1,-1) (-1,1)};
\addplot [red,thick,dashed] coordinates {(-1,-1) (-1,-1.3)};
\addplot [red,thick,dashed] coordinates {(-1,1) (-1,1.3)};

\addplot [->,teal] coordinates {(1.1,0.8) (1.4,0.8)};
\addplot [->,teal] coordinates {(0.9,0.8) (0.6,0.8)};
\node [teal] at (axis cs: 1.3,1) {$\frac{\sigma}{2\varepsilon_0}$};
\node [teal] at (axis cs: 0.7,1) {$-\frac{\sigma}{2\varepsilon_0}$};

\addplot [->,teal] coordinates {(-0.3,0.8) (-0.6,0.8)};
\node [teal] at (axis cs: -0.5,1) {$-\frac{\sigma}{2\varepsilon_0}$};

\addplot [->,teal] coordinates {(-1.1,0.8) (-1.4,0.8)};
\node [teal] at (axis cs: -1.3,1) {$-\frac{\sigma}{2\varepsilon_0}$};

\node[] at (axis cs: -1.2,0.2) {$-\frac{\Delta}{2}$};
\node[] at (axis cs: 1.1,0.2) {$\frac{\Delta}{2}$};
\node [teal] at (axis cs: 1.04,1.4) {$\sigma$};
\node [red] at (axis cs: -1.06,1.4) {$-\sigma$};

\addplot [->,red] coordinates {(-1.4,-0.8) (-1.1,-0.8)};
\node [red] at (axis cs: -1.3,-1) {$\frac{\sigma}{2\varepsilon_0}$};

\addplot [->,red] coordinates {(-0.3,-0.8) (-0.6,-0.8)};
\node [red] at (axis cs: -0.5,-1) {$-\frac{\sigma}{2\varepsilon_0}$};

\addplot [->,red] coordinates {(-0.3,-0.8) (-0.6,-0.8)};
\node [red] at (axis cs: -0.5,-1) {$-\frac{\sigma}{2\varepsilon_0}$};

\addplot [->,red] coordinates {(0.9,-0.8) (0.6,-0.8)};
\node [red] at (axis cs: 0.7,-1) {$-\frac{\sigma}{2\varepsilon_0}$};

\addplot [->,red] coordinates {(1.4,-0.8) (1.1,-0.8)};
\node [red] at (axis cs: 1.3,-1) {$-\frac{\sigma}{2\varepsilon_0}$};

\end{axis}
\end{tikzpicture}
\caption{Andamento del campo attraverso due lastre cariche}
\label{fig:lastre_indefinite}
\end{figure}

Siano due lastre poste a distanza $\Delta$ con carica $\sigma$ e $-\sigma$,
si avrà un campo pari a $\frac{\sigma}{2\varepsilon_0}$ a destra della
lastra $\sigma$ e $-\frac{\sigma}{2\varepsilon_0}$ nel lato sinistro della lastra.
Viceversa la lastra sinistra $-\sigma$ avrà verso opposto, applicando la sovrapposizione degli 
effetti si vede che il campo è nullo all'esterno dei due piani e pari a $-\sigma/\varepsilon_0$ 
nel mezzo.
$$
E_z = \begin{cases}
0, & |z| > \frac{\Delta}{2}\\
-\frac{\sigma}{\varepsilon_0}, & |z| \leq \frac{\Delta}{2}
\end{cases}
$$
Questa situazione è analoga a quella presente in un condensatore.

\section{Elettrostatica - Potenziale}
Si riportano le equazioni dell'elettrostatica
$$
\oiint_\Sigma \vec{E}\cdot\hat{n} dS = \frac{Q_{\Omega_\Sigma}}{\varepsilon_0}\ \ \forall\ \Sigma \text{ chiusa}
$$
$$
\oint_\Gamma \vec{E}\cdot\hat{t} dl =0\ \ \forall\ \Gamma \text{ chiusa}
$$
In forma locale diventano
$$
\nabla\cdot\vec{E} = \frac{\rho}{\varepsilon_0} \text{ in } \Omega \text{ punti regolari}
$$
L'equazione di Faraday-Neumann diventa:
$$
\nabla \times \vec{E} = 0 \text{ in } \Omega \text{ punti regolari}
$$
Le condizioni di raccordo si scrivono come:
$$
\begin{aligned}
\hat{n}\cdot (\vec{E}_2 - \vec{E}_1) &= \frac{\sigma}{\varepsilon_0} \text{ su } \partial\Omega\\
\hat{n}\times (\vec{E}_2 - \vec{E}_1) &= 0 \text{ su } \partial\Omega
\end{aligned}
$$

Se il dominio $\Omega$ è a connessione lineare semplice si ottiene l'equazione di \textit{Poisson}
$$
\nabla\times\vec{E} = 0 \Rightarrow \vec{E} = - \nabla V \text{ in } \Omega \Rightarrow \nabla^2V = -\frac{\rho}{\varepsilon_0} \text{ in } \Omega
$$
Sulla frontiera di $\Omega$: (condizione di raccordo per la funzione potenziale)
$$
\hat{n}\cdot (\vec{E}_2 - \vec{E}_1) = \hat{n}\cdot\left(-\nabla V_2 + \nabla V_1\right) =
\frac{\sigma}{\varepsilon_0} \Leftrightarrow
-\frac{\partial V_2}{\partial n} + \frac{\partial V_1}{\partial n} = \frac{\sigma}{\varepsilon_0}
$$
Continuità della componente tangente del campo elettrico:
$$
\hat{n}\times\left(\vec{E}_2-\vec{E}_1\right) = 0 \Rightarrow \hat{n}\times\left(-\nabla V_2 + \nabla V_1\right) = 0 \Rightarrow \hat{n}\times\left(-\nabla V_2 + \nabla V_1\right)\cdot\hat{m} = 0
$$
con $\hat{m}$ versore qualsiasi non parallelo ad $\hat{n}$, si prende ad esempio il versore $\hat{m}$
ortogonale alla linea chiusa rettangolare che interseca la superficie del dominio, 
il versore $\hat{t}$ sarà dunque dato dal prodotto vettoriale $\hat{t} = \hat{n}\times\hat{m}$ 
e risulterà tangente alla linea chiusa secondo il verso antiorario.
Sfruttando la proprietà circolare del prodotto misto:
$$
\begin{aligned}
\hat{t}\cdot\left(\nabla V_2 - \nabla V_1\right)  = 0 &\Rightarrow \frac{\partial V_2}{\partial \hat{t}} - \frac{\partial V_1}{\partial \hat{t}} = 0 \\
\frac{\partial}{\partial \hat{t}}(V_2-V_1)  = 0 &\Rightarrow V_2 = V_1
\text{ lungo la direzione }\hat{t}
\end{aligned}
$$
La continuità di $\hat{n}\times\vec{E}$ implica la continuità della funzione potenziale attraverso le superfici data l'arbitrarietà di $\hat{t}$.

\subsection{Funzioni armoniche}
Supponiamo che la carica $\rho$ sia nulla in $\Omega$, allora
$$
\nabla^2 V = 0 \text{ in } \Omega \text{ equazione di Laplace}
$$
le soluzioni dell'equazione di Laplace sono funzioni armoniche in $\Omega$.

\paragraph{Proprietà delle funzioni armoniche}
\subparagraph{Proprietà di unicità della soluzione}
Supponiamo di analizzare un problema di valori al contorno (BVP = \textit{Boundary Value Problem})

Problema di Dirichlet per l'equazione di Laplace:
$$
\begin{cases}
\nabla^2 V = 0 \text{ in } \Omega\\
\left.V\right|_{\partial\Omega} = h \text{ su } \partial\Omega
\end{cases}
$$
Ammette una e una sola soluzione.
L'unicità si dimostra con l'identità di Green:
$$
\nabla\cdot(f\nabla f) = f\nabla^2f + (\nabla f)^2
$$
Si integra utilizzando il teorema della divergenza:
$$
\begin{aligned}
\iiint_\Omega \nabla\cdot(V\nabla V) &= \cancel{\iiint_\Omega V\nabla^2 V dV} + \iiint_\Omega (\nabla V)^2 dV\\
\iint_{\partial\Omega} V \frac{\partial V}{\partial n}dS &= \iiint_\Omega (\nabla V)^2 dV
\end{aligned}
$$
Questo vale per una qualunque soluzione del problema.

Supponiamo $V_1$ e $V_2$ soluzioni. Def $\Delta V = V_1 - V_2$ 
$$
\begin{cases}
\nabla^2 (\Delta V) = 0 & \text{ in }\Omega \\
\left.\Delta V\right|_{\partial \Omega} = 0 \text{ su } \partial \Omega
\end{cases}
$$
Applicando la precedente identià di Green:
$$
\iint_{\partial \Omega} \Delta V \frac{\partial \Delta V}{\partial n} dS =
\iiint_\Omega \left[\nabla(\Delta V)\right]^2 dV = 0
$$
Perchè $\Delta V = 0$ sulla frontiera del dominio.
Il secondo termine può essere nullo se la funzione integranda è nulla in $\Omega$ ma se il dominio
è connesso:
$$
\nabla(\Delta V) = 0 \text{ in } \Omega \Rightarrow \Delta V = \text{cost in } \Omega \stackrel{\Delta V = 0 \text{ in }\partial\Omega}{\Rightarrow}
\Delta V = 0 \text{ in } \Omega \Rightarrow V_1 = V_2 \text{ in } \Omega
$$
L'unicità della soluzione continua a valere anche se $\Omega$ è illimitato assumendo che V sia 
normale all'infinito ($V \to 0 \text{ se } r \to \infty$).

\subparagraph{Principio del massimo delle funzioni armoniche}
Una funzione armonica in un dominio $\Omega$ assume valori estremi (i massimi e i minii) sulla
frontiera, se $V$ è costante su una superficie $\Sigma$, necessariamente deve essere costante
nella regione racchiusa $\Omega_\Sigma$ (se definita nella suddetta regione).

Se si riesce ad individuare una superficie sulla quale la funzione armonica è costante, nel caso
del potenziale si chiamano superfici equipotenziali, necessariamente all'interno della regione
racchiusa da questa superficie la funzione potenziale è costante.

\subparagraph{Problema di Neumann per l'equazione di Laplace}
Considerato sempre un dominio $\Omega$ ricerchiamo una funzione armonica in omega con derivata 
sulla frontiera assegnata di valore $g$
$$
\begin{cases}
\nabla^2 V = 0 \text{ in } \Omega\\
\left.\frac{\partial V}{\partial\hat{n}}\right|_{\partial \Omega} = g \text{ su } \partial \Omega
\end{cases}
$$
Si può dimostrare che le soluzioni del teorema di Neumann differiscono per una
costante additiva, invocando l'identità di Green:
$$
\begin{aligned}
\iiint_{\Omega} \left[\nabla(\Delta V)\right]^2 dV &= \iint_{\partial\Omega} \Delta V \frac{\partial\Delta V}{\partial \hat{n}} dS \stackrel{\partial\Delta V/\partial n = 0 }{=} 0\\
\nabla(\Delta V) &= 0 \text{ in } \Omega \Rightarrow \Delta V = \text{ cost in }\Omega 
\end{aligned}
$$

\subsection{Calcolo della funzione potenziale in condizioni di simmetria}
\paragraph{Carica puntiforme}
Sia posta una carica $Q$ nell'origine e $P_0$ un punto distante $r_0$ dall'origine e $P$ distante $r$ dall'origine.
Si suppone di estendere il raggio $r$ fino alla lunghezza di $r_0$ e di congiungere il punto $P'$
ottenuto con $p_0$ mediante una linea che giace sulla circonferenza di pari raggio.

$$
\vec{E}(P) = \frac{Q}{4\pi \varepsilon_0}\frac{1}{r^2} \vec{e}_r
$$
Integrando $\vec{E}$ lungo la curva $\gamma$ da $P$ a $P_0$
$$
\int_{P_\gamma P_0} \vec{E}\cdot \hat{t} dl = \int_{P_\gamma P_0} -\nabla V\cdot \hat{t} dl = 
\int_r^{r_0} \frac{Q}{4\pi\varepsilon_0}\frac{1}{r^2} \vec{e}_r\cdot \hat{t} dr + 
\cancel{\int_{P'}^{P_0} \vec{E}\cdot\hat{t} dl}
$$
ma $\vec{E}\cdot\hat{t} = 0$ perchè i due vettori sono ortogonali
$$
\int_r^{r_0} \frac{Q}{4\pi\varepsilon_0}\frac{1}{r^2}dr = \frac{Q}{4\pi\varepsilon_0}\left[-\frac{1}{r}\right]_r^{r_0} = \frac{Q}{4\pi\varepsilon_0}\left(\frac{1}{r} - \frac{1}{r_0}\right)
$$
$r_0$ e quindi $P_0$ funge da riferimento per la funzione potenziale, poichè $\vec{E} = -\nabla V$ 
tutte le funzioni potenziali che differiscono per una costante additiva costituiscono lo stesso
campo elettrico, in questo caso la costante è proprio $-\frac{Q}{4\pi\varepsilon_0}\frac{1}{r_0}$.
Si può quindi scegliere la costante di integrazione pari a $0$.

In genere quando le cariche sono al finito, conviene scegliere come funzione potenziale
$$
V(r) = \frac{Q}{4 \pi \varepsilon_0}\frac{1}{r}
$$

Tutte le scelte possibili sono valide, il potenziale è solo un artificio matematico
necessario a calcolare il campo che è la vera grandezza fisica.

Per ricavare il campo a partire dal potenziale è necessario calcolare il gradiente:
$$
\vec{E} = -\nabla V = \frac{Q}{4\pi\varepsilon_0}\frac{1}{r^2}\vec{e}_r
$$
pari proprio al campo della carica puntiforme.

Sia $Q$ una carica nell'origine e $P$ il raggio vettore che indica il punto in cui si vuole
calcolare il campo, le superfici ad $r$ costante hanno tutte lo stesso potenziale mentre
il campo è diretto in direzione radiale.

\paragraph{Potenziale generato da una distribuzione volumetrica $\rho$}
Sia $\Omega$ una regione carica con densità $\rho(p')$, con $p'$ interno ad $\Omega$ e raggio
vettore $\vec{r}_{p'}$, sia il punto in cui si vuole calcolare il campo $p$
allora la densità di carica elementare sarà $\rho(p')dV$, di conseguenza il potenziale in $p$
dovuto a $\rho(p')$ sarà:
$$
dV = \frac{\rho(p')dV}{4\pi\varepsilon_0}\frac{1}{|\vec{r}_p - \vec{r}_{p'}|} 
\stackrel{\text{PSE}}{\Rightarrow} \frac{1}{4\pi\varepsilon_0} \iiint_\Omega \frac{\rho(p')}{|\vec{r}_p - \vec{r}_{p'}|} dV = V(p) \text{ potenziale di Coulomb}
$$

$V(p)$ è continua $\forall P \in \mathbb{R}^3$ compresi i punti appartenenti ad $\Omega$, la 
singolarità è eventualmente integrabile dato che il volume infinitesimo $dV$ ha ordine 3 mentre la 
funzione integranda va come $\frac{1}{r}$.

\paragraph{Potenziale prodotto da distribuzione superficiale $\sigma$}
Sia $S$ la superficie con distribuzione superficiale $\sigma$ $p'$ un punto appartenente alla 
superficie e $p$ il punto in cui si calcola il potenziale.
$$
dV = \frac{\sigma(p')dS}{4\pi\varepsilon_0} \frac{1}{|\vec{r}_p - \vec{r}_{p'}|} 
\stackrel{\text{PSE}}{\Rightarrow}
\frac{1}{4\pi\varepsilon_0} \iint_S \frac{\sigma(p')}{|\vec{r}_p - \vec{r}_{p'}|} dS = V(p)
$$
Anche questa funzione è continua $\forall\ p \in \mathbb{R}^3$

\paragraph{Distribuzione lineare con densità $\lambda(p')$}
con $p'\in\gamma$ 
$$
dV = \frac{\lambda(p')dl}{4\pi\varepsilon_0}\frac{1}{|\vec{r}_p-\vec{r}_{p'}|} 
\stackrel{\text{PSE}}{\Rightarrow} V(p) = \frac{1}{4\pi\varepsilon_0} 
\int_\gamma \frac{\lambda(p')}{|\vec{r}_p-\vec{r}_{p'}|}dl
$$
Questa funzione è continua $\forall p \notin \gamma $ dato che diverge logaritmicamente su $\gamma$.

\subsection{Operatori differenziali in coordinate curvilinee}
Sia $\Delta\Omega$ un volumetto elementare in $P$ in cui si pone un sistema di coordinate
locali $l_1,l_2,l_3$, il volumetto ha dimensioni $dl_1,\ dl_2,\ dl_3$.
Siano $(u_1,u_2,u_3)$ un sistema di coordinate curvilinee ortogonali, 
gli spostamenti elementari saranno:
\begin{align*}
dl_1 &= h_1du_1\\
dl_2 &= h_2du_2\\
dl_3 &= h_3du_3
\end{align*}
Con $h_1,h_2,h_3$ i fattori metrici che dipendono dal punto.

Le superfici coordinate elementari:
\begin{align*}
dS_1 &= dl_2dl_3 = h_2h_3du_2du_3\\
dS_2 &= dl_3dl_1 = h_3h_1du_3du_1\\
dS_3 &= dl_1dl_2 = h_1h_2du_1du_2
\end{align*}

Volume elementare:
$$
dV = dl_1dl_2dl_3 = h_1h_2h_3du_1du_2du_3
$$

\paragraph{Operatore Divergenza in coordinate curvilinee}
Operando la definizione intrinseca la divergenza è il rapporto fra il flusso attraverso la 
superficie del volume elementare e il volume elementare stesso:

$$
\vec{v}(p) \in \mathbb{C}^1(\Omega) :
$$
$$
\nabla\cdot\vec{v} = \frac{\iint_{\partial\Delta\Omega}\vec{v}\cdot\hat{n}dS}{\text{Vol}(\Delta \Omega)}
= \frac{1}{dV} \left[\begin{aligned}
&\vec{v}\left(u_1+\frac{du_1}{2},u_2,u_3\right)\cdot\vec{e}_1\cdot
dS_1\left(u_1+\frac{du_1}{2},u_2,u_3\right) - \\
&\vec{v}\left(u_1-\frac{du_1}{2},u_2,u_3\right)\cdot\vec{e}_1\cdot
dS_1\left(u_1-\frac{du_1}{2},u_2,u_3\right) + \\
&\vec{v}\left(u_1,u_2+\frac{du_2}{2},u_3\right)\cdot\vec{e}_2\cdot
dS_2\left(u_1,u_2+\frac{du_2}{2},u_3\right)  - \\
&\vec{v} \left(u_1,u_2-\frac{du_2}{2},u_3\right)\cdot\vec{e}_2\cdot
dS_2\left(u_1,u_2-\frac{du_2}{2},u_3\right) + \\
&\vec{v} \left(u_1,u_2,u_3+\frac{du_3}{2}\right)\cdot\vec{e}_3\cdot
dS_3\left(u_1,u_2,u_3+\frac{du_3}{2}\right) - \\
&\vec{v} \left(u_1,u_2,u_3-\frac{du_3}{2}\right)\cdot\vec{e}_3\cdot
dS_3\left(u_1,u_2,u_3-\frac{du_3}{2}\right)
\end{aligned} \right]
$$
Si considerano i primi due termini ricordando che $dV = h_1h_2h_3du_1du_2du_3$
$$
\nabla\cdot\vec{v} = \frac{1}{h_1h_2h_3} \left[\frac{v_1(u_1+\frac{du_1}{2},u_2,u_3)dS_1(u_1+\frac{du_1}{2},u_2,u_3) - v_1(u_1-\frac{du_1}{2},u_2,u_3)}{du_1du_2du_3}\right]
$$
1:23:50
\paragraph{Distribuzione di carica superficiale con simmetria sferica}
$$
\frac{1}{r^2}\frac{\partial}{\partial r} \left(r^2 \frac{\partial V}{\partial r}\right) = 0
$$

A B C D costanti di integrazione eccetera AAAAAHAHAHAAA


\include{13_elettrostatica_conduttori}
\subparagraph{Problema generale dell'elettrostatica}

Siano $N$ conduttori carichi $c_1,c_2,\ldots ,c_i$ e le rispettive normali alle superfici
$\hat{n}_i$, ogni conduttore carico avrà un potenziale $V_i$. Trovare il potenziale $V(P)$ 
dato dagli stessi e le cariche contenute.
Si suppone che $V(P) = V_k$ in $c_k$ e su $\partial c_k$.
La carica contenuta dal conduttore $k$, $Q_k$ è data da

$$
Q_k = \int_{\partial_{C_k}} \varepsilon_0 \vec{E}\cdot\hat{n}_k dS
$$

Sia $\Omega = \mathbb{R}^3 - c_1 - c_2 - \ldots  - c_k$ la regione esterna che non contiene i
conduttori, si ha un problema di Dirichlet esterno per l'Equazione di Laplace.

\begin{equation}
\begin{cases}
\nabla^2 V = 0 & \text{in } \Omega\\
V|_{\partial_{C_k}} = V_k & k=1,\ldots ,N \\
\lim_{P\to \infty}V(P)=0 & \text{normale all' } \infty
\end{cases}
\label{eq:problema_dirichelet}
\end{equation}
In queste condizioni si dimostra che la soluzione esiste ed è unica.
La soluzione generale si costruisce con il PSE considerando N BVPs ausiliari:
$$
\begin{cases}
\nabla^2 V^{(i)} = 0 &\text{in } \Omega\\
V^{(i)}|_{\partial_{C_i}} = 1,\ v^{(i)}|_{\partial_{C_k}} = 0 &  k \neq i
\end{cases}
$$
Anche le soluzioni per queste equazioni sonno univocamente determinate per i teoremi di unicità
delle equazioni di Laplace.
La soluzione generale del problema \ref{eq:problema_dirichelet} sarà:
\begin{align*}
&V(P) = V_1V^{(1)}(P) + V_2V^{(2)}(P) + \ldots  + V_NV^{(N)}(P) \\
& \nabla^2 V = 0 \ \ \text{ in }\Omega
\end{align*}
I laplaciani sono tutti pari a 0 quindi anche la loro somma sarà pari a 0.
Si può ora considerare la carica depositata sul conduttore $C_i$:
$$
Q_i = \int_{\partial_{C_i}} \varepsilon_0 \vec{E} \cdot \hat{n}_i dS = \int_{\partial_{C_i}} -\varepsilon_0 \frac{\partial V}{\partial n_i} dS = \left(\int_{\partial_{C_i}} - \varepsilon_0 \frac{\partial V}{\partial n_i}^{(1)} dS\right)V_1 + \ldots  + \left(\int_{\partial_{C_i}} - \varepsilon_0 \frac{\partial V}{\partial n_i}^{(N)} dS\right)V_N
$$
Gli integrali $i$-esimi vengono chiamati coefficienti di capacità ed esprimono il contributo
di carica fornito al conduttore $i$-esimo per mezzo del potenziale $N$-esimo.
$$
[C_{ij}] = \frac{\si{\coulomb}}{\si{\volt}} = \si{\farad}
$$
In un generico sistema di $N$ conduttori, le cariche si possono esprimere mediante
un'applicazione lineare in $\mathbb{R}^N$
$$
\begin{cases}
Q_1 = C_{11}V_1 + \ldots + C_{1N}V_N \\
\vdots \\
Q_n = C_{n1}V_1 + \ldots + C_{nn}V_N
\end{cases}
$$

Il coefficiente $C_{ii}$ è la carica accumulata sul conduttore $c_i$ quando $V_i=1$ e $V_k = 0$
per $k\neq i$.
$$
\begin{aligned}
C_{ii} &= \iint_{\partial c_i} -\varepsilon_0\frac{\partial V^{(i)}}{\partial n_i} dS\\
C_{ij} &= \iint_{\partial c_i} -\varepsilon_0\frac{\partial V^{(j)}}{\partial n_i} dS & V_j = 1,\ 
V_k = 0,\ k\neq j
\end{aligned}
$$

Questi coefficienti soddisfano varie proprietà come:

\begin{description}
 \item[Reciprocità] Come $C_{ij} = C_{ji} $
 \item[Proprietà 2] $c_{ij} < 0 $, $c_{ii}>0$
 \item[Simmetria e dominanza diagonale] $C_{ii} > \sum_{i\neq j}|C_{ij}|$
\end{description}
La proprietà 2 si dimostra rappresentando due conduttori, il conduttore $j$ a 
potenziale unitario mentre il conduttore $i$ a potenziale nullo, per l'armonicità della funzione
potenziale tutte le linee di campo saranno uscenti dal conduttore $j$ ed entranti nel conduttore 
$i$ quindi $\vec{E}\cdot\hat{n}_i < 0$.
$$ %matrice dei coefficienti di capacità \underline{C}
\begin{bmatrix}
Q_1\\
\vdots \\
Q_N
\end{bmatrix} = 
\begin{bmatrix}
C_{11} & \ldots  & C_{1N} \\
C_{21} & \ldots  & C_{2N} \\
\vdots & \ldots  & \vdots \\
C_{N1} & \ldots  & C_{NN}
\end{bmatrix} \cdot
\begin{bmatrix}
V_1 \\
\vdots \\
V_N
\end{bmatrix}
$$

\textbf{Esempio} Elettrodi sferici a \textit{``grande''} distanza

Siano $P_1$ e $P_2$ i centri di una coppia di sfere conduttrici di raggio $r_1$ ed $r_2$ con potenziali $V_1$ e
$V_2$ e cariche $Q_1$ e $Q_2$.
Si ipotizza che la distanza $|P_1P_2|$ sia sufficientemente grande da rendere trascurabile il potenziale
prodotto da una sfera in corrispondenza dell'altra ossia facendo in modo che non si \textit{``vedano''} 
elettrostaticamente, ossia i loro potenziali e distribuzioni di cariche non si influenzino a vicenda.

Si sa inoltre che il potenziale di una sfera carica è:
$$
V = \frac{Q}{4 \pi \varepsilon_0 R}
$$
il calcolo si può effettuare per entrambe le sfere ottenendo:
$$
Q_1 = 4 \pi \varepsilon_0 R_1 V_1,\ Q_2 = 4 \pi \varepsilon_0 R_2 V_2
$$

Supponendo che $Q_1 = -Q_2$ (fenomeno di induzione completa)
$$
-Q_1 = 4 \pi \varepsilon_0 R_2 V_2 \Rightarrow V_1 - V_2 = \frac{Q_1}{4 \pi \varepsilon_0}
\left(\frac{1}{R_1} + \frac{1}{R_2}\right)
$$
Si definisce la capacità tra i due elettrodi:
$$
\frac{Q_1}{V_1 - V_2} = C = \frac{4 \pi \varepsilon_0}{\frac{1}{R_1}+\frac{1}{R_2}}
$$
In generale quando $Q_1 = - Q_2$ si può definire una capacità tra i due elettrodi che operativamente
può essere scritta come 
$$
C = \frac{\iint_{\partial c_1} \varepsilon_0 \vec{E}\cdot\hat{n}dS} {\int_{P_1}^{P_2}\vec{E}\cdot\hat{t} dl} = \frac{Q_1}{V_1 - V_2}
$$


\textbf{Caso generale}: la carica $Q_i$ dipende da tutti i potenziali $V_1, \ldots ,V_N$:
$$
Q_1 = C_{11}V_1 + C_{12}V_2 + \ldots + C_{1N}V_N = C_{11}V_1 + C_{12}V_2 + C_{12}V_1 - C_{12} V_1 +
\ldots  + C_{1N}V_N
$$
$$
Q_1 = \left(C_{11}+C_{12} + \ldots + C_{1N}\right) V_1 + (-C_{12})(V_1-V_2) + (-C_{13})(V_1-V_3)
+ \ldots + (-C_{1N})(V_1-V_N)
$$
La prima parentesi viene sostituita da $C_{11}^*$, $(-C_{12}) = C_{12}^*$ fino a $C_{1N}^*$,
vengono chiamate
\textit{capacità parziali} del sistema di conduttori, per evidenziare le capacità 
dovute all'interazione dei conduttori a due a due.
Queste definiscono nella maggior parte dei casi anche gli accoppiamenti parassiti
tra diverse capacità presenti in un circuito.

Alcune proprietà:
$$\begin{aligned}
C_{ii}^* &\stackrel{\text{def}}{=} \left.\frac{Q_i}{V_i}\right|_{V_k = 1}\ k=1\ldots ,n\\
C_{ij}^* &\stackrel{\text{def}}{=} \left.\frac{Q_i}{V_j}\right|_{V_j=-1}\ V_k = 0,\ k \neq j
\end{aligned}
$$
Si può mostrare che $C_{ii}^* =C_{ii} + \sum_{j\neq i}C_{ij} \geq 0,\ C{ij}^* = -C_{ij} > 0$

Caso in cui $N=2$:
$$
\begin{aligned}
Q_1 &= C_{11} V_1 + C_{12}V_2 = C_{11}V_1 + C_{12}V_2 + C_{12}V_1 - C_{12}V_1 = (C_{11}+C_{12})V_1 - C_{12}(V_1-V_2)\\
Q_2 &= C_{21}V_1 + C_22 V_2 = C_{21}^*(V_2-V_1) + C_{22}^*(V_2-\cancel{V_{\infty}})
\end{aligned}
$$
Supponendo che ci sia induzione completa: $Q_1 = - Q_2$
$$\begin{aligned}
Q_1 &= C_{11}^*V_1 + C_{12}^*(V_1-V_2) \\
-Q_1 &= C_{21}^*(V_2-V_1) + C_{22}^*V_2
\end{aligned}
$$
Sviluppando le due equazioni si ottiene:
$$\begin{aligned}
\frac{Q_1}{C_{11}^*} &= V_1 + \frac{C_{12}^*}{C_{11}^*}(V_1-V_2)\\
\frac{-Q_1}{C_{22}^*} &= \frac{C_{21}^*}{C_{22}^*}(V_2-V_1) + V_2
\end{aligned}
$$
Sottraendo le due equazioni si ottiene:
$$\begin{aligned}
&Q_1\left(\frac{1}{C_{11}^*}+\frac{1}{C_{22}^*}\right) = (V_1-V_2) + (V_1-V_2)\left(\frac{C_{12}^*}{C_{11}^*} + \frac{C_{21}^*}{C_{22}^*} \right) = (V_1 - V_2)\left[1+C_{12}^*\left(\frac{1}{C_{11}^*}+\frac{1}{C_{22}^*}\right)\right]\\
&\frac{Q_1}{V_1-V_2} = C = \frac{1 + C_{12}^*\left(\frac{1}{C_{11}^*} + \frac{1}{C_22}^*\right)}
{\frac{1}{C_{11}^*}+\frac{1}{C_{22}^*}} = C_{12}^* + \left(\frac{1}{C_{11}^*} + \frac{1}{C_{22}^*}\right)^{-1} = C
\end{aligned}
$$
Questa equazione suggerisce fisicamente che la capacità totale sia pari al parallelo della
capacità $C_{12}^*$ tra i due conduttori e la serie di due capacità con un punto in comune
all'infinito rappresentanti le capacità dei singoli conduttori rispetto a questo punto.

Per realizzare un condensatore che non sia influenzato dalla presenza di conduttori esterni
si sfrutta la proprietà di schermo elettrostatico, si realizza un'armatura in modo da contenere
l'altra in modo da mantenere pari a zero una delle due capacità rispetto all'infinito.
Un modo è quindi quello di chiudere un conduttore $C_1$ all'interno dell'altro $C_2$.
La capacità parziale del primo conduttore è nulla perché il campo all'interno del conduttore
più grande è nullo, di conseguenza, per la legge di Gauss anche la carica contenuta nel conduttore
1 sarà pari a 0.
$$\begin{aligned}
& \oiint_{\Sigma}\vec{E}\cdot\hat{n}dS = 0\ \ \forall \ \Sigma\\
&C_{11}^* = \left.\frac{Q_1}{V_1}\right|_{V_k = 1} = 0
\end{aligned}
$$
Quindi 
$$
\begin{aligned}
Q_1 &= C_{12}^*(V_1-V_2) \\
Q_2 &= C_{21}^*(V_2-V_1) + C_{22}^*V_2
\end{aligned}
$$
Di conseguenza la capacità $C$ sarà pari a $C_{12}^*$
$$
\begin{aligned}
Q_1 &= C(V_1-V_2) \\
Q_2 &= -C(V_1-V_2) + C_{22}^*V_2 = -Q_1 + C_{22}^*V_2
\end{aligned}
$$
La carica netta del sistema $C_1 \cup C_2$ sarà:
$$
Q_1+Q_2 = C_{22}^*V_2
$$
interamente distribuita sulla superficie esterna.

\subparagraph{Calcolo capacità condensatore piano}
Siano due lastre indefinite poste ad una distanza $d$ e con cariche opposte
$\sigma$ e $-\sigma$, supposta $\sigma$ positiva il campo elettrico è orientato dalla
lastra carica positivamente a quella negativa, di intensità pari a 
$$
\vec{E} = \frac{\sigma}{\varepsilon_0} \vec{e}_x
$$
Per calcolare la tensione tra le due armature e quindi la capacità si procede con un integrale di 
linea:
$$
V_1-V_2 = \int_{-\frac{d}{2}}^{\frac{d}{2}} \frac{\sigma}{\varepsilon_0} dx = \frac{\sigma}{\varepsilon_0} d
$$
Ricordando che $Q_1 = \sigma S$ 
$$
C = \frac{Q_1}{V_1-V_2} = \frac{\sigma S}{\varepsilon_0 d}
$$
Se $\frac{\sqrt{S}}{d} \gg 1 $ si può approssimare un condensatore piano finito con un modello
ideale che non tiene in considerazione gli effetti di bordo.

\subparagraph{Condensatore sferico}
Costituito da due sfere concentriche di raggio $R_1$ ed $R_2$
si può calcolare il potenziale per sovrapposizione:
$$
V_1 = V(R_1) = \frac{Q_1}{4\pi \varepsilon_0 R_1},\ \ V_2 = V(R_2) = \frac{Q_2}{4 \pi \varepsilon_0 R_2}
$$
La capacità
$$
C = \frac{Q_1}{V_1 - V_2} = \frac{\cancel{Q_1}}{\frac{\cancel{Q_1}}{4 \pi \varepsilon_0}\left(\frac{1}{R_1}-\frac{1}{R_2}\right)} = \frac{4 \pi \varepsilon_0}{\frac{1}{R_1}-\frac{1}{R_2}} = 4 \pi \varepsilon_0 \frac{R_1 R_2}{R_2 - R_1}
$$

\subparagraph{Condensatore cilindrico}
Costituito da due cilindri coassiali di raggi $R_1$ e $R_2$, si applica nuovamente il PSE
$$\begin{aligned}
V(r) &= -\frac{\sigma_1}{\varepsilon_0}R_1 \ln r - \frac{\sigma_2}{\varepsilon_0}R_2 \ln R_2\ \ R_1\leq r \leq R_2\\
V(R_1) &= V_1 = -\frac{\sigma_1}{\varepsilon_0} R_1 \ln R_1 - \frac{\sigma_2}{\varepsilon_0}R_2 \ln R_2\\
V(R_2) &= V_2 = -\frac{\sigma_1}{\varepsilon_0} R_1 \ln R_2 - \frac{\sigma_2}{\varepsilon_0}R_2 \ln R_2\end{aligned}
$$
Eseguendo la differenza dei potenziali
$$
V_1 - V_2 = -\frac{\sigma_1}{\varepsilon_0} R_1 \ln R_1 + \sigma_1\frac{R_1}{\varepsilon_0}\ln R_2
= \frac{\sigma_1 R_1}{\varepsilon_0}\ln\left(\frac{R_2}{R_1}\right) = \frac{2 \pi R_1 L \sigma_1}{2 \pi L \varepsilon_0}\ln \left(\frac{R_2}{R_1}\right)
$$
il numeratore è proprio $Q_1$
$$
C = \frac{Q_1}{V_1-V_2} = \frac{2 \pi \varepsilon_0 L }{\ln \left(\frac{R_2}{R_1}\right)}
$$

\subparagraph{Capacità parziali di due elettrodi sferici}
Presi due elettrodi a distanza $d$ di raggio $R_1$ ed $R_2$, supponendo che $d \gg R_1,R_2$
in maniera tale da supporre che il potenziale generato da una sfera sia approssimativamente
costante nei punti occupati dall'altra, confondiamo cioè il valore del potenziale sulla
superficie della sfera con quello al suo centro.

Applicando il PSE..

\begin{align*}
V_1' &= \frac{Q}{4 \pi \varepsilon_0 R_1}& V_2' &\simeq \frac{Q_1}{4 \pi \varepsilon_0 d}\\
V_1'' &\simeq \frac{Q_2}{4 \pi \varepsilon_0 d}& V_2'' &= \frac{Q_2}{4 \pi \varepsilon_0 R_2}
\end{align*}

$$
\begin{aligned}
V_1 &= V_1' + V_1'' = \frac{Q_1}{4\pi\varepsilon_0 R_1} + \frac{Q_2}{4 \pi \varepsilon_0 d} \\
V_2 &= V_2' + V_2'' = \frac{Q_1}{4\pi\varepsilon_0 d} + \frac{Q_2}{4 \pi \varepsilon_0 R_2}
\end{aligned}
$$
$$
\begin{bmatrix}
V_1 \\
V_2
\end{bmatrix} = \frac{1}{4\pi\varepsilon_0} \begin{bmatrix}
\frac{1}{R_1} & \frac{1}{d}\\
\frac{1}{d} & \frac{1}{R_2}
\end{bmatrix}\cdot
\begin{bmatrix}
Q_1\\
Q_2
\end{bmatrix}
$$

Si inverte la matrice precedentemente ottenuta:
$$
\begin{bmatrix}
Q_1\\
Q_2
\end{bmatrix} = \frac{4\pi \varepsilon_0}{\frac{1}{R_1R_2}-\frac{1}{d^2}} \begin{bmatrix}
\frac{1}{R_2} & -\frac{1}{d}\\
-\frac{1}{d} & \frac{1}{R_1}
\end{bmatrix}\begin{bmatrix}
V_1 \\
V_2
\end{bmatrix}
$$
Si ottengono quindi le capacità parziali:
\begin{align*}
&C_{11} = \frac{4\pi\varepsilon_0}{\frac{1}{R_1R_2}-\frac{1}{d^2}}\frac{1}{R_2} & C_{12} = 
\frac{4\pi\varepsilon_0}{\frac{1}{R_1R_2}-\frac{1}{d^2}}\left(-\frac{1}{d}\right)\\
&C_{22} = \frac{4\pi\varepsilon_0}{\frac{1}{R_1R_2}-\frac{1}{d^2}}\frac{1}{R_1} & C_{21} = 
\frac{4\pi\varepsilon_0}{\frac{1}{R_1R_2}-\frac{1}{d^2}}\left(-\frac{1}{d}\right)
\end{align*}
\begin{align*}
&C_{11}^* = C_{11} + C_{12} = \frac{4\pi\varepsilon_0}{\frac{1}{R_1R_2}-\frac{1}{d^2}}
\left(\frac{1}{R_2} - \frac{1}{d}\right)\\
&C_{22}^* = \frac{4\pi\varepsilon_0}{\frac{1}{R_1R_2}-\frac{1}{d^2}}
\left(-\frac{1}{d} + \frac{1}{R_1}\right)\\
&C_{12}^* = -\frac{4 \pi \varepsilon_0}{\frac{1}{R_1R_2}-\frac{1}{d^2}}\left(-\frac{1}{d}\right) =
\frac{4 \pi \varepsilon_0}{\left(\frac{1}{R_1R_2} - \frac{1}{d^2}\right)}\cdot\frac{1}{d} = \frac{4\pi\varepsilon_0 R_1R_2d}{d^2-R_1R_2}
\end{align*}

Se supponiamo che ci sia un fenomeno di induzione completa: 
$$
Q_2 = -Q_1 \Rightarrow C = C_{12}^* +
\left(\frac{1}{C_{11}^*} + \frac{1}{C_{22}^*}\right)^{-1}
$$
$$
C = \frac{Q_1}{V_1-V_2} = \frac{1}{\frac{1}{4\pi\varepsilon_0}}\frac{Q_1}{\frac{Q_1}{R_1} -
\frac{Q_1}{d}-\left(\frac{Q_1}{d}-\frac{Q_1}{R_2}\right)} =
\frac{4\pi\varepsilon_0}{\frac{1}{R_1} + \frac{1}{R_2} - \frac{2}{d}}
$$
Eseguendo il limite per $d\to \infty$ si vede che l'espressione precedente è coerente con quella 
ricavata nell'ipotesi in cui le cariche siano poste a grande distanza.


\subparagraph{Capacità per unità di lunghezza di una linea bifilare}

I due conduttori possono essere rappresentati mediante due sezioni cilindriche,
le cariche associate $Q_1$ e $Q_2$ si accumuleranno per unità di lunghezza e 
saranno quindi misurate in \si{\coulomb\per\meter}.
Applicando il PSE
\begin{figure}[h!]
\centering
 \includegraphics[width=0.5\linewidth]{conduttori_bifilari}
\caption{Sezione conduttori bifilari}
\end{figure} 

\begin{align*}
V_1' &= -\frac{\sigma_1 R_1}{\varepsilon_0} \ln R_1 & V_2' &= -\frac{\sigma_1 R_1}{\varepsilon_0} \ln d\\
V_1'' &= -\frac{\sigma_2 R_2}{\varepsilon_0} \ln d & V_2'' &= -\frac{\sigma_2 R_2}{\varepsilon_0} \ln R_2
\end{align*}
quindi moltiplicando e dividendo per $2\pi$
\begin{align*}
V_1 &= V_1' + V_1'' = -\frac{\sigma_1 2\pi R_1}{2\pi\varepsilon_0} \ln R_1 -\frac{\sigma_2 2\pi R_2}{2\pi\varepsilon_0} \ln d\\
V_2 &= V_2' + V_2'' = -\frac{\sigma_1 2\pi R_1}{2\pi\varepsilon_0} \ln d  -\frac{\sigma_2 2\pi R_2}{2\pi\varepsilon_0} \ln R_2
\end{align*}

Per ipotesi di induzione completa $(Q_1 = - Q_2)$:
$$
C = \frac{Q_1}{V_1-V_2} = \frac{Q_1}{-\frac{Q_1}{2\pi\varepsilon_0}\ln R_1 + \frac{Q_1}{2\pi\varepsilon_0}\ln d + \frac{Q_1}{2\pi\varepsilon_0}\ln d - \frac{Q_1}{2 \pi \varepsilon_0} \ln R_2} = \frac{2\pi\varepsilon_0}{\ln \left(\frac{d^2}{R_1R_2}\right)}
$$

Se si suppone che i raggi dei conduttori siano gli stessi, come accade solitamente nelle linee multifilari, la capacità della linea diventa
$$
C\cdot L = \frac{2\pi\varepsilon_0 L}{2 \ln \left(\frac{d}{R}\right)} = \frac{\pi \varepsilon_0 L}{\ln \left(\frac{d}{R}\right)} = 27.8\cdot 10^{-12} \frac{L}{\ln\left(\frac{d}{R}\right)}\ \ [\si{\farad}]
$$
Questa ipotesi è valida se non si considera l'influenza del suolo, per studiare l'effetto di questa 
presenza si usa invece il \textit{metodo delle immagini}.

\subsection{Metodo delle immagini}
È un metodo basato sull'unicità delle soluzioni per problemi di Dirichelet per le equazioni di
Poisson o Laplace (se il termine noto è nullo). Se si è in grado di determinare una funzione
armonica che soddisfa le soluzioni del problema, anche se questa non coincide con la reale 
configurazione delle cariche, questa sarà anche soluzione del problema con le condizioni al contorno
in esame in virtù del principio di unicità.

Supponiamo la presenza di una carica puntiforme di valore $q$ in $P'$ ad una certa quota $z$
e si suppone di avere un piano indefinito a potenziale 0.
Andrebbe in teoria risolto un problema di Dirichelet per l'equazione di Laplace nello spazio
$z>0$ e opportuna condizione al contorno.
$$
\begin{cases}
\nabla^2 V = 0\\
V(z=0)=0
\end{cases}
$$

In alternativa si suppone di porre nel suolo una carica di valore opposto a pari distanza dal suolo
della precedente, la soluzione del potenziale se si considerano solo queste due cariche sarà
$$
V(\vec{r}) = \frac{q}{4 \pi \varepsilon_0} \left[\frac{1}{|\vec{r}_p-\vec{r}_{p'}|}-\frac{1}{|\vec{r}_p-\vec{r}_{p''}|}\right]
$$
Data la simmetria tra le due cariche rispetto al piano di
terra, lo saranno anche le linee di campo e le superfici equipotenziali, in particolare
lo stesso piano di terra sarà una superficie equipotenziale.
\begin{figure}[h!]
\centering
 \includegraphics[width=0.5\linewidth]{metodo_immagini}
\caption{Metodo delle immagini}
\end{figure} 
La soluzione trovata come somma dei potenziali della carica reale e della carica immagine
è armonica e rispetta la condizione al contorno del problema di partenza $(V(z=0)=0)$ sarà
quindi soluzione del problema iniziale senza carica immagine.

\subparagraph{Linea bifilare in prossimità del terreno}
Utilizzando ancora il metodo delle immagini si può calcolare il campo generato da una linea
bifilare in prossimità del suolo.
\begin{figure}[h!]
\centering
 \includegraphics[width=0.5\linewidth]{linea_bifilare_terreno}
\caption{Metodo delle immagini su una linea bifilare}
\end{figure} 
Applicando il PSE
\begin{align*}
V_1 &= -\frac{\sigma_1 R_1 2\pi}{2\pi\varepsilon_0}\ln R_1 -
\frac{\sigma_2 R_2 2 \pi}{2 \pi \varepsilon_0}\ln d + 
\frac{\sigma_1 R_1 2 \pi}{2\pi \varepsilon_0} \ln(2 h_1) +
\frac{\sigma_2 R_2 2 \pi}{2\pi\varepsilon_0} \ln D \\
V_1 &= -\frac{Q_1}{2\pi\varepsilon_0} \ln\left(\frac{R_1}{2h_1}\right) - 
\frac{Q_2}{2\pi\varepsilon_0} \ln \left(\frac{d}{D}\right)\\
V_2 &= -\frac{Q_1}{2\pi\varepsilon_0} \ln\left(\frac{d}{D}\right) - 
\frac{Q_2}{2\pi\varepsilon_0} \left(\frac{R_2}{2h_2}\right)
\end{align*}
Nel caso di induzione completa si può calcolare la capacità
$$
C = \frac{\cancel{Q_1}}{\frac{\cancel{Q_1}}{2\pi\varepsilon_0}\left[-\ln\left(\frac{R_1}{2h_1}\right)+\ln\left(\frac{d}{D}\right)+\ln\left(\frac{d}{D}\right)-\ln\left(\frac{R_2}{2h_2}\right)\right]} = 
\frac{2\pi\varepsilon_0}{\ln\left[\left(\frac{d}{D}\right)^2\cdot\frac{4h_1h_2}{R_1R_2}\right]}
$$
Per capire cosa accade all'aumentare di $h_1$ e $h_2$, ossia aumentando
la distanza dal suolo, va trovata la relazione che lega $D$ alla 
distanza $d$.
$$
D^2 = (h_1+h_2)^2 + d^2 - (h_1-h_2)^2 = (h_1+h_2)^2 + d'^2 = 2h_1h_2 + d^2 + 2h_1h_2
$$ 
$$
D^2 = 4h_1h_2+d^2
$$
$$
C = \frac{2\pi\varepsilon_0}{\ln\left[\frac{d^2}{d^2+4h_1h_2} 
\frac{4h_1h_2}{R_1R_2} \right]} = \frac{2\pi\varepsilon_0}
{\ln\left[ \frac{d^2}{\frac{d^2}{4h_1h_2}+1}\cdot\frac{1}{R_1R_2} \right]}
$$
Se $4h_1h_2 \gg d^2,\ h_1,h_2 \gg \frac{d}{2} \Leftrightarrow \frac{h_1}{d},\frac{h_2}{d} \gg \frac{1}{2}$ , la capacità diventa
$$
C =\frac{2\pi\varepsilon_0}
{\ln\left[ \frac{d^2}{\cancel{\frac{d^2}{4h_1h_2}}+1}\cdot\frac{1}{R_1R_2} \right]} \simeq \frac{2\pi\varepsilon_0}{\ln\left(\frac{d^2}{R_1R_2}\right)}
$$
Soluzione identica a quella trovata nel caso precedente in
assenza del terreno.
Per confrontare la formula approssimata in assenza del terreno, rispetto a 
quella esatta si può fare un esempio:
$$
\begin{aligned}
R_1 &= R_2 = \SI{0.01}{\meter}\\
h_1 &= h_2 = \SI{10}{\meter} \\
d &= \SI{1}{\meter}
\end{aligned} \Rightarrow 
\begin{aligned}
C_{\text{esatt.}} &= \SI{6.039}{\pico\farad\per\meter} \\
C_{\text{appross.}} &= \SI{6.034}{\pico\farad\per\meter}
\end{aligned} \Rightarrow
\frac{\Delta C \% }{C} = -0.03 \%
$$
La formula approssimata sottovaluta la capacità per unità di lunghezza
dello 0.3 \textpertenthousand

(per mille)
\newpage
\subparagraph{Realizzazione di un induttore con un condensatore}
Sia dato un doppio bipolo con parametro $G$ conduttanza di
\textit{girazione}, le sue equazioni caratteristiche sono:
$$
\begin{cases}
i_1 = G v_2\\
i_2 = -G v_1
\end{cases}
$$
\begin{figure}[h!]
\centering
\includegraphics[width=0.3\linewidth]{giratore}
\caption{Giratore}
\end{figure}
Si realizza mediante dispositivi elettronici come 
amplificatori operazionali.

Connettendo ad una porta del giratore un condensatore la 
caratteristica diventa:
$$
\begin{aligned}
i_2 &= -C \frac{dv_2}{dt}\\
-Gv_1 &= -C\frac{d}{dt}\left(\frac{i_1}{G}\right)\\
v_1 &= \frac{C}{G^2}\frac{d}{dt}i_1 = L_{eq} \frac{d}{dt}i_1
\end{aligned}
$$
\begin{figure}[h!]
\centering
\includegraphics[width=0.3\linewidth]{giratore_con_condensatore}
\caption{Giratore con condensatore}
\end{figure}
\newpage
\section{Materiali dielettrici}
E il loro modello elettromagnetico

\begin{itemize}
 \item Cariche ``legate'' nei dielettrici
 \item Cariche ``libere'' nei conduttori
\end{itemize}
Si usano queste due distinzioni macroscopiche anche se è possibile
assistere a spostamenti macroscopici di cariche anche nei dielettrici
durante le scariche disruptive e la rottura della relativa 
\textit{rigidità dielettrica}.

Quando il dielettrico è sottoposto ad un campo esterno, questo esercita 
una forza sulle cariche deformando la struttura del materiale e 
producendo così un campo differente, questo fenomeno prende il nome
di \textit{polarizzazione elettrica} e corrisponde ad un piccolo
spostamento delle cariche legate nel dielettrico a causa di un campo
esterno.
\subsection{Il dipolo elettrico}
Il dipolo elettrico è costituito da due cariche uguali e opposte
situate a distanza $d$. In queste condizioni, potenziale e campo 
dipendono dal \textit{momento di dipolo}.
$$
\vec{p} = qd \frac{(\vec{r}_{p'}-\vec{r}_{p''})}{|\vec{r}_{p'}-\vec{r}_{p''}|}
$$
\begin{figure}[h!]
\centering
\includegraphics[width=0.2\linewidth]{dipolo_elettrico}
\caption{Dipolo elettrico}
\end{figure}

In questa configurazione è semplice calcolare il potenziale:
$$
V(P) = \frac{q}{4\pi\varepsilon_0}\left[ \frac{1}{|\vec{r}_p-\vec{r}_{p'}|} - \frac{1}{|\vec{r}_p-\vec{r}_{p''}|} \right] = 
\frac{q}{4\pi\varepsilon_0} \left[ \frac{|\vec{r}_p-\vec{r}_{p''}|-|\vec{r}_p-\vec{r}_{p'}|}
{|\vec{r}_p-\vec{r}_{p'}|\cdot|\vec{r}_p-\vec{r}_{p''}|} \right]
$$
Allontanandosi dal dipolo $|\vec{r}_p| \gg |\vec{r}_{p'}|,|\vec{r}_{p''}|$ quindi il potenziale
diventa:
$$
V(P) = \frac{q}{4\pi\varepsilon_0} \left[ \frac{|\vec{r}_p-\frac{d}{2}\vec{e}_z|-|\vec{r}_p + \frac{d}{2}\vec{e}_z|}
{|\vec{r}_p-\vec{r}_{p'}|\cdot|\vec{r}_p-\vec{r}_{p''}|} \right] = 
\frac{q}{4\pi\varepsilon_0} \left[ \frac{\sqrt{r_p^2 + \left(\frac{d}{2}\right)^2-r_p d \cos \vartheta} - \sqrt{r_p^2 + \left(\frac{d}{2}\right)^2 + r_p d \cos \vartheta}}
{|\vec{r}_p-\vec{r}_{p'}|\cdot|\vec{r}_p-\vec{r}_{p''}|} \right] 
$$
Sviluppando in serie il numeratore si ottiene
$$
 \sqrt{1 + \left(\frac{d}{2r_p}\right)^2- 
\frac{d}{r_p} \cos \vartheta} - \sqrt{1 + \left(\frac{d}{2r_p}\right)^2 + \frac{d}{r_p} \cos \vartheta} \simeq %rigo 2
\left(1 + \frac{d}{2r_p} \cos \vartheta\right) - \left(1 - \frac{d}{2r_p} \cos \vartheta\right)
\simeq d\cos\vartheta
$$

In conclusione il potenziale in un punto lontano dal dipolo sarà
$$
V(P) \simeq \frac{q}{4\pi\varepsilon_0} \frac{d\cos\vartheta}{r_p^2} = \frac{1}{4\pi\varepsilon_0}
\frac{\vec{p}\cdot\vec{r}_p}{r_p^3}
$$
$$
\vec{E}(P) = -\nabla V = \frac{3\left(\vec{p}\cdot\vec{r}_p\right)\vec{r}_p - r_p^2\vec{p}}{4\pi\varepsilon_0r_p^5}
$$
Si osserva quindi che la funzione potenziale decresce come $\frac{1}{r_p^2}$ mentre il 
campo decresce come $\frac{1}{r_p^3}$

\subparagraph{Dipolo in un campo elettrico esterno}
\begin{figure}[h!]
\centering
\includegraphics[width=0.2\linewidth]{dipolo_nel_campo}
\caption{Dipolo immerso in un campo elettrico $\vec{E}$}
\end{figure}
Supponiamo di avere un dipolo di momento $\vec{p}$ e applichiamo un campo esterno $\vec{E}$,
questo eserciterà una forza sulla carica $q$ ed una opposta sulla carica $-q$, la risultante delle 
due forze sarà nulla ma producono una coppia sul bipolo esprimibile come
$$
\left(\vec{r}_{p'} - \vec{r}_{p''}\right)\times q\vec{E} = \vec{p}\times\vec{E}
$$
La coppia tende a far ruotare il bipolo elettrico e l'interazione ammette due condizioni di 
equilibrio determinate da $\vec{p}\times\vec{E} = 0 $

Stabile se il dipolo è allineato con il campo, instabile se è allineato ma in verso opposto,
una piccola perturbazione lo farebbe ruotare di 180\textdegree

Questa caratteristica permette di discernere le sostanze in due categorie: apolari ($O_2$) e polari
($H_2O$) se queste presentano o meno momenti di dipolo permanenti.

Esistono vari tipi di polarizzazione: per \textit{deformazione} se una sostanza apolare, sotto 
l'azione di un campo elettrico, subisce lo spostamento delle cariche legate producendo 
un momento di dipolo.
Oppure per \textit{orientamento} che riguarda le sostanze polari che a temperatura ambiente
presentano un momento di dipolo medio nullo, pur essendo le singole molecole (o atomi) polari.
L'agitazione termica determina l'orientamento casuale dei dipoli elementari.

\begin{figure}[h!]
 \centering
 \includegraphics[width=0.4\linewidth]{polarizzazione_deformazione}
 \caption{Polarizzazione per deformazione}
\end{figure}


\begin{figure}[h!]
 \begin{subfigure}{.5\textwidth}
 \centering
 \includegraphics[width=0.45\linewidth]{sost_apolare_a_media_nulla}
 \caption{Sostanza apolare}
 \end{subfigure} 
 \begin{subfigure}{.5\textwidth}
\centering
 \includegraphics[width=0.8\linewidth]{sost_apolare_polarizzata}
 \caption{Sostanza polarizzata}
 \end{subfigure}
 \caption{Polarizzazione per orientamento}
\end{figure}

Dal punto di vista fisico i due fenomeni sono diversi ma dal punto di vista modellistico
entrambi hanno un momento di dipolo allineato con il campo esterno.
Si introduce quindi nel volume elementare $\Delta\Omega$ il momento di dipolo medio
$$
<\vec{p}> = \vec{P}(Q,t)\cdot \text{Vol}(\Delta\Omega)
$$
dove $\vec{P}(Q,t)$ è il vettore (intensità di) di polarizzazione elettrica ed ha il significato 
di una densità di momento di dipolo media
$$
<\vec{p}> = \frac{\sum_i N\vec{p}_i}{\text{Vol}(\Delta\Omega)}\cdot \text{Vol}(\Delta\Omega)
$$
Per descrivere le cariche legate all'interno del volumetto può essere comodo immaginare la materia
costituita da tanti dipolini elementari.


È fondamentale rappresentare i dipoli sulle interfacce
dei volumetti, una delle due cariche potrebbe trovarsi
all'esterno del volume di controllo (\textit{come una persona 
bloccata in metro con la gamba fuori}).

Si determina quindi la carica di polarizzazione, ossia la 
carica netta presente in $\Delta\Omega$, tutti i dipoli
all'interno del volume hanno carica netta nulla, perchè 
contengono appunto entrambe le cariche, gli unici che 
contribuiscono sono quelli che hanno una carica all'esterno 
del volume.

L'unico contributo è quindi quello della frontiera di
$\Delta\Omega$ ossia dei dipoli tagliati.

$$
Q_{pol} = -\oiint_{\partial\Delta\Omega} \vec{P}\cdot\hat{n}dS
$$
Il segno $-$ è giustificato dal fatto che il segno del vettore
dipolo punta verso la carica positiva che se si trova all'esterno
del volumetto, ossia il vettore dipolo ë concorde alla normale
alla superficie, lascerà una carica negativa all'interno del 
volume.
Sulla faccia esterna si accumulerà quindi una carica 
$$
\sigma_{pol} = \vec{P}\cdot\hat{n}
$$
Per calcolare la densità di carica per polarizzazione si esegue il 
seguente limite:
$$
\frac{Q_{pol}}{\text{Vol}(\Delta\Omega)} = 
\frac{-\oiint_{\partial\Delta\Omega} \vec{P}\cdot\hat{n}dS}
{\text{Vol}(\Delta\Omega)} \stackrel{\text{Vol}(\Delta\Omega)\to 0}{\Rightarrow} \rho_{pol}(Q,t) = -\nabla\cdot\vec{P}(Q,t)
$$
(In seguito verrà omessa la dipendenza dal tempo perché interessati 
all'elettrostatica)

Considerata una superficie $S$ di discontinuità tra due materiali,
si può calcolare la condizione di raccordo tra le cariche usando 
la tecnica della superficie ``a monetina''
$$
-\hat{n}\cdot(\vec{P}_2-\vec{P}_1) = \sigma_{pol}
$$
dove $\sigma_{pol}$ è la quantità di carica per unità di superficie.

Nel caso in cui si abbia polarizzazione solo nel mezzo $(1)$
allora si ricade nella precedente
$$
\hat{n}\cdot\vec{P}_1 = \sigma_{pol}
$$

Introdotto il vettore polarizzazione è uqindi possibile esprimere
i materiali polarizzati come sedi addizionali di distribuzioni di 
cariche, sia di volume che di superficie.

Il passo successivo è inserire le distribuzioni di carica per 
polarizzazione nelle equazioni di Maxwell

\subsection{Equazioni di Maxwell in presenza di dielettrici}
Si distinguono quindi le cariche ``libere'' da quelle ``legate''
Presa una superficie chiusa $\Sigma$, la legge di Gauss
$$
\oiint_\Sigma \vec{E}\cdot\hat{n}dS = \frac{1}{\varepsilon_0}(Q_{lib}+Q_{pol}) = \frac{Q_{lib}}{\varepsilon_0} - \frac{1}{\varepsilon_0} \oiint_{\Sigma} \vec{P}\cdot\hat{n}dS \ \ \forall\Sigma
$$
Di conseguenza la quantità di carica libera si ricava con
$$
\oiint_{\Sigma} (\varepsilon_0\vec{E}+\vec{P})\cdot \hat{n}dS = 
Q_{lib}
$$
Si introduce quindi un nuovo vettore $\vec{D} = \varepsilon_0\vec{E} + \vec{P}$ chiamato campo spostamento elettrico proprio perché associato allo spostamento delle cariche legate.
$[\vec{D}] = \si{\coulomb\per\meter^2}$

Le equazioni dell'Elettrostatica diventano:
\begin{align*}
&\oiint_{\Sigma}\vec{D}\cdot\hat{n}dS = Q_{lib}\ \ \forall\Sigma \\
&\oint_{\Gamma} \vec{E}\cdot\hat{t}dl = 0 \ \ \forall \Gamma
\end{align*}

Utilizzando il campo di spostamento risultano solo le cariche 
libere, ci si disinteressa quindi delle cariche di polarizzazione,
si hanno però due campi da determinare $\vec{D}$ ed $\vec{E}$.
Continua a valere il principio di conservazione della carica, 
consideriamo in questo caso solo le cariche di polarizzazione.
$$
\oiint_\Sigma\vec{J}_{pol}\cdot\hat{n}dS = 
- \iiint_{\Omega_\Sigma}\frac{\partial}{\partial t} \rho_{pol}dV
= -\iiint_{\Omega_\Sigma} -\frac{\partial}{\partial t} 
(\nabla\cdot\vec{P})dV 
$$
Applicando il teorema di Schwarz
$$
-\iiint_{\Omega_\Sigma} -\frac{\partial}{\partial t} 
(\nabla\cdot\vec{P})dV = \iiint_{\Omega_\Sigma} \nabla\cdot\left(\frac{\partial\vec{P}}{\partial t}\right)dV
$$
e ancora applicando il teorema della divergenza
$$
\iiint_{\Omega_\Sigma} \nabla\cdot\left(\frac{\partial\vec{P}}{\partial t}\right)dV = \oiint_{\Sigma}\frac{\partial \vec{P}}{\partial t} \cdot \hat{n}dS = i_{\Sigma_{pol}} = \oiint_\Sigma\vec{J}_{pol}\cdot\hat{n}dS
$$
Si definisce quindi a livello locale
$$
\vec{J}_{pol} = \frac{\partial \vec{P}}{\partial t}
$$

\paragraph{Legge di Ampére-Maxwell}
$$
\oint_{\Gamma} \vec{B}\cdot\hat{t} dl = \mu_0 \iint_{S_\Gamma}\left( \vec{J}_{lib} + \vec{J}_{pol} + \varepsilon_0\frac{\partial 
\vec{E}}{\partial t}\right) \cdot \hat{n}dS =
\mu_0 \iint_{S_\Gamma}\left[\vec{J}_{lib} + 
\frac{\partial}{\partial t} \left(\varepsilon_0\vec{E}+\vec{P}\right)\right] \cdot \hat{n}dS 
$$
$$
\oint_{\Gamma} \vec{B}\cdot \hat{t} dl = \mu_0 \iint_{S_\Gamma} 
\left(\vec{J}_{lib} + \frac{\partial \vec{D}}{\partial t}\right)\cdot
\hat{n}dS\ \ \forall\ \Gamma
$$
Di conseguenza il termine $\frac{\partial \vec{D}}{\partial t}$ prende
il nome di densità di corrente di spostamento.

\paragraph{Equazioni di Maxwell nei dielettrici in forma locale}
In un volume
\begin{align*}
&\nabla \cdot\vec{D} = \rho_{lib}\ \ \text{in }\Omega\\
&\nabla \times \vec{E} = 0
\end{align*}
Su una superficie di discontinuità
\begin{align*}
&\hat{n}\cdot\left(\vec{D}_2-\vec{D}_1\right) = \sigma_{lib}\\
&\hat{n}\times\left(\vec{E}_2-\vec{E}_1\right) = 0
\end{align*}
Le incognite delle equazioni sono quindi i campi $\vec{D}$ ed 
$\vec{E}$ mentre le sorgenti le cariche libere.

Per chiudere il problema e renderlo ``ben posto'' va specificata
la \textit{relazione costitutiva} del materiale, che leghi
$\vec{D}$ ed $\vec{E}$.

In generale
$$
\vec{D} = \mathbf{D}[\vec{E}]
$$
Dove $D$ è un funzionale agente ``sulla storia'' di $\vec{E}$ 
.
In realtà è associato un funzionale per ogni singola componente
del campo $\vec{D}$.
Il legame può essere non locale nello spazio e nel tempo ossia:
$\vec{D}(\hat{t},Q)$ dipende dalla storia di $\vec{E}$ in tutti 
i punti dello spazio e per $t < \hat{t}$, il legame è quindi
\textbf{anisotropo}, $\vec{D}$ ed $\vec{E}$ possono non essere
paralleli nello stesso punto.

Trascurando la non località, il legame è espresso comunque da una 
funzione non lineare:
$$
\vec{D} = \mathbf{D}(\vec{E}),\ \ \mathbf{D}: \mathbb{R}^3 \to \mathbb{R}^3
$$
Il caso semplice è avere la funzione $\mathbf{D}$ lineare, 
inizialmente tutti i materiali rispondono in 
maniera lineare alle sollecitazioni del campo elettrico, come
il primo termine di una serie di Taylor. Il funzionale $\mathbf{D}$ 
sarà un tensore del secondo ordine: una matrice dipendente dal 
punto e il materiale si dirà non omogeneo.

Se il materiale è \textit{isotropo} allora il tensore è una matrice
diagonale, ossia $\vec{E}$ e $\vec{D}$ sono paralleli.
$$
\mathbf{D} = \varepsilon(Q)I,\ \ I\text{ tensore identità}
$$
$$
\vec{D} = \varepsilon \begin{pmatrix}
                       1 & 0 & 0 \\
                       0 & 1 & 0 \\
                       0 & 0 & 1
                      \end{pmatrix} \cdot \vec{E} = \varepsilon\vec{E}
$$

Se il materiale è omogeneo, $\varepsilon(Q) = $ cost. il materiale
è costituito da una costante chiamata \textit{costante dielettrica}
del materiale, solitamente maggiore di $\varepsilon_0$, per questo
motivo si rappresenta solitamente la caratteristica di un materiale
mediante la costante dielettrica relativa
$$
\varepsilon_r = \frac{\varepsilon}{\varepsilon_0}
$$
\newpage
Valori tipici possono essere:

\begin{table}[H]
\centering
\begin{tabular}[]{c|c}
 Materiale & $\varepsilon_r$ \\ \hline
 Aria & $1 + \SI{6e-4}{}$ \\
 $\text{H}_2\text{O}$ dist. & $80$ \\
 Vetro & $5\div7$
 \end{tabular}
 \caption{Costante dielettrica relativa per alcuni materiali}
\end{table}

L'introduzione della costante dielettrica introduce un legame tra il 
campo elettrico e quello di polarizzazione, ricordando che
$$
\vec{D} = \varepsilon \vec{E}
$$
si ottiene:
$$
\vec{D} = \varepsilon_0\vec{E} + \vec{P} \Rightarrow \vec{P} 
= (\varepsilon-\varepsilon_0)\vec{E} = (\varepsilon_r-1)\varepsilon_0 \vec{E} = \varepsilon_0\chi_e\vec{E}
$$
dove $\chi_e = (\varepsilon_r -1)$ è la suscettività dielettrica.
Dal punto di vista fisico, introdurre il vettore spostamento
elettrico con una relazione costitutiva lineare, omogenea ed 
isotropa, consiste nel dire che il momento di dipolo per unità di 
volume $\vec{P}$ è direttamente proporzionale al campo elettrico
secondo il coefficiente $\varepsilon_0\chi_e$ variabile in funzione
del materiale.

Se il campo elettrico applicato è superiore alla rigidità 
dielettrica, si ha una scarica violenta nel materiale.

Nell'aria a pressione atmosferica la rigidità è \SI{30}{\kilo\volt\per\centi\meter} = \SI{3e6}{\volt\per\meter}

Per vetro e silicati il valore è $40 \div 60 \cdot 10^6$
\si{\volt\per\meter} 

Questi materiali sono fondamentali per le performance dei condensatori,
è possibile aumentare la capacità di un condensatore (a pari geometria)
introducendo un dielettrico tra le armature. Per un condensatore
piano ad esempio:
\begin{align*}
\vec{E} &= -\nabla V \\
\vec{D} &= \varepsilon\vec{E} = -\varepsilon\nabla V \\
\hat{n}\cdot (\vec{D}_2-\vec{D_1}) = \sigma_{lib} &\Rightarrow -\varepsilon\frac{\partial V}{\partial x} = \varepsilon E_x = \sigma_{lib}
\end{align*}
Calcolando la differenza di potenziale tra le due armature
$$
V_1-V_2 = \int_{-\frac{d}{2}}^{\frac{d}{2}}E_x dx =
\int_{-\frac{d}{2}}^{\frac{d}{2}}\frac{\sigma_{lib}}{\varepsilon}dx = 
\left[\frac{\sigma_{lib}}{\varepsilon}x \right]_{-\frac{d}{2}}^{\frac{d}{2}} = \frac{\sigma_{lib}}{\varepsilon}d
$$
$$
Q_1 = \sigma_{lib}S \Rightarrow C = \frac{Q_1}{V_1-V_2} = 
\varepsilon\frac{\sigma_{lib}S}{\sigma_{lib}d} = \varepsilon\frac{S}{d}
= \varepsilon_r C_{\text{vuoto}}
$$

\include{17_condensatori_con_dielettrici}
\subsection{Formulazione debole dell'equazione di Poisson (FEM)}
\begin{equation*}
\begin{cases}
\nabla^2u = f & \text{in }\Omega\\
\left. u \right|_{\partial\Omega} = g
\end{cases}
\end{equation*}
$$
u \in C^2(\Omega),\ f \in C^0(\Omega),\ g\in C^0(\partial\Omega)
$$
In queste condizione la soluzione esiste ed è unica.
La soluzione di classe $C^2$ viene chiamata \textbf{forte} in termini di regolarità
ossia è molto regolare.

Normalmente quando si applica il \textit{metodo degli elementi finiti} (FEM) 
i requisiti sulla funzione incognita $u$ sono meno stringenti e si
parla di soluzione \textbf{debole}.

Si considera uno spazio di funzioni \textit{test} chiamate funzioni
``peso''  $p \in C^1_0\ : \ p(\partial\Omega) = 0$

Moltiplicando ambi i membri dell'equazione di Poisson e integrando si ottiene:
$$
\iiint_\Omega p\nabla^2 u dV = \iiint_\Omega p f dV
$$
Ricordando l'identità vettoriale $p\nabla^2 u = \nabla\cdot \left(p\nabla u\right) - \nabla p\cdot \nabla u $

Sostituendo si ottiene:
$$
\iiint_\Omega p f dV = \iiint_{\Omega} \left(\nabla\cdot\left(p\nabla u\right)-\nabla p\cdot\nabla u\right)dV
$$
Applicando il teorema della divergenza
$$
\iiint_\Omega p f dV = \cancel{\iint_{\partial\Omega} p \frac{\partial u}{\partial n}dS} - \iiint_\Omega \nabla p\cdot\nabla u\ dV
$$
Siccome la funzione $p$ test scelta è nulla sulla frontiera, il primo termine della differenza si azzera:
$$
\iiint_\Omega p f dV = \iiint_\Omega -\nabla p\cdot\nabla u\ dV\ \ \forall\ p\in C^1_0(\Omega)
$$
Quella appena ottenuta è proprio la formazione debole del BVP iniziale.

Introducendo $u_\partial : u_\partial = g\ \text{su } \partial\Omega\ : \ u = \hat{u} + u_\partial$ ossia prolungando
la funzione $g$ all'interno del dominio, si può ricavare la funzione $\hat{u}$ differenza tra le due.

Si vede dunque che $\hat{u}$ è soluzione di un BVP per la stessa equazioni con condizioni di
Dirichlet omogenee, ossia $\hat{u} = 0\ \text{su } \partial\Omega$.

Sostituendo la funzione $u$ estesa nell'integrale precedente si ottiene
$$
-\iiint_\Omega \nabla p \cdot \nabla\hat{u}\  dV = \iiint_\Omega p f dV +
\iiint_\Omega \nabla p \cdot \nabla u_\partial\ dV\ \forall\ p\in C^1_0(\Omega)
$$
I due integrali a destra dell'uguaglianza sono termini noti, l'espressione rappresenta
una famiglia di equazioni integrali nell'incognita $\hat{u}$ al variare
di $p$ nello spazio delle funzioni test.

\newpage
\subparagraph{Metodo di Galerkin}
Consiste nel ricercare una soluzione espressa mediante la base di uno spazio
a dimensione finita chiamato ad esempio $W^0_M$ dove $M$ è il numero di
gradi di libertà da utilizzare ossia quanto più accurata sarà la soluzione,
se al limite $M$ fosse infinito avrei una soluzione identica al continuo.

Supposta una base $\hat{u}_M$ per questo spazio di funzioni che si 
annullano sulla frontiera a dimensione finita $M$
$$
\hat{u}_M = \sum_{k = 1}^{M} a_k w_k \ \ \text{ con } w_k \text{ funzioni di base di } W_m^0
$$
Sostituendo nell'integrale del paragrafo precedente
$$
-\iiint_\Omega \nabla w_h \cdot \nabla \left(\sum_{k=1}^{M} a_k w_k \right) dV = 
\iiint_\Omega w_h f dV + \iiint_\Omega \nabla w_h \cdot \nabla u_\partial\ dV\ \ h=1,...,M
$$
I coefficienti $a_k$ per linearità si possono tirar fuori dall'integrale
$$
-\sum_{k=1}^M \left(\iiint_\Omega \nabla w_h \cdot \nabla w_k\ dV \right) a_k = 
\iiint_\Omega w_h f dV + \iiint_\Omega \nabla w_h \cdot \nabla u_\partial\ dV
$$
1:27:47


Work in progress...
\subparagraph{Si ringraziano infinitamente le persone che hanno contribuito}
\begin{itemize}
\item Tommaso Pio Pergamo
\end{itemize}
\end{document}
