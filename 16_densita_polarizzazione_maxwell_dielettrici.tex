
È fondamentale rappresentare i dipoli sulle interfacce
dei volumetti, una delle due cariche potrebbe trovarsi
all'esterno del volume di controllo (\textit{come una persona 
bloccata in metro con la gamba fuori}).

Si determina quindi la carica di polarizzazione, ossia la 
carica netta presente in $\Delta\Omega$, tutti i dipoli
all'interno del volume hanno carica netta nulla, perché
contengono appunto entrambe le cariche, gli unici che 
contribuiscono sono quelli che hanno una carica all'esterno 
del volume.

L'unico contributo è quindi quello della frontiera di
$\Delta\Omega$ ossia dei dipoli tagliati.

$$
Q_{pol} = -\oiint_{\partial\Delta\Omega} \vec{P}\cdot\hat{n}dS
$$
Il segno $-$ è giustificato dal fatto che il segno del vettore
dipolo punta verso la carica positiva che se si trova all'esterno
del volumetto, ossia il vettore dipolo è concorde alla normale
alla superficie, lascerà una carica negativa all'interno del 
volume.
Sulla faccia esterna si accumulerà quindi una carica 
$$
\sigma_{pol} = \vec{P}\cdot\hat{n}
$$
Per calcolare la densità di carica per polarizzazione si esegue il 
seguente limite:
$$
\frac{Q_{pol}}{\text{Vol}(\Delta\Omega)} = 
\frac{-\oiint_{\partial\Delta\Omega} \vec{P}\cdot\hat{n}dS}
{\text{Vol}(\Delta\Omega)} \stackrel{\text{Vol}(\Delta\Omega)\to 0}{\Rightarrow} \rho_{pol}(Q,t) = -\nabla\cdot\vec{P}(Q,t)
$$
(In seguito verrà omessa la dipendenza dal tempo perché interessati 
all'elettrostatica)

Considerata una superficie $S$ di discontinuità tra due materiali,
si può calcolare la condizione di raccordo tra le cariche usando 
la tecnica della superficie ``a monetina''
$$
-\hat{n}\cdot(\vec{P}_2-\vec{P}_1) = \sigma_{pol}
$$
dove $\sigma_{pol}$ è la quantità di carica per unità di superficie.

Nel caso in cui si abbia polarizzazione solo nel mezzo $(1)$
allora si ricade nella precedente
$$
\hat{n}\cdot\vec{P}_1 = \sigma_{pol}
$$

Introdotto il vettore polarizzazione è quindi possibile esprimere
i materiali polarizzati come sedi addizionali di distribuzioni di 
cariche, sia di volume che di superficie.

Il passo successivo è inserire le distribuzioni di carica per 
polarizzazione nelle equazioni di Maxwell.

\subsection{Equazioni di Maxwell in presenza di dielettrici}
Si distinguono quindi le cariche ``libere'' da quelle ``legate''
Presa una superficie chiusa $\Sigma$, la legge di Gauss
$$
\oiint_\Sigma \vec{E}\cdot\hat{n}dS = \frac{1}{\varepsilon_0}(Q_{lib}+Q_{pol}) = \frac{Q_{lib}}{\varepsilon_0} - \frac{1}{\varepsilon_0} \oiint_{\Sigma} \vec{P}\cdot\hat{n}dS \ \ \forall\Sigma
$$
Di conseguenza la quantità di carica libera si ricava con
$$
\oiint_{\Sigma} (\varepsilon_0\vec{E}+\vec{P})\cdot \hat{n}dS = 
Q_{lib}
$$
Si introduce quindi un nuovo vettore $\vec{D} = \varepsilon_0\vec{E} + \vec{P}$ chiamato campo spostamento elettrico proprio perché associato allo spostamento delle cariche legate.
$[\vec{D}] = \si{\coulomb\per\meter^2}$

Le equazioni dell'Elettrostatica diventano:
\begin{align*}
&\oiint_{\Sigma}\vec{D}\cdot\hat{n}dS = Q_{lib}\ \ \forall\Sigma \\
&\oint_{\Gamma} \vec{E}\cdot\hat{t}dl = 0 \ \ \forall \Gamma
\end{align*}

Utilizzando il campo di spostamento risultano solo le cariche 
libere, ci si disinteressa quindi delle cariche di polarizzazione,
si hanno però due campi da determinare $\vec{D}$ ed $\vec{E}$.
Continua a valere il principio di conservazione della carica, 
consideriamo in questo caso solo le cariche di polarizzazione.
$$
\oiint_\Sigma\vec{J}_{pol}\cdot\hat{n}dS = 
- \iiint_{\Omega_\Sigma}\frac{\partial}{\partial t} \rho_{pol}dV
= -\iiint_{\Omega_\Sigma} -\frac{\partial}{\partial t} 
(\nabla\cdot\vec{P})dV 
$$
Applicando il teorema di Schwarz
$$
-\iiint_{\Omega_\Sigma} -\frac{\partial}{\partial t} 
(\nabla\cdot\vec{P})dV = \iiint_{\Omega_\Sigma} \nabla\cdot\left(\frac{\partial\vec{P}}{\partial t}\right)dV
$$
e ancora applicando il teorema della divergenza
$$
\iiint_{\Omega_\Sigma} \nabla\cdot\left(\frac{\partial\vec{P}}{\partial t}\right)dV = \oiint_{\Sigma}\frac{\partial \vec{P}}{\partial t} \cdot \hat{n}dS = i_{\Sigma_{pol}} = \oiint_\Sigma\vec{J}_{pol}\cdot\hat{n}dS
$$
Si definisce quindi a livello locale
$$
\vec{J}_{pol} = \frac{\partial \vec{P}}{\partial t}
$$

\paragraph{Legge di Ampére-Maxwell}
$$
\oint_{\Gamma} \vec{B}\cdot\hat{t} dl = \mu_0 \iint_{S_\Gamma}\left( \vec{J}_{lib} + \vec{J}_{pol} + \varepsilon_0\frac{\partial 
\vec{E}}{\partial t}\right) \cdot \hat{n}dS =
\mu_0 \iint_{S_\Gamma}\left[\vec{J}_{lib} + 
\frac{\partial}{\partial t} \left(\varepsilon_0\vec{E}+\vec{P}\right)\right] \cdot \hat{n}dS 
$$
$$
\oint_{\Gamma} \vec{B}\cdot \hat{t} dl = \mu_0 \iint_{S_\Gamma} 
\left(\vec{J}_{lib} + \frac{\partial \vec{D}}{\partial t}\right)\cdot
\hat{n}dS\ \ \forall\ \Gamma
$$
Di conseguenza il termine $\frac{\partial \vec{D}}{\partial t}$ prende
il nome di densità di corrente di spostamento.

\paragraph{Equazioni di Maxwell nei dielettrici in forma locale}
In un volume
\begin{align*}
&\nabla \cdot\vec{D} = \rho_{lib}\ \ \text{in }\Omega\\
&\nabla \times \vec{E} = 0
\end{align*}
Su una superficie di discontinuità
\begin{align*}
&\hat{n}\cdot\left(\vec{D}_2-\vec{D}_1\right) = \sigma_{lib}\\
&\hat{n}\times\left(\vec{E}_2-\vec{E}_1\right) = 0
\end{align*}
Le incognite delle equazioni sono quindi i campi $\vec{D}$ ed 
$\vec{E}$ mentre le sorgenti le cariche libere.

Per chiudere il problema e renderlo ``ben posto'' va specificata
la \textit{relazione costitutiva} del materiale, che leghi
$\vec{D}$ ed $\vec{E}$.

In generale
$$
\vec{D} = \mathbf{D}[\vec{E}]
$$
Dove $D$ è un funzionale agente ``sulla storia'' di $\vec{E}$.

In realtà è associato un funzionale per ogni singola componente
del campo $\vec{D}$.
Il legame può essere non locale nello spazio e nel tempo ossia:
$\vec{D}(\hat{t},Q)$ dipende dalla storia di $\vec{E}$ in tutti 
i punti dello spazio e per $t < \hat{t}$, il legame è quindi
\textbf{anisotropo}, $\vec{D}$ ed $\vec{E}$ possono non essere
paralleli nello stesso punto.

Trascurando la non località, il legame è espresso comunque da una 
funzione non lineare:
$$
\vec{D} = \mathbf{D}(\vec{E}),\ \ \mathbf{D}: \mathbb{R}^3 \to \mathbb{R}^3
$$
Il caso semplice è avere la funzione $\mathbf{D}$ lineare, 
inizialmente tutti i materiali rispondono in 
maniera lineare alle sollecitazioni del campo elettrico, come
il primo termine di una serie di Taylor. Il funzionale $\mathbf{D}$ 
sarà un tensore del secondo ordine: una matrice dipendente dal 
punto e il materiale si dirà non omogeneo.

Se il materiale è \textit{isotropo} allora il tensore è una matrice
diagonale, ossia $\vec{E}$ e $\vec{D}$ sono paralleli.
$$
\mathbf{D} = \varepsilon(Q)I,\ \ I\text{ tensore identità}
$$
$$
\vec{D} = \varepsilon \begin{pmatrix}
                       1 & 0 & 0 \\
                       0 & 1 & 0 \\
                       0 & 0 & 1
                      \end{pmatrix} \cdot \vec{E} = \varepsilon\vec{E}
$$

Se il materiale è omogeneo, $\varepsilon(Q) = $ cost. il materiale
è costituito da una costante chiamata \textit{costante dielettrica}
del materiale, solitamente maggiore di $\varepsilon_0$, per questo
motivo si rappresenta solitamente la caratteristica di un materiale
mediante la costante dielettrica relativa
$$
\varepsilon_r = \frac{\varepsilon}{\varepsilon_0}
$$
\newpage
Valori tipici possono essere:

\begin{table}[H]
\centering
\begin{tabular}[]{c|c}
 Materiale & $\varepsilon_r$ \\ \hline
 Aria & $1 + \SI{6e-4}{}$ \\
 $\text{H}_2\text{O}$ dist. & $80$ \\
 Vetro & $5\div7$
 \end{tabular}
 \caption{Costante dielettrica relativa per alcuni materiali}
\end{table}

L'introduzione della costante dielettrica introduce un legame tra il 
campo elettrico e quello di polarizzazione, ricordando che
$$
\vec{D} = \varepsilon \vec{E}
$$
si ottiene:
$$
\vec{D} = \varepsilon_0\vec{E} + \vec{P} \Rightarrow \vec{P} 
= (\varepsilon-\varepsilon_0)\vec{E} = (\varepsilon_r-1)\varepsilon_0 \vec{E} = \varepsilon_0\chi_e\vec{E}
$$
dove $\chi_e = (\varepsilon_r -1)$ è la suscettività dielettrica.
Dal punto di vista fisico, introdurre il vettore spostamento
elettrico con una relazione costitutiva lineare, omogenea ed 
isotropa, consiste nel dire che il momento di dipolo per unità di 
volume $\vec{P}$ è direttamente proporzionale al campo elettrico
secondo il coefficiente $\varepsilon_0\chi_e$ variabile in funzione
del materiale.

Se il campo elettrico applicato è superiore alla rigidità 
dielettrica, si ha una scarica violenta nel materiale.

Nell'aria a pressione atmosferica la rigidità è \SI{30}{\kilo\volt\per\centi\meter} = \SI{3e6}{\volt\per\meter}

Per vetro e silicati il valore è $40 \div 60 \cdot 10^6$
\si{\volt\per\meter} 

Questi materiali sono fondamentali per le performance dei condensatori,
è possibile aumentare la capacità di un condensatore (a pari geometria)
introducendo un dielettrico tra le armature. Per un condensatore
piano ad esempio:
\begin{align*}
\vec{E} &= -\nabla V \\
\vec{D} &= \varepsilon\vec{E} = -\varepsilon\nabla V \\
\hat{n}\cdot (\vec{D}_2-\vec{D_1}) = \sigma_{lib} &\Rightarrow -\varepsilon\frac{\partial V}{\partial x} = \varepsilon E_x = \sigma_{lib}
\end{align*}
Calcolando la differenza di potenziale tra le due armature
$$
V_1-V_2 = \int_{-\frac{d}{2}}^{\frac{d}{2}}E_x dx =
\int_{-\frac{d}{2}}^{\frac{d}{2}}\frac{\sigma_{lib}}{\varepsilon}dx = 
\left[\frac{\sigma_{lib}}{\varepsilon}x \right]_{-\frac{d}{2}}^{\frac{d}{2}} = \frac{\sigma_{lib}}{\varepsilon}d
$$
$$
Q_1 = \sigma_{lib}S \Rightarrow C = \frac{Q_1}{V_1-V_2} = 
\varepsilon\frac{\sigma_{lib}S}{\sigma_{lib}d} = \varepsilon\frac{S}{d}
= \varepsilon_r C_{\text{vuoto}}
$$
