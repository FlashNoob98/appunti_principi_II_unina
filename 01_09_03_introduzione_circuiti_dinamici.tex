\section{Introduzione ai circuiti dinamici}
Circuiti lineari tempo varianti (LTV)

Tutti i circuiti che contengono componenti come
resistori, induttori e condensatori, generatori indipendenti di tensione e corrente ai quali assumiamo 
una tensione impressa o una corrente impressa.
Vi sono inoltre bipoli tempo-varianti come interruttori che si chiudono o si aprono.

Preso un generico circuito, vogliamo determinare una tensione o un'intensità di corrente di un generico
bipolo.
\textit{(Nella vita bisogna sempre porsi un obiettivo)}

\textbf{Strumenti a disposizione:}

\begin{itemize}
\item Equazioni di interconnessione:
        LKC $ sum i_{k}(t) = 0 per ogni t $
        LKT  $ per ogni maglia del grafo sum_h (\pm)v_n(t) = 0 per ogni t$
\item Relazioni caratteristiche dei bipoli coerenti con la scelta dei versi delle grandezze (convenzione del generatore o dell'utilizzatore)
per ogni resistore avremo una caratteristica $V_r =R$ per ogni induttore $V_l = l\frac{di}{dt} [\si{\henry}]$ per ogni C $i_C = C\frac{dv_c}{dt} [\si{\farad}]$
per ogni generatore di tensione $v_e = e(t)$ e $i_j = j(t)$
per gli interruttori in chiusura a $t=0$ {t<0 i = 0 per ogni v; t> 0 v = 0 per ogni i}
per gli interruttori in apertura (duale) 
\end{itemize}

Le equazioni di interconnessione non  %27:11

NOTA: per ogni induttore $ P^a(t) = v_l\cdot i_l = L \frac{di_l}{dt}\cdot i_l = \frac{d}{dt} [\frac{1}{2}Li_l^2] = \frac{dW_m}{dt} $ energia immagazzinata
nel campo magnetico dell'induttore
$\Delta W^a(t_1,t_2) = \int_t1^t2 P^a(\tau)d\tau = W_m (t_2) - W_m(t_1) $
Se in un certo istante di tempo l'induttore presenta una certa energia in Joule [\si{\joule}] quella sarà la massima energia estraibile
dall'induttore.

Per ogni condensatore $P^a(t) = \frac{d}{dt} W_e$, $W_e(t) = \frac{1}{2} C_{V_c}^2$

Si ha quindi un sistema di equazioni circuitali in cui si ha una parte algebrica con le caratteristiche adinamiche, ossia con caratteristiche
non differenziali e non integrali, più una parte differenziale data dai bipoli dinamici come condensatori e induttori
In letteratura un sistema simile si indica con DAE (Differential Algebric Equation), differente
dalla ODE (Ordinary Differential Equation).

Le tecniche utilizzate in teoria dei circuiti mirano a trasformare una DAE in una ODE, ossia 
per formulare l'equazioni circuitali come equazioni differenziali ordinaria, per operare questa trasformazione si fa riferimento
alla dinamica delle sole {variabili di stato}: $i_l(t) v_c(t)$

la loro conoscenza (delle variabili di stato) permette di esprimere tutte le altre variabili attraverso relazioni algebriche, vengono definite
variabili slave, perchè subordinate alle prime (?) %42:38


Per eseguire ciò si richiama una procedura generale per l'analisi di circuiti lineari tempo varianti (LTV)
Questa procedura si basa su un'analisi a intervallo, si partiziona l'asse dei tempi in intervalli in ciascuno dei quali esiste un circuito 
tempo invariante (LTI) equivalente a quello di partenza

Figura 46:17

le grandezze di stato come la tensione $V_c$ nell'istante $t=0$?

Supponiamo che i generatori indipendenti abbiano grandezze (potenza) \textbf{limitate}, ossia le tensioni impresse $e(t)$ e le correnti impresse $j(t)$,
in questa ipotesi sappiamo che le variabili di stato sono funzioni \textbf{continue} ossia:
\begin{equation*}
\begin{split}
i_L (0^+) & = i_L(0^-) \\ 
V_C (0+) & = V_C(0^-)
\end{split}
\end{equation*}

La soluzione si determina trovando la dinamica delle variabili di stato in ciascun circuito ausiliario "incollando" le soluzioni utilizzando la proprietà di continuità
% poco chiaro vedi 52:41

Il problema di risolvere circuiti lineari tempo varianti si scompone nel risolvere i circuiti lineari tempo invarianti, che si scompone nel risolvere
circuiti lineari del primo ordine (circuiti RC o RL)
Figure 55:26

Un bipolo adinamico lineare e un bipolo dinamico fanno subito pensare all'utilizzo dei teoremi di Thevenin e Norton, permettendo la riduzione del circuito 
adinamico ad un semplice generatore con un resistore equivalente.

Applicando ad esempio la LKT $e_0(t) = R_{th} C \frac{dV_C}{dt} + V_C$ equazione di stato
$V_C(t=0) = V_0   \Rightarrow V_c(t) = [V_0-v_{cp}(0)] e ^{-\frac{t}{\tau}}$ dove $\tau = R_{th}C$
 
Dopo un intervallo pari a $4\sim5 \tau$ si assume il processo di carica o scarica terminato

Circuito RL $i_l(t) = [I_0-Ilp(0)]e^{-\frac{t}{\tau}} + Ilp(t) con \tau = \frac{L}{R_th}, I_0 = i_L (t=0)$

Osservazione: la soluzione generale del circuito RC, ossia la dinamica di $V_c(t)$ può essere espressa come la somma di due termini $V_{C_{tr}}(t)$ e $V_{C_p}(t)$ ossia
uno transitorio, che tende a svanire se attendiamo un tempo sufficiente lungo, porta con se la \textit{memoria} dello stato iniziale, memoria che viene
persa quando $t>4\sim5\tau$, ammesso che $\tau$ sia positiva; esistono infatti alcune combinazioni di elementi circuitali si comportano come un resistore negativo.

Il secondo termine è quello di regime permanente, che ovviamente non ha memoria dello stato iniziale ma dipende solo dalla nuova configurazione
del circuito (termine forzato).

La decomposizione in regime transitorio e permanente è una decomposizione generale che vale per qualsiasi circuito, a patto di complicare la matematica,
per quanto riguarda il circuito transitorio.

Si parla inoltre di evoluzione libera ed evoluzione forzata 
$$\begin{cases}
e_0(t) = R_thC\frac{dV_c}{dt} + V_C \\
V_c(0) = V_0
\end{cases}$$

si scompone in:

$$\begin{cases} %evoluzione libera
0 = R_thC\frac{dV_c}{dt} + V_C \\ 
V_c(0) = V_0
\end{cases}$$

$$\begin{cases} %evoluzione forzata
e_0(t) = R_thC\frac{dV_c}{dt} + V_C \\
V_c(0) = \SI{0}{\volt}
\end{cases}$$

Trattazione analoga (duale) per il circuito RL

\section{Circuiti lineari tempo invarianti del II ordine}
Divisi in (RC, RL, RLC), la procedura generale di risoluzione richiede tre step:
\begin{enumerate}
 \item Determinazione delle equazioni di stato
 \item Determinazione delle condizioni iniziali
 \item Soluzione del problema di Cauchy
\end{enumerate}

Esempio: Circuito 1:18:00

LA dinamica dello stato per t<0 è di facile risoluzione, $i_L(t) = 0$, $v_C(t) = 0$
per t >  0 invece si analizza il circuito:
Equazioni di interconnessione: LKC nel nodo evidenziato in rosso (tra i due resistori)
LKC: $i_1 = i_c+i_L$
LKT: $E = R_1 i_1 + V_c$
$V_c = R_2 i_L + v_L$
$v_L = L\frac{di_L}{dt}$
$i_c = C\frac{dV_c}{dt}$

Ricaviamo I1 dalla seconda e sostituiamola nella prima,
ora vanno ricavate le variabili di stato ottenendo
$$
\begin{cases}
i_C = \frac{E}{R_2} - \frac{v_C}{R_1} - i_L \\
v_L = v_C - R_2 i_L
\end{cases}
$$
Sostituendo le equazioni caratteristiche dei bipoli dinamici:

$$
\begin{cases}
C\frac{dv_C}{dt} = \frac{E}{R_2} - \frac{v_C}{R_1} - i_L \\
L\frac{di_L}{dt} = v_C - R_2 i_L
\end{cases}
$$

Le condizioni iniziali sono 
$$\begin{cases}
v_C(0^+) = v_C(0^-) = 0\\
i_L(0^+) = i_L(0^-) = 0
\end{cases}
$$
Per risolvere queste equazioni conviene ridurre l'equazione del secondo ordine ad una sola delle incognite,
ad esempio si sostituisce nell'equazione che definisce $i_C$, $v_C$ ricavata dalla equazione 2.
In questo modo si ha un'unica equazione in cui compaiono le grandezze relative all'induttore

Si ottiene 
\begin{equation}
LC \frac{d^2i_L}{dt^2} + CR_2\frac{di_L}{dt} = \frac{E}{R_1} - \frac{L}{R_1}\frac{di_L}{dt} - \frac{R_2}{R_1}i_L-i_L
\end{equation}
%ora raccogli i termini
Controlli da eseguire: controllo dimensionale, positività dei coefficienti altrimenti il circuito non sarebbe dissipativo.

L'integrale generale è scritto come somma dell'integrale dell'omogenea associata e dell'integrale particolare.
Polinomio caratteristico:
\begin{equation}
 \lambda^2 + (R_2/L + 1/R_1 C)\lambda + (1+\frac{R_2}{R1})\frac{1}{LC} = 0
\end{equation}

Vedi figura 01:38 modi naturali aperiodici smorzati

Caso 2 radici reali e coincidenti, caso piuttosto patologico
la radice viene determinata con $-\sigma$ avremo un modo esponenziale decrescente $e^{-\sigma t} $ con $\tau = \frac{1}{\sigma}$
e un modo pari a  $te^{-\sigma t}$


Caso 3 modo periodico smorzato
$\lambda_{1,2} = -\sigma \pm j\omega d$
$i_{L_0}(t) = e^{-\sigma t} [K_1 \cos (\omega_d t) + K_2 \sin(\omega_d t)]$

Restano da determinare le costanti di integrazione imponendo le condizioni iniziali,
\begin{equation*}
\begin{cases}
i_L(0^+) = i_L(0^-) \\
\frac{d_{i_L}}{dt}(0^+) = \frac{1}{L}[v_C(0^+) - R_2I_L(0^+)] = 0
\end{cases}
\end{equation*}

Inserisci ultima equazione
2 pi su omega d distana tra 2 picchi 
