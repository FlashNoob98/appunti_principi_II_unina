
Si può sfruttare il fatto che il potenziale $V_2$ deve essere normale all'infinito, ossia
$\lim_{r\to\infty} V_2(r) = 0 \Rightarrow D = 0$.

La $V_1(r)$ deve essere invece definita e continua per ogni valore di $r$, in particolare per $r=0$
ciò implica che $A=0$.

Quindi 
$$V_2(r) = -\frac{C}{r} \ \  V_1(r) = B$$

Si devono utilizzare altre due condizioni per ricavare la costante: la prima è la continuità del 
potenziale, e l'altra è la discontinuità della componente normale del campo elettrico 
sulla superficie.

$$
V_1(R) = V_2(R) = B = -\frac{C}{R}
$$

$$
\left(\frac{\partial V_2}{\partial r} - \frac{\partial V_1}{\partial r}\right)_{r=R} = -\frac{\sigma_0}{\varepsilon_0} \stackrel{\partial V_1 = 0}{=}
\frac{C}{R^2} \Rightarrow C = -\frac{\sigma_0}{\varepsilon_0}R^2
$$
Quindi
$$
\begin{aligned}
V_2(r) &= \frac{\sigma}{\varepsilon_0}\frac{R^2}{r}\\
V_1(r) &= \frac{\sigma}{\varepsilon_0}R
\end{aligned}
$$
$$
V(r) = \begin{cases}
\frac{\sigma_0}{\varepsilon_0}R,\ 0\leq r \leq R \\
\frac{\sigma_0}{\varepsilon_0}\frac{R^2}{r},\ r\geq R
\end{cases}\Rightarrow
E_r(r) = -\frac{\partial V}{\partial r} =
\begin{cases}
0,\ 0\leq r \leq R \\
\frac{\sigma_0}{\varepsilon_0}\frac{R^2}{r^2},\ r\geq R
\end{cases}
$$

Dividendo ambo i membri per $4\pi R$
$$
V(r) = \begin{cases}
\frac{\sigma_0}{4\pi R \varepsilon_0}4\pi R^2 = \frac{Q}{4\pi\varepsilon_0 R},\ 0\leq r \leq R \\
\frac{\sigma_0}{4\pi r \varepsilon_0}4\pi R^2 = \frac{Q}{4\pi\varepsilon_0}\frac{1}{r},\ r\geq R
\end{cases}
\Rightarrow E_r(r) =\begin{cases}
0,\ 0\leq r \leq R \\
\frac{Q}{4\pi\varepsilon_0}\frac{1}{r^2},\ r\geq R
\end{cases}
$$

Per $r\geq R$ la distribuzione di carica si comporta come se la carica $Q = 4 \pi R^2 \sigma_0$ fosse concentrata
nell'origine degli assi.

Il potenziale è quindi costante all'interno della sfera e decresce con andamento $1/r$ all'esterno
della stessa, ossia per $r \geq R$ come mostrato in figura \ref{fig:potenziale_sfera_carica}.
\begin{figure}[h] % plot di un grafico 2D andamento del potenziale elettrico
\centering
\begin{tikzpicture}
\begin{axis}[
%axis lines = left,
axis x line = center,
axis y line = middle,
xlabel = $r$,
ylabel = $V(r)$,
ymax = 1.5,
ymin = 0,
xtick = {0,1},
xticklabels = {$0$,$R$},
ytick = {0,1,2},
yticklabels = {$0$ , $\frac{Q}{4\pi\varepsilon_0 R}$, }
]

\addplot [
domain = 0:1,
samples = 2,
]{1};

\addplot[
domain = 1:4,
samples = 60,
] {x^-1};

\node[] at (axis cs: 2.4,0.6) {$\sim \frac{Q}{4\pi\varepsilon_0 r}$};
\end{axis}
\end{tikzpicture}
\caption{Andamento del potenziale attraverso la sfera carica}
\label{fig:potenziale_sfera_carica}
\end{figure}

Ricordando che $\vec{E} = -\nabla V$ ma il potenziale è in questo caso una funzione 
monodimensionale, il gradiente si calcola con una semplice derivata, il risultato
è mostrato in figura \ref{fig:campo_sfera_carica}.
\begin{figure}[h] % plot di un grafico 2D andamento del campo elettrico
\centering
\begin{tikzpicture}
\begin{axis}[
%axis lines = left,
axis x line = center,
axis y line = middle,
xlabel = $r$,
x label style ={anchor=north},
ylabel = $E(r)$,
ymax = 1.5,
ymin = 0,
xtick = {0,1},
xticklabels = {$0$,$R$},
ytick = {0,1,2},
yticklabels = {$0$ , $\frac{\sigma_0}{\varepsilon_0}$, }
]

\addplot [
color = red,
style = ultra thick,
domain = 0:1,
samples = 2,
]{0};

\addplot[
color = red,
style = thick,
domain = 1:4,
samples = 60,
] {x^-2};

\addplot [dashed,color = red] coordinates {(1,0) (1,1)};

\node[] at (axis cs: 2,0.6) {$\sim \frac{Q}{4\pi\varepsilon_0 r^2}$};
\end{axis}
\end{tikzpicture}
\caption{Andamento del campo attraverso la sfera carica}
\label{fig:campo_sfera_carica}
\end{figure}

\newpage
\subparagraph{Distribuzione superficiale con simmetria cilindrica}
Si supponga di avere una superficie cilindrica indefinita lungo l'asse $z$, di raggio $R$.
La densità di carica è costante sulla superficie: $\sigma(R,\varphi,z) = \sigma_0$, ne consegue che
analogamente alla sfera, la funzione potenziale dipende solo da $r$: $V(p)=V(r)$.

Definite con $1$ la regione interna al cilindro e $2$ quella esterna si modella il problema:
$$
\begin{cases}
\nabla^2 V_1 = 0, & 0\leq r\leq R\\
V_1(R) = V_2(R) & \text{continuità su }\sigma \\
\frac{\partial V_2}{\partial r} - \frac{\partial V_1}{\partial r} = -\frac{\sigma_0}{\varepsilon_0} & \text{su } \Sigma\\
\nabla^2 V_2 = 0, & r\geq R\\
\lim_{r\to \infty}V_2(r) = 0 & \text{normale all'infinito}
\end{cases}
$$

Eseguiamo il Laplaciano di $V$ in coordinate cilindriche:
$$
\frac{1}{r}\frac{\partial}{\partial r} \left(r\frac{\partial V}{\partial r}\right) = 0 \begin{cases}
\frac{1}{r}\left(r \frac{\partial^2 V_1}{\partial r^2} + 
\frac{\partial V_1}{\partial r}\right) = 0\Rightarrow V''_1 + 
\frac{1}{r}V'_1 = 0\\
\frac{1}{r}\left(r \frac{\partial^2 V_2}{\partial r^2} + 
\frac{\partial V_2}{\partial r}\right) = 0\Rightarrow V''_2 + 
\frac{1}{r}V'_2 = 0
\end{cases}
$$

Risolvendo le equazioni differenziali:
\begin{align*}
V_1(r) &= A \ln r + B\\
V_2(r) &= C \ln r + D
\end{align*}

\begin{align*}
V_1(r) \text{ definito e continuo } 
\forall r\leq R & \Rightarrow A = 0 \Rightarrow V_1(r) = B 
\text{ costante}\\
V_2(r) \text{ normale all'infinito} &\Rightarrow D = \ln r_0\\
V_1(r) = V_2(r) &= B = -\frac{\sigma_0}{\varepsilon_0}R\ln R + D\\
\frac{\partial V_2}{\partial r} - \frac{\partial V_1}{\partial r} &= -\frac{\sigma_0}{\varepsilon_0}
\Rightarrow C = -\frac{\sigma_0 R}{\varepsilon_0}
\end{align*}
quindi
$$
V(r) = \begin{cases}
-\frac{\sigma_0 R}{\varepsilon_0} \ln R + D,& r < R\\
-\frac{\sigma_0 R}{\varepsilon_0} \ln r + D,& r \geq R
\end{cases}
$$
ma
$$
V_1(R) = V_2(R) = 0\Rightarrow D = \frac{\sigma_0 R}{\varepsilon_0} \ln R
$$
Derivando
$$
E_r (r) = -\frac{\partial V}{\partial r} = \begin{cases}
0 & r < R\\
\frac{\sigma_0 R}{\varepsilon_0}\frac{1}{r}& r\geq R
\end{cases}
$$

Si confronta il campo trovato con la legge di Gauss per ricavare la densità di carica $\lambda$.

$$
\oiint_\Sigma \vec{E}\cdot\hat{n}dS = \frac{Q_{\Omega_\Sigma}}{\varepsilon_0}
$$
$$
2 \pi r L E_r = \frac{\sigma}{\varepsilon_0} 2 \pi R L,\ r > R
$$
$$
E_r = \frac{1}{2 \pi \varepsilon_0} \frac{\sigma R 2 \pi}{r} = \frac{\lambda}{2\pi\varepsilon_0}\frac{1}{r}
$$
$$
\lambda = \sigma 2 \pi R
$$
Come se la carica fosse contenuta tutta sull'asse di simmetria del cilindro.


\subparagraph{Distribuzione di carica superficiale a simmetria piana}
Potenziale di un doppio strato di cariche.

Prese due lastre cariche $\sigma_0$ e $-\sigma_0$ poste ad una distanza $\Delta/2$ dall'asse $Z$.
Per simmetria la funzione potenziale sarà funzione di $x$.
$$
V(P) = V(x)
$$
Dividendo lo spazio in 3 regioni, si definiscono 3 funzioni potenziali:
$$
\begin{cases}
\nabla^2V_1 = 0 &x < \frac{\Delta}{2}\\
\nabla^2V_2 = 0 & |x| < \frac{\Delta}{2}\\
\nabla^2V_3 = 0 & x > \frac{\Delta}{2}\\
V_1(-\frac{\Delta}{2}) = V_2(-\frac{\Delta}{2}) \\
V_2(\frac{\Delta}{2}) = V_3(\frac{\Delta}{2}) \\
-\left.\frac{\partial V_3}{\partial x}\right|_{\frac{\Delta}{2}} + \left. \frac{\partial V_2}{\partial x}\right|_{\frac{\Delta}{2}} = \frac{\sigma_0}{\varepsilon_0} \\
+\left. \frac{\partial V_1}{\partial x}\right|_{-\frac{\Delta}{2}}  \left.-\frac{\partial V_2}{\partial x}\right|_{-\frac{\Delta}{2}} = -\frac{\sigma_0}{\varepsilon_0}\
\end{cases}
$$
Analizzando la prima equazione:

\begin{align*}
&\nabla^2V = \frac{d^2V}{dx^2} = 0 \Rightarrow \\
&V_1(x) = Ax+B \\
&V_2(x) = Cx+D \\
&V_3(x) = Ex + F \\
&V_1(x) \to 0, \ x\to -\infty \Rightarrow A = 0 \\
&V_3(x) \to 0, \ x\to +\infty \Rightarrow E = 0 
\end{align*}

Imponendo le condizioni di raccordo:
$$
\begin{aligned}
-\frac{\partial V_3}{\partial x} + \frac{\partial V_2}{\partial x} = \frac{\sigma_0}{\varepsilon_0} 
&\Leftrightarrow \cancel{-E} + C = \frac{\sigma_0}{\varepsilon_0}\\
\frac{\partial V_1}{\partial x} - \frac{\partial V_2}{\partial x} = -\frac{\sigma_0}{\varepsilon_0}
&\Leftrightarrow \cancel{-A} - C = -\frac{\sigma_0}{\varepsilon_0}
\end{aligned} \Rightarrow C = \frac{\sigma_0}{\varepsilon_0} \Rightarrow V_2(x) = \frac{\sigma_0}{\varepsilon_0}x + D
$$

Imponiamo la continuità del potenziale sulle due superfici di carica
\begin{align*}
V_2\left(\frac{\Delta}{2}\right) = \frac{\sigma_0}{\varepsilon_0} \frac{\Delta}{2} + D = V_3\left(\frac{\Delta}{2}\right) = F\\
V_1\left(-\frac{\Delta}{2}\right) = B = V_2\left(-\frac{\Delta}{2}\right) = -\frac{\sigma_0}{\varepsilon_0} \frac{\Delta}{2} + D = \frac{\sigma_0}{\varepsilon_0} \frac{\Delta}{2} - D \Rightarrow D = 0
\end{align*}
Tirando le somme come mostrato in figura \ref{fig:potenziale_lastre_cariche}:
$$
V(x) = \begin{cases}
-\frac{\sigma_0}{\varepsilon_0}\frac{\Delta}{2} & x \leq -\frac{\Delta}{2} \\
\frac{\sigma_0}{\varepsilon_0}x & |x| \leq \frac{\Delta}{2} \\
\frac{\sigma_0}{\varepsilon_0} \frac{\Delta}{2} & x\geq \frac{\Delta}{2}
\end{cases}
$$
\begin{figure}[h] % plot di un grafico 2D andamento del potenziale elettrico
\centering
\begin{tikzpicture}
\begin{axis}[
%axis lines = left,
axis x line = center,
axis y line = middle,
xlabel = $x$,
ylabel = $V(x)$,
ymax = 1.5,
ymin = -1.5,
xtick = {-1,0,1},
xticklabels = {$-\frac{\Delta}{2}$,$0$,$\frac{\Delta}{2}$},
ytick = {-1,0,1},
yticklabels = {$-\frac{\sigma_0}{\varepsilon_0}\frac{\Delta}{2}$,$0$ , $\frac{\sigma_0}{\varepsilon_0}\frac{\Delta}{2}$ },
xticklabel style = {color = red}
]

\addplot [
domain = -1:1,
samples = 2,
color = teal,
]{x};

\addplot[
domain = 1:2,
samples = 2,
color = teal,
] {1};
\addplot[
domain = -2:-1,
samples = 2,
color = teal,
] {-1};

\addplot [dashed,red] coordinates {(1,0) (1,1)};
\addplot [dashed,red] coordinates {(-1,0) (-1,-1)};

\node[] at (axis cs: 2.4,0.6) {$\sim \frac{Q}{4\pi\varepsilon_0 r}$};
\end{axis}
\end{tikzpicture}
\caption{Andamento del potenziale attraverso le lastre cariche}
\label{fig:potenziale_lastre_cariche}
\end{figure}

Analizzando il campo $E_x$
$$
E_x = -\frac{\partial V}{\partial x} = \begin{cases}
0 & x < -\frac{\Delta}{2} \\
-\frac{\sigma}{\varepsilon_0} & |x| < \frac{\Delta}{2} \\
0 & x > \frac{\Delta}{2}
\end{cases}
$$

\begin{figure}[h] % plot di un grafico 2D andamento del potenziale elettrico
\centering
\begin{tikzpicture}
\begin{axis}[
%axis lines = left,
axis x line = center,
axis y line = middle,
xlabel = $x$,
ylabel = $E(x)$,
ymax = 1,
ymin = -1.5,
xtick = {-1,0,1},
xticklabels = {$-\frac{\Delta}{2}$,$0$,$\frac{\Delta}{2}$},
ytick = {-1,0},
yticklabels = {$-\frac{\sigma_0}{\varepsilon_0}$,$0$ ,  },
xticklabel style = {color = red}
]

\addplot [
domain = -1:1,
samples = 2,
color = teal,
]{-1};

\addplot[
domain = 1:2,
samples = 2,
color = teal,
] {0};
\addplot[
domain = -2:-1,
samples = 2,
color = teal,
] {0};

\addplot [dashed,red] coordinates {(1,0) (1,-1)};
\addplot [dashed,red] coordinates {(-1,0) (-1,-1)};

\node[] at (axis cs: 2.4,0.6) {$\sim \frac{Q}{4\pi\varepsilon_0 r}$};
\end{axis}
\end{tikzpicture}
\caption{Andamento del campo attraverso le lastre cariche}
\label{fig:campo_lastre_cariche}
\end{figure}

Passando al limite per $\Delta \to 0^+$ si ottiene uno strato doppio di carica di valori
uguali ed opposti, come uno strato superficiale continuo di dipoli elementari.

$$
V(x) = \begin{cases}
-\frac{\tau}{2\varepsilon_0} & x<0 \\
\frac{\tau}{2\varepsilon_0} & x>0
\end{cases}
$$
con $\sigma_0\Delta = \tau$ il momento di dipolo per unità di superficie.
Nei punti regolari $E_x(x) = 0\ \forall x \neq 0 $ ossia il potenziale è discontinuo
attraverso lo strato mentre il campo no.
Confrontandolo con le equazioni dello strato semplice carico:
$$
E_x(x) = \begin{cases}
\frac{\sigma_0}{2\varepsilon_0} & x > 0\\
- \frac{\sigma_0}{2 \varepsilon_0} & x < 0
\end{cases} \Rightarrow V(x) = \begin{cases}
-\frac{\sigma_0}{2\varepsilon_0}x & x > 0 \\
\frac{\sigma_0}{2\varepsilon_0}x & x<0
\end{cases}
$$

\subsection{Elettrostatica in presenza di conduttori}
Si suppone che ci siano N conduttori (elettrodi) nello spazio in \textit{``equilibrio''}.
Si parla di conduttore ohmico, ossia che soddisfa la legge di Ohm in forma locale:
$$
\vec{J}(P) = \gamma(P)\vec{E}(P)
$$
oppure
$$
\vec{E}(P) = \eta(P)\vec{J}(P)
$$
con $\gamma$ ed $\eta$ conducibilità e resistività elettrica.
I un conduttore in equilibrio $\vec{J} = 0$ o equivalentemente $\vec{E} = 0$, ciò implica 
che la funzione potenziale è costante in $\Omega$ fin sulla frontiera del mezzo conduttore.

$\partial \Omega$ è una superficie equipotenziale di $V \stackrel{\text{Principio del max.}}{\Rightarrow} V = \text{ cost in } \Omega$.

Siccome $\vec{E} = -\nabla V$ è quindi sempre perpendicolare alla frontiera di un conduttore,
il modulo e il verso di $\vec{E}$ subito al di fuori di una regione conduttiva $\Omega$ 
si ricavano applicando le condizioni di raccordo per il campo elettrico:
$$
\hat{n}\cdot (\vec{E}_2-\vec{E}_1) = \frac{\sigma}{\varepsilon_0}
$$
ma $\vec{E}_1 = 0$ perché il conduttore è all'equilibrio, quindi 
$$
\vec{E}_2 = \frac{\sigma}{\varepsilon_0} \hat{n}
$$
Questo risultato prende il nome di Teorema di Coulomb e permette di calcolare il campo elettrico al 
di fuori di una superficie conduttiva all'equilibrio senza conoscerne l'effettiva distribuzione di 
carica.
Con la stessa legge di Gauss si può dimostrare che non c'è alcuna carica concentrata all'interno
del conduttore dato che si otterrebbe flusso nullo, ciò implica che la carica è interamente
distribuita sulla superficie.

\subsection{Capacità di un conduttore isolato} (rispetto all'infinito)
Preso un conduttore generico carico, il campo al suo interno sarà pari a 0, eseguendo un integrale
esteso alla sua frontiera:
$$
\iint_{\partial \Omega}\sigma dS = \iint_{\partial \Omega} \varepsilon_0 \vec{E}\cdot \hat{n} dS
$$
eseguiamo il rapporto con l'integrale di linea del campo elettrico da un punto sulla superficie all'
infinito dove si assume il campo nullo.

$$
\frac{\iint_{\partial \Omega}\sigma dS }{\int_P^\infty \vec{E}\cdot\hat{t}dl } = \frac{Q}{V} \stackrel{\text{def}}{=} C
$$
si ottiene la capacità del conduttore isolato per definizione.
Si osserva che il potenziale $V$ sarà necessariamente proporzionale alla carica $Q$, si 
conclude che il rapporto è indipendente dalla carica contenuta nel corpo, ma dipenderà soltanto da
parametri materiali e geometrici.
\subparagraph{Esempio}
Elettrodo sferico di raggio $R$

Sarà presente una distribuzione di cariche $\sigma$ uniforme e costante sulla superficie della sfera
e per simmetria sappiamo che il potenziale $V$ dipenderà solo dalla distanza $r$ dalla sfera.
$$
V(r) = \frac{Q}{4 \pi \varepsilon_0}\frac{1}{r},\ r\geq R \stackrel{r=R}{\Rightarrow} V(R) = 
\frac{Q}{4\pi\varepsilon_0 R}
$$
Quindi 
$$
C = \frac{Q}{V(R)} = 4\pi\varepsilon_0 R
$$
Si può facilmente stimare la capacità del pianeta Terra, considerato come un conduttore ideale
conoscendo le costanti e approssimando il raggio a circa \SI{6.4e6}{\meter} e $\varepsilon_0 = 
\SI{8.85e-12}{\farad\per\meter}$ si ha un risultato di circa \SI{711}{\micro\farad}.

\subparagraph{Caso generale di N conduttori}
Determinare il campo $\vec{E}$ con N conduttori in condizioni elettrostatiche.
Siano indicati con $C_1$, $C_2$ ... $C_k$ $k=1,...,N$ gli N conduttori e i rispettivi vettori normali
$\hat{n}_1$, $\hat{n}_2$ ... $\hat{n}_k$.
Ogni conduttore avrà un suo potenziale $V_k$ in $C_k$ e sulla frontiera $\partial C_k$, la quantità di carica sul k-esimo conduttore:
$$
Q_k = \iint_{\partial c_k} \varepsilon_0 \vec{E}\cdot\hat{n} dS
$$
Sia $\Omega$ la regione di spazio non occupata dai conduttori $\Omega = \mathbb{R}^3 - C_1 - C_2 -... - C_k$, allora
$$
\nabla^2 V = 0 \text{ in } \Omega 
$$
e deve rispettare anche
$$
\left. V \right|_{c_k} = V_k \text{ su } \partial \Omega_k
$$
e $V$ normale all'infinito.
Per il teorema di unicità della soluzione, questa è univocamente determinata.

Per determinare la soluzione si caratterizza $V(p)$ mediante il principio di sovrapposizione degli effetti:
si considerano $n$ problemi di valori al contorno (BVP) ausiliari per altrettante funzioni
potenziali $V^{(k)},\ k=1,...,N$
$$
\begin{cases}
\nabla^2 V^{(i)} = 0 \text{ in } \Omega \\
\left.V^{(i)}\right|_{\partial c_i} =  1,\ \left.V^{(k)}\right|_{\partial c_k} = 0,\ k \neq i
\end{cases}
$$

La soluzione del problema omogeneo è:
$$
V(P) = V_1 V^{(1)}(P) + V_2 V^{(2)}(P) + ... + V_NV^{(N)}(P)
$$
è proprio la soluzione unica del problema di partenza.
