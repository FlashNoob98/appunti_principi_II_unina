
\section{Introduzione ai campi stazionari e instazionari}
\subsection{Richiami di analisi vettoriale}
\paragraph{Sistemi di riferimento e coordinate}
Per descrivere una proprietà nello spazio, si utilizza di solito un riferimento composto da 
una terna ortogonale di assi indicati con $L_1$ L2 e L3 con origine comune in $O$.
Si descrive un punto nello spazio $P$ con un raggio vettore che parte dall'origine e 
raggiunge il punto $P$.

Questo vettore può essere determinato con una terna di scalari $(u_1,u_2,u_3)$ che sono le coordinate del 
punto $P$.

La scelta più consona è quella di introdurre un sistema di coordinate cartesiane tali che 
$$
P \rightarrow (x,y,z)\ \vec{OP} = x\vec{e_x} + y\vec{e_y} + z\vec{e_z}
$$
con $\vec{e_x},\ \vec{e_y},\ \vec{e_z}$ i versori degli assi coordinati $e_1,\ e_2,\ e_3$.

In alternativa si possono utilizzare le coordinate \textbf{cilindriche}:
Il punto $P$ è ancora rappresentato da 3 scalari $(r,\varphi, z)$ e dato dalla combinazione di queste 
coordinate.
$$
\vec{OP} = r\vec{e_r} + \varphi\vec{e_{\varphi}} + z\vec{e_z}
$$

$$
\begin{cases}
r = \sqrt{x^2+y^2}\\
\sin\varphi = \frac{y}{\sqrt{x^2+y^2}},\ \cos\varphi = \frac{x}{\sqrt{x^2+y^2}}\\
z = z
\end{cases}
$$
Queste variabili possono essere ottenute in MATLAB con i seguenti comandi:
cart2pol e $\varphi= $ atan2(y,x) o viceversa pol2cart.

Un altro sistema di riferimento comunemente utilizzato è quello delle coordinate \textbf{sferiche}.
$$
P\rightarrow (r,\theta,\varphi)\ \ \vec{OP} = r\vec{e_r} + \theta\vec{e_\theta} + \varphi \vec{e_\varphi}
$$
con $(\vec{e_r},\vec{e_\theta},\vec{e_\varphi})$ terna levogira

$$
\begin{cases}
x = r\sin\theta\cos\varphi\\
y = r\sin\theta\sin\varphi\\
z = r\cos\theta
\end{cases}
$$
anche in questo caso è possibile utilizzare la funzione MATLAB ``\textit{sph2cart}''.
Formule inverse:
$$
\begin{cases}
r = \sqrt{x^2+y^2+z^2} \\
\cos\theta = \frac{z}{\sqrt{x^2+y^2+z^2}}\\
\sin\theta = \frac{\sqrt{x^2+y^2}}{\sqrt{x^2+y^2+z^2}} \\
\cos\varphi  = \frac{x}{\sqrt{x^2+y^2}},\ \sin\varphi = \frac{y}{\sqrt{x^2+y^2}}
\end{cases}
$$
\newpage
\textbf{Spostamenti elementari:}
Supponiamo uno spostamento lungo la direzione $e_1$ associamo un coefficiente ``metrico'' del 
sistema di coordinate. RIVEDI!!
$$
\begin{cases}
de_1 = h_1 du_1\\
de_2
\end{cases}
$$
Questi fattori ``aggiustano'' le dimensioni delle relazioni in metri dovute a variazioni di coordinate
ad esempio in radianti.

Per le coordinate cartesiane i coefficianti metrici sono tutti uguali tra loro e pari ad 1.
$$
h_1 = h_2 = h_3 = 1 \Rightarrow
\begin{cases}
de_1 = dx \\
de_2 = dy \\
de_3 = dz
\end{cases}
$$

Aggiungi parte integrali

Ripetiamo l'analisi per le \textbf{coordinate cilindriche:}
$ (u_1,u_2,u_3) = (r,\varphi,z)$
Supponiamo uno spostamento associato alla variazione della coordinata $r$, il raggio vettore
viene incrementato di una quantità $dr = dl_1 \Rightarrow h_1 =1$.

Effettuando una variazione $d\varphi$ invece il raggio vettore percorrerà un arco pari a $rd\varphi = dl_2 \Rightarrow h_2 = r$ a pari variazione di arco ci sarà uno spostamento maggiore proporzionale alla
distanza dall'origine $r$.

$$
dS_1 = rd\varphi dz\ \ dS_2 =dzdr\ \ dS_3 = rdrd\varphi
$$
$$
dV = rdr\ d\varphi\ dz
$$

In \textbf{coordinate sferiche} si ha $()=()$

Ad una variazione $dr$ si ha uno spostamento lungo il raggio vettore $\vec{OP}$ quindi anche in questo
caso il fattore metrico sasrä pari ad 1.

Ad una variazione della variabile $\theta$ detta anche co-latitudine, corrisponde una rotazione
pari a $dl_2 = rd\theta \Rightarrow h_2 = r$.

Ad una variazione di $\phi$ si ha un arco $dl_3 = r\sin\theta d \varphi \Rightarrow h_3 = r\sin\theta$

$$\begin{cases}
dS_1 = r^2\sin\theta\ d\theta\ d\varphi \\
dS_2 = r \sin \theta \\
dS_3 = r d \theta dr
\end{cases}
$$
$$
dV = r^2 \sin\theta\ dr\ d\theta\ d\varphi
$$

\paragraph{Campi scalari}
Si intende con campo scalare una funzione $f:\Omega \in R^3 \to R$ con $\Omega$ sufficientemente 
regolare.

Un campo \textbf{vettoriale} invece è una funzione $\vec{v} : \Omega \in R^3 \to R^3$
che può essere espressa mediante 3 campi scalari $v_1(P),v_2(P),v_3(P)$ che sono le componenti di
$\vec{v}(P)$ nel sistema di coordinate scelto.

$$
\vec{v}(x,y,z) = v_x(x,y,z)\vec{e_x} + v_y(x,y,z)\vec{e_y} + v_z(x,y,z)\vec{e_z}
$$
$$
\vec{v}(r\theta,\varphi) ECC
$$

Richiamiamo per i campi scalari la superficie (curva) di livello per il caso bidimensionale 
(tridimensionale):
$$
f(P) \in C^1(\Omega)
$$
una superficie di livello è definita da:
$$
f(x,y,z) = f_0
$$
In ogni punto $P$ passa una ed una sola superficie di livello.

Nel caso bidimensionale $f(x,y) = f_0$.


\paragraph{Linea di forza di un campo vettoriale}
Sia $\vec{v(P)} : \Omega \to R^3$ e sia la linea $\gamma$ tangente in ogni punto a $\vec{v(P)}$,
è per definizione una linea vettoriale di $\vec{v}$.
L'orientamento dei versori tangenti della linea $\gamma$ descrivono direzione e verso di $\vec{v}$ 
(non il modulo).

\textbf{Tracciamento delle linee vettoriali} di un campo $\vec{v}(P)$ assegnato:

Riferendosi allo riferimento degli assi cartesiani appena definiti ...

$$
\frac{v_y(P_0)}{v_x(P_0)} = \frac{dy}{dx}(P_0)
$$
Si può quinid risolvere una ODE
$$
\begin{cases}
\frac{dy}{dx} = \frac{v_y(x,y,z)}{v_x(x,y,z)} \\
\frac{dz}{dx} = \frac{v_z(x,y,z)}{v_x(x,y,z)}\\
y(x_0) = y_0 \\
z(x_0) = z_0
\end{cases}
$$
Si definisce una superficie vettoriale di $\vec{v}(P)$
$$
S\in \Omega: \vec{u}(P)\cdot\vec{v}(P) = 0 \forall P \in S
$$
I versori normali $\vec{u}$ saranno ortogonali in ogni punto della superficie $S$.

Def: Tubo di flusso di un campo $\vec{v}$, sia $\Gamma$ una linea chiusa che non sia una linea di forza
sulla quale vengono definiti dei punti, si suppone che ci siano delle linee vettoriali che attraversano
questi punti. Si definisce questo ``tubo'' in modo tale che le linee vettoriali siano tangenti sulla sua faccia laterale.
\newpage
Circuitazioni e flussi di campi vettoriali

\textbf{Circuitazione} (o circolazione) di $\vec{v}$ sulla linea chiusa $\Gamma$, supponiamo un punto $Q$ sulla 
linea e il versore tangente $\hat{t}(Q)$ e una linea vettoriale $\vec{v}(Q)$ con un certo angolo
$\alpha$ rispetto a $\hat{t}(Q)$ allora:

$$
\oint_{\Gamma} = \vec{v}\cdot\hat{t}dl 
$$

\textbf{Flusso} di $\vec{v}$ uscente da una superficie chiusa $ETA$

$$
\oiint_{ETA}\vec{v}\cdot\hat{n}dS
$$

\textbf{Operatori differenziali}

\textbf{Gradiente} siano date due superfici di livello $f(P) = f_0$ e $f(P) = f_0 + \Delta f$, consideriamo la distanza tra $P\in S_1$ lungo la retta normale alla superficie $S_0$ passante per il punto 
$P_0$ ed individuo un punto $P$ su $S_1$, si considera ora il rapporto $\frac{f(P)\cdot f(P_0)}{d(p,p_0)}$ è un rapporto incrementale che con il limite di $P\to P_0$ si definisce la derivata parziale lungo la
direzione normale passante per il punto $P_0$.

Aggiungi def Gradiente

Gradiente di coordinate curvilinee
$$
df = \nabla f (P_0) \cdot dl_i\vec{e_i}
$$
$$
\frac{1}{h_i} \frac{\partial f}{\partial u_i} = \left[\nabla f (P)\right]\cdot e_i
$$
Aggiungi ulteriori esempi con le altre coordinate cilindriche e sferiche al variare dei fattori metrici.


