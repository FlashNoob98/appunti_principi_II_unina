
Data la risposta all'ingresso impulsivo, si può calcolare la risposta a qualsiasi ingresso, mediante
l'uso dell'integrale di convoluzione.
Osservando attentamente l'integrale si vede che la risposta impulsiva gode della proprietà per la quale
$$
h(t) = 0 \ t<0,\ h(t) \neq 0 \forall t\ \geq 0 \Rightarrow t - \tau \geq 0 \Rightarrow \tau \leq t
$$

Se l'ingresso $x(t) = 0,\ t< t_0$ allora possiamo affermare che la $y(t)$ avrà come estremi di integrazione
$t_0^-$ e $t$, con il - si sottintende la possibilità che siano presenti $\delta(t_0)$ in quell'istante 
di tempo.

\paragraph{Esempi dell'utilizzo dell'integrale di convoluzione}
Circuito RC parallelo con forzamento esponenziale, forzato da un generatore di corrente.
$$
j(t) = I e ^{\frac{t}{\tau}} u(t)
$$
Per studiare la risposta del circuito bisogna per prima cosa determinare la $V_{cf}(t)$, mediante
lo studio della risposta impulsiva.
Per determinare la risposta impulsiva si forza il circuito con una $\delta(t)$, utilizzando i metodi 
precedenti.

$i_c = \delta(t)$
$$
v_c(0^+) = \frac{1}{C}\int_{0^-}^{0^+} i_c(\tau)d\tau + V_c(0^-) = \frac{1}{C}
$$
Per determinare la risposta impulsiva si determina l'evoluzione libera spegnendo il generatore,
ci sarà un parallelo RC con la seguente equazione di stato:
$$
\begin{cases}
RC\frac{dV_c}{dt} + v_C = 0 \\
V_c(0^+) = \frac{1}{C}
\end{cases}
$$
Quindi 
$$
V_{c_{lib}} = h(t) = \frac{1}{C} e^{-\frac{t}{RC}},\ t\geq 0
$$
$$
h(t) = \frac{1}{C}e^{-\frac{t}{RC}}\cdot u(t) \forall t
$$

Utilizzando l'integrale di convoluzione per calcolare la risposta forzata:

$$
V_{cf}(t) = \int_{0^-}^{t}Ie^{\frac{\tau}{T}} \cdot \frac{1}{C} e^{-\frac{t-\tau}{RC}} d\tau = 
\frac{I}{C} \int_{0^-}^{t} e^{\frac{\tau}{t}}\cdot e^{-\frac{t}{RC}}\cdot e^{\frac{\tau}{RC}} d\tau
$$

$$
= \frac{I}{C} e^{-\frac{t}{RC}}\cdot \frac{e^{t\left(\frac{1}{T}+\frac{1}{RC}\right)}-1}{\frac{1}{RC}+\frac{1}{T}}
$$

Circuito RC serie con forzamento in tensione a rampa lineare $e(t)$
$$
e(t) = \begin{cases}
0\ t<0 \\
\frac{t}{T},\ 0\leq t\leq T\\
0\ t>0
\end{cases}
$$
Possiamo rappresentare questa funzione mediante l'uso di due funzioni $u(t)$

$$
e(t) = \frac{t}{T}\left[u(t) - u(t-T)\right]
$$
Determiniamo quindi la risposta $h(t)$ mediante la risposta al gradino $g(t)$
Le equazione di stato è:
$$
\begin{cases}
1 = RC\frac{dV_c}{dt} + V_c \\
V_c(0) = 0 
\end{cases}
\Rightarrow V_c(t) = 1 - e^{-\frac{t}{RC}},\ t\geq 0$$
Quindi 
$$
g(t) = (1 - e^{-\frac{t}{RC}})u(t) \forall t
$$
ma
$$
h(t) = \frac{dg}{dt} = \frac{1}{RC}e^{-\frac{t}{RC}},\ t\geq 0
$$
Determiniamo ora la risposta forzata $v_{cf}(t)$:
$$
V_{cf}(t) = \int_{0^-}^{t} e(\tau) h(t-\tau)d\tau = \int_{0^-}^{t}\frac{\tau}{T}\left[u(\tau) - u(t-\tau)\right]
\cdot \frac{1}{RC} e ^{-\frac{t-\tau}{RC}}\cdot u(t-\tau)d\tau
$$
Svolgiamo ora l'integrale osservando che $e(\tau) \neq 0 $ solo se $t \in [0,T]$, separiamo quindi
il calcolo dell'integrale in due eventualità:
$$
t\leq T:\ \int_{0^-}^{t}\frac{\tau}{T}\frac{1}{RC}e^{-\frac{t-\tau}{RC}}d\tau = \frac{1}{TRC}e^{-\frac{t}{RC}}\int_{0^-}^{t}\tau e^{\frac{\tau}{RC}}d\tau 
$$
utilizzando l'integrazione per parti:
$$
\left(f\cdot g\right)' = f'g + g'f
$$
$$
f = \tau\ g' = e^{\frac{\tau}{RC}}
$$
$$
\frac{1}{TRC}e^{-\frac{t}{}RC}\int_{0^-}^{t}\frac{d}{dt}\left[\tau e^{\frac{\tau}{RC}}\cdot RC \right] -
RC e^{\frac{\tau}{RC}} d\tau =
$$
$$
\frac{1}{TRC} e^{-\frac{t}{RC}} \left\{ \left[t e^{\frac{t}{RC}}\cdot RC \right] - \left[(RC)^2 e^{\frac{\tau}{RC}} \right]_{0^-}^{t}  \right\} =
$$
$$
\frac{t}{T} - \frac{RC}{T}\left(1-e^{-\frac{t}{RC}}\right)
$$

Secondo caso:
$$
t > T:\ \int_{0^-}^{T} \frac{\tau}{T}\frac{1}{RC} e^{-\frac{t-\tau}{RC}}d\tau =  e^{-\frac{t-T}{RC}}-\frac{RC}{T}\left(e^{-\frac{t-T}{RC}} -e^{-\frac{t}{RC}} \right)
$$

\section{Metodo dei fasori}
Il metodo dei fasori si basa sul concetto che conoscendo l'andamento di regime con grandezze isofrequenziali, si può risolvere il sistema traspotando le grandezze sinusoidali nel dominio simbolico
dei fasori, si rapprentano cioè le grandezze descrittive dei bipoli medianti numeri complessi.

Mediante la L-Trasformata si possono usare ancora una volta equazioni nel dominio complesso ma per 
studiare reti nel regime transitorio.

\paragraph{Analisi delle reti dinamiche LTI con Laplace}
Consideriamo un blocco contenenti tutte le equazioni circuitali nel dominio del tempo,
otteniamo tutte le equazioni di stato di induttori e condensatori.
Questo risultato si ottiene mediante i metodi di risoluzione delle ODE, per arrivare alla conoscenza
della dinamica delle \textit{variabili di stato}, se sono poi interessato ad altre variabili, attravarso
operazioni \textit{algebriche lineari} si conosce la dinamica di utte le variabili del circuito.

In alternativa, sfruttando la trasformata di Laplace, le equazioni ODE vengono trasformate nel dominio
di Laplace, saranno equazioni algebriche lineari anzichè differenziali, risolvendo quindi solo un sistema
di equazioni lineari si conosce direttamente la trasformata delle variabili di stato, effettuando quindi
il procedimento inverso di antitrasformazione si ricavano le variabili di stato nel dominio del tempo,
l'antitrasformazione può essere invece posticipata ed eseguita dopo aver ricavato le generiche variabili
del circuito, ancora una volta con operazioni algebriche.

La soluzione di un sistema di equazioni differenziali è notevolmente più complesso che risolvere un 
sistema lineare, ecco il vantaggio dell'utilizzo della trasformata di Laplace.

Definizione della trasformata di Laplace di una funzione $f(t)\ \in\ [0,+\infty[\to R$:
$$
L[f(t)] = F(s) = \int_{0^-}^{+\infty} f(t) e^{-st}dt \ F:s \in\ C \to C
$$
ammesso che l'integrale improprio converga, per assicurarci che ciò accade si pone la condizione sufficiente, verificata nella stragrande delle situazioni:
$$
\text{Se} \left|f(t) \right| \leq Me^{\alpha t} ,\ M,\alpha \in R \text{ costanti}
$$
allora l'integrale converge, dim.:
$$
\int_{0^-}^{+\infty} f(t) e^{-st}dt \leq \int_{0^-}^{+\infty} \left|f(t)\right| e^{-st}dt \leq
\int_{0^-}^{+\infty} Me^{\alpha t} e^{-st}dt = M\int_{0^-}^{+\infty} e^{(\alpha-s)t}dt
$$
condizione verificata per ogni $\Re \left\{ s\right\} > \alpha$.

Definizione dell'antitrasformata:
$$
L^{-1}\left[F(S) \right] = f(t) = \frac{1}{2\pi i} \lim_{T\to+\infty} \int_{\gamma-iT}^{\gamma+iT}
e^{st}F(s)ds
$$
con $\Re\left\{s\right\} = \gamma$ tale che tutte le singolarità di $F(s)$ si trovino a sinistra di $\gamma$.

\paragraph{Proprietà della L-trasformata} Ricordando quelle utilizzate nel metodo dei fasori:
\begin{itemize}
\item Unicità: $\forall f(t) \in [0,+\infty[\ \exists!\ F(s) = L[F(s)]$

considerate $F(s)$ e $G(s)$, se $F(s) = G(s) \Rightarrow f(t) = g(t)$ quasi ovunque, ossia:
$$\int_{0^-}^{+\infty}|f(t) - g(t)|dt = 0$$

\item Linearità: date $f_1(t)$ ed $f_2(t) :\ [0,+\infty[ \rightarrow R,\ k_1,k_2 \in R$ allora
$$L[k_1f_1(t) + k_2f_2(t)] = k_1F_1(s) + k_2F_2(s) $$
si dimostra con la proprietà di linearità dell'integrale

\item Traslazione nel dominio di Laplace: sia data $f(t) \in [0^-,+\infty[,\ L[f(t)] = F(s)$ e
consideriamo $F(s-\lambda),\ \lambda\ \in\ C$ 
$$F(s-\lambda) = \int_{0^-}^{+\infty}f(t) e^{-(s-\lambda)t}dt = \int_{0^-}^{+\infty} (f(t)e^{\lambda t})e^{-st} dt \Rightarrow F(s-\lambda) = L[f(t) e^{\lambda t}]$$ con $\Re\{s\} > \lambda$ 

\item Derivazione: $f'(t)  = \frac{d}{dt}f(t),\ L[f(t)] = F(s)$ allora 
$$L\left[\frac{d}{dt}f(t)\right] = \int_{0^-}^{+\infty}f'e^{-st}dt \stackrel{\text{x parti}}{=} 
\left[fe^{-st}\right]_0^{+\infty} - \int_{0^-}^{+\infty}(-s)e^{-st}\cdot f dt =$$
$$= \left(\lim_{t\to\infty}\left[fe^{-st}\right]-f(0^-)\right) + sF(s) = sF(s) - f(0^-)$$
$$
L\left[\frac{d}{dt}f(t)\right] = sF(s) - f(0^-)
$$
\item Prodotto di convoluzione nel dominio di Laplace, facendo leva sul teorema di Borel:
$$
L\left[f*g(t)\right] =\int_{0^-}^{+\infty}\left(\int_{0^-}^{t}f(\tau)g(t-\tau)d\tau\right) e^{-st}dt = F(s)\cdot G(s)
$$
per dimostrare questo teorema si scambiano le variabili di integrazione $t$ e $\tau$ sfruttando i teoremi
di \href{https://it.wikipedia.org/wiki/Teorema_di_Fubini}{Fubini} e Tonelli
\end{itemize}
\newpage
Trasformate notevoli di frequente utilizzo nei circuiti:
\begin{itemize}
\item Esponenziale: $f(t) = e^{\lambda t},\ \lambda\ \in R $ 
$$
L\left[e^{\lambda t}\right] = \int_{0^-}^{+\infty} e^{-(s-\lambda) t} dt = \left[\frac{1}{\lambda -s}e^{(\lambda -s)t}\right]_{0^-}^{+\infty} = \frac{1}{\lambda -s} \left[\lim_{t\to\infty}e^{(\lambda-s)t}-1\right] =
\frac{1}{s-\lambda}
$$
\item Funzione gradino $u(t)$:
$$
L[u(t)] = \int_{0^-}^{+\infty}e^{0t}e^{-st}dt = \frac{1}{s}
$$
\item Delta di Dirac $\delta(t)$:
$$
L[\delta(t)] = \int_{0^-}^{+\infty}\delta(t) e^{-st}dt = e^{-s\cdot 0} = 1
$$
\item Funzioni sinusoidali, $\cos(\omega t)=\frac{e^{j\omega t}+e^{-j\omega t}}{2},\ \sin(\omega t) = \frac{e^{j\omega t}-e^{-j\omega t}}{2j}$
$$
L[\cos(\omega t)] = \frac{1}{2} L[e^{j\omega t}] + \frac{1}{2} L[e^{-j\omega t}] = \frac{s}{s^2+\omega^2},\ \Re\{s\} > 0
$$
$$
L[\sin(\omega t)] = \frac{1}{2j}\left(\frac{1}{s-j\omega}-\frac{1}{s+j\omega}\right) = \frac{\omega}{s^2+\omega^2},\ \Re\{ s\} > 0
$$
\end{itemize}
