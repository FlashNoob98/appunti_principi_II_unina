resistività = $\eta$
\subsection{Conduzione nei metalli}
I portatori di carica sono elettroni ``liberi'' ossia in grado di percorrere distanze
macroscopiche sotto l'azione di una forza sotto l'azione di un campo elettrico.
In un metallo a riposo c'è un moto dei portatori di carica ma la velocità media di deriva (drift)
$$
\vec{v}\ ^- = 0
$$
cioò non implica che le particelle siano ferme.

Nel caso in cui la velocità media sia diversa da 0 si ha una velocità delle particelle più bassa(?).

Considerando un conduttore cilindrico di sezione di area $S=\SI{1}{\milli\meter^2}$, considerando un'
intensità di corrente pari ad \SI{1}{\ampere}, quindi
$$
J = \frac{i}{S} = 10^6\si{\ampere\per\meter^2} = n^-q_e\vec{v}
$$

La densità $n^-$ è pari a $10^23/\si{\centi\meter^3} = $
$$
v^- = \frac{J}{nqe} =  \frac{10^6}{10^23\cdot10^6\cdot ()} \simeq 10^-6 \si{\meter\per\second} = \SI{1}{\micro\meter\per\second}
$$

17:20


\subsection{Equazioni di Maxwell nel vuoto}
Ci si possono esprimere tutti i fenomeni di elettromagnetismo macroscopico, in presenza di distribuzioni 
di carica e correnti nello spazio vuoto.

Forniscono tutte le possibili circuitazioni e i possibili flussi uscenti da superfici chiuse per i campi
$\vec{E}$ e $\vec{B}$.

In virtù del teorema di decomposizione di Helmoltz si garantisce che $\vec{E}$ e $\vec{B}$ 
sono univocamente determinati (assumendoli normali all'infinito).

\paragraph{Legge di gauss per il campo elettrico $\vec{E}$}
Consideriamo una superficie chiusa $\Sigma$ nello spazio con bordo $\Omega_\Sigma$
la prima equazione di Maxwell afferma: 
\begin{equation}
\oiint_\Sigma \vec{E}\cdot\hat{n}dS = \frac{1}{\varepsilon_0}Q_{\Omega_\Sigma} = \frac{1}{\varepsilon_0}
\iiint_{\Omega_\Sigma} \rho dV \ \ \forall \Sigma,\ \forall t
\end{equation}

$$
\left[\oiint \vec{E}\cdot\hat{n}dS\right] = \si{\volt\per\meter\cdot\meter^2} = \si{\volt\cdot\meter}
$$

\paragraph{Legge di Gauss per il campo $\vec{B}$} o solenoidalità del campo $\vec{B}$,
afferma che il flusso attraverso una superficie aperta dipende solo dall'orlo (flusso concatenato ad una
linea).
\begin{equation}
\oiint_{\Sigma} \vec{B}\cdot\hat{n}dS = 0\ \forall \Sigma, \forall t
\end{equation}

$$
\left[\oiint \vec{B}\cdot\hat{n}dS\right] = \si{\tesla\cdot\meter^2} = \si{\weber}
$$
Un'interpretazione di questa legge, in analogia a quella per il campo elettrico, possiamo affermare
che non è possibile isolare una ``carica'' magnetica netta, ossia un monopolo magnetico.

Le restanti equazioni di Maxwell riguardano le circuitazioni, definiscono i fenomeni di induzione
elettromagnetica ed induzione magnetoelettrica.

\paragraph{Legge di Faraday-Neumann-Lenz}
Consideriamo una linea chiusa qualunque $\Gamma$ ed una qualsiasi superficie $S_\Gamma$ di
orlo $\Gamma$.
\begin{equation}
\oint_{\Gamma} \vec{E}\cdot\hat{t}dl = - \iint_{S_\Gamma} \frac{\partial \vec{B}}{\partial t}
\cdot\hat{n} dS\ \ \forall\Gamma,\ S_\Gamma\in\Omega,\ \forall t
\end{equation}
$\Gamma$ ed $S_\Gamma$ possono anche essere variabili nel tempo.

Se in una regione di spazio c'è un campo $\vec{B}$ variabile nel tempo, esso induce necessariamente
un campo elettrico nella stessa regione di spazio.
Si può semplificare l'equazione supponendo che le superfici $\Gamma$ ed $S_\Gamma$ siano ferme
$$
\oint_{\Gamma} \vec{E}\cdot\hat{t}dl = -\frac{d}{dt} \iint_{S_\Gamma}\vec{b}\cdot\hat{t}dS = 
-\frac{d}{dt} \Phi_\Gamma
$$
Presi due punti qualsiasi $A$ e $B \in R^3$ supponiamo che esistono infinite linee $\gamma$, il terzo 
termine è il suo opposto.

La differenza fra tensione elettrica calcolata tra due linee che connettono gli stessi punti nello 
spazio è pari alla variazione di flusso di campo di induzione magnetica concatenato alle due linee.

$$
v_{AB\gamma} - v_{AB\gamma'} = -\frac{d\Phi_\Gamma}{dt}
$$
In generale il campo elettrico NON è conservativo.

\paragraph{Legge di Ampére-Maxwell} (Induzione magnetoelettrica)

\begin{equation}
\oint_\Gamma \vec{B}\cdot\hat{t} dl = \mu_0 \iint_{S_\Gamma} \left(\vec{J} + \varepsilon_0\frac{\partial\vec{E}}{\partial t}\right)\cdot \hat{n} dS
\end{equation}

$\mu_0$ è la permeabilità magentica del vuoto $\mu_0 = 4\pi\cdot10^{-7}\si{\henry\per\meter} $
anche se a partire dalla revisione delle unità di misura del 2019 è una quantita misurata e non più una costante universale (anche se coincide con la precedente definizione per le prime 10 cifre significative)

$$
\frac{1}{\sqrt{\varepsilon_0\mu_0}} = c \simeq 3\cdot10^{8}\ \si{\meter\per\second}
$$

questa legge è la duale della legge di Faraday-Neumann-Lenz. Si può riscrivere come:
$$
\oint_\Gamma\vec{B}\cdot\hat{t}dl = \mu_0 i_{S_\Gamma}(t) + \mu_0 \iint_{S_\Gamma} \frac{\partial}{\partial t}(\varepsilon_0 \vec{E})\cdot\hat{n}dS
$$
Il secondo termine prende il nome di intensità della corrente di spostamento, quindi il termine
$\left(\varepsilon_0 \vec{E}\right)$ è chiamato densità di corrente di spostamento.

Se le linee e le superfici sono ferme:
$$
\oint_\Gamma\vec{B}\cdot\hat{t}dl = \mu_0\iint_{S_\Gamma}\vec{J}\cdot\hat{n}dS + 
\mu_0\frac{d}{dt}\iint_{S_\Gamma} \varepsilon_0 \vec{E}\cdot\hat{n} dS
$$
Si vede che il flusso di $\vec{E}$ ha le dimensioni di una carica elettrica 59:00

\paragraph{Principio di conservazione della carica}
Combinando linearmente le equazioni d ...

\begin{equation}
\oiint_\Sigma\vec{J}\cdot\hat{n}dS = - \iiint_{\Omega_\Sigma}\frac{\partial \rho}{\partial t} dV
\ \ \forall\Sigma,\ \forall t
\end{equation}

Se le superfici sono ferme
$$
i_{\Sigma}(t) = \oiint_\Sigma\vec{J}\cdot\hat{n}dS = -\frac{d}{dt} \iiint_{\Omega_\Sigma} \rho dV = -\frac{dQ_{\Omega_\Sigma}}{dt}\ \forall\Sigma,\ \forall t
$$

Consideriamo invece il campo di corrente totale, $\vec{J} + \varepsilon_0\frac{\partial\vec{E}}{\partial t}$, con superfici ferme si ha:
$$
\oiint_{\Sigma} \left(\vec{J} + \varepsilon_0\frac{\partial\vec{E}}{\partial t}\right)\cdot\hat{n}dS = 
-\frac{dQ_{\Omega_\Sigma}}{dt} + \frac{d}{dt} \oiint_{\Sigma} \varepsilon_0\vec{E}\cdot\hat{n}dS =
$$
$$
= -\frac{dQ_{\Omega_\Sigma}}{dt} + \varepsilon_0\frac{d}{dt}\left[\frac{Q_{\Omega_\Sigma}}{\varepsilon_0}\right] = 0\ \ \forall\Sigma
$$
Si conclude che il campo $\vec{J} + \varepsilon_0\frac{\partial\vec{E}}{\partial t} $ ossia
il campo di corrente totale è solenoidale.


\subsection{Il limite stazionario delle equazioni di Maxwell}
Si ottengono i modelli quasi-stazionari, un'estensione del modello interamente stazionario.
Si ha condizione stazionaria se le grandezze (campi, distribuzioni di cariche e correnti) sono 
costanti nel tempo.
$$
\oiint_{\Sigma}\vec{E}\cdot\hat{n}dS = \frac{Q_{\Omega_\Sigma}}{\varepsilon_0} \ \forall \Sigma
$$
$$
\oiint_{\Sigma}\vec{B}\cdot\hat{n}dS = 0 \ \forall\Sigma
$$
$$
\oint_{\Gamma} \vec{E}\cdot\hat{t}dl = 0 \ \ \forall \Gamma
$$
$$
\oint_{\Gamma}\vec{B}\cdot\hat{t} dl = \mu_0 i_{S_\Sigma}\ \forall \Sigma,S_\Sigma
$$
aggiungi cons.carica
Nel limite stazionario $\vec{E}$ e $\vec{B}$ sono disaccoppiati.

termine noto
Inoltre sono presenti come termini noti la carica elettrica e le correnti, ossia le sorgenti
per il campo elettrico sono le cariche elettriche, mentre le sorgenti del campo di induzione magnetica
sono le cariche in moto, ossia le correnti elettriche.

Il campo $\vec{E}$ è conservativo per la circuitazione

Il campo $\vec{J}$ diventa solenoidale, o conservativo per il flusso.

Questo riconduce a
$$
\vec{E} = -\nabla\varphi
$$
è possibile ricavare il campo elettrico utilizzando una sola incognita $(\varphi)$ piuttosto che
conoscere le funzioni di tutte le componenti di $\vec{E}$ nello spazio.

Consideriamo una linea aperta che connetta due punti $A$ e $B$, in condizioni stazionarie, è l'integrale di una forma differenziale esatta

$$
\int_{A\gamma B}\vec{E}\cdot\hat{t}dl \stackrel{\text{def}}{=} v_{AB\gamma} \stackrel{\text{staz.}}{=}
\int_{A\gamma B}-\nabla\varphi\cdot\hat{t}dl =\varphi(A)-\varphi(B)
$$
Ciò equivale a dire che la tensione di due punti dello spazio non dipende dalla linea $\gamma$ ed è
esprimibile come differenza di potenziale attraverso la $\varphi$.

\paragraph{Lavoro della forza elettromagnetica} Sulle cariche in moto in un volume elementare.
Sui portatori di carica agisce la forza di Lorentz
$$
\vec{F} = q\left(\vec{E}+\vec{v}\times\vec{B}\right)
$$

Se si considera il lavoro elementare $\delta L$ compiuto nello spostamento di una singola carica in
$\Delta\Omega$
$$
\delta L = \vec{F}\cdot\hat{t}dl = q\left(\vec{E}+\vec{v}\times\vec{B}\right)\cdot \hat{t} dl = q\vec{E}\cdot\vec{v}dt
$$

Per calcolare il lavoro ottenuto su tutte le particelle
$$
dL = \vec{E}\cdot\left(N^+q_pv^+ + N\ ^-q_e\vec{v}\ ^-\right)dt = \vec{E}\cdot\vec{J}\ \text{Vol}(\Delta\Omega)dt
$$

$$
\vec{E}\cdot\vec{J} Sistema\ qua
$$

Se si considera un materiale conduttore ohmico, ossia che soddisfa la legge di Ohm in forma locale
\begin{align*}
\begin{matrix}
\vec{J} = \gamma\vec{E} \\
\vec{E} = \eta\vec{J}
\end{matrix}\ 
\Rightarrow \vec{E}\cdot\vec{J} = \gamma\left|\vec{E}^2\right|
= \eta\left|\vec{J}^2\right| \geq 0 \ \forall t
\end{align*}
Quest'utlimo risultato rappresenta l'energia necessaria all'avanzamento delle particelle, chiamato
Effetto Joule.


\subparagraph{Sorgenti elementari dei campi elettromagnetici}
Ci si occupa ora del ``secondo'' blocco del modello Maxwell-Lorentz, si analizzano le ulteriori 
distribuzioni delle sorgenti:

Sia data una struttura con una grandezza preponderante rispetto alle latre due, viene definita 
\textit{trave}.
In elettromagnetismo ha senso considerare distribuzioni di cariche e correnti che abbiano una oppure
due dimensioni prevalenti.

Nel primo caso si hanno densità lineari di carica e corrente.

Nel secondo caso si hanno densità superficiali di carica e corrente.

