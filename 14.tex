CORREGGI APPUNTI PRECEDENTI SUL POTENZIALE ALL'INFINITO

\subparagraph{Problema generale dell'elettrostatica}

Siano $N$ conduttori carichi, il potenziale $V(P)$ dato dagli stessi.

La carica contenuta dal conduttore $k$, $Q_k$ è data da

$$
Q_k = \int_{\partial_{C_k}} \varepsilon_0 \vec{E}\cdot\hat{n}_k dS
$$

Sia $\Omega$ la regione esterna che non contiene i conduttori, si ha un problema di Dirichelet
esterno per l'Equazione di Laplace.

$$
\begin{cases}
\nabla^2 V = 0\\
V|_{\partial_{C_k}} = V_k \\
V \text{ normale all' } \infty
\end{cases}
$$

La soluzione generale si costruisce con il PSE considerando N BVPs ausiliari
$$
\begin{cases}
\nabla^2 V^{(i)} = 0 \text{ in} \Omega\\
V^{(i)}|_{\partial_{C_k}} = 1,\ v^{(i)}|_{\partial_{C_k}} = 0,\ k \neq i
\end{cases}
$$

Possiamo ora considerare la carica depositata sul conduttore $C_i$:
$$
Q_i = \int_{\partial_{C_i}} \varepsilon_0 \vec{E} \cdot \hat{n}_i dS = \int_{\partial_{C_i}} -\varepsilon_0 \frac{\partial V}{\partial n_i} dS = \left(\int_{\partial_{C_i}} - \varepsilon_0 \frac{\partial V}{\partial n_i}^{(1)} dS\right)V_1 + ... + \left(\int_{\partial_{C_i}} - \varepsilon_0 \frac{\partial V}{\partial n_i}^{(n)} dS\right)V_n
$$
Gli integrali $i$-esimi vengono chiamati coefficienti di capacità
$$
[C_{ij}] = \frac{\si{\coulomb}}{\si{\volt}} = \si{farad}
$$

$$
\begin{cases}
Q_1 = C_{11}V_1 + \ldots + C_{1n}V_n \\
\vdots \\
Q_n = C_{n1}V_1 + \ldots + C_{nn}V_n
\end{cases}
$$

30:55 Aggiungi coefficienti $C_{ii}$

Questi coefficienti soddisfano varie proprietà come:

\begin{description}
 \item[Reciprocità] Come $C_{ij} = C_{ji} $
 \item[proprietà 2]
 \item[Simmetrica e dominanza diagonale] $C_{ii} > \sum_{i\neq j}|C_{ij}|$
\end{description}

\textbf{Esempio} Elettrodi sferici a \textit{``grande''} distanza

Siano $P_1$ e $P_2$ i centri di una coppia di sfere di raggio $r_1$ ed $r_2$ con potenziali $V_1$ e
$V_2$ e cariche $Q_1$ e $Q_2$.
Si ipotizza che la distanza $|P_1P_2|$ sia sufficientemente grande da rendere trascurabile il potenziale
prodotto da una sfera in corrispondenza dell'altra ossia facendo in modo che non si \textit{``vedano''} 
elettrostaticamente, ossia i loro potenziali e distribuzioni di cariche non si influenzino a vicenda.

Si sa inoltre che il potenziale di una sfera carica è:
$$
V = \frac{Q}{4 \pi \varepsilon_0 R}
$$
il calcolo si può effettuare per entrambe le sfere ottenendo:
$$
Q_1 = 4 \pi \varepsilon_0 R_1 V_1,\ Q_2 = 4 \pi \varepsilon_0 R_2 V_2
$$

Supponiamo che $Q_1 = -Q_2$ (fenomeno di induzione completa)
$$
-Q_1 = 4 \pi \varepsilon_0 R_2 V_2 \Rightarrow V_1 - V_2 = \frac{Q_1}{4 \pi \varepsilon_0}
\left(\frac{1}{R_1} + \frac{1}{R_2}\right)
$$
Si definisce la capacità tra i due elettrodi:
$$
\frac{Q_1}{V_1 - V_2} = C = \frac{4 \pi \varepsilon_0}{\frac{1}{R_1}+\frac{1}{R_2}}
$$
In generale quando $Q_1 = - Q_2$ si può definire una capacità tra i due elettrodi che operativamente
può essere scritta come 
$$
C = \frac{\iint_{\partial C_1} \varepsilon_0 \vec{E}\cdot\hat{n}dS} {\int_{P_1}^{P_2}\vec{E}\cdot\hat{t} dl} = \frac{Q_1}{V_1 - V_2}
$$


\textbf{Caso generale}: la carica $Q_i$ dipende da tutti i potenziali $V_1, \ldots ,V_n$:
$$
Q_1 = C_{11}V_1 + C_{12}V_2 + \ldots + C_{1n}V_n = C_{11}V_1 + C_{12}V_2 + C_{12}V_1 - C_{12} V_1 +
... + C_{1n}V_n
$$
$$
Q_1 = \left(C_{11}+C_{12} + \ldots + C_{1n}\right) V_1 + (-C_{12})(V_1-V_2) + (-C_{13})(V_1-V_3)
+ \ldots + (-C_{1n})(V_1-V_n)
$$
La prima parentesi viene sostituita da $C_{11}^*$, $(-C_{12}) = C_{12}^*$ eccetera, vengono chiamate
\textit{capacità parziali} del sistema di conduttori, per evidenziare le capacità 
dovute all'interazione dei conduttori a due a due.
Queste definiscono nella maggior parte dei casi anche gli accoppiamenti parassiti
tra diverse capacità presenti in un circuito.

Alcune proprietà: 01:18:00 RIVEDI
$$
C_{ii}^* = \left.\frac{Q_i}{V_i}\right|_{V_k} = 1,\ k=1...,n
$$
$$
C_{ij}^* = \left.\frac{Q_i}{V_j}\right|_{V_j}=1,\ V_k = 0,\ k \neq j
$$
Si può mostrare che $c_{ii}^* =c_{ii} + \sum_{j\neq i}c_{ij} \geq 0$


Sviluppiamo il caso $N = 2$, supponiamo il fenomeno di induzione completa $(Q_1 = -Q_2)$
$$
Q_1 = C_{11}^*V_1 + C_{12}(V_1-V_2)
$$
$$
-Q_1 = C_{21}^*(V_2-V_1) + C_{22}^*
$$

PAUSA 01:23:00

\subparagraph{Calcolo capacità condensatore piano}


\subparagraph{Condensatore sferico}
Costituito da due sfere concentriche di raggio $R_1$ ed $R_2$
si può calcolare il potenziale per sovrapposizione:
$$
V_1 = V(R_1) = \frac{Q_1}{4\pi \varepsilon_0 R_1},\ \ V_2 = V(R_2) = \frac{Q_2}{4 \pi \varepsilon_0 R_2}
$$
La capacità
$$
C = \frac{Q_1}{V_1 - V_2} = \frac{\cancel{Q_1}}{\frac{\cancel{Q_1}}{4 \pi \varepsilon_0}\left(\frac{1}{R_1}-\frac{1}{R_2}\right)} = \frac{4 \pi \varepsilon_0}{\frac{1}{R_1}-\frac{1}{R_2}} = 4 \pi \varepsilon_0 \frac{R_1 R_2}{R_2 - R_1}
$$

\subparagraph{Condensatore cilindrico}
Costituito da due cilindri coassiali di raggio $R_1$ e $R_2$
$$
V(r) = -\frac{\sigma_1}{\varepsilon_0}R_1 \ln r - \frac{\sigma_2}{\varepsilon_0}R_2 \ln R_2\ \ R_1\leq r \leq R_2
$$
$$
V(R_1) = V_1 = -\frac{\sigma_1}{\varepsilon_0} R_1 \ln R_1 - \frac{\sigma_2}{\varepsilon_0}R_2 \ln R_2
$$
$$
V(R_2) = V_2 = -\frac{\sigma_1}{\varepsilon_0} R_1 \ln R_2 - \frac{\sigma_2}{\varepsilon_0}R_2 \ln R_2
$$
Eseguendo la differenza dei potenziali
$$
V_1 - V_2 = -\frac{\sigma_1}{\varepsilon_0} R_1 \ln R_1 + \sigma_1\frac{R_1}{\varepsilon_0}\ln R_2
= \frac{\sigma_1 R_1}{\varepsilon_0}\ln\left(\frac{R_2}{R_1}\right) = \frac{2 \pi R_1 L \sigma_1}{2 \pi L \varepsilon_0}\ln \left(\frac{R_2}{R_1}\right)
$$
$$
C = \frac{Q_1}{V_1-V_2} = \frac{2 \pi \varepsilon_0 L }{\ln \left(\frac{R_2}{R_1}\right)}
$$

\subparagraph{Capacità parziali di due elettrodi sferici}
Presi due elettrodi a distanza $d$ di raggio $R_1$ ed $R_2$, supponendo che $d \gg R_1,R_2$
in maniera tale da supporre che il potenziale prodotto da una sfera sia costante nei punti 
occupati dall'altra.

Applicando il PSE..

\begin{align*}
V_1' &= \frac{Q}{4 \pi \varepsilon_0 R_1}& V_2' &\simeq \frac{Q_1}{4 \pi \varepsilon_0 d};\\
V_1'' &\simeq \frac{Q_2}{4 \pi \varepsilon_0 d}& V_2'' &= \frac{Q_2}{4 \pi \varepsilon_0 R_2}
\end{align*}
Continua ultima parte con la matrice
