
\section{Introduzione ai campi stazionari e instazionari}
\subsection{Richiami di analisi vettoriale}
\paragraph{Sistemi di riferimento e coordinate}
Per descrivere una proprietà nello spazio, si utilizza di solito un riferimento composto da 
una terna ortogonale di assi indicati con $L_1$ $L_2$ e $L_3$ con origine comune in $O$.
Si descrive un punto nello spazio $P$ con un raggio vettore che parte dall'origine e 
raggiunge il punto $P$.

\begin{figure}[H] %esempio punto P nello spazio
\centering
\begin{tikzpicture}[scale=3,tdplot_main_coords]
\draw [thick,->] (0,0,0) -- (1,0,0) node[anchor= north east]{$l_1$};
\draw [thick,->] (0,0,0) -- (0,1,0) node[anchor= north west]{$l_2$};
\draw [thick,->] (0,0,0) -- (0,0,1) node[anchor= south]{$l_3$};
\tdplotsetcoord{P}{0.8}{50}{45};
\coordinate (O) at (0,0,0);
\draw [-stealth,color=red] (O) -- (P) node[anchor = west]{$P$};

\draw [dashed,color=red] (O) -- (Px);
\draw [dashed,color=red] (O) -- (Py);
\draw [dashed,color=red] (O) -- (Pz);
\draw [dashed,color=red] (Px) node[anchor = south east]{$x$} -- (Pxy);
\draw [dashed,color=red] (Py) node[anchor = south west]{$y$} -- (Pxy);
%\draw [dashed,color=red] (Px) -- (Pxz);
%\draw [dashed,color=red] (Pz) -- (Pxz);
%\draw [dashed,color=red] (Py) -- (Pyz);
\draw [dashed,color=red] (O) -- (Pxy);
%\draw [dashed,color=red] (Pz) -- (Pyz);
\draw [dashed,color=red] (Pxy) -- (P);
%\draw [dashed,color=red] (Pxz) -- (P);
%\draw [dashed,color=red] (Pyz) -- (P);
\draw [dashed,color=red] (Pz) node[anchor = east]{$z$} -- (P);
\end{tikzpicture}
\end{figure}

Questo vettore può essere determinato con una terna di scalari $(u_1,u_2,u_3)$ che sono le 
coordinate del punto $P$.

La scelta più consona è quella di introdurre un sistema di coordinate cartesiane tali che 
$$
P \rightarrow (x,y,z)\ \ \vec{OP} = x\vec{e_x} + y\vec{e_y} + z\vec{e_z}
$$
con $\vec{e_x},\ \vec{e_y},\ \vec{e_z}$ i versori degli assi coordinati $l_1,\ l_2,\ l_3$.

In alternativa si possono utilizzare le coordinate \textbf{cilindriche}:

\begin{figure}[H] %esempio punto P coordinate cilindriche
\centering
\begin{tikzpicture}[scale=3,tdplot_main_coords]
\draw [thick,->] (0,0,0) -- (1,0,0) node[anchor= north east]{$l_1$};
\draw [thick,->] (0,0,0) -- (0,1,0) node[anchor= north west]{$l_2$};
\draw [thick,->] (0,0,0) -- (0,0,1) node[anchor= south]{$l_3$};
\tdplotsetcoord{P}{0.8}{50}{45};
\coordinate (O) at (0,0,0);
\draw [-stealth,color=red] (O) -- (P) node[anchor = west]{$P$};

\draw [color=red] (O) -- (Pxy) node[anchor = north]{$r$};
\tdplotdrawarc[color=blue,->]{(O)}{0.2}{0}{45}{anchor=north}{$\varphi$};
%\draw [dashed,color=red] (O) -- (Px);
%\draw [dashed,color=red] (O) -- (Py);
%\draw [dashed,color=red] (O) -- (Pz);
%\draw [dashed,color=red] (Px) node[anchor = south east]{$x$} -- (Pxy);
%\draw [dashed,color=red] (Py) node[anchor = south west]{$y$} -- (Pxy);
%\draw [dashed,color=red] (Px) -- (Pxz);
%\draw [dashed,color=red] (Pz) -- (Pxz);
%\draw [dashed,color=red] (Py) -- (Pyz);
%\draw [dashed,color=red] (Pz) -- (Pyz);
\draw [dashed,color=red] (Pxy) -- (P);
%\draw [dashed,color=red] (Pxz) -- (P);
%\draw [dashed,color=red] (Pyz) -- (P);
\draw [dashed,color=red] (Pz) node[anchor = east]{$z$} -- (P);
\end{tikzpicture}
\end{figure}

Il punto $P$ è ancora rappresentato da 3 scalari $(r,\varphi, z)$ e dato dalla combinazione di queste 
coordinate.
$$
\vec{OP} = r\vec{e_r} + \varphi\vec{e_{\varphi}} + z\vec{e_z}
$$

$$
\begin{cases}
r = \sqrt{x^2+y^2}\\
\sin\varphi = \frac{y}{\sqrt{x^2+y^2}},\ \cos\varphi = \frac{x}{\sqrt{x^2+y^2}}\\
z = z
\end{cases}
$$
Queste variabili possono essere ottenute in MATLAB con i seguenti comandi:
\verb|cart2pol| e $\varphi= $ \verb|atan2(y,x)| o viceversa \verb|pol2cart|.
$$
\begin{cases}
x = r\cos\varphi \\
y = r\sin\varphi \\
z = z
\end{cases}
$$

Un altro sistema di riferimento comunemente utilizzato è quello delle coordinate \textbf{sferiche}.

\begin{figure}[H] %esempio punto P coordinate sferiche
\centering
\begin{tikzpicture}[scale=3,tdplot_main_coords]
\draw [thick,->] (0,0,0) -- (1,0,0) node[anchor= north east]{$l_1$};
\draw [thick,->] (0,0,0) -- (0,1,0) node[anchor= north west]{$l_2$};
\draw [thick,->] (0,0,0) -- (0,0,1) node[anchor= south]{$l_3$};
\tdplotsetcoord{P}{0.8}{50}{45}; %coordinate punto P
\tdplotsetthetaplanecoords{45}; %coordinata phi per determinare il piano di theta
\tdplotdrawarc[tdplot_rotated_coords,color=blue,->]{(0,0,0)}{0.5}{0}{50}{anchor=south}{$\theta$};
\coordinate (O) at (0,0,0);
\draw [-stealth,color=red] (O) -- (P) node[anchor = west]{$P$};
\draw (0,0.18,0.13) node[color=red]{$r$};
\draw [dashed,color=red] (O) -- (Pxy);
\tdplotdrawarc[color=blue,->]{(O)}{0.2}{0}{45}{anchor=north}{$\varphi$};
%\draw [dashed,color=red] (O) -- (Px);
%\draw [dashed,color=red] (O) -- (Py);
%\draw [dashed,color=red] (O) -- (Pz);
%\draw [dashed,color=red] (Px) node[anchor = south east]{$x$} -- (Pxy);
%\draw [dashed,color=red] (Py) node[anchor = south west]{$y$} -- (Pxy);
%\draw [dashed,color=red] (Px) -- (Pxz);
%\draw [dashed,color=red] (Pz) -- (Pxz);
%\draw [dashed,color=red] (Py) -- (Pyz);
%\draw [dashed,color=red] (Pz) -- (Pyz);
\draw [dashed,color=red] (Pxy) -- (P);
%\draw [dashed,color=red] (Pxz) -- (P);
%\draw [dashed,color=red] (Pyz) -- (P);
%\draw [dashed,color=red] (Pz) node[anchor = east]{$z$} -- (P);
\end{tikzpicture}
\end{figure}

$$
P\rightarrow (r,\theta,\varphi)\ \ \vec{OP} = r\vec{e_r} + \theta\vec{e_\theta} + \varphi \vec{e_\varphi}
$$
con $(\vec{e_r},\vec{e_\theta},\vec{e_\varphi})$ terna levogira

$$
\begin{cases}
x = r\sin\theta\cos\varphi\\
y = r\sin\theta\sin\varphi\\
z = r\cos\theta
\end{cases}
$$
anche in questo caso è possibile utilizzare la funzione MATLAB \verb|sph2cart|.

Formule inverse:
$$
\begin{cases}
r &= \sqrt{x^2+y^2+z^2} \\
\cos\theta &= \frac{z}{\sqrt{x^2+y^2+z^2}}\\
\sin\theta &= \frac{\sqrt{x^2+y^2}}{\sqrt{x^2+y^2+z^2}} \\
\cos\varphi  &= \frac{x}{\sqrt{x^2+y^2}},\ \sin\varphi = \frac{y}{\sqrt{x^2+y^2}}
\end{cases}
$$

\subsection{Spostamenti elementari}
Si supponga uno spostamento lungo la direzione $l_1$ associando un coefficiente ``metrico'' pari 
alla distanza percorsa nel sistema di coordinate.

\begin{align*}
dl_1 &= h_1 du_1 \\
dl_2 &= h_2 du_2 \\
dl_3 &= h_3 du_3
\end{align*}

Questi fattori ``aggiustano'' le dimensioni delle relazioni in metri dovute a variazioni di 
coordinate differenti ad esempio in radianti.

Per le coordinate cartesiane i coefficienti metrici sono tutti uguali tra loro e pari ad 1.
$$
h_1 = h_2 = h_3 = 1 \Rightarrow
\begin{cases}
dl_1 = dx \\
dl_2 = dy \\
dl_3 = dz
\end{cases}
$$
Si suppone di costruire un volumetto elementare attorno il punto $P$, se ne può calcolare
l'area delle facce e il suo volume.
Si indica con $dS_1$ la superficie perpendicolare all'asse $l_1$, essa sarà pari a 
$dS_1 = dl_2\cdot dl_3 = dy\cdot dz$, si riportano per completezza le tre superfici:
\begin{align*}
dS_1 &= dl_2\cdot dl_3 = dy\cdot dz \\
dS_2 &= dl_3\cdot dl_1 = dz\cdot dx \\
dS_3 &= dl_1\cdot dl_2 = dx\cdot dy
\end{align*}
Il volume elementare invece si ricava con:
$$
dV = dl_1\cdot dl_2 \cdot dl_3 = dx\cdot dy\cdot dz
$$

Ripetiamo l'analisi per le \textbf{coordinate cilindriche:}
$ (u_1,u_2,u_3) = (r,\varphi,z)$

Supponiamo uno spostamento associato alla variazione della coordinata $r$, il raggio vettore
viene incrementato di una quantità $dr = dl_1 \Rightarrow h_1 =1$.

Effettuando una variazione $d\varphi$ invece il raggio vettore percorrerà un arco pari 
a $rd\varphi = dl_2 \Rightarrow h_2 = r$ per una variazione di arco ci sarà uno 
spostamento proporzionale alla distanza dall'origine $r$.
\begin{align*}
dr = dl_1 &\Rightarrow h_1 = 1\\
r d\varphi = dl_2 &\Rightarrow h_2 = r \\
dz = dl_3 &\Rightarrow h_3 = 1
\end{align*}
Superfici elementari:
\begin{align*}
dS_1 &= r\cdot d\varphi\cdot dz\\
dS_2 &= dz\cdot dr\\
dS_3 &= r\cdot dr\cdot d\varphi
\end{align*}
Volume infinitesimo:
$$
dV = r\cdot dr\cdot d\varphi\cdot dz
$$

In \textbf{coordinate sferiche} si ha $(u_1,u_2,u_3)=(r,\theta,\varphi)$

Ad una variazione $dr$ si ha uno spostamento lungo il raggio vettore $\vec{OP}$ quindi anche in questo
caso il fattore metrico sarà pari ad 1.

Ad una variazione della variabile $\theta$ detta anche co-latitudine, corrisponde una rotazione
pari a $dl_2 = r\cdot d\theta \Rightarrow h_2 = r$.

Ad una variazione di $\phi$ si ha un arco $dl_3 = r\cdot\sin\theta\cdot d \varphi \Rightarrow h_3 = r\cdot\sin\theta$
\begin{align*}
dl_1 &= dr & h_1 &=1 \\
dl_2 &= r\cdot d\theta & h_2 &=r \\
dl_3 &= r\cdot\sin\theta\cdot d\varphi & h_3 &= r\cdot \sin\theta 
\end{align*}

Superfici infinitesime:
\begin{align*}
dS_1 &= r^2\cdot\sin\theta\cdot d\theta\cdot d\varphi \\
dS_2 &= r\cdot \sin\theta \\
dS_3 &= r\cdot d \theta\cdot dr
\end{align*}
Volume infinitesimo:
$$
dV = r^2\cdot \sin\theta\cdot dr\cdot d\theta\cdot d\varphi
$$

\subsection{Richiami sui campi}
\paragraph{Definizione}
Si intende con campo \textbf{scalare} una funzione $f:\Omega \in R^3 \to R$ con $\Omega$ 
sufficientemente regolare.

Un campo \textbf{vettoriale} invece è una funzione $\vec{v} : \Omega \in R^3 \to R^3$
che può essere espressa mediante 3 campi scalari $v_1(P),v_2(P),v_3(P)$ che sono le componenti
di $\vec{v}(P)$ nel sistema di coordinate scelto.

In coordinate \textit{cartesiane}:
$$
\vec{v}(x,y,z) = v_x(x,y,z)\vec{e_x} + v_y(x,y,z)\vec{e_y} + v_z(x,y,z)\vec{e_z}
$$

In coordinate \textit{sferiche}:
$$
\vec{v}(r,\theta,\varphi) = v_r(r,\theta,\varphi)\vec{e_r} + 
v_\theta(r,\theta,\varphi)\vec{e_\theta} + v_\varphi(r,\theta,\varphi)\vec{e_\varphi}
$$

Richiamiamo per i campi scalari la superficie (curva) di livello per il caso bidimensionale 
(tridimensionale):
$$
f(P) \in C^1(\Omega)
$$
una superficie di livello è definita da:
$$
f(x,y,z) = f_0
$$
In ogni punto $P$ passa una ed una sola superficie di livello.

Nel caso bidimensionale $f(x,y) = f_0$.


Si definisce la \textbf{linea di forza} di un campo vettoriale.
Sia $\vec{v(P)} : \Omega \to R^3$ e sia la linea $\gamma$ tangente in ogni punto a $\vec{v(P)}$,
è per definizione una linea vettoriale di $\vec{v}$.
L'orientamento dei versori tangenti della linea $\gamma$ descrivono direzione e verso di $\vec{v}$ 
(non il modulo).

Tracciamento delle linee vettoriali di un campo $\vec{v}(P)$ assegnato:

Riferendosi alle componenti del campo vettoriale secondo gli assi cartesiani appena definiti
si vede che lo spostamento differenziale $dx$ e $dy$ associato al campo vettoriale soddisfa la 
relazione:
$$
\frac{v_y(P_0)}{v_x(P_0)} = \frac{dy}{dx}(P_0)
$$
Estendendo questo concetto per tutti i punti della linea che si vuole tracciare
possiamo risolvere un problema ai valori iniziali per l'equazione differenziale ordinaria nella
forma:
$$
\begin{cases}
\frac{dy}{dx} = \frac{v_y(x,y,z)}{v_x(x,y,z)} \\
\frac{dz}{dx} = \frac{v_z(x,y,z)}{v_x(x,y,z)}\\
y(x_0) = y_0 \\
z(x_0) = z_0
\end{cases}
$$
Questa equazione restituisce la linea di campo a partire dal punto $P_0$ in forma parametrica
rispetto ad $x$.

Si definisce una \textbf{superficie vettoriale} di $\vec{v}(P)$
$$
S\in \Omega: \hat{n}(P)\cdot\vec{v}(P) = 0\ \forall\ P \in S
$$
Sia data una superficie aperta in cui è contenuto un campo vettoriale $\vec{v}$,
i versori normali $\hat{n}$ saranno ortogonali in ogni punto della superficie $S$ e quindi al 
campo $\vec{v}$.

Il \textbf{tubo di flusso} di un campo $\vec{v}$: 
sia $\Gamma$ una linea chiusa che non sia una linea di forza
sulla quale vengono definiti dei punti, si suppone che ci siano delle linee vettoriali che attraversano
questi punti. Si definisce questo ``tubo'' in modo tale che le linee vettoriali siano tangenti sulla sua faccia laterale.
\newpage
\subsection{Circuitazioni e flussi di campi vettoriali}

\textbf{Circuitazione} (o circolazione) di $\vec{v}$ sulla linea chiusa $\Gamma$, si suppone un punto $Q$ sulla 
linea e il versore tangente $\hat{t}(Q)$ e una linea vettoriale $\vec{v}(Q)$ con un certo angolo
$\alpha$ rispetto a $\hat{t}(Q)$ allora:

$$
\oint_{\Gamma}  \vec{v}\cdot\hat{t}dl = 
\oint_{\Gamma}  \left|\vec{v}(Q)\right|\cdot\cos\alpha\ dl
$$
con $\hat{t}dl$ elemento di linea orientata.

\textbf{Flusso} di $\vec{v}$ uscente da una superficie chiusa $\Sigma$

$$
\oiint_{\Sigma}\vec{v}\cdot\hat{n}\ dS = \oiint_{\Sigma} \left|\vec{v}(Q)\right|\cos\alpha\ dS
$$

\subsection{Operatori differenziali}

\textbf{Gradiente} siano date due superfici di livello $f(P) = f_0 = S_0$ e
$f(P) = f_0 + \Delta f = S_1$, consideriamo la distanza $d(P,P_0)$ con 
$P\in S_1$ lungo la retta normale alla superficie $S_0$ passante per il punto 
$P_0$ ed individuo un punto $P$ su $S_1$, si considera ora il rapporto $\frac{f(P)\cdot f(P_0)}{d(p,p_0)}$ è un rapporto incrementale che con il limite:
$$
\lim_{P\to P_0} \left[\frac{f(P)\cdot f(P_0)}{d(p,p_0)}\right] \stackrel{\text{\text{def}}}{=} \left.\frac{\partial f}{\partial n}\right|_{P_0} 
$$
definisce la derivata parziale lungo la direzione normale passante per il punto $P_0$.

Si definisce il \textbf{gradiente} di $f$ (indicato con $\nabla f$) il vettore con modulo pari a 
$\left.\frac{\partial f}{\partial n}\right|_{P_0}$, verso pari a quello della
normale $n$ nella direzione delle $f$ crescenti.
È quindi possibile definire una generica variazione di $f$ con:
$$
df = \nabla f(P_0)\cdot d\vec{r}
$$
con $\vec{dr}$ uno generico spostamento $\vec{r_q} - \vec{r_{p_0}},\ q\in\Omega$.

\paragraph{Gradiente nei sistemi di coordinate curvilinee}
Si consideri uno spostamento elementare lungo ciascuna delle direzioni coordinate
$dl_1,\ dl_2,\ dl_3$ allora la variazione
$$
df = \nabla f (P_0) \cdot dl_i\vec{e_i} = \nabla f(P_0) \cdot h_i du_i\vec{e_i} 
$$
ma ricordando che il prodotto tra il gradiente e il versore della coordinata i-esima è pari
alla derivata direzionale lungo quella coordinata
$$
\frac{1}{h_i} \frac{\partial f}{\partial u_i} = \left[\nabla f (P_0)\right]\cdot \vec{e_i}
$$
Ad esempio nelle coordinate cartesiane:
$$
\nabla f(P) = \frac{\partial f}{\partial x}\vec{e_x} + \frac{\partial f}{\partial y}\vec{e_y} + 
\frac{\partial f}{\partial z}\vec{e_z}
$$

Oppure nelle coordinate cilindriche:
$$
\nabla f(P) = \frac{\partial f}{\partial r}\vec{e_r} + \frac{1}{r}\frac{\partial f}{\partial \varphi}\vec{e_\varphi} + \frac{\partial f}{\partial z}\vec{e_z}
$$

Coordinate sferiche:
$$
\nabla f(P) = \frac{\partial f}{\partial r}\vec{e_r} + \frac{1}{r}\frac{\partial f}{\partial \theta}\vec{e_\theta} + \frac{1}{r\sin\theta} \frac{\partial f}{\partial \varphi}\vec{e_\varphi}
$$


Si aggiunge inoltre l'operatore \textbf{divergenza} di un campo vettoriale $\nabla\cdot\vec{v}$.

Sia $\Sigma$ una superficie chiusa che delimita una regione di spazio $\Omega_\Sigma$ contenente 
il punto $P\in\Omega$, sia $\hat{n}$ il versore normale alla
superficie, se ne definisce il flusso uscente alla superficie $\Phi_\Sigma$
$$
\oiint_\Sigma \vec{v}\cdot \hat{n}dS = \Phi_\Sigma
$$
Effettuando il limite sul volume 
$$
\lim_{\text{Vol}(\Omega_\Sigma)\to 0} \frac{\oiint_\Sigma \vec{v}\cdot \hat{n}dS}{\text{Vol}(\Omega_\Sigma)}
\stackrel{\text{def}}{=} \nabla\cdot \vec{v}(P)
$$
con la condizione che il volume infinitesimo contenga ancora il punto $P$ ed il limite esista e 
sia finito.

In coordinate cartesiane la divergenza di $\vec{v}(P)$
$$
\nabla\cdot\vec{v}(P) = \frac{\partial v_x}{\partial x} + \frac{\partial v_y}{\partial y} + \frac{\partial v_z}{\partial z}  
$$


\paragraph{Teorema della divergenza}
Il flusso uscente dalla superficie $\Sigma$ è pari all'integrale esteso al volume 
$\Omega_\Sigma$ della divergenza.
$$
\oiint_\Sigma \vec{v}\cdot \hat{n} dS = \iiint_{\Omega_\Sigma} \nabla\cdot \vec{v} d V
$$

\paragraph{Campi solenoidali e indivergenti}

Un campo $\vec{V}(P)$ si dice \textbf{solenoidale} nel dominio $\Omega$ se $\forall$ superficie chiusa $\Sigma$
$$
\oiint_\Sigma \vec{v}\cdot \hat{n} dS = 0\ \ \forall\ \Sigma \in \Omega
$$

Per i campi solenoidali, il flusso attraverso una superficie aperta dipende solo dall'orlo
della superficie (flusso concatenato ad una linea).
$$
\iint_{S_\Gamma'} \vec{v}\cdot\hat{n}dS = \iint_{S_\Gamma''} \vec{v}\cdot\hat{n}dS 
$$
con $\Gamma$ l'orlo delle due superfici $S_\Gamma'$ e $S_\Gamma''$.

Campo \textbf{indivergente}: $\vec{v}(P)$ indivergente in $\Omega$ se $div \vec{v}(P) = 0 \ \forall P \in \Omega$

I campi indivergenti soddisfano il principio di deformazione della superficie:

Si supponga di avere un dominio $\Omega$ contenente due superfici, $\Sigma_1$ contenente $\Sigma_2$
allora:
$$
\iint_{\Sigma_1} \vec{v}\cdot\hat{n}dS = \iint_{\Sigma_2}\vec{v}\cdot\hat{n}dS 
$$
le due superfici sono deformabili con continuità una nell'altra senza mai uscire da $\Omega$.

Determinata la regione di spazio compresa tra le due superfici $\Omega_{\Sigma_1\Sigma_2}$,
effettuando l'integrale della divergenza esso sarà pari a 0 (perché nulla la divergenza) ma per 
il teorema della divergenza esso sarà pari all'integrale sulle 2 superfici:
$$
\iiint_{\Omega_{\Sigma_1\Sigma_2}} \nabla\cdot\vec{v}dV = 0 = \oiint_{\Sigma_1}\vec{v}\cdot\hat{n}dS - \oiint_{\Sigma_2}\vec{v}\cdot\hat{n}dS
$$

Quando la \textbf{solenoidalità} è equivalente alla divergenza?

Se $\vec{v}(P)$ è solenoidale in $\Omega$ allora il flusso è pari a zero in ogni $\Sigma$, $\oiint_\Sigma \vec{v}\cdot\hat{n}dS = 0 \ \forall \Sigma $ allora
anche il flusso sulle superfici utilizzate per calcolare il limite che conduce alla divergenza del campo vettoriale ricade in questa circostanza, quindi $\nabla\cdot\vec{v}$ per $p\to 0$ è pari a 0 in ogni punto
del dominio $\Omega$, quindi la solenoidalità implica l'indivergenza.
L'indivergenza diventa una condizione necessaria per la solenoidalità.

Per dimostrare il viceversa è necessario richiedere un'ipotesi sul dominio, ossia che esso sia
a connessione superficiale semplice (semplicemente connesso)
ogni superficie appartenente ad $\Omega$ può essere ridotta ad un punto con continuità senza uscire da $\Omega$,
allora questa condizione è sufficiente per dire che $\vec{v}(P)$ è solenoidale in $\Omega$.

Un esempio di campo non solenoidale è quello con un dominio multi-connesso, ossia un dominio
al quale viene sottratta un'intera regione di spazio al suo interno.
Se anche la divergenza del campo in un punto di questo dominio è pari a 0, il campo non è 
solenoidale perché l'integrale del flusso non è pari a 0.

Nella regione non inclusa nel dominio potrebbero essere infatti presenti delle cariche
elettriche, e la divergenza non sarebbe nulla.
