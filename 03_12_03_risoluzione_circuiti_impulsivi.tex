
Approfondiamo bene questo concetto!

Circuito RC serie, supponiamo che la tensione imposta sia un gradino unitario,
l'equazione di stato sarà:
$$
\begin{cases}
u(t) = RC \frac{dv_c}{dt} + v_c \\
v_c(0^+) = 0
\end{cases}
\Rightarrow v_c(t) = 
\begin{cases}
0\ ,\ t<0 \\
1-e^{-\frac{t}{\tau}}\ ,\ t\geq 0
\end{cases}
$$

Se si considera la funzione $U_\Delta$ come una funzione che vale $0$ per $t , \frac{-\Delta}{2}$, cresce
linearmente fino ad $1 \text{ con $t$ pari a }\frac{\Delta}{2}$, vale $1$ per $t > \frac{\Delta}{2}$,
allora se si esegue la differenza di due funzioni $U_\Delta$ traslate di $\pm\frac{\Delta}{2}$ si ottiene 
una porta trapezoidale, diventa una porta per $\Delta \rightarrow 0$.

19:32 riprende le formule della volta precedente sulla derivazione del gradino.

\paragraph{Circuiti RC ed RL semplici}
Si prendono 2 circuiti modello: il circuito RC parallelo e il circuito RL serie.

Siano i generatori impulsivi: $j(t) = Q\delta(t)$ e $e(t) = \Phi\delta(t)$.
$$
j(t) = \frac{v_c}{R} + i_c = Q\delta(t) = \frac{v_c}{R} + C\frac{dv_c}{dt}
$$
Si integra la funzione nell'istante in cui è centrata la $\delta(t)$ ossia:
$$
\int_{0^-}^{0^+} Q\delta(\tau)d\tau  = \int_{0^-}^{0^+}\frac{v_c}{R}d\tau + C[v_c(0^+)-v_c(0^-)]
\Leftrightarrow Q = Cv_c(0^+) \ \ [Q] = C \text{ (Coulomb)}
$$
$$
\int_{0^-}^{0^+} Q\delta(\tau)d\tau = \text{ Coulomb}
$$
Ciò significa che la $\delta(t)$ ha la dimensione di \si{\per\second} per essere coerente con l'integrale
e restituire una quantità finita.

Caso duale con circuito RL:

$$
e(t) = R\cdot i_L + v_L
$$
$$
\Phi\delta(t) = R\cdot i_L + L\frac{di_L}{dt}
$$
$$
\Phi\int_{0^-}^{0^+} \delta(\tau)d\tau = L i_L(0^+)\ , \ i_L(0^-) = 0
$$

Anche in questo caso per avere la dimensione del flusso in \si{\weber} per $\Phi$ allora la $\delta(t)$ 
avrà le dimensioni di \si{\per\second}.

\paragraph{Procedura generale per la risoluzione di circuiti con generatori impulsivi}
Si ha un circuito dinamico semplice al quale è collegato un generatore impulsivo,
si determina il circuito resistivo associato, ossia vengono sostituiti i condensatori con 
generatori di tensione e gli induttori con generatori di corrente.

Si ottiene un circuito parziale in cui si spengono i generatori interni (anche quelli equivalenti ai bipoli dinamici) e si lascia agire solo il generatore impulsivo.

Un ulteriore circuito è ottenuto facendo l'esatto contrario e spegnendo quindi il generatore impulsivo.

Si applica il PSE (Principio di Sovrapposizione degli effetti)
$$
\begin{cases}
v_L = v_L' + v_L''\\
i_C = i_C' + i_C''
\end{cases}
$$

Dal circuito C' si ricavano $v_L'$ e $i_C'$ causate dai generatori impulsivi 41:41

Viceversa dal circuito C'' si valutano le variabili di stato utilizzando come condizioni iniziali
le variabili ottenute in C'.

$$
v_C(0^+) = \frac{1}{C} \int_{0^-}^{0^+} i_c'(\tau)d\tau \ , \ v_c(0^-) = 0
$$
$$
i_L(0^+) = \frac{1}{L} \int_{0^-}^{0^+} i_L'(\tau)d\tau \ , \ i_L(0^-) = 0
$$
L'integrale delle variabili di stato del circuito C'' è pari a 0 dato che la funzione integranda è limitata
e l'intervallo è infinitesimo.

Infine si risolve il circuito C'' usando le variabili di stato appena calcolate in C'.

Esercizio 

Conclusione dell'esercizio: le variabili di stato possono essere discontinue ma limitate, le altre
variabili possono invece essere anche impulsive.

\paragraph{Risposta forzata di un circuito LTI}
Si suppone di avere un solo generatore esterno, se ne vogliono determinare le variabili ai capi
di un solo bipolo, si parla di filtro o sistema SISO \textit{(Single Input Single Output)},
sia $x(t)$ l'ingresso e $y(t)$ l'uscita, si deve supporre che il circuito sia a stato 0, ossia
le sue variabili di stato sono tutte nulle (circuito precedentemente spento).

Si parla di approssimazione \textit{PieceWise-Constant} ossia costante a tratti dell'ingresso
$x(t)$.
Si partiziona l'asse dei tempi in tanti intervalli di ampiezza $\Delta t$ centrati negli istanti
di tempo $t_k = k\Delta t\ ,\ k\ \in\ Z$.

Costruiamo un'approssimazione $x_\Delta(t)$ costante di valore $x(t_k)$ in $\left[t_k-\frac{\Delta t}{2}\ ,\ 
t_k+\frac{\Delta t}{2}\right]$.
La funzione $x(t)$ può quindi essere rappresentata come somma di impulsi rettangolari successivi,
sfruttando la funzione $\Pi_{\Delta}(t)$, sarà quindi:
$$
x_{\Delta}(t) = \sum_{k = -\infty}^{+\infty} x(t_k) \Pi_\Delta(t-t_k)\Delta t
$$
Eseguendo il limite per $\Delta t \rightarrow 0^+$ in ipotesi di sufficiente regolarità:
$$
\lim_{\Delta t \to 0^+} x_\Delta(t) = x(t) = \int_{-\infty}^{+\infty} x(\tau) \delta (t-\tau)
d\tau
$$
Questa conclusione richiama la proprietà di campionamento della $\delta(t)$.

Se vogliamo calcolare la risposta di $x(t)$ allora:
$$
\text{Risp}\left\{x(t)\right\} = y(t) = \text{Risp} \left\{\int_{-\infty}^{+\infty} x(\tau)\delta(t-\tau)
d\tau\right\}
$$
per linearità e tempo-invarianza si ottiene
$$
y(t) = \int_{-\infty}^{+\infty} x(\tau)h(t-\tau) d\tau
$$
chiamato integrale di convoluzione, $h(t)$ è la risposta impulsiva per la variabile di uscita.

Si parla di prodotto di convoluzione tra due funzioni $f,g \ \in\ R$
$$
f * g(t) = \int_{-\infty}^{+\infty} f(\tau)g(t-\tau)d\tau
$$
Si può calcolare la risposta impulsiva... rivedi 1:57:00
