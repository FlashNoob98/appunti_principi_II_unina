\subparagraph{Problema generale dell'elettrostatica}

Siano $N$ conduttori carichi $c_1,c_2,\ldots ,c_i$ e le rispettive normali alle superfici
$\hat{n}_i$, ogni conduttore carico avrà un potenziale $V_i$. Trovare il potenziale $V(P)$ 
dato dagli stessi e le cariche contenute.
Supponiamo che $V(P) = V_k$ in $c_k$ e su $\partial c_k$.
La carica contenuta dal conduttore $k$, $Q_k$ è data da

$$
Q_k = \int_{\partial_{C_k}} \varepsilon_0 \vec{E}\cdot\hat{n}_k dS
$$

Sia $\Omega = \mathbb{R}^3 - c_1 - c_2 - \ldots  - c_k$ la regione esterna che non contiene i
conduttori, si ha un problema di Dirichelet esterno per l'Equazione di Laplace.

\begin{equation}
\begin{cases}
\nabla^2 V = 0 & \text{in } \Omega\\
V|_{\partial_{C_k}} = V_k & k=1,\ldots ,N \\
\lim_{P\to \infty}V(P)=0 & \text{normale all' } \infty
\end{cases}
\label{eq:problema_dirichelet}
\end{equation}
In queste condizioni sappiamo che la soluzione esiste ed è unica.
La soluzione generale si costruisce con il PSE considerando N BVPs ausiliari:
$$
\begin{cases}
\nabla^2 V^{(i)} = 0 &\text{in } \Omega\\
V^{(i)}|_{\partial_{C_i}} = 1,\ v^{(i)}|_{\partial_{C_k}} = 0 &  k \neq i
\end{cases}
$$
Anche le soluzioni per queste equazioni sonno univocamente determinate per i teoremi di unicità
delle equazioni di Laplace.
La soluzione generale del problema \ref{eq:problema_dirichelet} sarà:
\begin{align*}
&V(P) = V_1V^{(1)}(P) + V_2V^{(2)}(P) + \ldots  + V_NV^{(N)}(P) \\
& \nabla^2 V = 0 \ \ \text{ in }\Omega
\end{align*}
I laplaciani sono tutti pari a 0 quindi anche la loro somma sarà pari a 0.
Possiamo ora considerare la carica depositata sul conduttore $C_i$:
$$
Q_i = \int_{\partial_{C_i}} \varepsilon_0 \vec{E} \cdot \hat{n}_i dS = \int_{\partial_{C_i}} -\varepsilon_0 \frac{\partial V}{\partial n_i} dS = \left(\int_{\partial_{C_i}} - \varepsilon_0 \frac{\partial V}{\partial n_i}^{(1)} dS\right)V_1 + \ldots  + \left(\int_{\partial_{C_i}} - \varepsilon_0 \frac{\partial V}{\partial n_i}^{(N)} dS\right)V_N
$$
Gli integrali $i$-esimi vengono chiamati coefficienti di capacità ed esprimono il contributo
di carica fornito al conduttore $i$-esimo per mezzo del potenziale $N$-esimo.
$$
[C_{ij}] = \frac{\si{\coulomb}}{\si{\volt}} = \si{\farad}
$$
In un generico sistema di $N$ conduttori, le cariche si possono esprimere mediante
un'applicazione lineare in $\mathbb{R}^N$
$$
\begin{cases}
Q_1 = C_{11}V_1 + \ldots + C_{1N}V_N \\
\vdots \\
Q_n = C_{n1}V_1 + \ldots + C_{nn}V_N
\end{cases}
$$

Il coefficiente $C_{ii}$ è la carica accumulata sul conduttore $c_i$ quando $V_i=1$ e $V_k = 0$
per $k\neq i$.
$$
\begin{aligned}
C_{ii} &= \iint_{\partial c_i} -\varepsilon_0\frac{\partial V^{(i)}}{\partial n_i} dS\\
C_{ij} &= \iint_{\partial c_i} -\varepsilon_0\frac{\partial V^{(j)}}{\partial n_i} dS & V_j = 1,\ 
V_k = 0,\ k\neq j
\end{aligned}
$$

Questi coefficienti soddisfano varie proprietà come:

\begin{description}
 \item[Reciprocità] Come $C_{ij} = C_{ji} $
 \item[Proprietà 2] $c_{ij} < 0 $, $c_{ii}>0$
 \item[Simmetria e dominanza diagonale] $C_{ii} > \sum_{i\neq j}|C_{ij}|$
\end{description}
La proprietà 2 si dimostra rappresentando due conduttori, il conduttore $j$ a 
potenziale unitario mentre il conduttore $i$ a potenziale nullo, per l'armonicità della funzione
potenziale tutte le linee di campo saranno uscenti dal conduttore $j$ ed entranti nel conduttore 
$i$ quindi $\vec{E}\cdot\hat{n}_i < 0$.
$$ %matrice dei coefficienti di capacità \underline{C}
\begin{bmatrix}
Q_1\\
\vdots \\
Q_N
\end{bmatrix} = 
\begin{bmatrix}
C_{11} & \ldots  & C_{1N} \\
C_{21} & \ldots  & C_{2N} \\
\vdots & \ldots  & \vdots \\
C_{N1} & \ldots  & C_{NN}
\end{bmatrix} \cdot
\begin{bmatrix}
V_1 \\
\vdots \\
V_N
\end{bmatrix}
$$

\textbf{Esempio} Elettrodi sferici a \textit{``grande''} distanza

Siano $P_1$ e $P_2$ i centri di una coppia di sfere conduttrici di raggio $r_1$ ed $r_2$ con potenziali $V_1$ e
$V_2$ e cariche $Q_1$ e $Q_2$.
Si ipotizza che la distanza $|P_1P_2|$ sia sufficientemente grande da rendere trascurabile il potenziale
prodotto da una sfera in corrispondenza dell'altra ossia facendo in modo che non si \textit{``vedano''} 
elettrostaticamente, ossia i loro potenziali e distribuzioni di cariche non si influenzino a vicenda.

Si sa inoltre che il potenziale di una sfera carica è:
$$
V = \frac{Q}{4 \pi \varepsilon_0 R}
$$
il calcolo si può effettuare per entrambe le sfere ottenendo:
$$
Q_1 = 4 \pi \varepsilon_0 R_1 V_1,\ Q_2 = 4 \pi \varepsilon_0 R_2 V_2
$$

Supponiamo che $Q_1 = -Q_2$ (fenomeno di induzione completa)
$$
-Q_1 = 4 \pi \varepsilon_0 R_2 V_2 \Rightarrow V_1 - V_2 = \frac{Q_1}{4 \pi \varepsilon_0}
\left(\frac{1}{R_1} + \frac{1}{R_2}\right)
$$
Si definisce la capacità tra i due elettrodi:
$$
\frac{Q_1}{V_1 - V_2} = C = \frac{4 \pi \varepsilon_0}{\frac{1}{R_1}+\frac{1}{R_2}}
$$
In generale quando $Q_1 = - Q_2$ si può definire una capacità tra i due elettrodi che operativamente
può essere scritta come 
$$
C = \frac{\iint_{\partial c_1} \varepsilon_0 \vec{E}\cdot\hat{n}dS} {\int_{P_1}^{P_2}\vec{E}\cdot\hat{t} dl} = \frac{Q_1}{V_1 - V_2}
$$


\textbf{Caso generale}: la carica $Q_i$ dipende da tutti i potenziali $V_1, \ldots ,V_N$:
$$
Q_1 = C_{11}V_1 + C_{12}V_2 + \ldots + C_{1N}V_N = C_{11}V_1 + C_{12}V_2 + C_{12}V_1 - C_{12} V_1 +
\ldots  + C_{1N}V_N
$$
$$
Q_1 = \left(C_{11}+C_{12} + \ldots + C_{1N}\right) V_1 + (-C_{12})(V_1-V_2) + (-C_{13})(V_1-V_3)
+ \ldots + (-C_{1N})(V_1-V_N)
$$
La prima parentesi viene sostituita da $C_{11}^*$, $(-C_{12}) = C_{12}^*$ fino a $C_{1N}^*$,
vengono chiamate
\textit{capacità parziali} del sistema di conduttori, per evidenziare le capacità 
dovute all'interazione dei conduttori a due a due.
Queste definiscono nella maggior parte dei casi anche gli accoppiamenti parassiti
tra diverse capacità presenti in un circuito.

Alcune proprietà:
$$\begin{aligned}
C_{ii}^* &\stackrel{\text{def}}{=} \left.\frac{Q_i}{V_i}\right|_{V_k = 1}\ k=1\ldots ,n\\
C_{ij}^* &\stackrel{\text{def}}{=} \left.\frac{Q_i}{V_j}\right|_{V_j=1}\ V_k = 0,\ k \neq j
\end{aligned}
$$
Si può mostrare che $C_{ii}^* =C_{ii} + \sum_{j\neq i}C_{ij} \geq 0,\ C{ij}^* = -C_{ij} > 0$

Caso in cui $N=2$:
$$
\begin{aligned}
Q_1 &= C_{11} V_1 + C_{12}V_2 = C_{11}V_1 + C_{12}V_2 + C_{12}V_1 - C_{12}V_1 = (C_{11}+C_{12})V_1 - C_{12}(V_1-V_2)\\
Q_2 &= C_{21}V_1 + C_22 V_2 = C_{21}^*(V_2-V_1) + C_{22}^*(V_2-\cancel{V_{\infty}})
\end{aligned}
$$
Supponiamo che ci sia induzione completa: $Q_1 = - Q_2$
$$\begin{aligned}
Q_1 &= C_{11}^*V_1 + C_{12}^*(V_1-V_2) \\
-Q_1 &= C_{21}^*(V_2-V_1) + C_{22}^*V_2
\end{aligned}
$$
Sviluppando le due equazioni si ottiene:
$$\begin{aligned}
\frac{Q_1}{C_{11}^*} &= V_1 + \frac{C_{12}^*}{C_{11}^*}(V_1-V_2)\\
\frac{-Q_1}{C_{22}^*} &= \frac{C_{21}^*}{C_{22}^*}(V_2-V_1) + V_2
\end{aligned}
$$
Sottraendo le due equazioni si ottiene:
$$\begin{aligned}
&Q_1\left(\frac{1}{C_{11}^*}+\frac{1}{C_{22}^*}\right) = (V_1-V_2) + (V_1-V_2)\left(\frac{C_{12}^*}{C_{11}^*} + \frac{C_{21}^*}{C_{22}^*} \right) = (V_1 - V_2)\left[1+C_{12}^*\left(\frac{1}{C_{11}^*}+\frac{1}{C_{22}^*}\right)\right]\\
&\frac{Q_1}{V_1-V_2} = C = \frac{1 + C_{12}^*\left(\frac{1}{C_{11}^*} + \frac{1}{C_22}^*\right)}
{\frac{1}{C_{11}^*}+\frac{1}{C_{22}^*}} = C_{12}^* + \left(\frac{1}{C_{11}^*} + \frac{1}{C_{22}^*}\right)^{-1} = C
\end{aligned}
$$
Questa equazione suggerisce fisicamente che la capacità totale sia pari al parallelo della
capacità $C_{12}^*$ tra i due conduttori e la serie di due capacità con un punto in comune
all'infinito rappresentanti le capacità dei singoli conduttori rispetto a questo punto.

Per realizzare un condensatore che non sia influenzato dalla presenza di conduttori esterni
si sfrutta la proprietà di schermo elettrostatico, si realizza un'armatura in modo da contenere
l'altra in modo da mantenere pari a zero una delle due capacità rispetto all'infinito.
Un modo è quindi quello di chiudere un conduttore $C_1$ all'interno dell'altro $C_2$.
La capacità parziale del primo conduttore è nulla perchè il campo all'interno del conduttore
più grande è nullo, di conseguenza, per la legge di Gauss anche la carica contenuta nel conduttore
1 sarà pari a 0.
$$\begin{aligned}
& \oiint_{\Sigma}\vec{E}\cdot\hat{n}dS = 0\ \ \forall \ \Sigma\\
&C_{11}^* = \left.\frac{Q_1}{V_1}\right|_{V_k = 1} = 0
\end{aligned}
$$
Quindi 
$$
\begin{aligned}
Q_1 &= C_{12}^*(V_1-V_2) \\
Q_2 &= C_{21}^*(V_2-V_1) + C_{22}^*V_2
\end{aligned}
$$
Di conseguenza la capacità $C$ sarà pari a $C_{12}^*$
$$
\begin{aligned}
Q_1 &= C(V_1-V_2) \\
Q_2 &= -C(V_1-V_2) + C_{22}^*V_2 = -Q_1 + C_{22}^*V_2
\end{aligned}
$$
La carica netta del sistema $C_1 \cup C_2$ sarà:
$$
Q_1+Q_2 = C_{22}^*V_2
$$
interamente distribuita sulla superficie esterna.

\subparagraph{Calcolo capacità condensatore piano}
Siano due lastre indefinite poste ad una distanza $d$ e con cariche opposte
$\sigma$ e $-\sigma$, supposta $\sigma$ positiva il campo elettrico è orientato dalla
lastra carica positivamente a quella negativa, di intensità pari a 
$$
\vec{E} = \frac{\sigma}{\varepsilon_0} \vec{e}_x
$$
Per calcolare la tensione tra le due armature e quindi la capacità si procede con un integrale di 
linea:
$$
V_1-V_2 = \int_{-\frac{d}{2}}^{\frac{d}{2}} \frac{\sigma}{\varepsilon_0} dx = \frac{\sigma}{\varepsilon_0} d
$$
Ricordando che $Q_1 = \sigma S$ 
$$
C = \frac{Q_1}{V_1-V_2} = \frac{\sigma S}{\varepsilon_0 d}
$$
Se $\frac{\sqrt{S}}{d} \gg 1 $ si può approssimare un condensatore piano finito con un modello
ideale che non tiene in considerazione gli effetti di bordo.

\subparagraph{Condensatore sferico}
Costituito da due sfere concentriche di raggio $R_1$ ed $R_2$
si può calcolare il potenziale per sovrapposizione:
$$
V_1 = V(R_1) = \frac{Q_1}{4\pi \varepsilon_0 R_1},\ \ V_2 = V(R_2) = \frac{Q_2}{4 \pi \varepsilon_0 R_2}
$$
La capacità
$$
C = \frac{Q_1}{V_1 - V_2} = \frac{\cancel{Q_1}}{\frac{\cancel{Q_1}}{4 \pi \varepsilon_0}\left(\frac{1}{R_1}-\frac{1}{R_2}\right)} = \frac{4 \pi \varepsilon_0}{\frac{1}{R_1}-\frac{1}{R_2}} = 4 \pi \varepsilon_0 \frac{R_1 R_2}{R_2 - R_1}
$$

\subparagraph{Condensatore cilindrico}
Costituito da due cilindri coassiali di raggi $R_1$ e $R_2$, si applica nuovamente il PSE
$$\begin{aligned}
V(r) &= -\frac{\sigma_1}{\varepsilon_0}R_1 \ln r - \frac{\sigma_2}{\varepsilon_0}R_2 \ln R_2\ \ R_1\leq r \leq R_2\\
V(R_1) &= V_1 = -\frac{\sigma_1}{\varepsilon_0} R_1 \ln R_1 - \frac{\sigma_2}{\varepsilon_0}R_2 \ln R_2\\
V(R_2) &= V_2 = -\frac{\sigma_1}{\varepsilon_0} R_1 \ln R_2 - \frac{\sigma_2}{\varepsilon_0}R_2 \ln R_2\end{aligned}
$$
Eseguendo la differenza dei potenziali
$$
V_1 - V_2 = -\frac{\sigma_1}{\varepsilon_0} R_1 \ln R_1 + \sigma_1\frac{R_1}{\varepsilon_0}\ln R_2
= \frac{\sigma_1 R_1}{\varepsilon_0}\ln\left(\frac{R_2}{R_1}\right) = \frac{2 \pi R_1 L \sigma_1}{2 \pi L \varepsilon_0}\ln \left(\frac{R_2}{R_1}\right)
$$
il numeratore è proprio $Q_1$
$$
C = \frac{Q_1}{V_1-V_2} = \frac{2 \pi \varepsilon_0 L }{\ln \left(\frac{R_2}{R_1}\right)}
$$

\subparagraph{Capacità parziali di due elettrodi sferici}
Presi due elettrodi a distanza $d$ di raggio $R_1$ ed $R_2$, supponendo che $d \gg R_1,R_2$
in maniera tale da supporre che il potenziale generato da una sfera sia approssimativamente
costante nei punti occupati dall'altra, confondiamo cioè il valore del potenziale sulla
superficie della sfera con quello al suo centro.

Applicando il PSE..

\begin{align*}
V_1' &= \frac{Q}{4 \pi \varepsilon_0 R_1}& V_2' &\simeq \frac{Q_1}{4 \pi \varepsilon_0 d}\\
V_1'' &\simeq \frac{Q_2}{4 \pi \varepsilon_0 d}& V_2'' &= \frac{Q_2}{4 \pi \varepsilon_0 R_2}
\end{align*}

$$
\begin{aligned}
V_1 &= V_1' + V_1'' = \frac{Q_1}{4\pi\varepsilon_0 R_1} + \frac{Q_2}{4 \pi \varepsilon_0 d} \\
V_2 &= V_2' + V_2'' = \frac{Q_1}{4\pi\varepsilon_0 d} + \frac{Q_2}{4 \pi \varepsilon_0 R_2}
\end{aligned}
$$
$$
\begin{bmatrix}
V_1 \\
V_2
\end{bmatrix} = \frac{1}{4\pi\varepsilon_0} \begin{bmatrix}
\frac{1}{R_1} & \frac{1}{d}\\
\frac{1}{d} & \frac{1}{R_2}
\end{bmatrix}\cdot
\begin{bmatrix}
Q_1\\
Q_2
\end{bmatrix}
$$
