
Funzioni generiche $t\cdot f(t)$ con $f(t)$ le funzioni precedentemente analizzate.
Esponenziale:
$$
L[te^{\lambda t}] = \int_{0^-}^{+\infty} t e^{\lambda t}e^{-s t} dt = \int_{0^-}^{+\infty}t e^{(\lambda -s)t}
dt = 
$$
$$
= \left[\frac{e^{(\lambda -s )t }}{\lambda-s}\cdot t\right]_{0^-}^{+\infty} - \int_{0^-}^{+\infty}1\cdot\frac{1}{\lambda -s}e^{(\lambda-s)t}dt = 
$$
$$
= 0 + \frac{1}{s-\lambda}\cdot\frac{1}{s-\lambda} = \frac{1}{s-\lambda} ,\ \Re\{s\} > \lambda
$$

Coseno:
$$
L[t\cos(\omega t)] = L\left[t \frac{e^{j\omega t}+e^{-j\omega t}}{2}\right] = \frac{1}{2}\frac{1}{(s-j\omega)^2} + \frac{1}{2}\frac{1}{(s+j\omega)^2} = 
$$
$$
= \frac{1}{2}\frac{1}{s^2-\omega^2-2j\omega s} + \frac{1}{2}\frac{1}{s^2-\omega^2+2j\omega s} =
\frac{1}{2}\frac{s^2-\omega^2+2j\omega s +s^2 -\omega^2 -2j\omega s}{(s^2-\omega^2)^2+4\omega^4s^2} =
$$
$$
= \frac{s^2-\omega^2}{(s^2+\omega^2)^2}
$$

Seno:
$$%%Rivedi il rigo qui sotto
L[t\sin(\omega t)] = \frac{1}{2j} \left[\frac{1}{(s-j\omega)^2}-\frac{1}{s+j\omega}^2\right] =
\frac{1}{2j}\left[\frac{s^2-\omega^2+2j\omega s-s^2+\omega^2+2j\omega s}{(s^2-\omega^2)^2+4\omega^2s^2}\right] = 
$$
$$
= \frac{2\omega s}{(s^2+\omega^2)^2}
$$
Funzioni che descrivono moti periodici smorzati, sfruttando la proprietà di traslazione
$f(t) = e^{-\sigma t}\cos(\omega t),\  e^{-\sigma t}\sin(\omega t)$:
$$
L[e^{-\sigma t}\cos(\omega t)] = L[\cos(\omega t)](s+\sigma) = \frac{s+\sigma}{(s+\sigma)^2+\omega^2}\ \Re\{s\} > -\sigma
$$
$$
L[e^{-\sigma t}\sin(\omega t)] = L[\sin(\omega t)](s+\sigma) = \frac{\omega}{(s+\sigma)^2+\omega^2}\ \Re\{s\} > -\sigma
$$
\paragraph{Circuito RL serie con L-trasformata}
Sia dato il circuito RL serie, se ne voglia determinare la risposta impulsiva $h(t) = i_L(t)$ dato l'ingresso
$e(t) = \delta(t)$
$$
\begin{cases}
\delta(t) = R\cdot i_L + L\frac{di_L}{dt} \\
i_L(0^-) = 0
\end{cases}
$$
Si applica la trasformata di Laplace ad entrambi i membri della prima equazione, sfruttando anche
la proprietà di derivazione:
$$
1 = RI_L(s) + L(sI_L(s) - i_L(0^-))
$$
$$
L[i_L(t)] = I_L(s) = \frac{1}{R+sL} = \frac{1}{L}\frac{1}{\frac{R}{L}+s}
$$
L'antitrasformata sarà invece pari a:
$$
L^{-1}[I_L(s)] = \frac{1}{L} L^{-1}\left[\frac{1}{S+\frac{R}{L}}\right] = \frac{1}{L}e^{-\frac{t}{\tau}},\ 
\tau = \frac{L}{R}
$$

Si vede ora la risposta al gradino, $e(t) = u(t)$:
$$
u(t) = R\cdot i_L+ L\frac{di_L}{dt} \rightarrow \frac{1}{s} = RI_L + sLI_L 
$$
$$
I_L = \frac{1}{s(R+sL)} = \frac{A}{s} + \frac{B}{R+sL} = \frac{AR+sLA+sB}{s(R+sL)}
$$
Si ricavano i valori di $A$ e $B$ sfruttando il principio di identità dei polinomi:
$$
LA+B = 0,\ AR = 1
$$

$$
A = \frac{1}{R},\ B = -\frac{L}{R}
$$
In conclusione:
$$
I_L(s) = \frac{1}{R}\left[\frac{1}{s}-\frac{L}{R+SL}\right] = \frac{1}{R}\left[\frac{1}{s}-
\frac{L}{L\left(\frac{R}{L}+s\right)}\right]
$$
Antitrasformando si ottiene:
$$
i_L(t) = \frac{u(t)}{R} - \frac{1}{R}e^{-\frac{R}{L} \text{eccetera}}
$$

\paragraph{Equazioni circuitali nel dominio della L-trasformata}
Si considera ancora la classe dei circuiti dinamici lineari tempo-invarianti (LTI),
nel dominio del tempo sono necessarie le LKT e le LKC, le caratteristiche dei bipoli e dei generatori.
Questo sistema è un DAE, va trasformato in un ODE nelle variabili di stato per poter essere risolto.

Se trasformo tutte le equazioni che provengono dalle leggi di Kirchoof ottengo ancora le stesse equazioni
ma nel dominio di Laplace:
$$
\sum_{k} (\pm)I_k(s) = 0
$$
$$
\sum_k (\pm)V_k(s) = 0
$$
Ancora le equazioni dei bipoli:
$$
V_R(s) = RI_R(s)
$$
$$
V_L(s) = sLI_L(s) - Li_L(0^-)
$$
$$
I_C(s) = sCV_C(s) - Cv_c(0^-)
$$

Per trattare i circuiti in evoluzione forzata minuto(46) si può associare un'impedenza equivalente
ai bipoli, ad esempio l'impedenza operatoriale dell'induttore diventa
$$
Z_L(s) = sL,\ V_L(s) = Z_LI_L(s)
$$
$$
Z_C = \frac{1}{sC}
$$
Questo risultato ci permette di utilizzare tutti i metodi precedentemente visti per la risoluzione
di un circuito, si esegue un'antitrasformazione alla fine per ritornare nel dominio del tempo.
Abbiamo posto però l'ipotesi di trovarsi in evoluzione forzata, ossia supponendo uno stato iniziale 
nullo, cosa accade se ciò non è vero?

Stato non zero iniziale:
$$
V_L(s) = sLI_L - L i_L(0^-)
$$
$$
V_L(s) = Z_LI_L+E_0
$$
Si modella il bipolo dinamico con un generatore di tensione in serie, impulsivo.
Si può fare un ragionamento simile con il condensatore al quale si affianca un generatore di corrente
in parallelo $J_{cc}$
$$
I_C(s) = sCV_C(s) - Cv_C(0^-)
$$
$$
J_{cc} = Cv_C(0^-)
$$

\paragraph{Funzione di trasferimento e legame con la risposta impulsiva}
Sia preso un generico circuito LTI, viene forzato con un generatore di tensione, se ne analizzano
le grandezze su un singolo bipolo interno al circuito, considerato quindi come un SISO $h(t)$,
l'uscita del sistema può essere calcolata mediante l'integrale di convoluzione della risposta impulsiva.
$$
y(t) = \int_{0^-}^{t} x(\tau)h(t-\tau)d\tau
$$
Si trasporta il fenomeno nel dominio di Laplace e si considera lo stesso circuito sostituito dai bipoli
operatoriali, si può ancora considerare il sistema come un SISO con ingresso pari a $X(s)$ e uscita
pari a $Y(s)$.
Questo circuito è a-dinamico lineare con un solo generatore, quindi tutte le grandezze sono proporzionali
al singolo forzamento, posso quindi affermare che $Y(s) = H(s)X(s)$ con $H(s)$ un coefficiente di 
proporzionalità.

$H(s)$ viene ricavato, mediante il teorema di Borel:
$$
L[f*g(t)] = F(s)\cdot G(s)
$$
$$
y(t) = \int_{0^-}^{t} x(\tau)h(t-\tau) d\tau \Rightarrow H(s) = L[h(t)]
$$
$H(s)$ si chiama quindi \textit{funzione di trasferimento} del circuito ed è definita come il rapporto
tra la trasformata dell'uscita e quella dell'ingresso e coincide con la trasformata della risposta
impulsiva del circuito:
$$
H(s) \stackrel{\text{def}}{=} \frac{Y(s)}{X(s)} = L[h(t)]
$$

Esempio di applicazione, circuito LC forzato in risonanza:

Si ha un generatore ideale di tensione, un induttore ideale e un condensatore ideale senza perdite,
questo circuito rappresenta il limite ideale 
La risonanza è associata ad una specifica pulsazione pari a:
$$
\omega_r = \frac{1}{\sqrt{LC}}
$$
e il forzamento è pari ad un segnale con pulsazione pari alla pulsazione di risonanza
$$
e(t) = E_m \sin(\omega_r t)
$$
Il circuito non può essere analizzato con il metodo dei fasori dato che non è dissipativo e l'energia 
immagazzinata nei bipoli dinamici non tende a 0 in evoluzione libera per $t\to \infty$.

Si utilizza quindi l'analisi nel dominio di Laplace:
$$
I(s) = \frac{E(s)}{sL+ \frac{1}{SC}}
$$
ma
$$
E(s) = \frac{\omega_r}{s^2+\omega_r^2} = \frac{\omega_r}{s^2+\frac{1}{LC}} = L[e(t)]
$$
$$
I(s) = \frac{E_m\omega_r s C}{(s^2+\frac{1}{LC})(s^2+\frac{1}{LC})} = \frac{E_m\omega_r\frac{sC}{LC}}{(s^2+\frac{1}{LC})^2} = \frac{E_m\omega_r}{L} \frac{s}{(s^2+\frac{1}{LC})^2}
$$
Riprendendo la trasformata del seno:
$$
L[t\sin(\omega_r t)] = \frac{2\omega s}{(s^2+\omega^2)^2}
$$
riprendendo il calcolo di $I(s)$:
$$
= \frac{E_m\omega_r}{2L\omega_r}\cdot \frac{2\omega_r s}{(s^2+\omega_r^2)^2} = \frac{E_m}{2L}\frac{2\omega_r s}{(s^2+\frac{1}{LC})^2}
$$
Antitrasformando:
$$
i(t) = L^{-1}[I(s)] = \frac{E_m}{2L} t \sin(\omega_r t)
$$

\newpage
\paragraph{Calcolo della funzione di trasferimento per un circuito del secondo ordine}
Circuito minuto 1:40:00
Siamo interessati come uscita alla $v(t)$ dato l'ingresso $e(t)$, ci trasferiamo ancora una volta dal 
dominio del tempo a quello di Laplace.
Si assegnano i parametri del circuito:
$R_1 = \SI{5}{\ohm}\ R_2=R_3=\SI{10}{\ohm}\ C =\SI{0.5}{\farad}\ L=\SI{1}{\henry} $
Sostituiamo i bipoli con impedenze, troviamo la $Z_{eq}$ pari a $5 + s$ in serie con $\frac{1}{0.2+0.5s}$
La tensione in uscita sarà la partizione della tensione in ingresso tra queste due impedenze:
$$
V(s) = E(s)\frac{Z_{eq}^{(2)}}{Z_{eq}^{(2)}+Z_{eq}^{(1)}} = \frac{1}{(0.2+0.5 s)(\frac{1}{0.2+0.5 s}+5+s)} = \frac{1}{2+2.75+0.5s^2}
$$
Si trovano gli zeri del polinomio, saranno due radici reali e distinte,
la $H(s)$ sarà quindi:
$$
H(s) = \frac{k_1}{s+0.886} + \frac{k_2}{s+4.514} = \frac{k_1 5 +4.514 k_1 + k_2 s + 0.886 k_2 }{s^2+5.45+4} =
$$
$$
= \frac{2}{(s^2+5s+4)},\ 
\begin{cases}k_1+k_2 = 0\\
4.514k_1 + 0.886k_2 = 2
\end{cases}
$$
$$
k_1 = \frac{2}{4.514-0.886} = 0.551 = -k_2
$$
In conclusione 
$$
H(s) = \frac{0.551}{s+0.886} - \frac{0.551}{s+4.514}
$$
antitrasformando:
$$
L^{-1}[H(s)] = 0.551\left(e^{-0.886 t}-e^{-4.514 t}\right) = h(t)
$$
